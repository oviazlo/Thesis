\chapter{Physics of the Standard Model and Beyond}
\label{chap:Theory}

%% Restart the numbering to make sure that this is definitely page #1!
\pagenumbering{arabic}

%% Note that the citations in this chapter use the journal and
%% arXiv keys: I used the SLAC-SPIRES online BibTeX retriever
%% to build my bibliography. There are also quite a few non-standard
%% macros, which come from my personal collection. You can have them
%% if you want, or I might get round to properly releasing them at
%% some point myself.

\section{Introduction to theory chapter}

% physics --> trying to describe everything with laws and models from fundametal principa and ideally we want to create an absolute model which will describe everything.

Particle physics as any field of physics is based on the two pillars: experiment and theory.
The aim of the particle physics is to construct a universal theory which will explain behavior of the universe.
To do so one need to have an understanding of the compositeness of the universe and the laws how these components interact between themselves.
In order to do so one need to have possibility to quantify observed effect by making precise measurements.
By collecting enough experimental results one aim to construct a model which will be able to describe all measurements done.
More different measurements model is able describe more confidence one can have that model works and that it can be used to predict 
unmeasured effects yet. This was the case with so-called Standard Model (SM) which became extremely popular in last century because
it was able to describe hundreds of new observed particle in the particle colliding experiments.
The SM describes with remarkable precision three types of particle interactions with only gravitational type left undescribed.
Thus a big effort is ongoing to try to incorporate the gravity into the SM to obtain an complete model.

However, despite the fact that SM agrees amazingly well with experimental evidences and even if one will not take gravity into account 
there are are many reasons to think that SM is not complete.
These reasons are based on set of unsolved questions which SM cannot address (e.g. neutrino masses, dark matter, matter-antimatter asymmetry).
Because all known elementary particles fits well in the SM, thus problems make us believe that there are
potentially new physics (and new particles which correspond to it) in TeV or above-TeV regimes which can solve these problems.
Start of the Large Hadron Collider (LHC) provide us possibility to look inside the TeV energy frontier.
This is why many analysis are focused to investigate new energy regime and look for possible deviations from the SM which can be hints to new physics.
Since SM was well established more than 30 years ago many theorists spent tremendous amount of time to create
plethora of the models which extend SM and address multiple unsolved problems. These models can predict final signatures and selections which are most sensitive to 
the possible new physics and motivate strategies of the searches.

This chapter contain brief description of the SM model, it's problems and consider a few of Beyond Standard Model (BSM) models which can address some unsolved SM questions.

\section{Standard Model}

\subsection{Elementary particles}

% *****************
% INTRO
% *****************

The SM is a very successful model which describes all the known particles in existence to a remarkable degree of accuracy.
All of the SM particles are fundamental particles (have no internal structure) that make up the matter and forces in the universe.
The elementary particles can be classified into two groups: 
\begin{itemize}
 \item the spin-1/2 fermions which obey Fermi-Dirac statistics
 \item the bosons which obey Bose-Einstein statistics and have an integer spin values.
\end{itemize}

% *****************
% FERMIONS
% *****************
Matter in the universe is made up from fermions, which are classified into three generations.
Each generation consists from two electromagnetically charged quarks and one lepton as well as an associated neutral neutrino.
Fermions from different generations have the same charges but differ by the mass.
One quark from the generation have charge +2/3$e$ while other have -1/3$e$. Lepton carry an integer charge -1$e$.
There are experimental evidences which support that the number of fermion generation has to be equal three~\cite{three_lepton_generations}.
Each higher generation consist from more heavy particles which tend to decay to the lighter particles from lower generations which can be
interpreted as an explanation why matter is made from the first generation particles.
In addition to the electrical charge quarks also carry colour charge, 
which allows for the same flavour quarks to coexist inside one hadron without violating the Pauli principle.
There are three different colour within SM namely red, green and blue.
Number of colours have been experimentally confirmed from the measurements of the ratio of the hadronic cross section 
to the $\mu^+\mu^-$ production cross section in the electron-positron annihilation. \toDo[add link]
Each of the fermions have an associated anti-particle, which is the particle with opposite charges but with identical mass and spin.
List of fermions and their properties are summarized in \TableRef{tab:fermions}.

\begin{table*}[!ht]
\begin{center}
\begin{tabular}{c||c|c|c||c|c|c}
\multirow{2}{*}{Generation} & \multicolumn{3}{c||}{Leptons} & \multicolumn{3}{c}{Quarks} \\
\cline{2-7}
 & Flavor & Mass [GeV] & Charge [$e$] &  Flavor & Mass [GeV] & Charge [$e$] \\
\hline
\multirow{2}{*}{1st} & $\nu_e$ & $<2\times10^{-9}$ & 0 & up & $2.2^{+0.6}_{-0.4}\times10^{-3}$ & 2/3 \\
 & $e$ & 0.000511 & -1 & down & $4.7^{+0.5}_{-0.4}\times10^{-3}$ & -1/3 \\
\cline{1-7} 
\multirow{2}{*}{2nd} & $\nu_{\mu}$ & $<0.00019$ & 0 & charm & $1.27\pm0.03$ & 2/3 \\
 & $\mu$ & 0.106 & -1 & strange & $96^{+8}_{-4}\times10^{-3}$ & -1/3 \\
\cline{1-7}
\multirow{2}{*}{3rd} & $\nu_{\tau}$ & $<0.0182$ & 0 & top & $174.2\pm1.4$ & 2/3 \\
 & $\tau$ & 1.777 & -1 & bottom & $4.18^{+0.04}_{-0.04}$ & -1/3 \\
\end{tabular}
\end{center}
 \caption{The fermion particle generations with their electrical charges and masses.}
\label{tab:fermions}
\end{table*}

% *****************
% BOSONS
% *****************
The remaining particles described by the SM are bosons. Those particle have an integer values of spin.
List of all known elementary boson is shown in \TableRef{tab:bosons}.
There are many experimental evidences that proof existence of bosons.
The first evidence of the Z and W bosons were done by the UA1 and UA2 collaborations~\cite{ARNISON1983103, BAGNAIA1983130}
which made the first measurements of their masses.
The Higgs boson have been discovered the latest by the ATLAS and CMS collaborations in 2012~\cite{Aad2012tfa, Chatrchyan2012ufa}.

\begin{table}[h!]
\begin{center}
{
\begin{tabular}{|l|c|c|c|c|c|c|}\cline{1-7}
 Boson & Mass &  Charge [$e$] & Spin  & Interaction & Range & Act on \\ \hline
    photon  &  0  &  0 & 1 &  Electromagnetism & $ \infty $  &  charge \\ \hline
  8 gluons  &  0  &  0 & 1 &  Strong & $10^{-15}$~m &  colour \\ \hline
%     $W^{\pm}$   &  80.4~GeV  &  $\pm$1 & 1 &  \multirow{2}{*}{Weak}  & \multirow{2}{*}{$10^{-18}$~m} & \multirow{2}{*}{isospin \toAsk[+hypercharge?] }  \\
    $W^{\pm}$   &  80.4~GeV  &  $\pm$1 & 1 &  \multirow{2}{*}{Weak}  & \multirow{2}{*}{$10^{-18}$~m} & isospin  \\
    $Z$   &  91.2~GeV  &  0 & 1 &  &  & \toAsk[+hypercharge?]  \\ \hline
    Higgs   &  125~GeV  &  0 & 0 &  &  &   \\ 
\hline
\end{tabular}
}
\caption{\label{tab:bosons}The Standard Model bosons with their masses and `charges', and corresponding interaction types. }
\end{center}
\end{table}




\subsection{Type of interactions and fields}
% quanta of interaction fields

The SM is the Quantum Field Theory that describes the interactions between the particles.
There are three fundamental forces (other than gravitation) which are incorporated into the SM framework: the electromagnetic, weak and strong forces. These forces are mediated between the matter particles (elementary or composite) by carrier particles with integer spin (spin 1), which are called gauge bosons.
Each fundamental force has its own gauge boson: the electromagnetic force is mediated via an exchange of massless photons, the strong force is mediated by gluons, while the weak force is transmitted by massive W and Z bosons.
In the quantum field theories the forces are interpreted as dynamics of the quantized, relativistic and locally interacting fields.
The electromagnetic, weak and strong forces have different field strength and work over different ranges (see \TableRef{tab:bosons}). 
Both types of matter particles, leptons and quarks, interact through the electromagnetic and weak forces (except neutrinos, which interact only weekly), whereas only quarks interact strongly.

The SM combines two main theories built to describe all of three fundamental interactions, the force-carrier bosons and the matter particles. The first one is Glashow-Salam-Weinberg (GSW) theory which unifies the electromagnetic and weak interactions. The second theory, quantum chromodynamics (QCD), describes the strong interactions of quarks and gluons. Both GSW and QCD theories are constrained by principles of local gauge invariance of the fields based on the $U(1)_Y\times SU(2)_{L}$ weak isospin and hypercharge, and $SU(3)_C$ color symmetry groups, respectively. Thus, the SM is a Quantum Field Theory based on $U(1)_Y\times SU(2)_{L}\times SU(3)_C$ symmetry group with $1+3+8=12$ generators that correspond to 12 massless gauge bosons (see \TableRef{tab:bosons}), if the gauge symmetry is unbroken. Remarkably that both GSW and QCD theories, unlike the QED, are non-abelian that determines the property of the self-interactions between the corresponding gauge boson fields. In particular, self-interaction of gluon field leads to an important characteristic of the QCD such as the assymptotic freedom and confinement.

The weak W and Z bosons are experimentally observed to be massive, indicating that the electro-weak sector of the SM is spontaneously broken. 
In the SM the spontaneous symmetry breaking is implemented by the Higgs mechanism. In this approach,
a doublet of complex fields is introduced which interacts with all the SM particles and has a potential with infinite number of degenerate vacuum states. 
\toAsk[is it true for all SM particles? what about neutrinos?]
The choice of one particular ground state with non-zero vacuum expectation value, $v$, breaks the $SU(2)$ symmetry and results into one scalar neutral particle (Higgs boson) that must be present in the SM. As it was mentioned above the existence of the Higgs boson was recently confirmed experimentally.
The Higgs mechanism allows to give masses not only to charged and neutral gauge bosons, but also to the fermions through the corresponding Yukawa coupling terms.

% TODO check is paragraph below is well connected to the text above
SM have been tested by hundreds of measurements by many different experiments.
All measurements were in remarkably well agreement with theoretical calculation.
As an example comparison of the latest measurements of different SM process cross sections by the ATLAS collaboration with theoretical are shown in
\FigureRef{fig:SM_theory_vs_data}.

\begin{figure}[]
  \centering
  \includegraphics[width=0.99\textwidth]{intro/ATLAS_a_SMSummary_TotalXsect.eps}
  \caption{
  Summary of Standard Model total production cross section measurements done by the ATLAS experiment compared to the corresponding theoretical expectations. 
  All theoretical expectations were calculated at next-to-leading order or higher. 
  }
  \label{fig:SM_theory_vs_data}
\end{figure}



\subsection{Symmetries}
% symmetries --> their combination --> SM (which combine three type of interactions)

\subsection{Experimental confirmation of SM}
% great agreement between SM and experimental results

\section{Physics Beyond SM}

\subsection{Problems of the SM}

Despite of the described problems above, the incredible accuracy of the SM thus far measured up to TeV scale leads to the understanding that SM 
is simply incomplete rather than incorrect. 
This is way a first step to build a new model which will address some of the problems is firstly to verify that it agrees with the SM predictions. 
This is why many new models aim to expand SM rather than to provide completely new approach.
Such models are typically called Beyond Standard Models (BSM).

\subsection{Sensitive final states and Searches of the BSM physics}
% Oleg: (look in paper with prospects of BSM searches)
% say about my final state (with leptons)

\subsection{List of considered in thesis BSM models and their description}

