\chapter{Physics of the Standard Model and Beyond}
\label{chap:Theory}

%% Restart the numbering to make sure that this is definitely page #1!
\pagenumbering{arabic}

%% Note that the citations in this chapter use the journal and
%% arXiv keys: I used the SLAC-SPIRES online BibTeX retriever
%% to build my bibliography. There are also quite a few non-standard
%% macros, which come from my personal collection. You can have them
%% if you want, or I might get round to properly releasing them at
%% some point myself.

\section{Introduction}
% TODO add one or two sentences
Particle physics studies the elementary constituents of the universe and how they interact with each other. One of many questions which it attempts to address is the origin of our universe and what it consists of. 
Like any field of science, particle physics is based on the two pillars: experiment and theory.
By making experimental observations, one obtains certain information, and by systematizing this information and making relations between the pieces, one aims to develop a theory or a model which would explain everything.
The more different observations the model can explain, the more confidence one can have that the model works and that it can be used to predict as-yet unmeasured effects. 

This was the case with the so-called Standard Model (SM) which became extremely popular in the last century because it was able to describe hundreds of newly observed particles in collider experiments.
The SM describes with remarkable precision three types of particle interactions, 
with only the gravitational interaction left unincluded.
Thus a significant effort is ongoing to try to incorporate gravity into the SM to obtain a complete model.

However, despite the fact that the SM agrees amazingly well with experimental measurements, and even if gravity is not taken into account, 
there are many reasons to think that the SM is not complete.
There is a set of unsolved questions which the SM cannot address (e.g. neutrino masses, dark matter, matter-antimatter asymmetry).
Because all known elementary particles fit well in the SM, these problems make us believe that there is
potentially new physics (and correspondingly new particles) in TeV or above-TeV regimes which can solve these problems.
% TODO Will comment: I think either "provides a possibility" or "provides us with a possibility" are correct, but "provides us a possibility" seems not quite right to me 
The Large Hadron Collider (LHC) provides us with a possibility to explore a TeV energy frontier.
This is why many analyses are focused on an investigation of new energy regime and search for possible deviations from the SM which can be hints of new physics.
Since the SM was well established more than 30 years ago, many theorists spent a tremendous amount of time and effort to create
a plethora of models which extend the SM and address multiple unsolved problems. These models can predict signatures and criteria which are most sensitive to the possible new physics and motivate strategies of the searches.

This chapter contains a brief description of the SM, its problems and 
some Beyond Standard Model (BSM) models which have been developed to address several of these problems.

\section{The Standard Model}

\subsection{Elementary particles}

% *****************
% INTRO
% *****************

% TODO Will: You don't define "e" when you use it to talk about quark and lepton charges. Maybe talk about lepton charge first as this ~ defines it 

The SM is a very successful model which describes all the known particles in existence to a remarkable degree of accuracy.
All of the SM particles are fundamental particles (they have no internal structure) that make up the matter and forces in the universe.
The elementary particles can be classified into two groups: 
% TODO Will comment:
% - the spin-1/2 fermions which & the bosons which > "the spin-1/2 fermions, which" and "the bosons, which" - if you miss the comma, it suggests there are other fermions which do not obey Fermi-Dirac statistics 
\begin{itemize}
 \item the spin-1/2 fermions, which obey Fermi-Dirac statistics,
 \item the bosons, which obey Bose-Einstein statistics and have integer spin values.
\end{itemize}

% *****************
% FERMIONS
% *****************
The matter in the universe is made of fermions, which are classified into three generations.
Each generation consists of two electromagnetically charged quarks and one charged lepton as well as an associated neutral neutrino.
Fermions from different generations have the same charges but differ by mass.
One quark from a generation has charge +2/3$e$, while the other has -1/3$e$. Charged leptons carry an integer charge -1$e$.
% TODO Will: Technically Z measurements support the hypothesis that there are three *light* generations 
Experimental evidence supports the hypothesis that the number of fermion generations has to be three~\cite{three_lepton_generations}.
% TODO Will comment: - "Each higher generation" - you haven't mentioned any hierarchy of generations thus far 
Each higher generation consists of more heavy particles which decay to lighter particles from lower generations, and that can be
interpreted as an explanation of 
% why matter is made from the first generation particles.
why the observed stable matter in the universe is made exclusively from particles of the first generation.
% TODO Will: I would also say it's a bit more than "allows" - it's what holds hadrons together and stops electromagnetism blasting them apart 
In addition to electric charge, quarks also carry colour charge, which allows two otherwise identical quarks to jointly occupy an energy and spin state inside a hadron.

% TODO Will: - different “colours” within the SM -> I think I would say "colour charges" without quotation marks, but put quotation marks the first time you mention colour charge i.e. " "colour" charge" 
There are three different ``colours'' within the SM, namely: red, green and blue.
The number of colours has been experimentally confirmed by measurements of the 
ratio of hadronic to $\mu^+\mu^-$ production cross sections
in the electron-positron annihilation~\cite{pdg_2014}. 
Each fermion has an associated antiparticle, which is the particle with opposite charges but with identical mass and spin.
A list of fermions and their properties is presented in \TableRef{tab:fermions}.

% TODO Will Table 1.1: "nu_e" isn't really a flavour. Maybe "flavour" -> "particle"? 
\begin{table*}[!ht]
\begin{center}
\begin{tabular}{c||c|c|c||c|c|c}
\multirow{2}{*}{Generation} & \multicolumn{3}{c||}{Leptons} & \multicolumn{3}{c}{Quarks} \\
\cline{2-7}
 & Flavour & Mass [GeV] & Charge [$e$] &  Flavour & Mass [GeV] & Charge [$e$] \\
\hline
\multirow{2}{*}{1st} & $\nu_e$ & $<2\times10^{-9}$ & 0 & up & $2.2^{+0.6}_{-0.4}\times10^{-3}$ & 2/3 \\
 & $e$ & 0.000511 & -1 & down & $4.7^{+0.5}_{-0.4}\times10^{-3}$ & -1/3 \\
\cline{1-7} 
\multirow{2}{*}{2nd} & $\nu_{\mu}$ & $<0.00019$ & 0 & charm & $1.27\pm0.03$ & 2/3 \\
 & $\mu$ & 0.106 & -1 & strange & $96^{+8}_{-4}\times10^{-3}$ & -1/3 \\
\cline{1-7}
\multirow{2}{*}{3rd} & $\nu_{\tau}$ & $<0.0182$ & 0 & top & $174.2\pm1.4$ & 2/3 \\
 & $\tau$ & 1.777 & -1 & bottom & $4.18^{+0.04}_{-0.04}$ & -1/3 \\
\end{tabular}
\end{center}
 \caption{The fermion particle generations with their electrical charges and masses.}
\label{tab:fermions}
\end{table*}

% *****************
% BOSONS
% *****************
The remaining particles described by the SM are bosons. These particles have integer values of spin.
A list of all known elementary bosons is shown in \TableRef{tab:bosons}.
Many experimental measurements obtained with the help of colliders prove the existence of the bosons.
The first evidence of the Z and W bosons was obtained by the UA1 and UA2 collaborations~\cite{ARNISON1983103, BAGNAIA1983130}
which made the first measurements of their masses.
The gluon has been experimentally confirmed 
in electron-positron annihilation at the PETRA storage ring by observation
of the three-jet topology~\cite{three_jet_event}.
The Higgs boson was discovered by the ATLAS and CMS collaborations in 2012~\cite{Aad2012tfa, Chatrchyan2012ufa}.

\begin{table}[h!]
\begin{center}
{
\begin{tabular}{|l|c|c|c|c|c|c|}\cline{1-7}
 Boson & Mass &  Charge [$e$] & Spin  & Interaction & Range & Interact with \\ \hline
    photon  &  0  &  0 & 1 &  Electromagnetism & $ \infty $  &  charge \\ \hline
  8 gluons  &  0  &  0 & 1 &  Strong & $10^{-15}$~m &  colour \\ \hline 
%     $W^{\pm}$   &  80.4~GeV  &  $\pm$1 & 1 &  \multirow{2}{*}{Weak}  & \multirow{2}{*}{$10^{-18}$~m} & \multirow{2}{*}{isospin \toAsk[+hypercharge?] }  \\
    $W^{\pm}$   &  80.4~GeV  &  $\pm$1 & 1 &  \multirow{2}{*}{Weak}  & \multirow{2}{*}{$10^{-18}$~m} & weak isospin  \\
    $Z$   &  91.2~GeV  &  0 & 1 &  &  & + hypercharge \\ \hline
    Higgs   &  125~GeV  &  0 & 0 &  &  &   \\ 
\hline
\end{tabular}
}
\caption{\label{tab:bosons}The Standard Model bosons with their masses and `charges', and corresponding interaction types. }
\end{center}
\end{table}




\subsection{Types of interactions and fields}
% quanta of interaction fields

The SM is a Quantum Field Theory that describes interactions between particles.
There are three fundamental forces which are incorporated into the SM framework: the electromagnetic, the weak and the strong forces. These forces are mediated between matter particles (elementary or composite) by carrier particles with spin 1, which are called gauge bosons.
Each fundamental force has its own gauge boson(s): the electromagnetic force is mediated via an exchange of massless photons, the strong force is mediated by massless gluons, while the weak force is transmitted by massive W and Z bosons.
% TODO Snizhko suggest just to not use this sentence...
% TODO comment it after Will will read it.
In quantum field theories the forces are given by the dynamics of the quantised, relativistic and locally interacting fields.
% \toAsk[is ``interpreted as dynamics'' correct?]
The electromagnetic, weak and strong forces have different strengths and act over different ranges (see \TableRef{tab:bosons}). 
Both types of fundamental matter particles, leptons and quarks, interact through the electromagnetic and weak forces (except neutrinos, which interact only weakly), whereas only quarks interact strongly.

The SM combines two main theories built to describe all three fundamental interactions, the force-carrier bosons and the matter particles. The first one is the Glashow-Salam-Weinberg (GSW) theory which unifies the electromagnetic and weak interactions. The second theory, quantum chromodynamics (QCD), describes the strong interactions of quarks and gluons. Both GSW and QCD theories are constrained by principles of local gauge invariance of the fields based on the $U(1)_Y\times SU(2)_{L}$ weak isospin and hypercharge, and $SU(3)_C$ color symmetry groups, respectively. Thus, the SM is a Quantum Field Theory based on the $U(1)_Y\times SU(2)_{L}\times SU(3)_C$ symmetry group with $1+3+8=12$ generators that correspond to 12 massless gauge bosons (see \TableRef{tab:bosons}), if the gauge symmetry is unbroken. 
% TODO Oxana was not sure about word ``determines''... consult with Oleg!
Both GSW and QCD theories, unlike the quantum electrodynamics (QED), are non-abelian, which determines the property of self-interactions between the corresponding gauge boson fields. In particular, self-interaction of the gluon field leads to essential characteristics of the QCD, such as asymptotic freedom and confinement.

% TODO Will: "indicating that the electro-weak symmetry of the SM is spontaneously broken" - I'm not sure it directly indicates that, more that SSB is a way to give masses to the EW bosons while maintaining the gauge invariance of the EW lagrangian. But I'm not certain about that distinction. 
The weak W and Z bosons have been experimentally proven to be massive, indicating that the electro-weak symmetry of the SM is spontaneously broken. 
In the SM the spontaneous symmetry breaking is implemented by the Higgs mechanism. 
% TODO Will: "doublet of complex fields is introduced which interacts with all the SM particles" > not neutrinos unless there are RH ones, which are not included in the SM. And certainly not photons or gluons. Ah you get to this in the next paragraph. But still I wouldn't say "all". 
In this approach, a doublet of complex fields is introduced which interacts with the SM particles and has a potential with an infinite number of degenerate vacuum states. 
The choice of one particular ground state with nonzero vacuum expectation value, $v$, breaks the $SU(2)$ symmetry and results in one scalar neutral particle (Higgs boson) that must be present in the SM. As was mentioned above, the existence of the Higgs boson has been confirmed recently.

% TODO cross-check validity of this statement
The Higgs mechanism gives masses not only to $Z$ and $W$ gauge bosons but also to fermions through the corresponding Yukawa coupling terms. 
% TODO Will: "flips the left-handed fermions into right-handed fermions" - this is a bit of a vague statement. I would say "it couples left and right handed fermions" 
An important property of the Higgs field is that it flips the left-handed fermions into right-handed fermions and vice versa. 
As right-handed neutrinos have not been observed and, thus, are not included in the SM, the left-handed
neutrinos can not interact with Higgs boson and remain massless within the SM. 
However, as will be discussed further, there is experimental evidence which proves that neutrinos are massive.

% TODO check is paragraph below is well connected to the text above
The SM has been tested in thousands of measurements by many different experiments.
All measurements are in remarkably good agreement with theoretical predictions.
For example, comparison of the latest measurements of different SM process cross sections by the ATLAS collaboration with theoretical predictions is shown in
\FigureRef{fig:SM_theory_vs_data}.

\begin{figure}[]
  \centering
  \includegraphics[width=0.99\textwidth]{intro/ATLAS_a_SMSummary_TotalXsect.eps}
  \caption{
  Summary of the Standard Model total production cross section measurements made by the ATLAS experiment, compared to the corresponding theoretical expectations~\cite{sm_atlas_public_plots_2016}. 
  Theoretical predictions and their errors are shown with gray bars, while experimental
  measurements are shown with hollow markers with colored bars, which represent measurement errors.
  All theoretical expectations were calculated at next-to-leading order or higher in perturbative QCD.
  }
  \label{fig:SM_theory_vs_data}
\end{figure}


% \begin{figure}[]
%   \centering
%   \includegraphics[width=0.4\textwidth]{intro/ew_fit.pdf}
%   \caption{
%   ??? Taken from~\cite{ew_fit_sm_success}.
%   }
%   \label{fig:ew_fit}
% \end{figure}




% \subsection{Symmetries}
% symmetries --> their combination --> SM (which combine three type of interactions)

% \subsection{Experimental confirmation of SM}
% great agreement between SM and experimental results

\section{Physics Beyond the Standard Model}

\subsection{Problems of the Standard Model}

The SM is currently the best description of the micro-world. It predicted many particles before their discovery. However, despite its great phenomenological success, the SM does not describe the full picture and is believed to be incomplete. The following $big$ $questions$~\cite{Gershtein:2013iqa} remain unsolved within the SM:
\begin{itemize}
\item \textit{The particle content of the dark matter.} An existence of the dark matter is confirmed in the astrophysics and cosmology. The most compelling hypothesis is that the dark matter is made out of massive neutral particles weakly interacting with the matter (WIMPs). They are expected to have a mass of order less than a TeV. In this scenario, the WIMPs can be directly produced at high energy colliders.
\item \textit{Origin of the mass.} The Higgs Mechanism is introduced in the SM ad hoc. The Higgs boson is the first scalar fundamental particle observed in nature. It gives masses to the fermions, $W$, and $Z$ bosons. However, the SM does not tell us why this happens. It is still not clear whether this particle is fundamental or composite, or if there are other Higgs bosons.
% TODO - harmonize it with Little Higgs model description... 
% TODO - I remember that I didn't agree with Oleg about description of this problem!!!
\item \textit{Hierarchy problem.} The mass of the Higgs boson contains large contributions from radiative loop corrections. In particular, it is sensitive to heavy particles of the SM (as well as to hypothetical particles of new physics that might lie at the TeV scale). 
% TODO I don't like the sentence below...
% TODO only some BSM models take care of this problem, by introdusing superpartners
% which give the same radioative correction and they cancel out with each other...
% but other models are not designed to deal with this model, so they will not change anything!!!
In the SM these corrections are quadratically divergent which leads to an unnaturally high mass of the Higgs boson. It is possible to restore the Higgs mass to a proper value through $fine-tuning$, but this is considered to be unnatural. Some models which predict new physics and new particles at TeV or above scale allow avoiding this problem
by compensating problematic terms in the Higgs mass formula (e.g. little Higgs model~\cite{Brak}).
\item \textit{Origin of matter-antimatter asymmetry.} Amounts of matter and antimatter created after the Big Bang are expected to be equal. However, the universe visible from Earth is made almost entirely of the matter.
A possible explanation could come from new physics predicting baryon number violation, CP-violation and new scalar particles at the TeV scale.
\item \textit{Origin of mass hierarchy of fermion masses.} The mass of the top quark is almost $10^6$ times larger than masses of up and down quarks which calls for an explanation. Discovery of new particles would provide additional clues to this puzzle.
\item \textit{Neutrino mass.} The experimental results on neutrino oscillations~\cite{Fukuda:1998fd} confirm that neutrinos are massive particles. However, in the SM no mass term for the neutrino particles is incorporated. An extension of the SM with a model containing a massive right-handed, sterile neutrino can solve this problem. In such a model, the SM neutrinos acquire mass and the so-called seesaw mechanism explains the smallness of their masses~\cite{Mohapatra:1979ia}.
% TODO Oxana commented that graviton is not something specific to a String Theory, but just
% an analog of the force carrier as gauge bosons of the SM.
\item \textit{Gravity is not part of the SM.} The Quantum Field Theory, which provides a theoretical framework for the SM, is used to describe the micro-world, while the General Relativity, used to describe gravity, works in the macro-world and cannot be easily quantized. It is not an easy task to fit them into a single framework of the SM comfortably. However, a unification is possible in the context of the String Theory.
\end{itemize}

These questions unsolved by the SM motivate us to continue searches for new physics beyond the SM at the TeV scale and higher.

% \subsection{Strategies for finding new physics}
% % Oleg: (look in paper with prospects of BSM searches)
% % say about my final state (with leptons)
% 
% \toDo[This subsection is in progress. Preliminary content of the subsection:]
% \begin{itemize}
%  \item search of new physics is done mainly with a high energetic object or rare final states.
%  \item Electrons and muons are most well-measured objects at colliders.
%  \item Jets are more complicated objects to use in the analysis than $e$ and $\mu$.
%  \item Transverse missing momentum - powerful object for the search of new physics, the only way to measure weak particles.
% \end{itemize}


% TODO write something like this:
% General-purpose particle detectors measure the kinematics of all particles with strong or electromagnetic couplings,
% as well as the ‘missing’ momentum due to weakly interacting particles. This gives them the ability to discover
% almost any possible decay of new particles, as well as new stable particles.
% TODO also describe my final states...
% TODO general approach to look for new physics - deviation from the SM...
% TODO direct and undirect searches...

\subsection{Models beyond the Standard Model}
\label{subsec:bsm_models}
Despite the described problems above, the incredible accuracy of the SM, thus far measured up to TeV scale, leads to the understanding that SM 
is simply incomplete rather than incorrect. 
This is why a first step to build a new model which could address some of the problems is first to verify that it agrees with the SM predictions. 
This is why many new models aim to expand the SM rather than to provide an entirely new approach. Such models are typically called Beyond Standard Models (BSM).
There are a plethora of models which address SM problems in many different ways. 
This thesis covers two searches. One of them is focused on the search for new physics with same-sign dilepton signature. Such final state can be produced in many BSM models, e.g. models which predict double charged Higgs or Majorana neutrino. Another one is focused on the search for new heavy spin-1 gauge boson, namely $\PWprime$, with lepton plus missing transverse energy signature. BSM models which potentially can be observed in these searches are discussed below.

% GUT based models, particularly LRSM
In general, there are many possibilities and ways to go beyond the SM. 
Some famous examples can be found in ref.~\cite{Ellis:2011jb}.
Many models are based on the idea of Grand Unified Theory~\cite{GUT_bigPaper},
which seeks to find a simple symmetry group which are based on the SM symmetry group
and contains all known interactions~\cite{Langacker:1984dc,Cvetic:1995zs}.
One of such models is the model which is based on the $SO(10)$ group, which leads to intermediate symmetry:
\begin{equation}
SO(10) \to SU(3)_C \times SU(2)_L \times SU(2)_R \times U(1)_{B-L}
\end{equation}
It yields the so-called Left-Right Symmetric Model (LRSM), which adds
right-handed weak interaction to the SM and accordingly new right-handed $W_R$ and $Z_R$ gauge bosons.
The breaking of $SU(2)_R \times U(1)_{B-L} \to U(1)_Y$ occurs due to a triplet of complex Higgs fields, consisting of $\Delta^0_R, \Delta^+_R$ and $\Delta^{++}_R$, at a high energy scale~\cite{Azuelos:2004mwa}.
Doubly-charged Higgs $\Delta^{++}_R$ can decay to two same-sign lepton pair which makes it a signal candidate for the same-sign dilepton analysis described in \ChapterRef{chap:SS}. Spontaneous breaking of this symmetry provides mass to the right-handed $W_R$.
Since $SU(2)_R$ symmetry is broken at higher energy scale than $SU(2)_L$, it makes
$W_R$ a signal candidate for the analysis described in \ChapterRef{chap:Wprime}.
This model addresses a few SM problems. Firstly, it assumes Majorana nature of the neutrinos and assigns mass to them by a seesaw mechanism and allows scenarios in which the neutrino masses are naturally light~\cite{Mohapatra:1979ia}. 
Secondly, LRSM provides spontaneous parity breaking~\cite{Grimus:1993fx}, while in the SM parity is broken explicitly.

% Little Higgs model
Another set of models is little Higgs models.
These models address hierarchy and fine-tuning problems described above.
The idea of the models is that the Higgs boson becomes a pseudo-Goldstone boson due to some global symmetry breaking at a TeV energy scale.
The quadratic divergence corrections present in the SM Higgs scalar mass calculation from one-loop contributions from the SM top quark and gauge bosons will be canceled 
by the identical contributions from new heavy gauge bosons and new heavy fermionic states (to oppose the top contribution) introduced by the model~\cite{Han:2003wu,Brak}.
These models assume production of doubly charged Higgs and new heavy charged gauge bosons which are particles of interest for the searches presented in this thesis as well.

% Mention other models
In addition to the models described above there are many other theories and models like Kaluza-Klein, Zee-Babu, SUSY and other which either assume new charged spin-1 boson or same-sign dilepton final state, which proves the high potential of the BSM searches presented in the thesis.