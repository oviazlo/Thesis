\chapter{LUCID}
\label{chap:LUCID}

% ToWRITE:
% \begin{itemize}
%  \item what this chapter is about
%  \item List of publications related to this topic
%  \item My personal contribution related to the described topic
% \end{itemize}

This chapter describe LUCID detector which was built specially for Run-2 phase of a LHC program.
It cover aspects as design of the detector and its key components, assembly and testing of the new detector, 
operation and performance of the detector during 2015-2016 years of data taking.
Special attention is given for the calibration system of the detector and developing of the calibration procedure.

During the design phase a lot of tests were done in order to find out optimal parameters of detector components.
Also a lot of test were done in order to undersdant behaviour of detector componenets and to ???
During the assembly and installation phase all parts of the detector were taken together. During this phase bunch of tests were done
to make sure that all components perform as they should.
Also though-out testing of the system was done to make sure that no damage was done during the installation of the detector.
During operation phase which is still ongoing a lot of studies were held to understand performance of the detector and calibration system.

I contributed in each step described above. I took part in developing of LUCID design particularly design of the calibration system. 
I made series of test to find out optimal parameters of LED and Laser diffusers used to distribute evenly an LED and laser signal and deliver it to all detector channels.
Also a lot of time I spent on undertanding behaviour of LED system as well as PMT and PIN-diode behaviour on it.
Also a few tests were done with Bi-207 radioactive source which is used as one of the wasy to calibrate PMT gain.
Also I participated in detector assembly in the clean room and did testing of the detector during this process.
Also testing of LED and Laser diffuser were done in order to cross check integrity of fibes and homogenity of signal between all PMTs.
Also temperature stress test was done in order to undertand maximum possible temperature allowed to not destroy detector during a beam pipe bake out procedure.
In the operational phase test main focus was given for undestanding of aging of PMTs and possibility to improve calibration system.

LUCID group published paper with description of choice and charaterization of photomultipliers for the new LUCID detector for Run-2~\cite{Alberghi:2016tad}.
My contribution was in understanding and developing of absolute calibration using a Bi-207 source.

During my PhD studies I hold a role of ATLAS Forward detectors Run Coordinator with main focus on LUCID for a 5 months.

% \section{Plan of the LUCID chapter}
% 
% \begin{itemize}
%  \item Lucid operation (should I write about it? if yes - what can I write?)
%  \begin{itemize}
%   \item performance of calibration system.
%   \item HV changes in 2015/2016. As an interesting fact how PMTs are aging.
%   \item interesting observations from LUCID operatins.
%   \item performance of detector - luminosity in 2015 (as a conlusion of LUCID chapter).
%  \end{itemize}
%  
%  \item PMT temperature test (should I describe experimental setup and procedure of development a circuit?);
%  \begin{itemize}
%   \item have a look on temperature change during the nights...
%  \end{itemize}
% 
%  \item Fiber boiling test (see sent letter by october 15, 2014):
%  \begin{itemize}
%   \item picture of bake-out phase
%  \end{itemize}
%  
%  \item Testing of LUCID after installation:
%  \begin{itemize}
%   \item testing of LED diffuser and integrity of fibers
%  \end{itemize}
%  
%  \item Detector design tests:
%  \begin{itemize}
%   \item Understanding LED and PIN-diode:
%   \begin{itemize}
%    \item tests charge or PMT/PIN vs. LED DAC (no conclution reached?)
%    \item angular distribution (lucid$\_$july22.pdf)
%    \item distance measurements 
%    \begin{itemize}
%     \item was done to proof that PMT is linear with light intensity
%     \item this because we saw that PMT charge is not linear vs. LED DAC.
%     \item conclusion: LED is not linear vs. DAC
%    \end{itemize}
%    
%    \item LED frequency test
%   \end{itemize}
%   
%   \item LED difuser geometry (no presentation?).
%   \item Laser diffuser distance.
%   \item Filter choice for LED diffuser (lucid$\_$september10.pdf; also ask Carla about 100 mV restriction).
% %   \item 
% %   \item 
% %   \item 
%  \end{itemize}
%  
%  \item Time tests. Long LED runs. (Though Anders also did it and Vincent is using his results to show, but I also was working for quite some time on it, so I think I should write about it as well):
%  \begin{itemize}
%   \item find presentation of Anders. Check what he did.
%   \item from my presentation (lucid15$\_$apr15.pdf) it looks like we have strong effect in high rate and no effect in low rate.
%  \end{itemize}
% \end{itemize}
% 
% \newpage

\section{LUCID}
LUCID is a luminosity monitor with two detectors placed around the beam-pipe on both forward ends of the ATLAS detector. 
Each detector consists of 16 photomultipliers and 4 quartz fiber bundles. The photomultipliers (PMTs) detect charged 
particles that traverse their quartz windows, where Cherenkov light is produced. Cherenkov light is produced in 
the fiber bundles as well and carried to PMTs that are protected by shielding some 2 meters away 
(see \FigureRef{fig:LucidDrawing}). To increase the detector lifetime, only a subset of the PMTs is used at a 
given time, the others being available as spares. In addition, 4 PMTs have a reduced window opening to decrease 
their acceptance and thus avoid saturation of some luminosity algorithms.

\begin{figure}
\centering
\begin{subfigure}{.5\textwidth}
  \centering
  \includegraphics[width=\linewidth]{LUCID/LUCIDdesign.png}
\end{subfigure}%
\begin{subfigure}{.5\textwidth}
  \centering
  \includegraphics[width=\linewidth]{LUCID/FourPMTs_zoomed.png}
\end{subfigure}
\caption{(left) Schematic drawing of one of the two detectors, showing the position of the photomultipliers 
and quartz fibers with respect to the LHC beampipe; (right) A quarter of one of the detectors. All tubes are 
placed inside mu-metal shielding to protect the PMTs from a stray field. Cooling pipes were installed in order 
to protect the PMTs from overheating during the beampipe bake-out procedure. Three of four tubes have fiber connectors, which
tranfser LED and laser pulses for calibration. The fourth tube is equipped with a Bi-207 source and is completely 
sealed.}
\label{fig:LucidDrawing}
\end{figure}


New readout electronics have been built that consist of VME boards that digitize the PMT signals with FADCs. 
The electronics record hits if the pulseheight is above a threshold and integrate the pulses in each 25 ns 
interval that correspond to a LHC bunch crossing. Figure \FigureRef{fig:pulseShape} shows a typical PMT signal shape in 
a physics run. The duration of the pulses is less than 25 ns.
With respect to the detector used in Run I~\cite{Aad:2013ucp}, the new LUCID has a reduced material budget, 
an increased dynamic 
range and will measure luminosity with additional algorithms based on PMT charge integration. It also has a 
completely new calibration system.

\begin{figure}
\centering
\includegraphics[width=.6\textwidth]{LUCID/pulseShape_run_267367_preliminary.eps}
\caption{Digitized pulse shape of a signal from one of the PMTs of the LUCID detector during a run recorded on 
the 10th of June 2015 at $\sqrt{s}$ = 13 TeV. The polarity of the pulse is inverted. The FADCs measure the 
pulse amplitude in time bins that are 3.125 ns long. Taken from Ref.~\cite{publicPlots}.}
\label{fig:pulseShape}
\end{figure}

\section{Calibration system}
The new LUCID uses R760 Hamamatsu PMTs, a smaller version of the previously used R762 model. These PMTs have a 
10 mm
quartz window diameter, while the old ones had a 14 mm diameter. A smaller PMT model has been chosen to reduce acceptance 
which will help to cope with the increased occupancy and to avoid saturation of the luminosity algorithms.
In addition, 4 PMTs per side have a specially reduced sensitive window with a 7 mm diameter which roughly 
corresponds to a factor 2 decrease in rate. They provide luminosity algorithms that will not saturate at 
increased luminosity.

The PMT gain calibration is monitored in 3 independent ways (see \FigureRef{fig:calibrationSystem}):
\begin{itemize}
 \item by LED signals carried by optical fibers;
 \item by laser signals transferred from the calibration system of ATLAS Tile Calorimeter;
 \item by radioactive sources (Bi-207).
\end{itemize}

\begin{figure}
\centering
\includegraphics[width=.7\textwidth]{LUCID/calibrationSystem.png}
\caption{The LUCID calibration system. 16 PMTs per side receives light from LEDs and the Tile laser calibration 
system. 
For redundancy, two fibers come from two different LED diffusers (with three LEDs each, monitored by 
PIN-diodes), and two fibers come from one laser diffuser. The four remaining PMTs in each detector are calibrated 
with Bi-207 sources.}
\label{fig:calibrationSystem}
\end{figure}

LED signals provide sharp peaks in the amplitude and charge distributions that are recorded by LUCID.
The stability of the PMT gain is controlled by measuring the mean value of these distributions. 
The stability of the LEDs themselves is controlled by PIN-diodes.
An alternative way of calibration is to use the ratio of the mean charge measured by PMTs with that of the 
PIN-diode. 
This charge is proportional to the LED intensity and by controlling charge ratio allows to rule out any 
dependence of the calibration results on LED intensity fluctuations.

The Tile calorimeter laser system provides an alternative source of stable light and is treated in the same way
as the LED signals in the calibration procedure. 
The stability of the laser signals is monitored by the Tile calibration system \cite{atlasGeneral}.

Bi-207 radioactive sources provide monoenergetic electrons from an internal conversion process with energies 
above the Cherenkov threshold in quartz. These electrons
have enough kinetic energy to penetrate the quartz window of PMT and produce signals similar to the signals 
from high 
energetic particles from physics runs. The mean of the charge and amplitude 
distributions from the Bi-207 sources are used in the same way as for the two methods described above. 
This method doesn't suffer from any instability issues~\cite{Alberghi:2016tad}.

The availability of three independent calibration methods increase the robustness of the calibration system 
and provide a possibility to cross-check calibration results between the methods.

\subsection{Laser diffuser}
\label{subsec:laserDiffuser}

% TODO Read here: https://www.rp-photonics.com/numerical_aperture.html

Stability of light source used for calibration is very crucial factor for calibration procedure.
That's why instead of relying on only one light source from LEDs it was chosen to employ laser light source 
provided and monitored by Tile calibration system \cite{atlasGeneral}. 
To distribute light between the channels one can't use the same diffuser as for LED light, 
due to the reason that laser light is very well collimated while it's not the case for LED.
In order to handle laser light new diffuser was introduced, shematic of which is shown in \FigureRef{fig:laserDiffuserSchematics}.
On one side of the diffuser there is a fiber bundle of 48 quartz fibers encased in ferrule connector of 2 mm diameter \FigureRef{fig:laserDiffuserMapping}.
On other side there is a single 0.6 mm fiber which deliver laser signal from Tile calibration system.

??? ... which depends from numerical aperture of used fiber. In this case fiber with numerical aperture equal to 0.22 was used.

It means that 2 mm light spot (which correspod to the size of fiber bundle) will be created in distance of 3.1 mm from surface of the fiber interface.
For calibration purpose light received by each channel have to be approximately equal between channel to channel. But intesity of light is not constant within light spot
that's why dedicated measurement are needed to find out best air gap distance.
Two condition have to be met:
\begin{itemize}
 \item light has to be evenly distributed between channels,
 \item preference are given for configurations in which dimmest channel receives maximum possible light.
\end{itemize}
Two fibers from bundle are used for each channel. For small distances between fiber with laser signal and diffuser biggest intensity are expected for central region of the diffuser, 
while for peripheral region - the lowest intensity is expeted. 
Diffuser are conventionally diveded to three regions: central, intermediate and peripheral. 
If one of the channel fiber from pair is in central or intermediate region another fiber from pair has to be placed on the most peripheral ring of diffuser.
And if both of fibers from pair are in peripheral region they are placed as close as possible to the center.
Pairs which include one fiber in central region are called central fibers and marked in orange color at \FigureRef{fig:laserDiffuserMapping}.
Pair with one fiber from intermediate region are marked with yelow and pairs marked with green color consists from both fibers in peripheral region.

\begin{figure}
\centering
\includegraphics[width=.6\textwidth]{LUCID/laserDiffuserAirGap.pdf}
\caption{Schematic drawing of fiber diffuser. ???}
\label{fig:laserDiffuserSchematics}
\end{figure}


\begin{figure}
\centering
\begin{subfigure}{.45\textwidth}
  \centering
  \includegraphics[width=0.9\linewidth]{LUCID/mapping_bundle1_color.png}
\end{subfigure}%
\begin{subfigure}{.45\textwidth}
  \centering
  \includegraphics[width=0.9\linewidth]{LUCID/mapping_bundle2_color.png}
\end{subfigure}
\caption{(left) Picture of fiber bundle connector which contains fibers which deliver laser signal to PMTs on side A detector. Numbers correpond to PMT number which fiber connected to; 
	 Fibers are artificially (???) divided to three categories based on distance to the center of the connect: central (orange coloro), intermidiate (yellow) and peripheral fibers (green).
	 ???
	 (right) Fiber bundle connector which correspond to Side C detector.}
\label{fig:laserDiffuserMapping}
\end{figure}

Measurement were done with different air gap distances for channels from each category. Due to identical structure of fiber bundle on side A and side C (or bundle 1 and 2), detiled measurements
were done only for one bundle and a few points were measured for the second bundle to verify results of first one. Amplitude of PMT signals were measured by osciloscope in 
ATLAS experimental hall. Measurement results are shown at \FigureRef{fig:laserDiffuserDistanceTest}. PMT 1 represent central fibers, PMT 7 - intermediate and PMT 16 and 19 - peripheral
fibers for bundle 1 and 2. With increasing distance signal amplitude are decreasing for central fibers which is expected. Biggest signal for peripheral fibers were obtained for distance of 4 cm.
Also homogenity within all categories are pretty good for 4 cm, so it was decided to use this value for final version of the diffuser.
% TODO conclusionts above are very short... Make it longer!!!






% It is a feryll connector of 2mm in diameter which contain all fibers which goes to PMTs \FigureRef{fig:laserDiffuserMapping}.
% In order to 
% 
% 
% But due to the fact that refractal index of fiber core is bigger than one for air ???
% But when light come out from fiber to the environment with another refractal index then fiber core 
% 
% light will ???
% New diffuser was introduced in order to handle laser light. ???
% In order to spread light one can use ???



% TODO describe some parameters of Tile system with reference to the atlasGeneral







\begin{figure}
\centering
\includegraphics[width=.6\textwidth]{LUCID/laserDiffuserDistanceTest.pdf}
\caption{Result of test. ??? air gap between quartz fiber with laser signal and fiber bundle }
\label{fig:laserDiffuserDistanceTest}
\end{figure}

