\chapter{LUCID}
\label{chap:LUCID}

\section{Plan of the LUCID chapter}

\begin{itemize}
 \item Lucid operation (should I write about it? if yes - what can I write?)
 \begin{itemize}
  \item performance of calibration system.
  \item HV changes in 2015/2016. As an interesting fact how PMTs are aging.
  \item interesting observations from LUCID operatins.
  \item performance of detector - luminosity in 2015 (as a conlusion of LUCID chapter).
 \end{itemize}
 
 \item PMT temperature test (should I describe experimental setup and procedure of development a circuit?);
 \begin{itemize}
  \item have a look on temperature change during the nights...
 \end{itemize}

 \item Fiber boiling test (see sent letter by october 15, 2014):
 \begin{itemize}
  \item picture of bake-out phase
 \end{itemize}
 
 \item Testing of LUCID after installation:
 \begin{itemize}
  \item testing of LED diffuser and integrity of fibers
 \end{itemize}
 
 \item Detector design tests:
 \begin{itemize}
  \item Understanding LED and PIN-diode:
  \begin{itemize}
   \item tests charge or PMT/PIN vs. LED DAC (no conclution reached?)
   \item angular distribution (lucid$\_$july22.pdf)
   \item distance measurements 
   \begin{itemize}
    \item was done to proof that PMT is linear with light intensity
    \item this because we saw that PMT charge is not linear vs. LED DAC.
    \item conclusion: LED is not linear vs. DAC
   \end{itemize}
   
   \item LED frequency test
  \end{itemize}
  
  \item LED difuser geometry (no presentation?).
  \item Laser diffuser distance.
  \item Filter choice for LED diffuser (lucid$\_$september10.pdf; also ask Carla about 100 mV restriction).
%   \item 
%   \item 
%   \item 
 \end{itemize}
 
 \item Time tests. Long LED runs. (Though Anders also did it and Vincent is using his results to show, but I also was working for quite some time on it, so I think I should write about it as well):
 \begin{itemize}
  \item find presentation of Anders. Check what he did.
  \item from my presentation (lucid15$\_$apr15.pdf) it looks like we have strong effect in high rate and no effect in low rate.
 \end{itemize}
\end{itemize}

\newpage

\section{LUCID}
LUCID is a luminosity monitor with two detectors placed around the beam-pipe on both forward ends of the ATLAS detector. 
Each detector consists of 16 photomultipliers and 4 quartz fiber bundles. The photomultipliers (PMTs) detect charged 
particles that traverse their quartz windows, where Cherenkov light is produced. Cherenkov light is produced in 
the fiber bundles as well and carried to PMTs that are protected by shielding some 2 meters away 
(see \FigureRef{fig:LucidDrawing}). To increase the detector lifetime, only a subset of the PMTs is used at a 
given time, the others being available as spares. In addition, 4 PMTs have a reduced window opening to decrease 
their acceptance and thus avoid saturation of some luminosity algorithms.

\begin{figure}
\centering
\begin{subfigure}{.5\textwidth}
  \centering
  \includegraphics[width=\linewidth]{LUCID/LUCIDdesign.png}
\end{subfigure}%
\begin{subfigure}{.5\textwidth}
  \centering
  \includegraphics[width=\linewidth]{LUCID/FourPMTs_zoomed.png}
\end{subfigure}
\caption{(left) Schematic drawing of one of the two detectors, showing the position of the photomultipliers 
and quartz fibers with respect to the LHC beampipe; (right) A quarter of one of the detectors. All tubes are 
placed inside mu-metal shielding to protect the PMTs from a stray field. Cooling pipes were installed in order 
to protect the PMTs from overheating during the beampipe bake-out procedure. Three of four tubes have fiber connectors, which
tranfser LED and laser pulses for calibration. The fourth tube is equipped with a Bi-207 source and is completely 
sealed.}
\label{fig:LucidDrawing}
\end{figure}


New readout electronics have been built that consist of VME boards that digitize the PMT signals with FADCs. 
The electronics record hits if the pulseheight is above a threshold and integrate the pulses in each 25 ns 
interval that correspond to a LHC bunch crossing. Figure \FigureRef{fig:pulseShape} shows a typical PMT signal shape in 
a physics run. The duration of the pulses is less than 25 ns.
With respect to the detector used in Run I~\cite{Aad:2013ucp}, the new LUCID has a reduced material budget, 
an increased dynamic 
range and will measure luminosity with additional algorithms based on PMT charge integration. It also has a 
completely new calibration system.

\begin{figure}
\centering
\includegraphics[width=.6\textwidth]{LUCID/pulseShape_run_267367_preliminary.eps}
\caption{Digitized pulse shape of a signal from one of the PMTs of the LUCID detector during a run recorded on 
the 10th of June 2015 at $\sqrt{s}$ = 13 TeV. The polarity of the pulse is inverted. The FADCs measure the 
pulse amplitude in time bins that are 3.125 ns long. Taken from Ref.~\cite{publicPlots}.}
\label{fig:pulseShape}
\end{figure}

\section{Calibration system}
The new LUCID uses R760 Hamamatsu PMTs, a smaller version of the previously used R762 model. These PMTs have a 
10 mm
quartz window diameter, while the old ones had a 14 mm diameter. A smaller PMT model has been chosen to reduce acceptance 
which will help to cope with the increased occupancy and to avoid saturation of the luminosity algorithms.
In addition, 4 PMTs per side have a specially reduced sensitive window with a 7 mm diameter which roughly 
corresponds to a factor 2 decrease in rate. They provide luminosity algorithms that will not saturate at 
increased luminosity.

The PMT gain calibration is monitored in 3 independent ways (see \FigureRef{fig:calibrationSystem}):
\begin{itemize}
 \item by LED signals carried by optical fibers;
 \item by laser signals transferred from the calibration system of ATLAS Tile Calorimeter;
 \item by radioactive sources (Bi-207).
\end{itemize}

\begin{figure}
\centering
\includegraphics[width=.7\textwidth]{LUCID/calibrationSystem.png}
\caption{The LUCID calibration system. 16 PMTs per side receives light from LEDs and the Tile laser calibration 
system. 
For redundancy, two fibers come from two different LED diffusers (with three LEDs each, monitored by 
PIN-diodes), and two fibers come from one laser diffuser. The four remaining PMTs in each detector are calibrated 
with Bi-207 sources.}
\label{fig:calibrationSystem}
\end{figure}

LED signals provide sharp peaks in the amplitude and charge distributions that are recorded by LUCID.
The stability of the PMT gain is controlled by measuring the mean value of these distributions. 
The stability of the LEDs themselves is controlled by PIN-diodes.
An alternative way of calibration is to use the ratio of the mean charge measured by PMTs with that of the 
PIN-diode. 
This charge is proportional to the LED intensity and by controlling charge ratio allows to rule out any 
dependence of the calibration results on LED intensity fluctuations.

The Tile calorimeter laser system provides an alternative source of stable light and is treated in the same way
as the LED signals in the calibration 
procedure. The stability of the laser signals is monitored by the Tile calibration system \cite{atlasGeneral}.

Bi-207 radioactive sources provide monoenergetic electrons from an internal conversion process with energies 
above the Cherenkov threshold in quartz. These electrons
have enough kinetic energy to penetrate the quartz window of PMT and produce signals similar to the signals 
from high 
energetic particles from physics runs. The mean of the charge and amplitude 
distributions from the Bi-207 sources are used in the same way as for the two methods described above. 
This method doesn't suffer from any instability issues.

The availability of three independent calibration methods increase the robustness of the calibration system 
and provide a possibility to cross-check calibration results between the methods.