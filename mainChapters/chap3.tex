\chapter{LUCID}
\label{chap:LUCID}

% ToWRITE:
% \begin{itemize}
%  \item what this chapter is about
%  \item List of publications related to this topic
%  \item My personal contribution related to the described topic
% \end{itemize}

This chapter describe LUCID detector which was built specially for Run-2 phase of a LHC program.
It cover aspects as design of the detector and its key components, assembly and testing of the new detector, 
operation and performance of the detector during 2015-2016 years of data taking.
Special attention is given for the calibration system of the detector and developing of the calibration procedure.

During the design phase a lot of tests were done in order to find out optimal design and parameters of detector components.
Also a lot of tests were done in order to understand behaviour and performance of detector components.
During the assembly and installation phase all parts of the detector were taken together. During this phase number of tests were done
to make sure that all components perform as they should.
Also overall testing of the system was done to make sure that no damage was done during the installation of the detector.
During operation phase which is still ongoing a lot of studies were held to understand performance of calibration system and detector overall.

I contributed in each step described above. I took part in developing of LUCID design particularly design of the calibration system. 
I made series of tests to find out optimal parameters of LED and Laser diffusers used to evenly distribute LED and laser signals and deliver it to all detector channels.
Also I spent a lot of time on understanding behaviour of LED system as well as PMT and PIN-diode signal behaviour.
Tests with Bi-207 radioactive source which is used as one of the way to calibrate PMT gain were done as well.
Also I participated in detector assembly in the clean room and did testing of the detector during this process.
Testing of LED and Laser diffuser were done in order to cross check integrity of fibers and homogenity of signals between all PMTs.
Also temperature stress-test was performed in order to undertand maximum possible temperature allowed to not destroy detector during a beam pipe bake out procedure.
In the operational phase tests main focus was given for undestanding of aging of PMTs and possibility to improve calibration system.

LUCID group published paper with description of choice and charaterization of photomultipliers for the new LUCID detector for Run-2~\cite{Alberghi:2016tad}.
My contribution was in understanding and developing of absolute calibration using a Bi-207 source.

During my PhD studies I hold a role of ATLAS Forward detectors Run Coordinator with main focus on LUCID for a 5 months.

% \section{Plan of the LUCID chapter}
% 
% \begin{itemize}
%  \item Lucid operation (should I write about it? if yes - what can I write?)
%  \begin{itemize}
%   \item performance of calibration system.
%   \item HV changes in 2015/2016. As an interesting fact how PMTs are aging.
%   \item interesting observations from LUCID operatins.
%   \item performance of detector - luminosity in 2015 (as a conlusion of LUCID chapter).
%  \end{itemize}
%  
%  \item PMT temperature test (should I describe experimental setup and procedure of development a circuit?);
%  \begin{itemize}
%   \item have a look on temperature change during the nights...
%  \end{itemize}
% 
%  \item Fiber boiling test (see sent letter by october 15, 2014):
%  \begin{itemize}
%   \item picture of bake-out phase
%  \end{itemize}
%  
%  \item Testing of LUCID after installation:
%  \begin{itemize}
%   \item testing of LED diffuser and integrity of fibers
%  \end{itemize}
%  
%  \item Detector design tests:
%  \begin{itemize}
%   \item Understanding LED and PIN-diode:
%   \begin{itemize}
%    \item tests charge or PMT/PIN vs. LED DAC (no conclution reached?)
%    \item angular distribution (lucid$\_$july22.pdf)
%    \item distance measurements 
%    \begin{itemize}
%     \item was done to proof that PMT is linear with light intensity
%     \item this because we saw that PMT charge is not linear vs. LED DAC.
%     \item conclusion: LED is not linear vs. DAC
%    \end{itemize}
%    
%    \item LED frequency test
%   \end{itemize}
%   
%   \item LED difuser geometry (no presentation?).
%   \item Laser diffuser distance.
%   \item Filter choice for LED diffuser (lucid$\_$september10.pdf; also ask Carla about 100 mV restriction).
% %   \item 
% %   \item 
% %   \item 
%  \end{itemize}
%  
%  \item Time tests. Long LED runs. (Though Anders also did it and Vincent is using his results to show, but I also was working for quite some time on it, so I think I should write about it as well):
%  \begin{itemize}
%   \item find presentation of Anders. Check what he did.
%   \item from my presentation (lucid15$\_$apr15.pdf) it looks like we have strong effect in high rate and no effect in low rate.
%  \end{itemize}
% \end{itemize}
% 
% \newpage

\section{Overview of LUCID detector}
\label{sec:LUCID}
LUCID is a luminosity monitor with two detectors placed around the beam-pipe on both forward ends of the ATLAS detector. 
Each detector consists of 16 photomultipliers and 4 quartz fiber bundles. The photomultipliers (PMTs) detect charged 
particles that traverse their quartz windows, where Cherenkov light is produced. Cherenkov light is produced in 
the fiber bundles as well and carried to PMTs that are protected by shielding some 2 meters away 
(see \FigureRef{fig:LucidDrawing}). To increase the detector lifetime, only a subset of the PMTs is used at a 
given time, the others being available as spares. In addition, 4 PMTs have a reduced window opening to decrease 
their acceptance and thus avoid saturation of some luminosity algorithms.

\begin{figure}
\centering
\begin{subfigure}{.5\textwidth}
  \centering
  \includegraphics[width=\linewidth]{LUCID/LUCIDdesign.png}
\end{subfigure}%
\begin{subfigure}{.5\textwidth}
  \centering
  \includegraphics[width=\linewidth]{LUCID/FourPMTs_zoomed.png}
\end{subfigure}
\caption{(left) Schematic drawing of one of the two detectors, showing the position of the photomultipliers 
and quartz fibers with respect to the LHC beampipe; (right) A quarter of one of the detectors. All tubes are 
placed inside mu-metal shielding to protect the PMTs from a stray field. Cooling pipes were installed in order 
to protect the PMTs from overheating during the beampipe bake-out procedure. Three of four tubes have fiber connectors, which
tranfser LED and laser pulses for calibration. The fourth tube is equipped with a Bi-207 source and is completely 
sealed.}
\label{fig:LucidDrawing}
\end{figure}


New readout electronics have been built that consist of VME boards that digitize the PMT signals with FADCs. 
The electronics record hits if the pulseheight is above a threshold and integrate the pulses in each 25 ns 
interval that correspond to a LHC bunch crossing. Figure \FigureRef{fig:pulseShape} shows a typical PMT signal shape in 
a physics run. The duration of the pulses is less than 25 ns.
With respect to the detector used in Run I~\cite{Aad:2013ucp}, the new LUCID has a reduced material budget, 
an increased dynamic 
range and will measure luminosity with additional algorithms based on PMT charge integration. It also has a 
completely new calibration system.

\begin{figure}
\centering
\includegraphics[width=.6\textwidth]{LUCID/pulseShape_run_267367_preliminary.eps}
\caption{Digitized pulse shape of a signal from one of the PMTs of the LUCID detector during a run recorded on 
the 10th of June 2015 at $\sqrt{s}$ = 13 TeV. The polarity of the pulse is inverted. The FADCs measure the 
pulse amplitude in time bins that are 3.125 ns long. Taken from Ref.~\cite{publicPlots}.}
\label{fig:pulseShape}
\end{figure}

\subsection{PMT sensors}

The new LUCID uses R760 Hamamatsu PMTs, a smaller version of the previously used R762 model. These PMTs have a 
10 mm quartz window diameter, while the old ones had a 14 mm diameter. A smaller PMT model has been chosen to reduce acceptance 
which will help to cope with the increased occupancy and to avoid saturation of the luminosity algorithms.
In addition, 4 PMTs per side have a specially reduced sensitive window with a 7 mm diameter which roughly 
corresponds to a factor 2 decrease in rate (see \FigureRef{fig:modPMT}). They provide luminosity algorithms that will not saturate at 
increased luminosity. Detailed description of choise and characterization of PMTs used in detector can be found in~\cite{Alberghi:2016tad}.

[ToAsk ???] Should I add additional information how choise of PMTs was done (as described in the paper~\cite{Alberghi:2016tad})?
% TODO also mention that below in the text there will be some description of tests done to understand behaviour of PMts?

\begin{figure}
\centering
\includegraphics[width=.6\textwidth]{LUCID/mod_PMT.jpg}
\caption{R760 Hamamatsu PMT with specially reduced sensitive window size used in the LUCID detector.}
\label{fig:modPMT}
\end{figure}


\subsection{LUCID electronics}


\section{Calibration system}

% TODO add reference to some book with desciption of PMT aging 



The PMT gain calibration is monitored in 3 independent ways (see \FigureRef{fig:calibrationSystem}):
\begin{itemize}
 \item by LED signals carried by optical fibers;
 \item by laser signals transferred from the calibration system of ATLAS Tile Calorimeter;
 \item by radioactive sources (Bi-207).
\end{itemize}

\begin{figure}
\centering
\includegraphics[width=.7\textwidth]{LUCID/calibrationSystem.png}
\caption{The LUCID calibration system. 16 PMTs per side receives light from LEDs and the Tile laser calibration 
system. 
For redundancy, two fibers come from two different LED diffusers (with three LEDs each, monitored by 
PIN-diodes), and two fibers come from one laser diffuser. The four remaining PMTs in each detector are calibrated 
with Bi-207 sources.}
\label{fig:calibrationSystem}
\end{figure}

LED signals provide sharp peaks in the amplitude and charge distributions that are recorded by LUCID.
The stability of the PMT gain is controlled by measuring the mean value of these distributions. 
The stability of the LEDs themselves is controlled by PIN-diodes.
An alternative way of calibration is to use the ratio of the mean charge measured by PMTs with that of the 
PIN-diode. 
This charge is proportional to the LED intensity and by controlling charge ratio allows to rule out any 
dependence of the calibration results on LED intensity fluctuations.

The Tile calorimeter laser system provides an alternative source of stable light and is treated in the same way
as the LED signals in the calibration procedure. 
The stability of the laser signals is monitored by the Tile calibration system \cite{atlasGeneral}.

Bi-207 radioactive sources provide monoenergetic electrons from an internal conversion process with energies 
above the Cherenkov threshold in quartz. These electrons
have enough kinetic energy to penetrate the quartz window of PMT and produce signals similar to the signals 
from high 
energetic particles from physics runs. The mean of the charge and amplitude 
distributions from the Bi-207 sources are used in the same way as for the two methods described above. 
This method does not suffer from any instability issues~\cite{Alberghi:2016tad}.

The availability of three independent calibration methods increase the robustness of the calibration system 
and provide a possibility to cross-check calibration results between the methods.

\subsection{LED diffuser}
\label{subsec:LEDDiffuser}

% TODO structure of the section:
% - to describe why we have 2 LED diffusers: why we use filters on one and don't use on another on
% - geometry studies
% - show LED stability plot over time (PIN-diode measurements)
% - filter studies - explain why we have to be below 100mV for PMTs (discuss this with Carla)
% - 
% - 


% light has to be evenly distributed among PMTs
% LED has to be monitored by PIN-diode in order to verify its stability
% geometrical constrain
%   taking into account size of PIN-diode
%   size of support structure for fibers
%   bending of fibers - whole LED diffuser has to be hosted in limited space and fiber cannot be bend to much.
% constrain during production of the diffuser:
%   complexity of drilling holes for fiber for some angles
%   gluing fibers to the diffuser


Despite a simple purpose of LED diffuser to evenly distribute light among PMTs exsistance of a number of constrains lead to necessity make dedicated study 
to define optimal parameters of the diffuser. Main points which were considered during design phase are:
\begin{itemize}
 \item light has to be evenly distributed among PMTs,
 \item LED has to be monitored by PIN-diode in order to verify its stability,
 \item geometrical constrains (size of PIN-diode case, limited space for the whole diffuser and limits on bending of fibers),
 \item production constrains (problems with drilling wholes at some angles, some position of fibers make glueing challenging).
\end{itemize}

To meet all these requirements radial design was proposed - PIN-diode is located centrally facing LED. Fibers surround Pin-diode case 
evenly with certain angle to the LED axis. Schematic sketch are shown in \FigureRef{fig:???}.
% TODO insert sketch used for testing.
Such geometry guarantee that distance between LED and fibers will be the same for all fibers and ???
  
As described above main purpose of LED diffuser is to evenly distribute light between PMT-channels. 

% consists from LED itself, set of fibers which deliver light to the PMTs and PIN-diode which is used to monitor
% stability of LED during long time. LED diffuser has to be design in such way, that all PMT channels have to receive the same amount of light.



% \begin{figure}
% \centering
% \includegraphics[width=.7\textwidth]{LUCID/LED_diffuser_schematic.pdf}
% \caption{Schematic drawing of LED diffuser.}
% \label{fig:LEDDiffuser}
% \end{figure}

\begin{figure}
\begin{subfigure}{.48\textwidth}
  \centering
  \includegraphics[width=\textwidth]{LUCID/LEDdiffuser_rightPart.png}
\end{subfigure}
\begin{subfigure}{.48\textwidth}
  \centering
  \includegraphics[width=.8\textwidth]{LUCID/photo_LED_diffuser_v2.jpg}
\end{subfigure}

\caption{LED diffuser. Schematic drawing (left), photo (right).}
\label{fig:LEDDiffuser}
\end{figure}


\subsection{Laser diffuser}
\label{subsec:laserDiffuser}

% TODO Read here: https://www.rp-photonics.com/numerical_aperture.html

Stability of light source used for calibration is very crucial factor for calibration procedure.
That's why instead of relying on only one light source from LEDs it was chosen to employ laser light source 
provided and monitored by Tile calibration system \cite{atlasGeneral} as alternative source of light. 
To distribute light between the channels one cannot use the same diffuser as for LED light, 
due to the reason that laser light is very well collimated while it's not the case for LED.
In order to handle laser light new diffuser was introduced, shematic of which is shown in \FigureRef{fig:laserDiffuserSchematics}.
Diffuser represent itself as a fiber bundle of 48 quartz fibers encased in ferrule connector of 2 mm diameter \FigureRef{fig:laserDiffuserMapping}.
Laser light is delivered from Tile calibration system by single fiber of 0.6 mm in diameter (depicted on the right at \FigureRef{fig:laserDiffuserSchematics}).

Light travels in the fiber by undergoing multiple total internal reflections from the interface between fiber core and cladding. 
Due to this, light from fiber will come out within a certain cone. The size of the cone is characterized by numerical aperture $NA$ of the fiber which is equal

\begin{equation}
\label{eq:numericalApperture}
 NA = n \sin{\theta_{max}} = \sqrt{n_{core}^2 - n_{cladding}^2},
\end{equation}

where $n$ is the refractive index of the medium (air in our case), $n_{core}$ is the refractive index of the fiber core, $n_{cladding}$ is the refractive index of the cladding
and $\theta_{max}$ is half-angle of light cone. That's why light spot produced by light from fiber will become bigger with distance from fiber edge. 
That's why to distribute light from fiber to diffuser one need to introduce some air gap between them as shown at \FigureRef{fig:laserDiffuserSchematics}.
Fiber used to deliver laser light has numerical aperture equal to 0.22, which correspond to 3.1 mm air gap distance needed to cover 2 mm surface of diffuser by light spot
(according to \EquationRef{eq:numericalApperture}).
For calibration purpose light received by each channel have to be approximately equal between channel to channel. But intesity of light is not constant within light spot
that's why dedicated measurement are needed to find out best air gap distance.
Two conditions have to be met:
\begin{itemize}
 \item light has to be evenly distributed between channels,
 \item preference are given for configurations in which dimmest channel receives maximum possible light.
\end{itemize}
Two fibers from bundle are used for each channel. For small distances between fiber with laser signal and diffuser biggest intensity are expected for central region of the diffuser, 
while for peripheral region - the lowest intensity is expeted. 
Diffuser are conventionally divided to three regions: central, intermediate and peripheral. 
If one of the channel fiber from pair is in central or intermediate region another fiber from pair has to be placed on the most peripheral ring of diffuser.
And if both of fibers from pair are in peripheral region they are placed as close as possible to the center of diffuser.
Pairs which include one fiber in central region are called central fibers and marked in orange color at \FigureRef{fig:laserDiffuserMapping}.
Pair with one fiber from intermediate region are marked with yellow and pairs marked with green color consists from both fibers in peripheral region.

\begin{figure}
\centering
\includegraphics[width=.6\textwidth]{LUCID/laserDiffuserAirGap.pdf}
\caption{Schematic drawing of coupling of laser diffuser (fiber bundle) with single quartz fiber which deliver laser light from Tile calibration system.}
\label{fig:laserDiffuserSchematics}
\end{figure}


\begin{figure}
\centering
\begin{subfigure}{.45\textwidth}
  \centering
  \includegraphics[width=0.9\linewidth]{LUCID/mapping_bundle1_color.png}
\end{subfigure}%
\begin{subfigure}{.45\textwidth}
  \centering
  \includegraphics[width=0.9\linewidth]{LUCID/mapping_bundle2_color.png}
\end{subfigure}
\caption{(left) Picture of laser diffuser (fiber bundle) with fibers which deliver laser signal to PMTs on side A detector. Numbers correpond to PMT number which fiber connected to; 
	 Two fibers from bundle connected to each PMT. Fiber pairs are divided to three categories based on distance from center of the connect to the closest fiber from pair: 
	 central (orange coloro), intermidiate (yellow) and peripheral fiber pair (green).
	 (right) laser diffuser which correspond to Side C detector.}
% \label{fig:laserDiffuserMapping}
\end{figure}

Measurement were done with different air gap distances for fiber pairs (and correspond PMTs) from each category. 
Due to identical structure of fiber bundle on side A and side C (or bundle 1 and 2), detailed measurements
were done only for one bundle and a few points were measured for the second bundle to verify results of first one. Amplitude of PMT signals were measured by osciloscope in 
ATLAS experimental hall. Measurement results are shown at \FigureRef{fig:laserDiffuserDistanceTest}. PMT 1 represent PMTs with central fiber pair connecter, 
PMT 7 - intermediate and PMT 16 and 19 - peripheral fibers for bundle 1 and 2. 
With increasing distance signal amplitude are decreasing for central fibers which is expected. 
Intermediate and peripheral fibers has maximums. Maximum for peripheral fibers correspond to the distance of 4 cm.
Homogenity within all categories are pretty good for 4 cm as well, so it was decided to use this value for final version of the diffuser.
% TODO conclusionts above are very short... Make it longer!!!

% TODO describe some parameters of Tile system with reference to the atlasGeneral


\begin{figure}
\centering
\includegraphics[width=.6\textwidth]{LUCID/laserDiffuserDistanceTest.pdf}
\caption{Measurement of PMT signal amplitude as a function of air gap distance for different catefories of fibers. Fiber pair connected to PMTs 1 represent central fiber pairs, to PMT 7 - intermediate, and to PMTs 16 and 19 - peripherl.}
\label{fig:laserDiffuserDistanceTest}
\end{figure}


\subsection{Bi207 calibration}
\label{subsec:bi207Calibration}



\subsection{Calibration performance during 2015-2016}
\label{subsec:calibPerformance}




% \section{Detector performance}
% \label{sec:DetPerf}
% 
% *** some text here ***
% 
% \subsection{Luminosity measurements}
% \label{subsec:lumMeas}


\section{Temperature measurements}
\label{sec:tempMeas}

In current section measurements of temperature dependence of PMT gain and behaviour of quartz fiber at high temperatures will be discussed.
LUCID detector is equipped with 18 temperature sensors per side which were installed to monitor temperature of key detector components during so called ``bake-out'' procedure of new
ATLAS beampipe. A few sensors are installed close to the PMTs and temperature is taken and archived every ??? minutes.
% TODO ask Davide how freuqent we archive temperature from sensors.
It is known that PMT is more sensitive to ambient temperature than ordinary electronic components (such as capacitors and resistors).
It caused by fact that PMT cathode quantum efficiency depends from temperature as well as gain of the dynode chain.
To estimate temperature effect on PMT gain dedicated measurement were done in 
the lab with R760 Hamamatsu tube which was coupled with Bi-207 radioactive 
source and describe later in this section.

Another test, described in this section, was focused on understanding how big 
temperature quartz fiber, used to provid LED and laser calibration signals, can 
tolerate during the bake-out procedure.

% TODO insert references to the sub-sections here?


TODO: describe scheme of temperature sensonrs installed in LUCID.

\subsection{Temperature controller}
\label{subsec:tempController}

In order to perform any temperature tests one need to have a reliable method to measure and control temperature in a testing area.
Measurements with PMTs are typically done in a light tight black boxes to make sure that no external light will accidently get on PMT photocathode.
Also black box is hermetically well sealed which prevent air from the box to leak outside which make it easy to control temperature inside the box with good precision.
A dedicated temperature controller, based Arduino Mega 2560 microcontroller
% TODO (reference???)
(programmed by Arduino software), was built.
To measure temperature LM35CAH sensor was used which is a precision integrated-circuit device with an output voltage linearly-proportional to the Centigrade temperature.
In order to interface the sensor with 
% readout input channel from 
Arduino board electrical circuit shown at \FigureRef{fig:tempReadOutCircuit} have been made, 
to match output voltage from sensor with readout of Arduino board.
As a heater element a simple chain of resistors were used which were dissipating heat produced by current flowing thorough them. 
Voltage on resistors was controlled by controller with help of MOSFET transistor. In order to verify that temperature will be homogenious within black box,
a pc fan was used. Heating profile was programmed in the Arduino microcontroller.
Measured temperature values were sent directly to pc by serial port or were stored on external microSD card, connected to the Arduino board.

\subsection{PMT gain dependence}
\label{subsec:pmtGainTempDep}

% TODO ask Vincent: do I need to explain how I subtract baseline for charge measurements? Or I can put it into appendix?
% TODO ask Vincent: do I need to describe all electronic chain here?

Due to the fact that interior of PMT is a vacuum and that heat conduct thought it very slowly temperature has to be changed really slowly in order to make sure that temperature of tube
reaches the same level as the ambient temperature.

One of possible component of systematic error of precise measurments of any detector is variation of temperature of environment.
Quite often it's really hard to keep temperature stable over long period of time in very large hall, especially large as area as experimental hall of the ATLAS detector.
That's why dedicated study was done to estimate temperature dependence of PMT gain used in LUCID detector.
Experimental setup consisted from thermal-insulated black box, PMT coupled with Bi-207 source and heater controlled by Arduino temperature controller.
Temperature measurments was done by Arduino controller as well.

PMT charge was digitized by Flash ADC (model ???) which was operated by ATLAS cental TDAQ framework (link???).
Shape of digitized pulse from Bi-207 source is shown at \FigureRef{fig:bi207DigitizedPulse}. 

In order to calculate charge and amplitude of each Bi-207 signal 

Results ???

In order to make sure that temperature is reasonable stable within one mean charge measurement (which correspond to one red point in \FigureRef{fig:PMTChargeTempDep}) 
very small temperature gradient was choosen, which was approximately 0.2\degree C per hour. 


\begin{figure}
\centering
\includegraphics[width=.7\textwidth]{LUCID/rndmPulse_1_mod.pdf}
\caption{Shape of digitized pulse from Bi-207 source. Something about baseline substraction??? }
\label{fig:bi207DigitizedPulse}
\end{figure}


\begin{figure}
\centering
\includegraphics[width=.7\textwidth]{LUCID/goodSlowTemp_Feb20_Feb22_charge.eps}
\caption{Temperature dependence of PMT gain represented by measurments of mean of charge distribution Bi-207 source signals for different temperature values.
Black dots correspond to temperature measurement; red dots correspond to charge mean measurements of Bi-207 source.}
\label{fig:PMTChargeTempDep}
\end{figure}


\subsection{Temperature test of calibration fibers}

% From https://indico.cern.ch/event/315665/contributions/1690160/attachments/605816/833747/LS2_LHC_paper_rev_final-1.pdf

In 2015 year during first long shutdown (LS1) a central ATLAS beampipe made from Ferum (???) 
was changed with a new one made from Beryllium and Aluminium to better deal with material activation and background.
% TODO ask someone wht was the main reason to change beampipe (and beampipe material)?
New beampipe has to be purified in order to prevent dust particles appear inside the pipe.
Beampipe is heated up to 350 \degree C for a few hours. This procedure is called beampipe bakeout.

The LUCID detector sits on supporting structure (gray structure in \FigureRef{???})

dedicated cooling has to be installed to protect PMTs and calibration fibers.
To understand how efficient cooling is needed 
one need to know temperature robustness of detector components.
% TODO what about temperature robustness of PMTs? do we know level at which they start to destroy? why we only did this test for fibers?
% Becuase used PMTs in detector are quite small it makes them quite easy to cool down. 
Calibration fibers goes from PMTs close to beampipe all along to LED and laser diffusers which are placed on top on monoblock structure 
%  TODO some other picture???
(see \FigureRef{fig:LucidDrawing} (left)).
That's why fibers can be potentially overheated at some parts. 
In order to understand which temperaure fibers can handle dedicated measurement was done. 
Idea pf the test was to estimate temperature threshold at which fibers start to loose their optical properties
by changing of optical properties of cladding and/or fiber core.

Experimental setup was consisted from box with thermoisolation properties, 
% TODO ``thermoisolation properties'' - rename it!!!
where bundle of fiber where placed.
LED was used as a light source on one side of of fiber and fiber was coupled with PMT on other side of the fiber to read out the signal.
To provide high temperature inside the box thermogun was used which was blowing hot air inside the box.
Temperature was measured with Arduino-based temperature controller described in~\ref{subsec:tempController}, with only temperature probe inside the box.
% TODO do we really use Arduino-based temperature controller in this test?

Firstly measurements were done with room temperature inside the box to use this value as a reference.
Then thermogun was switched on and temperature raised to around 95\degree C.
This temperature was kept constant for an slightly more than one hour to make sure that fiber was exposed to high temperature for
long enough time period to make changes bservable if any. 
% TODO end of previous sentence is a little bit weird... rephrase it!
In third phase of measurement temperature of air was increased to 110\degree C and it was kept on this level for one hour as well.
On the last phace thermogun was switched off and measurements were done for half an hour more.
Charge and amplitude measurements were done constantly with five-minute intervals. 
Temperature inside the box and charge measurements versus time are shown in \FigureRef{fig:fiberBakeOut}.
Charge measurements during second phase, when temperature was 95\degree C in the box are compatible with measurements done with room temperature.
And they kept stable during all phase two (which was one hour) which proofs that fibers can operate with this temperature.
But during third phase (110\degree C) significant decreasment of measured charge is observed, which demonstrates changes in the
optical properties of the fiber at this temperature. On the last phase temperature went back to room values and charge increased slightly but didn't reach 
nominal values which indicate unrecoverable damage of the fiber at 110\degree C temperature.

\begin{figure}
\centering
\includegraphics[width=.95\textwidth]{LUCID/fiberBakeOut.png}
\caption{???}
\label{fig:fiberBakeOut}
\end{figure}

This test demonstrated that there is big probability that calibration fibers will be damaged if they will be exposed 
with high temperatures (above 95\degree C) for long period of time.
But due to fact that bake-out procedure was planned to be done with around 220\degree C and early
simulations showed that expected temperature on LUCID area can be around 140\degree C 
% TODO find proof of this!!!
efficient cooling for fibers became mandatory requirement for the system.

Decision was to use water cooling. 
To make cooling of PMTs efficient dedicated metal support was designed to host PMTs shown in \FigureRef{fig:PMTMetalSupport}.
Water pipes were positioned under this support.
To protect calibration fibers from overheating they were bundled and glued around cooling pipe which was placed along 
the beampipe (see \FigureRef{fig:temperatureProbes_VJCone}).

Also to provide circulation of air in the detector region additional air cooling was installed. 
[according to Giulio air circulation was essential in order to make sure that some gas will not get into PMTs???]

To monitor temperature during bake-out procedure 18 temperature sensonrs per side were installed 
in important parts of the system as shown in \FigureRef{fig:metalSupportAndTempProbes} (right).

Overall temperature read by all sensors were not always lower than 40\degree C
An example of trending plot is shown in \FigureRef{fig:NEG_ctflangeA}, where emperatures in the cable tray 
and the beam pipe flange during the beampipe NEG activation bakeout at 220\degree C is shown.
It proofs that beampipe insulation and water cooling was very effective and no damage for LUCID detector components 
were done.


\begin{figure}
\centering
\begin{subfigure}{.6\textwidth}
  \centering
  \includegraphics[width=0.95\textwidth]{LUCID/PMT_metal_support.jpg}
\end{subfigure}%
\begin{subfigure}{.4\textwidth}
  \centering
  \includegraphics[width=0.95\linewidth]{LUCID/temperatureProbes_1_v2.jpg}
\end{subfigure}
\caption{???}
\label{fig:metalSupportAndTempProbes}
\end{figure}


% TODO describe:
% - water cooling - see in the figure
% - additional air cooling by blowing air with fans to make circulation of air around lucid detecor
% - temperature probes installed - add picture with probes
% - temperature plots during bakeout procedure

\begin{figure}
\centering
\includegraphics[width=.9\textwidth]{LUCID/temperatureProbes_2.jpg}
\caption{???}
\label{fig:temperatureProbes_VJCone}
\end{figure}



% TODO insert picture which show that LUCID sits on VJ cone around beampipe

\begin{figure}
\centering
\includegraphics[width=.9\textwidth]{LUCID/NEG_ctflangeA.png}
\caption{???}
\label{fig:NEG_ctflangeA}
\end{figure}
 
 

\section{The LUCID luminosity measurements}


Luminosity is measured by LUCID from a measurement of the number of PMT-hits, the number of bunch crossings 
with at least one PMT-hit and the integrated pulseheight (charge). These measurements are done over a time 
period called a luminosity block, which are typically 1 minute long and they are done for each of the 
individual bunch crossings in the LHC. 
The new electronics provides luminosity measurements using 124 different algorithms which take as input 
different combination of hits or charge from different tubes. Algorithms, which are based on PMT-hits from only 
one of the detectors (either A or C), are calculated by the LUCROD VME custom boards, while algorithms which depend on 
combination of hits from both detectors are calculated by the LUMAT boards (see 
\FigureRef{fig:Eletronics_schematics}).
The luminosity is proportional to the measured charge which is in turn proportional to the logarithm of the measured 
number of PMT-hits *** add log formula for lumi ***. The two types of measurements therefore have different limitations. 
The main issue with the 
charge measurement is PMT gain stability while the hit measurements can suffer from pile-up of several signals 
below threshold giving a signal above threshold.
% Figure~\ref{fig:hitCount} shows the number of PMT-hits from different LHC bunches. The large peaks correspond 
% to six trains of 
% six colliding bunches each, plus two isolated colliding bunches. Two smaller peaks that correspond to bunches 
% with only one beam are also seen. The baseline background level is due to the Bi-207 source used for monitoring 
% of the photomultiplier gain. 

\begin{figure}
\centering
\includegraphics[width=.61\textwidth]{LUCID/Eletronics_schematics.eps}
\caption{Bloc diagram of the LUCID electronics. Signals from all tubes are collected by 4 \mbox{LUCROD} cards 
(two per 
each side) that digitize the signals with FADCs. Some of the luminosity algorithms are calculated in the LUCRODs. 
Output of these calculations as well as a copy of all digitized PMT 
signals are sent to the LUMAT cards afterwards, which perform calculations of algorithms that combine data from 
both detectors and publish the results to the Information Server (IS) database.}
\label{fig:Eletronics_schematics}
\end{figure}


\subsection{Luminosity algorithms}



\subsection{First 13 TeV collisions at the LHC}
\label{sec:physics}

The LHC reported the first 13 TeV pp collision in May of 2015 and these were recorded by ATLAS and other LHC 
experiments. 
Starting from that time the new LUCID was succesfully operating and provided information about the 
luminosity delivered to ATLAS. 
% More than a femtobarn of luminosity is already recorded by ATLAS, which provides a
% sufficient amount of data about the performace of the LUCID detector with a high number of pp-interaction per bunch 
% crossing.

The PMT pulseheight distribution in a physics run is shown in the left plot of \FigureRef{fig:Pulseheight} (blue) 
together with the same distribution 
during a Bi-207 calibration run (red). In both distributions a peak due to Cherenkov photons is visible. The 
calibration distribution is cut due to the threshold in the electronics that define a PMT-hit.

In the right plot of \FigureRef{fig:Pulseheight} a comparison of the pulseheight distributions in a physics run 
with
low-$\mu$ (red) and high-$\mu$ (blue) are shown. The pulse height is shifted towards higher values when at high 
luminosity several particles traverse the photomultiplier window in the same bunch crossing.

\begin{figure}
\centering
\begin{subfigure}{.5\textwidth}
  \centering
  \includegraphics[width=\linewidth]{LUCID/LowMuWithBi_ampl_preliminary_v2.pdf}
  \label{fig:sub1}
\end{subfigure}%
\begin{subfigure}{.5\textwidth}
  \centering
  \includegraphics[width=\linewidth]{LUCID/ComparisonPhysRuns_ampl_preliminary_v2.pdf}
  \label{fig:sub2}
\end{subfigure}
\caption{Comparison of pulseheight distribution in a physics run with low-$\mu$ with the same distribution 
during a Bi-207 calibration run (left) and with distribution with high-$\mu$ physics run. Taken from Ref.~\cite{publicPlots}.}
\label{fig:Pulseheight}
\end{figure}

LUCID can measure luminosity in many ways and \FigureRef{fig:InternalConsistency} shows a comparison of the 
luminosity measured by an A and 
a C detector for different ATLAS data taking runs. The two measurements agree to better than 0.5$\%$.

The left plot of \FigureRef{fig:LumiVsTime} shows a measurement of the average number of inelastic pp collisions 
using different ATLAS 
luminometers and the right plot shows the ratio of this measurement with respect to a LUCID measurement. One of the 
detectors shows a deviation of up to 2$\%$ during this LHC fill but the other measurements are all in agreement 
with LUCID to better than 0.5$\%$.

The first month of data taking with the new detector therefore showed that LUCID could measure the relative 
luminosity with a precision of about 0.5$\%$.

\begin{figure}
\centering
\includegraphics[width=.7\textwidth]{LUCID/InternalConsistency_preliminary.pdf}
\caption{Fractional difference in measured luminosity between the forward (A) and backward (C) arms of the LUCID 
detector. The agreement between the two LUCID arms is better than 1$\%$. Taken from Ref.~\cite{publicPlots}.}
\label{fig:InternalConsistency}
\end{figure}


\begin{figure}
\centering
\begin{subfigure}{.5\textwidth}
  \centering
  \includegraphics[width=\linewidth]{LUCID/LumiVsTime_preliminary_v2.pdf}
  \label{fig:sub3}
\end{subfigure}%
\begin{subfigure}{.5\textwidth}
  \centering
  \includegraphics[width=\linewidth]{LUCID/DeviationSubsystems_preliminary_v2.pdf}
  \label{fig:sub4}
\end{subfigure}
\caption{(left) Average number of inelastic proton-proton collisions per bunch crossing during a 13 TeV fill; 
(right) Comparison of measured luminosity by different luminometers in ATLAS with respect to LUCID. Taken from Ref.~\cite{publicPlots}.}
\label{fig:LumiVsTime}
\end{figure}


\subsection{Luminosity measurements during 2015-2016}



