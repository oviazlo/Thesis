\chapter{Search for new charged bosons in final states with one muon and missing transverse momentum}
\label{chap:Wprime}

% PLAN
% <What this chapter about - short intro to Wprime analysis>
% <What was my personal contribution to the analysis> 


This chapter describes a search for new spin-1 heavy charged boson (called $\PWprime$) in the final state with one lepton and missing transverse momentum ($E_T^{miss}$).
The search was done with the first $\sqrt{s}$=13~TeV data collected by ATLAS  in 2015, corresponding to a luminosity of 3.2~fb$^{-1}$. The search has been performed in the muon and electron channels, however, this chapter focuses only on the muon channel while for the electron channel only the final results are shown, and the combination of both channels is presented.

%*******************************************************************************
% MOTIVATION
%*******************************************************************************


\section{Search strategy}
\label{sec:wprimeIntro} 

As described in \SectionRef{subsec:bsm_models}, there is a large number of models
which predict a new spin-1 charged gauge boson $\PWprime$. Thus it is not practical to perform a number of dedicated searches for all of them. Therefore, the so-called ``sequential'' Standard Model (SSM) is often used, as a reference benchmark model.
This model provides a clear interpretation of the experimental results and
is also used for comparing the results between experiments.
It assumes the $\PWprime$ boson to be a heavy ``copy'' of the SM $W$ with the same couplings to leptons, quarks and gauge bosons. 
This implies that a new decay channel (with respect to the SM $W$)
should be present: $\PWprime \to WZ$. This would be the dominant decay mode 
for high $\PWprime$ masses and lead to the $\PWprime$ width larger
than its mass at  $m_{\PWprime} > 500$~GeV. 
Some models, such as the Left-Right Symmetric model, described previously, predict this channel to be heavily suppressed, in a case of $m_{W_R} \gg m_{W_L}$.
The branching ratio for this channel is thus set in searches to be zero for any $\PWprime$ mass.
The branching ratio of $\PWprime \to \mu\nu$ or $e\nu$ as a function of $\PWprime$ mass
is shown in \FigureRef{fig:wprimeBR}. The rapid decrease at approximately 200 GeV corresponds to the decay channel $\PWprime \to tb$.
\begin{figure}[!htb]
  \centering
  \includegraphics[width=7.5cm]{Wprime/WprimeBR.eps}
  \caption{Branching ratio of $\PWprime \rightarrow e\nu$ or $\mu\nu$ as a function of the $\PWprime$ mass. Calculated with Pythia8 MC generator.}
  \label{fig:wprimeBR}
\end{figure}

Considering this model one can highlight three key features essential for this search:
\begin{itemize}
 \item Precise modeling of the background prediction. 
 The dominant part of the background originates from the charged-current Drell-Yan process and the analysis selection tests it up to few TeV.
 Thus, it is crucial to use the latest and most precise high-order calculations and corrections available at the time.
 \item High-$p_T$ lepton selection. Due to a simple event selection (one isolated lepton and missing $E_T$) 
 this analysis uses the most energetic lepton candidates available at 13 TeV center-of-mass energy collisions.
 This is why the understanding of the reconstruction efficiency and momentum resolution of high-$p_T$ leptons are very important for the analysis.
 \item Missing transverse energy, $E_T^{miss}$. Along with the lepton momentum, $E_T^{miss}$ is used in the calculation of the transverse invariant mass $m_T$ (which will be defined in \SectionRef{subsec:etmiss}),
 which is the signal discriminant and search variable in the analysis. For $m_T$ to be precisely reconstructed and modelled a good understanding of the missing $E_T$ reconstruction as well as good lepton momentum resolution is needed.
\end{itemize}

\section{$\PWprime$ signal in Monte Carlo}
\label{sec:wprimeSignal} 
% TODO rewrite it a little bit, cause it is directly copied from the note!!!

% TODO description of the flat sample:
% Wprime flat template sample which has a flat lepton-neutrino invariant mass spectrum via the removal of the breit-wigner term from pythia generation
% 
% the sample produced will have a mass spectrum which is essentially flat, with statistics at low to high mass, with no resonance shape. 
% This spectrum can then be almost exactly re-weighted to any Wprime resonance mass, given there are enough statistics in the flat spectrum

% MC generator
% TODO comment from Monika
Samples for the signal process $\PWprime \to \mu \nu$ are produced with the leading-order (LO) 
{\scshape pythia-8.183}~\cite{pythia8} generator for a series of $\PWprime$ masses. 
Additionally, the so-called flat sample is produced. This sample has a flat lepton-neutrino invariant mass spectrum.
Thus, it can be reweighted with the correct line shape to any desired $\PWprime$ mass by an appropriate reweighting function.
To verify the validity of this sample and validity of the reweighting procedure, comparisons with fixed mass samples were done.
% TODO Else asking for a reference for sentence above... should I show this comparison here? alos it was not me who did them.
% TODO make this fucking plot. however a priority is low.
The flat sample was generated with large statistics to cover
a wide range of transverse invariant mass $m_T$.

% TODO explain what is ``Wprime signal''... see comment from Oxana by Dec7
% comparison with W boson background (show plot of Wprime overlaid on top of W)
The invariant mass and transverse mass distributions for the $\PWprime$ samples with pole masses of 2, 3, 4 and 5 TeV are shown superimposed on top of the SM $W$ background in \FigureRef{fig:signal_with_W}.

\begin{figure}
\begin{subfigure}{.5\textwidth}
  \centering
  \includegraphics[width=\textwidth]{Wprime/Signal_onTopOf_W_invMass.eps}
\end{subfigure}%
\begin{subfigure}{.5\textwidth}
  \centering
  \includegraphics[width=\textwidth]{Wprime/Signal_onTopOf_W_mT.eps}
\end{subfigure}
\caption{Invariant mass (left) and transverse mass (right) spectrum of the $\PWprime$ signal on top of the $W$ background on generated MC level.}
  \label{fig:signal_with_W}
\end{figure}

%*******************************************************************************
% BKG PROCESSES AND MC
%*******************************************************************************


\section{Background processes}
\label{sec:wprimeBackgrounds}

In order to look for a potential signal from  new physics, one has to examine all other SM processes which contribute to the final state of interest.

% All background predictions are obtained with MC simulation, except for non-prompt
% lepton contribution, which arising due to jets and photon being misreconstructed
% as leptons.

Since this chapter is focused on the muon decay channel, the processes which produce a muon and missing transverse momentum will be discussed, 
however, in general, the same processes are relevant for the electron decay channel.

The dominant expected background in the analysis comes from the charged current Drell-Yan process.
The SM W decays to a lepton and a neutrino, which will be reconstructed as missing transverse momentum in the detector.
The contribution of this process appears like a Jacobian peak in the $m_T$ spectrum
with a maximum around 80 GeV and a slowly falling tail above 80 GeV.
% TODO probably I need to move it somewhere else, no?
Since $\PWprime$ conceptually is a heavier version of the SM W, it also appears in the transverse mass distribution as a Jacobian peak around the pole mass of the $\PWprime$ boson (as shown in \FigureRef{fig:signal_with_W}). 
% W boson bkg
The charged current Drell-Yan process is simulated with \powhegbox\ v2~\cite{Alioli:2010xd}and {\scshape pythia-8.186} generators at the next-to-leading-order (NLO) using the CT10~\cite{CT10} NLO PDFs. 
The cross section is corrected to the next-to-next-to leading order (NNLO) using the CT14NNLO PDF set by applying QCD and Electroweak (EW) mass-dependent $K$-factors to the MC generator cross sections.
% TODO include table or just plot is enough?
To get a sufficient statistics at the high transverse mass, several samples, binned in invariant mass of the lepton-neutrino pair, are used and are shown in \FigureRef{fig:wenu_notstacked}.

% TODO
% TODO replace these plots with my own!!! also change caption
% TODO
\begin{figure}[!htb]
\begin{subfigure}{.5\textwidth}
  \centering
  \begin{overpic}[width=\textwidth]{Wprime/WenuPlus_TruthLepBornElectrons_hist_invmass_notStacked.eps}
    \put (50,61) {\crule[white]{0.3\textwidth}{0.05\textwidth}}
  \end{overpic}
%   \includegraphics[width=\textwidth]{SS/support_note/limits/limit_ee_neg.eps}
\end{subfigure}%
\begin{subfigure}{.5\textwidth}
  \centering
  \begin{overpic}[width=\textwidth]{Wprime/WenuPlus_TruthLepBornElectrons_hist_mt_logX_notStacked.eps}
    \put (50,61) {\crule[white]{0.3\textwidth}{0.05\textwidth}}
  \end{overpic}
%   \includegraphics[width=\textwidth]{SS/support_note/limits/limit_ee_pos.eps}
\end{subfigure}
  \caption{Invariant mass and transverse mass spectrum of the invariant mass-binned W samples. The coloured lines show the different mass slices and the black line -- the sum of all, scaled up with a factor of 2.}
  \label{fig:wenu_notstacked}
\end{figure}

% Z boson bkg
The neutral current Drell-Yan process $Z/\gamma^* \to \mu \mu$ can contribute to the muon plus $E_T^{miss}$ final state if one of the muons is not properly reconstructed in the detector and thus contributes to the value of $E_T^{miss}$. 
In this case, it will not be used in the $E_T^{miss}$ calculation itself, but will contribute to the $E_T^{miss}$, as will be described in \SectionRef{subsec:etmiss}.
This process is simulated with the same MC generators and at the same order as the W boson production process.
% Contribution from taus
The contribution of the processes $W \to \tau \nu$ and $Z \to \tau \tau$, which can affect the muon channel, 
if the tau lepton decays to muon and neutrinos ($\tau^{-} \to \mu^{-} \overline{\nu_{\mu}} \nu_{\tau}$), are considered and simulated in the same way.

% Top
Another background which contributes to the final state of interest is the $t\bar{t}$ and single top production.
The top quarks decay immediately to a $W$ boson and a $b$ quark. The further leptonic decay of the $W$ provides an isolated muon and $E_T^{miss}$ from the the neutrino.
% TODO redo Feynman diagrams
Some Feynman diagrams of the top production processes are shown in \FigureRef{fig:single_top_feynman} and \FigureRef{fig:top_feynman}.
This background is simulated with \powhegbox\ and {\scshape pythia-6.428}~\cite{Pythia} at NLO using the CT10~\cite{CT10} NLO PDFs.
All these processes are considered as ``Top'' background in what follows.

\begin{figure}[!htb]
  \centering
  \includegraphics[width=.95\textwidth]{Wprime/top_2_feynman.png}
  \caption{Feynman diagrams for the single top production in the s-channel (left) and t-channel (right).}
  \label{fig:single_top_feynman}
\end{figure}

\begin{figure}[!htb]
  \centering
  \includegraphics[width=.52\textwidth]{Wprime/top_feynman.png}
  \includegraphics[width=.47\textwidth]{Wprime/top3_feynman.png}
  \caption{Feynman diagrams for the production of a top quark pair (left) and $W + t$.}
  \label{fig:top_feynman}
\end{figure}

% Diboson
More than one SM gauge boson can be produced in a single hard interaction, thus processes with $WW$, $WZ$ and $ZZ$ boson pairs produced in the final state are also contributing to the background.
Some Feynman diagrams of such processes are shown in \FigureRef{fig:diboson_feynman}.
Contribution to the muon plus $E_T^{miss}$ final state can come from decays like $WZ \to \Plepton \nu \nu \nu$ or $WZ \to \Plepton \nu q \overline{q}$.
These processes are simulated with {\scshape sherpa-2.1.1}~\cite{Sherpa} using the CT10 NLO PDFs.
They are called a ``diboson'' background in what follows. 

\begin{figure}[!htb]
  \centering
  \includegraphics[width=.95\textwidth]{Wprime/diboson_feynman.png}
  \caption{Feynman diagram for the diboson production.}
  \label{fig:diboson_feynman}
\end{figure}

% mT-binned samples
Only the inclusive samples for both diboson and top backgrounds were available from the official ATLAS production. 
This is why it was decided to produce samples binned in the transverse mass of lepton plus $E_T^{miss}$.
However, due to technical complications, these samples were not finished in time for the 2015 analysis, 
thus a dedicated extrapolation procedure to the high-$m_T$ region has been used to estimate these backgrounds.
It is planned to use these samples for the upcoming paper with the combined results of 2015 and 2016.

% Table with MC
The list of all background processes and used MC generators is shown in \TableRef{tab:MC_cross}.

\begin{table}[ht]
  \begin{center}
    \begin{tabular}{l|c|c|c}

      \hline
Process &  Generator&  PDF set & Normalization \\
&  + fragmentation/ &  & based on \\
&  hadronization & &\\
\hline\hline
&   &   \multirow{4}{*}{CT14NNLO~\cite{Dulat:2015mca}} & NNLO QCD \\
$W +$ jets, & \powhegbox\ v2~\cite{Alioli:2010xd} & &  with \vrap~\cite{vrap}, \\
$Z/\gamma^* +$ jets & + {\scshape pythia-8.186}~\cite{pythia8}  & &  NLO QED \\
 & & &  with \mcsanc~\cite{Bardin:2012jk,Bondarenko:2013nu} \\
\hline
\ttbar, t-channel $t$, & \powhegbox\ & \multirow{2}{*}{CT10} & \multirow{2}{*}{NLO QCD} \\
s-channel $Wt$ & + {\scshape pythia-6.428}~\cite{Pythia} & &  \\
\hline
\multirow{2}{*}{$WW, WZ, ZZ$} & \multirow{2}{*}{{\scshape sherpa-2.1.1}~\cite{Sherpa}} & \multirow{2}{*}{CT10} & \multirow{2}{*}{NLO QCD} \\
 & & &  \\
\hline
\hline
\multirow{2}{*}{$W^\prime \rightarrow \Plepton \nu$} & \multirow{2}{*}{{\scshape pythia-8.183}} &   \multirow{2}{*}{NNPDF2.3 LO} & NNLO QCD \\
& & &  with \vrap, \\
\hline
\end{tabular}
\end{center}
  \caption{List of MC generated samples used for background prediction. 
  The used MC generator, PDF set and order of cross section calculations used for the normalization are listed for each sample.
  }
\label{tab:MC_cross}
\end{table}

%*******************************************************************************
% SELECTION
%*******************************************************************************


\section{Event object selection}
\label{sec:wprimeSelection}

The analysis is based on \pp collision data collected in 2015 by the ATLAS detector with 13 TeV CM energy.
The integrated luminosity of the data sample corresponds to 3.2 fb$^{-1}$ and the mean number of interactions per bunch crossing was 14.

An event selected for the analysis has to have at least one reconstructed vertex with at least two tracks matched to it. 
If there are several vertices, the one with the highest
$\sum p^2_T$, where $p_T$ is the transverse momentum of the matched tracks, is chosen.
Events should have at least one muon candidate, and fire the single muon trigger, 
which requires the presence of one muon with $p_T > 50$ GeV.

\subsection{Lepton selection}
\label{subsec:lepton_selection}
% description of muon reconstruction in the ATLAS detector
The muon reconstruction in ATLAS is performed independently in ID and MS detectors. 
Information from the detectors is then combined to form a muon track~\cite{muon_performance_2015}.
The track reconstruction is done 
separately in the ID and MS, after which a global refit is done to form a combined track.
% TODO what is measured: $p_T$ or momentum? check muon paper
Muon $p_T$ is measured from the track curvature.

% As well as for electrons, there are different sets of identification criteria which provide different background supression, reconstruction efficiency and momentum measurement resolution: Loose, Medium, Tight and High$-pT$.

Muons of interest are high-$p_T$ isolated muons, with tracks that originate
from the primary vertex. Candidates have to satisfy the following set of criteria:
\begin{itemize}
 \item $p_T > 55$ GeV: to ensure a high and uniform trigger efficiency.
 \item $|\eta|<2.5$, excluding $1.01 < |\eta| < 1.1$: muons have to be within ID acceptance. 
 A small region is excluded in order to reject muons whose tracks in the muon spectrometer fall into poorly aligned chambers (relative barrel-endcap alignment).
%  TODO see comment from Monika to SS chapter.
 \item Transverse and longitudinal impact parameters $|d_0|/\sigma(d_0)<3$ and $|z_0\times sin\theta|<10$~mm: 
 this selection ensures that the muon was produced close to the primary vertex and rejects muons originating from decays of long-lived particles. The recommendation by the muon combined performance (MCP) working group has
 a more strict requirement $|z_0 \times sin \theta| < 0.5$ mm, however, as this criteria rejects many muons and just a little background it  
 was decided to relax this requirement to 10 mm as will be described in \SectionRef{subsec:wprime_cut_optimization}.
 \item Pass ``high-$p_T$'' set of muon identification criteria. 
 In the same way, as for electrons, ATLAS defines three sets of reference muon identification criteria designed for use in analyses: ``loose'', ``medium'',  ``tight''. These criteria are designed in a hierarchical way -- increasing the background rejection power at some cost to the identification efficiency.
 In 2015 a new ``high-$p_T$'' set has been introduced which aims to maximize the momentum resolution for tracks with $p_T > 100$~GeV~\cite{muon_performance_2015,ATLAS-CONF-2016-024}.
 It includes strict requirements on the MS part of the track, 
 which improves the $p_T$ resolution by approximately 30$\%$ in comparison with the ``tight'' criteria.  However, the muon reconstruction efficiency drops by 20$\%$ as well.
%  TODO describe why LooseTrackOnly isolation is enough
 \item Pass ``LooseTrackOnly'' isolation requirement.
 ATLAS defines seven muon isolation working points for use in analyses.
 They differ by used discriminating variables and by definition
 ~\cite{muon_performance_2015,ATLAS-CONF-2016-024}.
 ``LooseTrackOnly'' requirement provides 99$\%$ constant efficiency over the complete ($\eta$,$p_T$) phase space.
 The discriminating variable is the ratio of the sum of $p_T$ of all tracks (excluding the muon itself) with $p_T > 1.0$~GeV within $\Delta R = min(10$ GeV $/p_T^{\mu}, 0.3)$ 
 cone around the muon track, to the muon track $p_T^{\mu}$.
\end{itemize}  
 
\subsection{Optimization of the signal selection}
\label{subsec:wprime_cut_optimization}
% TODO additional lepton veto cut

To suppress the contribution from the neutral current Drell-Yan $Z/\gamma^*$ and $t\bar{t}$ processes,
in which two isolated leptons are expected in the final state, an additional lepton veto requirement
is applied. 
Events are rejected if a second muon with $p_T > 20$ GeV is found passing
either the high-$p_T$ or the medium identification criteria.
Events are also vetoed if an additional electron passing the following selection is found:
\begin{itemize}
\item $|\eta| < 2.47$, excluding barrel-endcap calorimeter transition region $1.37 < |\eta| < 1.52$
\item $p_T > 20$~GeV
\item Transverse impact parameter $|d_0|/\sigma(d_0) < 5$
\item Pass the medium muon identification criteria
\item Pass the ``Loose'' isolation criteria~\cite{muon_performance_2015,ATLAS-CONF-2016-024}.
In contrary to the ``LooseTrackOnly'' isolation described before, it uses an additional calorimeter-based discriminating variable to provide the same efficiency. The additional variable is the ratio of the sum of the transverse energy of topological clusters~\cite{Aad:2011he} (excluding the muon contribution itself) within $\Delta R = 0.2$ cone around the muon track, to the muon track $p_T^{\mu}$.
\item Electron must not overlap with the muon: $\Delta R(e,\mu)>0.1$. If it overlaps ($\Delta R(e,\mu)>0.1$)
it is assumed that the electron candidate arises from photon radiation from the muon and the event is kept.
\end{itemize}
These veto requirements lead to a significant reduction of the dimuon ($Z$) background
at high transverse mass as well as some reduction of the $t\bar{t}$ background.
The reduction of the total background level is approximately $10$--$15\%$ at high transverse mass. 
The signal efficiency is found to be essentially unaffected.
The possibility of using the ``loose'' identification working point for the additional muon veto was also considered. However, it was found to provide only a tiny improvement ($1$--$3\%$ additional reduction of the total background level) with respect to using ``medium'' working point, thus the latter was chosen to be used.

% TODO z sin_theta
As mentioned before, the requirement $|z_0 \times sin \theta| < 0.5$ mm should be applied by default. The main purpose of this requirement is to veto events with cosmic muons.
However, vetoing additional muon in the event we ``automatically'' discard most of the events with cosmic muons. After a dedicated study by the MCP group, it was found that the 
$|z_0 \times sin \theta|$ requirement can be loosened to 10 mm, without significant decrease of the cosmic muon rejection power.
Figure \ref{fig:Muon_LepVtxEff} shows the $d_0$ significance ($|d_0|/\sigma(d_0)$) and $|z_0 \times sin \theta|$ cut efficiencies. The efficiency shown in the left panel is for the recommended $|z_0 \times sin \theta|$ selection value of $0.5$ mm, while in the right panel -- for a looser
cut value of $10$~mm. The nominal cut value leads to a reduction of the selection efficiency by about $1$\% due to a wrong vertex being associated to the primary vertex. This efficiency is partially restored by the loosened cut.

\begin{figure}[]
  \centering
  \includegraphics[width=0.45\textwidth]{Wprime/IPplot05mm.eps}
  \includegraphics[width=0.45\textwidth]{Wprime/IPplot10mm.eps}
  \caption{$d_{0}$ and $|z_0 \times sin \theta|$ cut efficiencies. The $d_{0}$ efficiency is shown for the cut recommended by the tracking group. In the $|z_0 \times sin \theta|$ case the recommended cut of $0.5$ mm (left) and an alternative cut of $10$ mm (right) are shown. The efficiencies are calculated for combined muons in the
$\PWprime$ flat sample passing the medium or high-$p_T$ working point requirements.}
  \label{fig:Muon_LepVtxEff}
\end{figure}


\subsection{Transverse mass and missing transverse momentum}
\label{subsec:etmiss}
The missing transverse momentum, $E_T^{miss}$, is calculated following the ATLAS recommendation described in Ref.~\cite{met2015_1,met2015_2}.
$E_T^{miss}$ is calculated as a vector sum of the $p_T$ of selected objects:
\begin{itemize}
 \item muons which satisfy the analysis signal selection;
 \item electrons which satisfy the requirements described previously in
 \SectionRef{subsec:wprime_cut_optimization} with a stronger transverse momentum requirement of $p_T > 55$ GeV
 and passing tight electron identification criteria (which is the electron signal selection);
 \item tau leptons which satisfy the ``medium'' identification criteria~\cite{tau_id_8TeV} and $|\eta| < 2.5$, excluding  $1.37 < |\eta| < 1.52$ and $p_T > 20$~GeV requirements;
 \item photons which satisfy the ``tight'' identification criteria~\cite{photon_id_2011}, $|\eta| < 2.37$, excluding  $1.37 < |\eta| < 1.52$ and $p_T > 25$~GeV requirements;
 \item jets reconstructed with the anti-$k_t$ algorithm~\cite{jet_anti_kt} with the radius parameter of 0.4 in $|\eta| < 4.9$.
 Jets are calibrated using the method described in Ref.~\cite{jet_calib_syst_13TeV}.
 Only jets with $p_T > 20$~GeV are used;
 \item tracks originating from the primary vertex with $p_T > 0.5$ GeV and $|\eta| < 2.5$ but not belonging to any of the reconstructed physics objects listed above.
\end{itemize}
The missing transverse momentum is required to be larger than 55~GeV in order 
to balance the lepton transverse momentum cut.
This cut value allows one to significantly suppress 
the multi-jet background, which will be described below.

% TODO describe why if muon is not reconstructed it will contribute to Etmiss (see Magnars thesis, page 89)

The main variable of interest, $m_T$, which is used for statistical discovery analysis is defined as:
\begin{equation}
 m_\mathrm{T} = \sqrt{2 p_\mathrm{T} E_T^{miss} (1-\cos\varphi_{\Plepton\nu})}
\end{equation}
where $\varphi_{\Plepton\nu}$ is the angle between the muon momentum and the missing transverse momentum in the transverse plane.

The transverse mass has to be $m_\mathrm{T} > 110$~GeV which corresponds to twice the required  value of the muon
momentum and $E_T^{miss}$.

%*******************************************************************************
% BACKGROUND ESTIMATION
%*******************************************************************************

\section{Background estimation}
\label{sec:wprime_backgroundEstimation}
% TODO describe sources of fakes:

The background which arises from neutral and charged current Drell-Yan processes ($W$, $Z$), diboson production ($WW$,$WZ$,$ZZ$) and top background  (\ttbar, single top) is estimated with MC simulation.

Processes with multijet final state have a small chance 
to have a reconstructed muon present in these events.
% of being reconstructed as those which contain an isolated muon. 
There are two cases when jets can lead to a reconstructed muon. 
In the first case a hadron is not stopped in the calorimeter and passes through to the muon spectrometer and is misidentified as a muon. Another possibility is if a real muon originates from a jet. However, such muons are not produced in the primary vertex, because they originate from decays of heavy flavor hadrons, which have long lifetimes and travel for few hundreds of $\mu m$'s before they decay.
Thus, such muons from jets are called ``fake'' muons.

Due to the huge cross section of the processes which lead to multijet final states 
% it is very difficult and time consuming to simulate their production with MC simulation, 
it would be very difficult to model this background in the simulation,
this is why a data-driven method,
the so-called Matrix Method, is used to estimate the contribution of multijet processes to the signal muon selection.

% The backgrounds which have final state with a real muon originating from the primary vertex and 
% which can contribute to the signal selection are modelled with MC simulation.
% List of all considered processes is shown in \TableRef{tab:MC_cross}.
% Such muons are called ``real'' (prompt) muons.
% 
% However, processes with multijet final states (with one or more jets) will also contribute to
% the signal selection due to probability of wrong reconstruction of the jet activity as a muon.
% Such muon candidates are called ``fake'' (non-prompt) muons.
% The ``fake'' muon can be a real muon which originate from a heavy flavor hadron decay within a jet
% or from pion or kaon decays. But because they are not originate from the primary vertex they are not
% desired to be selected in the signal region and thus are called ``fake'' muons.
% The ``fake'' muon background is estimated using a data driven technique. 
% ``Fake'' muons are expected to be in general non-isolated, although some
% fraction does pass the isolation cut and ends up in the selected event sample. Isolation variables
% hence provide a strong separation of ``fake'' and ``real'' muons, and are essential to the data-driven
% estimation of the ``fake'' background.
% The ``fake'' background is estimated using the Matrix Method, which is presented below.

\subsection{The Matrix Method}
\label{subsec:matrix_method}
The goal of this method is to get an estimation of the ``fake'' muon contribution to the signal region.
The ``fake'' muons are expected to be non-isolated, however, a fraction of them pass the isolation cut. 
The idea of the method is to measure the probability (efficiency) of the ``fake'' muons
(which pass the loosened signal selection without isolation requirement) to pass the nominal muon signal selection without requiring isolation.
The same efficiency is measured for ``real'' muons, which originate from processes with muons from the primary vertex.

The matrix method provides a connection between the number of true ``fakes'' ($N_F$) and ``real'' muons ($N_R$) with the measurable quantities: the number of muon candidates which pass loose but fail tight selection criteria ($N_L$) and the number of candidates which pass the tight selection criteria ($N_T$), via \EquationRef{eq:mm1}.
\begin{equation}
  \left(\begin{array}{c} N_T \\ N_L \end{array}\right)&=
  \begin{pmatrix}
    \epsilon_R & \epsilon_F \\
    1-\epsilon_R & 1- \epsilon_F \\
  \end{pmatrix}
  \left(\begin{array}{c} N_R \\ N_F \end{array}\right)
  \label{eq:mm1}
\end{equation} 
The vector on the right hand side of the equation corresponds to the
truth quantities which are independent from each other.
This implies the quantities in the vector on the left hand side
have to be independent as well, thus $N_L$ should not contain any events
from $N_T$, and therefore the former value is defined as passing the loose selection but failing the tight one.

The matrix consists of efficiencies of ``fake'' and ``real'' muons passing the signal selection which are denoted as $\epsilon_F$ and $\epsilon_R$, respectively.
The efficiency is defined as the ratio of the number of ``fake''(``real'') muons which pass the tight selection,
$N_{tight}^{fake}$ ($N_{tight}^{real}$), to the number of muons which fail the tight but pass the loose selection,
$N_{loose}^{fake}$ ($N_{loose}^{real}$):
\begin{equation}
 \epsilon_F = \frac{N_{\textrm tight}^{\textrm fake}}{N_{\textrm loose}^{\textrm fake}} ,\qquad & \epsilon_R = \frac{N_{\textrm tight}^{\textrm real}}{N_{\textrm loose}^{\textrm real}}.
  \label{eq:mm2}
\end{equation}
The total number of muons passing the signal selection is given in the first line of the matrix:
\begin{equation}
 N_T&=N_{\textrm tight}^{\textrm real} + N_{\textrm tight}^{\textrm fake}=\epsilon_R N_R + \epsilon_F N_F\, ,
\label{eq:number_of_tight_muons}
\end{equation}
and consists of the fraction of ``fake'' muons which pass the signal selection and the fraction of ``real'' muons.
The value of interest is the true number of ``fake'' muons which pass the signal selection. One can express this quantity by inverting the matrix in \EquationRef{eq:mm1} and using the relation in \EquationRef{eq:number_of_tight_muons}:
\begin{equation}
\left(\begin{array}{c} N_R \\ N_F \end{array}\right)&=
\frac{1}{\epsilon_R(1-\epsilon_F)-\epsilon_F(1-\epsilon_R)}
\begin{pmatrix}
 1- \epsilon_F & -\epsilon_F \\
\epsilon_R-1 & \epsilon_R  \\
\end{pmatrix}
\left(\begin{array}{c} N_T \\ N_L \end{array}\right)
\end{equation} 
Thus, the true number of ``fake'' muons which pass the signal selection is:
\begin{equation}
N_{\textrm tight}^{\textrm fake} = \epsilon_F N_F=\frac{\epsilon_{F}}{\epsilon_{R}-\epsilon_{F}}\left(\epsilon_{R}(N_{L}+N_{T})-N_{T}\right) \,
  \label{eq:mm5}
\end{equation}
which is expressed by quantities which can be calculated from the data.

The cut used to distinguish between loose and tight muons is the isolation cut, and the loose muons
are thus defined as passing the signal muon selection cuts, except the isolation cut. Tight muons correspond
to the baseline selection.

The ``real'' muon efficiency is extracted from MC simulation, which reproduces well the efficiency of the isolation cut in data. The ``fake'' muon efficiency is measured in a control region designed to have a high purity of ``fake'' muons. The region is defined in the same way as the signal selection besides cuts on $E_T^{miss}$ and $m_T$, and requiring in addition:
\begin{itemize}
\item At least one jet with $p_T > 40$~GeV which does not overlap ($\Delta R > 0.2$)
with the selected muon.
\item Opening angle in the transverse plane between the muon and the $E_T^{miss}$, $\Delta\phi_{\mu,E_T^{miss}} < 0.5$.
\item No $Z$ candidate (any two muons with $80 < m_{\mu\mu} < 100$~GeV).
\item $d_0$ significance, $|d_0|/\sigma(d_0)$, greater than $1.5$.
\item $E_T^{miss} < 55$~GeV, ensuring that the control region does not overlap with the signal region.
\end{itemize}
This region is enhanced with ``fake'' muons, however, a significant ``real'' muon contamination is present in the region as well. This contribution, as predicted by MC simulation, is subtracted.

The obtained efficiencies are shown in \FigureRef{fig:matrix_method_efficiencies}.
\begin{figure}[]
  \centering
  \includegraphics[width=0.45\textwidth]{Wprime/realEffNominal.eps}
  \includegraphics[width=0.45\textwidth]{Wprime/fakeEffNominal.eps}
  \caption{Efficiency of the ``real'' (left) and ``fake'' (right) muons as a function of muon $p_T$ used in the data-driven matrix method to estimate contribution of the multijet background to the signal selection.}
  \label{fig:matrix_method_efficiencies}
\end{figure}

The systematic uncertainty was estimated for both ``real'' and ``fake'' muon efficiencies.
% TODO Monika - comments
The ``fake'' muon efficiency uncertainty was estimated by variation of the requirements for ``fake'' muon control region. These variations are:
% TODO rewrite list!!!
\begin{itemize}
\item Removing the $Z$ veto and $\Delta\phi_{\mu,\met}$ cuts.
\item Removing the $Z$ veto and $\Delta\phi_{\mu,\met}$ cuts, but tightening the $d_0$ significance cut to $2$.
\item Removing the $d_0$ significance cut. 
\item Using a tighter $d_0$ significance cut of $2$.
\item Removing the jet requirement.
\item Removing the jet requirement, but tightening the $d_0$ significance
cut to $2$.
\item Requiring $\met < 20~\GeV$.
\item Requiring $20 < \met < 55~\GeV$.
\end{itemize}
The ``fake'' efficiency was recalculated by using each of this requirements separately.
The effect on the ``fake'' muon efficiency is shown in \FigureRef{fig:matrix_method_systematics} (right).

Since the ``real'' muon efficiency is obtained from MC simulation one can use the efficiency obtained with the tag-and-probe method (using muon pairs in the invariant mass window $80 < m_{\mu\mu} < 102~\GeV$ from $Z\to\mu\mu$ decays) as a variation of the systematic uncertainty. 
A comparison of the efficiencies obtained with MC simulation and with the tag-and-probe method are shown in \FigureRef{fig:matrix_method_systematics} (left).

\begin{figure}[]
  \centering
  \includegraphics[width=0.45\textwidth]{Wprime/realEffVariations.eps}
  \includegraphics[width=0.45\textwidth]{Wprime/fakeEffVariations.eps}
  \caption{Systematic variations of the ``real'' (left) and ``fake'' (right) muon efficiencies as a function of muon $p_T$.}
  \label{fig:matrix_method_systematics}
\end{figure}

The impact of the systematic variations of the efficiencies on the final $m_T$ spectrum of the multijet background will be discussed in \SectionRef{subsec:multijet_systematcs}.

%*******************************************************************************
% VALIDATION REGIONS
%*******************************************************************************

\subsection{Multijet validation region}
% TODO review this section carefuly again!!!

% definition of the validation region
To test the data-driven multijet background prediction, one has to find a region in which its contribution will be enhanced. 
This region has to be kinematically close to the signal region.
Therefore, a validation region is defined in the same way as the signal selection but without the $E_T^{miss}$ and $m_T$ requirements. 
Two sets of validation regions are considered: with and without using isolation requirements. They correspond to the tight and loose muon definitions used in the matrix method.

% plots used for multijet CR
% TODO Monika - see comments from Dec14
The distributions of the variables used to define the enhanced ``fake'' muon control region are shown in \FigureRef{fig:muMMval1}.
A reasonable agreement within 10$\%$ for both the tight and loose distributions is observed for $\Delta\phi_{\mu,E_T^{miss}}$ and muon $d_0$ significance, as can be seen from the data over background prediction ratio plots in the bottom of each plot. The shape of the distributions is modelled well too for both the tight and loose regions.
The most obvious discrepancy is seen in the
distribution of the number of jets. The disagreement most likely arises from the modeling of jet emission in the $W +$ jets MC, where only one jet emission is included at the
matrix element level.
This in principle affects the MC subtraction in the ``fake'' muon control region.
However, the discrepancy is present only for $N_\mathrm{jet}\geq2$, while the control region is 
defined with the requirement $N_\mathrm{jet}\geq1$ and thus this discrepancy does not affect strongly the ``fake'' muon efficiency calculations.
Furthermore, one of the systematic variations of the ``fake'' muon efficiency was obtained with no $N_\mathrm{jet}$ cut at all. The effect of the variation of the $N_\mathrm{jet}$ requirement was found to be negligible. 

% TODO think about some additional description here!
% TODO check in the Magnars thesis
The $E_T^{miss}$, muon momentum and $m_T$ distributions are shown in \FigureRef{fig:muMMval2}.
The contribution of the ``fake'' background to the tight distributions (right-hand plots) is tiny. Thus, the agreement between background estimation and observation demonstrates that the background is well modelled by the MC simulation.
Since no $E_T^{miss}$ and $m_T$ cuts are applied there is significantly more statistics available than in the signal region. This allows to test the overall background prediction more precisely.

In the sample fulfilling the looser muon definition (left-hand plots) the multijet background is significantly enhanced which provides us with a handle to check the ``fake'' background estimate. 
A reasonable agreement within 10-15$\%$ for both the tight and loose selections is observed for all distributions.
The level of the agreement is comparable between loose and tight distributions, which shows the validity of the ``fake'' background estimation.

Taking into account the size of the systematic uncertainties, discussed in \SectionRef{sec:wprimeSystematics}, the data and background prediction agree reasonably well. 


% TODO Monika - see comments
% TODO Else - you could spend a few lines describing also the shape agreement in the plots.
% As one can note the multijet background is significantly enhanced in the looser muon sample with respect to the tight sample. 
% A reasonable agreement within 10$\%$ for both tight and loose distributions is observed for all distributions. Also, shape of the distribution is described reasonably well for both cases as well.
% Taking into account the size of the systematic uncertainties, discussed in \SectionRef{sec:wprimeSystematics}, the data and background prediction agrees reasonably well. 
% The tight muon sample includes the isolation requirement and thus the contribution from multijet background is significantly reduced. Also because no $E_T^{miss}$ and $m_T$ are applied 
% there are significantly more statistics available than in the signal region which allows to test the overall background prediction more precisely. In general, a reasonable agreement is observed. 

% TODO Else: I like the explanation, but think that you could spend a few lines describing also the shape agreement in the plots.

\begin{figure}[]
  \centering
  \includegraphics[width=0.49\textwidth]{Wprime/Njet40loose.eps}
  \includegraphics[width=0.49\textwidth]{Wprime/Njet40.eps}
  \includegraphics[width=0.49\textwidth]{Wprime/deltaPhiLoose.eps}
  \includegraphics[width=0.49\textwidth]{Wprime/deltaPhi.eps}
  \includegraphics[width=0.49\textwidth]{Wprime/d0sigLoose.eps}
  \includegraphics[width=0.49\textwidth]{Wprime/d0sig.eps}
  \caption{
  Distributions of the number of jets (top), $\Delta\phi_{\mu,\met}$ (middle), 
  and $d_0$ significance (bottom) in the inclusive loose (left) and tight (right) muon samples. 
  The distributions are considered before the $E_T^{miss}$ and $m_T$ cuts.
}
  \label{fig:muMMval1}
\end{figure}
\begin{figure}[]
  \centering
  \includegraphics[width=0.49\textwidth]{Wprime/METloose.eps}
  \includegraphics[width=0.49\textwidth]{Wprime/MET.eps}
  \includegraphics[width=0.49\textwidth]{Wprime/pTloose.eps}
  \includegraphics[width=0.49\textwidth]{Wprime/pT.eps}
  \includegraphics[width=0.49\textwidth]{Wprime/mTloose.eps}
  \includegraphics[width=0.49\textwidth]{Wprime/mT.eps}
  \caption{
  Distributions of the $E_T^{miss}$ (top), $p_T$ (middle), and $m_T$ (bottom)
  in the inclusive loose (left) and tight (right) muon samples. 
  The distributions are considered before the $E_T^{miss}$ and $m_T$ cuts.
}
  \label{fig:muMMval2}
\end{figure}


\subsection{Background extrapolation}
% TODO explain why it hurts to have huge statistical fluctuation for limit settings 
% TODO (it was somewhere in the note?)
% % from app_muonSmooth.tex:
% In a single-bin statistical analysis,
% it is straight forward to take into account the statistical uncertainty of the MC background
% estimates, as one can simply add it in quadrature to the systematic uncertainties on the
% background level in the search region. However, in a multi-bin analysis, results depend on
% the shape of the background distribution, and one should avoid propagating clearly unphysical features
% in this shape to the final results.

The MC simulation of the top and diboson background processes was available from the official ATLAS production only as inclusive samples.
They did not provide enough statistics in the high-$m_T$ region.
Therefore, these backgrounds were fitted in the low-$m_T$ region and extrapolated to obtain a smooth description in the high-$m_T$ region.

% TODO this is from Marcus --> change phrasing!!!
The fit was done with functions that were previously used to extrapolate the background, as for example in the $8~\TeV$ dilepton resonance search~\cite{Aad:2014cka}.
One choice is defined as:
\begin{equation}
 f(\mt) = e^{-a} m_\mathrm{T}^{b} m_\mathrm{T}^{c \log(m_\mathrm{T})}
  \label{eq:dijetfunc}
\end{equation}
The second one is:
\begin{equation}
 f(\mt) = \frac{a}{(m_\mathrm{T}+b)^{c}}
  \label{eq:powerlaw}
\end{equation}
These two functions were used to preform fits of the backgrounds in different ranges. 
The best fit according to the $\chi^{2}/N.d.o.f$ value has been used as a central value.
The systematic uncertainty is estimated as the envelope of all fits.
The statistical uncertainty of the fit was found to be negligible.

The starting value for the top background was varied in the range from $140$~GeV to $260$~GeV in steps of $20$~GeV. The end fit point was varied from $600$~GeV to $900$~GeV in steps of $25$~GeV.
For the diboson background these values were from $120$~GeV to $240$~GeV and from $500$~GeV to $700$~GeV, respectively, with the same step widths as for the top sample.
% TODO ``was stitched'' - somehow rewrite it, cause it's not clear
% The extrapolation was stitched to the background estimated by Monte Carlo at $\mt=600$~GeV in both cases.
The extrapolation is used in the $m_T$ spectrum starting from $\mt=600$~GeV for both top and diboson background estimations.

The fits and appropriate systematic uncertainty estimates are shown in 
\FigureRef{fig:mu_extrapolate_top} for the top and in \FigureRef{fig:mu_extrapolate_diboson} for the diboson backgrounds.
\begin{figure}[!htb]
  \centering
  \includegraphics[width=0.49\textwidth]{Wprime/top_extrapolate_fits.eps}
  \includegraphics[width=0.49\textwidth]{Wprime/top_extrapolate.eps}
  \caption{Fit and extrapolation of the top background. Both the full set of individual
fits (left) and the resulting central value and uncertainty (right) are shown.}
  \label{fig:mu_extrapolate_top}
\end{figure}
\begin{figure}[!htb]
  \centering
  \includegraphics[width=0.49\textwidth]{Wprime/diboson_extrapolate_fits.eps}
  \includegraphics[width=0.49\textwidth]{Wprime/diboson_extrapolate.eps}
  \caption{Fit and extrapolation of the diboson background. Both the full set of individual
fits (left) and the resulting central value and uncertainty (right) are shown.}
  \label{fig:mu_extrapolate_diboson}
\end{figure}

The multijet data-driven background estimation suffers from large statistical fluctuations in the high-$m_T$ region, thus the same fitting and extrapolation procedures are also used here.
% A simple power law fit is performed, which was found to be the most appropriate way according to the 8 TeV analysis~\cite{wprime_8TeV}:
% \begin{equation}
% \frac{dN}{d m_T} = a\, m_T^{-b}
% \end{equation}
The fits are performed in the ranges of $150$--$300~\GeV$ and $200$--$300~\GeV$.
The extrapolation was stitched to the multijet background estimate at $\mt=300$~GeV.

% TODO where it was stitched?

%*******************************************************************************
% SYSTEMATICS
%*******************************************************************************

\section{Systematic Uncertainties}
\label{sec:wprimeSystematics}
\subsection{Muon efficiency, resolution and scale}
The muon efficiency corrections are obtained by the MCP group using the tag-and-probe method on $Z\to\mu\mu$ and $J/\psi\to\mu\mu$ decays in data~\cite{MCP13TeV}. Systematic uncertainties
are derived from variations of the tag-and-probe selection, background subtraction and other parameters as defined by the group.~\cite{MCPrun1}.

The muon momentum corrections are obtained by fitting certain correction constants 
to match the invariant mass distribution in $Z\to\mu\mu$ and $J/\psi\to\mu\mu$ decays in MC
to that observed in data~\cite{MCP13TeV}. The dependence of the muon momentum on the fit parameters
is given by a model where each parameter is associated to a certain source of potential data/MC disagreement.
Systematic uncertainties are derived from variations of the fit procedure, alignment studies and other parameters as defined in~\cite{MCPrun1}.

\subsection{Jet energy scale and resolution}
% TODO write something more (about jet energy scale)
The jet energy scale and resolution uncertainties enter the analysis through 
the $\met$ calculation, since the $\met$ is calculated using calibrated jets. 
The uncertainties for the jet energy scale and resolution are provided 
by the ATLAS JetEtMiss working group ~\cite{jet_calib_syst_13TeV, JESUncer13TeV}. 
% A reduced set of uncertainties with three nuisance parameters is chosen for the jet energy scale. This reduced set of
% nuisance parameters simplifies the correlations between the different sources of the jet energy scale uncertainty (JET\_GroupedNP\_1, JET\_GroupedNP\_2, JET\_GroupedNP\_3).
% Four scenarios of correlation models are provided by the JetEtMiss group. The final result of an analysis using the reduced set must not
% depend on a specific choice of correlation model. 
The jet energy scale uncertainty has been tested for different recommended scenarios 
and was found to be negligible for all of them.\\
No resolution smearing is applied in the default scenario. 
According to the working group recommendation, effect of the smearing has to be used as a systematic uncertainty.\\ The jet uncertainties are fully correlated between the electron and muon channel.

\subsection{Missing transverse energy scale and resolution}
The uncertainties for the $\met$ scale and resolution are provided by the JetEtMiss group~\cite{met2015_1}. They enter
the analysis through the soft term in the $\met$ calculation, 
which corresponds to the energy deposits in the calorimeter not associated with
any reconstructed physics objects (leptons, photons, jets).
The uncertainties cover differences between data and MC and are only applied to MC. 
The $\met$ uncertainties are fully correlated between the electron and muon channel. 
The jet, electron and muon energy/momentum uncertainties are affecting the $\met$
calculation. These uncertainties are propagated to the $\met$ calculation in the same way.
% TODO see comment from Oxana about sentence above!

\subsection{Background estimate uncertainty}

The uncertainties of the charged and neutral current Drell-Yan processes were estimated by variations of the value of $\alpha_s$ and electroweak corrections as well as by using different PDF error sets and estimating the difference between these PDF sets.
The $\alpha_s$ influence was estimated by varying $\alpha_s$ by $\pm 0.0003$. This corresponds to the 90\% CL uncertainty. The effect on the $W$ background 
was $3\%$ at most and the effect is therefore neglected. 
The variation of the electroweak corrections was estimated to be larger than $3\%$ and was
taken into account during extraction of limits.
The PDF uncertainty of the CT14NNLO PDF is one of the main theory uncertainties
and it was calculated by using 90$\%$ CL PDF error set.
The uncertainty related to the choice of the PDF set used was estimated by comparing
results with NNPDF3.0~\cite{Ball:2014uwa}.
The difference between CT14 and HERAPDF2.0 is not considered as the PDF does not include high Bjorken-$x$ data. 
% TODO fix above about x. comment from Oxana

The uncertainty of the ``Top'' and ``Diboson'' backgrounds modelling consists of the theoretical cross section uncertainty and the high-$m_T$ extrapolation uncertainty. The former uncertainty affects the total background
prediction by less than 3$\%$ and thus is neglected; the latter one becomes considerable at the high-$m_T$ region and is taken into account during the limit setting step.

The detailed description of the theoretical uncertainties on the MC cross section can be found in~\cite{Aaboud:2016zkn}.

\subsection{Trigger and luminosity}
The systematic uncertainty of the trigger efficiency is evaluated by the ATLAS trigger group and it is related to the trigger
efficiency of muons which is dependent on $\eta$ and $p_T$.
The luminosity uncertainty was estimated in the same way as described in \SectionRef{sec:lucid_performance}, however, the current analysis is using a preliminary luminosity
uncertainty (which was available at the time of the publication of the first 13 TeV paper) equal to 5$\%$. It was obtained from the preliminary calibration of the luminosity scale done using data from van der Meer scans in August 2015.

\subsection{Multijet background}
\label{subsec:multijet_systematcs}

The systematic variations of the ``real'' and ``fake'' efficiency used in the matrix method were
described previously in \SectionRef{subsec:matrix_method}.
The multijet background has a very small contribution to the muon signal selection, as can be seen in 
\FigureRef{fig:muMMfinal} (left), where the fraction of the multijet background to the total background
after the final selection is shown as a function of $m_T$. Thus the effect from the systematic
variations of the matrix method efficiencies on the total background is small as well, as shown in
\FigureRef{fig:muMMfinal} (right). 
The systematic uncertainty on the total background is at the level of 1$\%$ for $m_T<4$~TeV and less than 2$\%$ for $m_T>4$~TeV.

\begin{figure}[!htb]
  \centering
  \includegraphics[width=0.49\textwidth]{Wprime/totalBGfrac.eps}
  \includegraphics[width=0.49\textwidth]{Wprime/totalBGsys.eps}
  \caption{The fraction that the fake muon background constitutes of the total background
as a function of \mt\ (left) and the effect of systematics variations on the total background
level as function of \mt\ (right). The power law fits are used for the fake muon background above
$\mt=300~\GeV$. In the left plot, the fit range $150$--$300~\GeV$ is used, and in the right plot, the
range $150$--$300~\GeV$ corresponds to solid lines while $200$--$300~\GeV$ -- to dashed lines.}
  \label{fig:muMMfinal}
\end{figure}

\subsection{Summary}
\TableRef{tab:syst} lists various systematic uncertainty sources
and their sizes for the background and for the $\PWprime$ signal with $m_T(\PWprime)=$ 2 and 4 TeV at transverse masses equal to 2 and 4~TeV.
All uncertainties below $3$\% have been neglected so far
since they do not affect the final result of the statistical analysis. 
The remaining experimental and theoretical systematics are applied to the background.
Only the experimental uncertainties are applied to the signal. 

\begin{table}
\begin{center}
\centering
\small
\begin{tabular}{l|cc}
\toprule
Source &  Background  &  Signal  \\
\midrule
Trigger &\syspair{3}{4} & \syspair{4}{4}\\
Lepton reconstruction  &\multirow{2}{*}{\syspair{5}{8}} & \multirow{2}{*}{\syspair{5}{7}}\\
and identification & & \\
Lepton isolation &\syspair{5}{5} & \syspair{5}{5}\\
Lepton momentum &\multirow{2}{*}{\syspair{3}{11}} & \multirow{2}{*}{\syspair{1}{4}}\\
scale and resolution & & \\
$E_T^{miss}$ resolution and scale &\syspair{<0.5}{<0.5} &\syspair{<0.5}{<0.5}\\
Jet energy resolution &\syspair{1}{2} &\syspair{<0.5}{<0.5}\\
\midrule
Multijet background & \syspair{1}{1} & {\sc n/a} ({\sc n/a})\\
Diboson \& top-quark bkg. &\syspair{5}{15} & {\sc n/a} ({\sc n/a})\\
PDF choice for DY &\syspair{<0.5}{1} & {\sc n/a} ({\sc n/a})\\
PDF variation for DY &\syspair{8}{12} & {\sc n/a} ({\sc n/a})\\
Electroweak corrections &\syspair{4}{6} & {\sc n/a} ({\sc n/a})\\
\midrule
Luminosity &\syspair{5}{5} &\syspair{5}{5}\\
\midrule
Total &\syspair{14}{25} & \syspair{9}{12}\\
\bottomrule
\end{tabular}
\end{center}
\caption{Systematic uncertainties on the expected number of events as evaluated at $m_T = $ 2 (4)~\TeV, both for signal events 
with a \wpssm\ mass of 2~(4)~\TeV\ and for background. Uncertainties estimated to have an impact
$< 3\%$ on the expected number of events in both channels and for all values of $m_T$ are not listed.
Uncertainties that are not applicable are denoted ``{\sc n/a}''. \label{tab:syst}}
\end{table}

%*******************************************************************************
% SIGNAL REGION
%*******************************************************************************
\section{Signal Region}
\label{sec:wprimeSignalRegion}
The muon $\eta$, $\phi$, $p_T$, and $E_T^{miss}$ distributions in the signal region are shown in \FigureRef{fig:mu_results_etaphi} and \FigureRef{fig:mu_results_ptmet}. 
The dominant contribution to the signal region originates from the W boson background.
No visible excess and good agreement between data and background estimate are observed.

The basis for the statistical analysis and the main distribution of interest 
is the transverse mass distributions which are shown in \FigureRef{fig:MT_mu_Wprime}.
The resonant $\PWprime$ signal overlaid on the background prediction is shown as well.
As one can see from the data over background prediction ratio plot, the data is systematically above the total background prediction
in the low-$m_T$ region but are still within the $\pm 1 \sigma$ uncertainty band, which is dominated by the $E_T^{miss}$ systematic uncertainty at low $m_T$.

\begin{figure}[]
  \centering
  \includegraphics[width=0.49\textwidth]{Wprime/muon_eta.eps}
  \includegraphics[width=0.49\textwidth]{Wprime/muon_phi.eps}
  \caption{
  Muon \eta\ (left) and $\phi$ (right) distributions after the final selection. The uncertainty band in the ratio plot includes all systematic uncertainties which are included in the statistical analysis except the integrated luminosity uncertainty ($5\%$).
}
  \label{fig:mu_results_etaphi}
\end{figure}

\begin{figure}[]
  \centering
  \includegraphics[width=0.49\textwidth]{Wprime/muon_pT.eps}
  \includegraphics[width=0.49\textwidth]{Wprime/muon_MET.eps}
 \caption{
 Muon $p_T$ (left) and $E_T^{miss}$ (right) distributions after the final selection. The uncertainty band in the ratio plot includes all systematic uncertainties which are included in the statistical analysis except the integrated luminosity uncertainty ($5\%$).
}
  \label{fig:mu_results_ptmet}
\end{figure}


\begin{figure}[]
  \centering
  \includegraphics[width=0.65\textwidth]{Wprime/muon_mT.eps}
  \caption{
  Muon $m_T$ distribution after final selection. 
  Shown is the total background estimate with resonant $\PWprime$ signal overlaid for various pole masses. 
  The uncertainty band in the ratio plot includes all systematic uncertainties which are included in the statistical analysis except the integrated luminosity uncertainty (5$\%$).}
  \label{fig:MT_mu_Wprime}
\end{figure}

\TableRef{tab:muBkgData}
shows the contributions of individual backgrounds as well as the total background
and the data in different $\mt$ regions. The quoted uncertainties include both systematic and
statistical uncertainties except the uncertainty on the integrated luminosity ($5\%$).
One can observe a good agreement between the data and the total background prediction in all $\mt$ regions. The charged-current Drell-Yan is the dominant contribution in the high-$m_T$ region which is more than 90$\%$ of the total background for $m_T>1$ TeV. No events with $m_T > 3$ TeV are observed in the data.

\begin{table}[]
  \centering
  \scriptsize
  \begin{tabular}{|c|c|c|c|c|c|c|c|}
    
    \multirow{2}{*}{Process} & \multicolumn{7}{c|}{$m_T$ [\GeV]} \\
& $110$--$150$ & $150$--$200$ & $200$--$400$ & $400$--$600$ & $600$--$1000$ & $1000$--$3000$ & $3000$--$7000$ \\ \hline 
$W$ & $98100\pm10000$ & $21000\pm2000$ & $7700\pm400$ & $476\pm30$ & $110\pm9$ & $13.0\pm1.2$ & $0.051\pm0.010$ \\ 
Top & $9900\pm700$ & $5410\pm340$ & $3090\pm140$ & $120\pm6$ & $13\pm5$ & $0.44\pm0.32$ & $0.00005\pm0.00030$ \\ 
$Z/\gamma^*$ & $7700\pm1000$ & $2130\pm250$ & $840\pm70$ & $37\pm4$ & $7.6\pm1.8$ & $0.64\pm0.06$ & $0.0037\pm0.0007$ \\ 
Diboson & $1140\pm80$ & $588\pm33$ & $326\pm14$ & $20.6\pm1.2$ & $3.8\pm2.1$ & $0.4\pm0.4$ & $0.002\pm0.008$ \\ 
Multi-jet & $1350\pm40$ & $551\pm23$ & $180\pm10$ & $5.6\pm1.0$ & $0.85\pm0.21$ & $0.078\pm0.028$ & $0.00038\pm0.00022$ \\ \hline 
Total SM & $118000\pm12000$ & $29700\pm2600$ & $12100\pm600$ & $660\pm40$ & $135\pm11$ & $14.6\pm1.4$ & $0.058\pm0.013$ \\ \hline 
Data & $131672$ & $31980$ & $12393$ & $631$ & $121$ & $15$ & $0$ \\ 
\end{tabular}
\caption{Contributions of individual backgrounds with uncertainties for different $m_T$ regions.
The uncertainties include both statistical and systematic uncertainties, and all weights are included
so that the total background level can be compared to data. The systematic uncertainty includes all systematic 
uncertainties which are included in the statistical analysis except the uncertainty
on the integrated luminosity ($5\%$). For the multi-jet background, only the statistical uncertainty is shown,
since the multi-jet systematics are not included in the statistical analysis.}
\label{tab:muBkgData}
\end{table}

% TODO something more here???
% TODO see comments from Else

%*******************************************************************************
% XSEC and MASS LIMITS
%*******************************************************************************

\section{Cross section and mass limits}

% TODO describe look for deviation
To search for a $\PWprime$ signal-like excess in the data a log likelihood ratio test is performed using the RooStat~\cite{RooStat_project} framework.
% TODO is sentence below complete? "probabilities OF over all``?
The likelihood function is constructed as the product of Poisson probabilities of all $m_T$ bins in the search region.
% The likelihood function is constructed as the product of Poisson probabilities of over all $m_T$ bins in the search region.
The effect of systematic uncertainties is described by nuisance parameters in the likelihood function.

% TODO total efficiency... maybe include it later...
% The total efficiency (the product of acceptance and reconstruction efficiency)
% needed by the statistical analysis to calculate the number of expected events is shown in \FigureRef{fig:AccEff_mu}.
% The slow decrease of the total efficiency from 1 TeV can be explained by the shape
% of the $\PWprime$ signal, which becomes more prolonged in the low-$mT$ region with increasing of the $\PWprime$ pole mass as shown in \FigureRef{???}.
% \toDo[insert plot]
% TODO !!! insert fucking plot 

% \begin{figure}[]
%   \centering
%   \includegraphics[width=0.65\textwidth]{Wprime/acceptance.eps}
%   \caption{
%   Total reconstruction efficiency ($\varepsilon_{tot}$) of $\PWprime \to \mu \nu$ as a function of simulated $\PWprime$ mass.
%   }
%   \label{fig:AccEff_mu}
% \end{figure}

% Statistical analysis demonstrates that no excess larger than 2$\sigma$ is observed.
% Therefore we proceed to set upper limit on the cross section for the production of $\PWprime$ times branching ratio has been set. 
Since no significant deviations have been observed, the upper limit on the cross section for the production of $\PWprime$ times branching ratio has been set. 
A Bayesian approach has been used for the limit settings, and the limits were calculated with the Bayesian Analysis Toolkit~\cite{BAT}.

% \toDo[add something here???]

Upper limits are set on the cross section times branching ratio, $\PWprime \rightarrow \Plepton\nu$, at 95\% C.L. 
The limits for the electron, muon, 
and combined lepton channels are presented in \TableRef{tab:limits_mass_wp}, \FigureRef{fig:wprime_limits} and \FigureRef{fig:wprime_limits_combined}.
The theoretical cross section curves as a function of $\PWprime$ mass are shown as well.
The lower mass limits of the model correspond to the intersection of the theoretical curve with the expected cross section limit. The obtained mass limits are summarized in \TableRef{tab:limits_mass_wp}. The mass limit has improved by 800 GeV in comparison with the previous ATLAS search reported in Ref.~\cite{wprime_8TeV}.

\begin{figure}[]
  \centering
\includegraphics[width=0.49\textwidth]{Wprime/Limit_xsec_wprime_m_Sys.eps}
\includegraphics[width=0.49\textwidth]{Wprime/Limit_xsec_wprime_e_Sys.eps}
\caption{$\PWprime$ cross section limit results for the muon (left) and electron (right) channels.}
\label{fig:wprime_limits}
\end{figure}


\begin{figure}[]
  \centering
\includegraphics[width=0.65\textwidth]{Wprime/Limit_xsec_wprime_comb_Sys.eps}
\caption{Combined $\PWprime$ cross section limit results.}
\label{fig:wprime_limits_combined}
\end{figure}


\begin{table}[]
  \centering
  \begin{tabular}{c|cc}
    \hline
    \hline
    &  \multicolumn{2}{c}{$m_{\PWprime}$ lower limit [\TeV]} \\
    Decay     &  Expected & Observed \\
    \hline
    \wpe  & 3.99 & 3.96 \\
    \wpmu & 3.72 & 3.56 \\
    \wpl  & 4.18 & 4.07 \\
    \hline
    \hline
  \end{tabular}
  \caption{Expected and observed 95\% CL lower limit on the \wpssm\ mass in the electron and muon channels and their combination.}
  \label{tab:limits_mass_wp}
\end{table}



\section{Summary and outlook}
\label{sec:wprimeConclusion}

% general search summary
A search for a new heavy gauge boson $\PWprime$ has been performed in the final state with a muon and missing transverse energy using 3.2 fb$^{-1}$ of $\sqrt{s}=13$~TeV \pp collision data collected in 2015. This has also been reported in Ref.~\cite{Aaboud:2016zkn}. 
Upper limits on the cross section for SSM $\PWprime$ production have been set as a function of the $\PWprime$ pole mass. A significantly stronger exclusion mass limit has been obtained in comparison to previous ATLAS searches~\cite{atlas_7tev_pub_1fb,atlas_7tev_pub,wprime_8TeV} as shown in \FigureRef{fig:Wprime_allResults}.
For comparison, the mass limits for three ATLAS analyses using $\sqrt{s}=8$ and 13~TeV data are shown in \TableRef{tab:Wprime_limit_vs_years_ATLAS} and the results from the CMS collaboration shown are in \TableRef{tab:Wprime_limit_vs_years_CMS}. As can be seen, the measured mass limits are compatible between the two experiments.

\begin{figure}[]
 \includegraphics[width=0.8\textwidth]{Wprime/combWprime_summary.eps}
  \caption{
Cross section limits for $\PWprime$ searches performed by ATLAS
normalized to the SM background cross section prediction.
The region above each curve is excluded at 95\% CL.
}
  \label{fig:Wprime_allResults}
\end{figure}


\begin{table*}[]
\begin{center}
\begin{tabular}{c||c|c||c|c||c|c}
 & \multicolumn{6}{c}{95\%  C.L. upper limit [TeV]} \\
 & \multicolumn{2}{c||}{20.3 fb$^{-1}$ at 8 TeV} & \multicolumn{2}{c||}{3.2 fb$^{-1}$ at 13 TeV} & \multicolumn{2}{c}{13.3 fb$^{-1}$ at 13 TeV} \\
 \hline
Signal & Expected & Observed & Expected & Observed & Expected & Observed \\
%[+0.05in]
\hline
\rule{0pt}{3ex}
$\PWprime \to e\nu$ & 3.13 & 3.13 & 3.99 & 3.96 & 4.59 & 4.64 \\
\hline
$\PWprime \to \mu\nu$ & 2.97 & 2.97 & 3.72 & 3.56 & 4.33 & 4.19 \\
\hline
$\PWprime \to \Plepton\nu$ & 3.17 & 3.24 & 4.18 & 4.07 & 4.77 & 7.74 \\
\end{tabular}
\end{center}
 \caption{Upper limit at 95\% C.L. on mass of SSM $\PWprime$ for three ATLAS analyses using $\sqrt{s} =$ 8 and 13 TeV \pp collision data~\cite{wprime_8TeV,Aaboud:2016zkn,wprime_atlas_13TeV_13fb}.}
\label{tab:Wprime_limit_vs_years_ATLAS}
\end{table*}


\begin{table*}[]
\begin{center}
\begin{tabular}{c||c|c||c|c}
 & \multicolumn{4}{c}{95\%  C.L. upper limit [TeV]} \\
 & \multicolumn{2}{c||}{19.7 fb$^{-1}$ at 8 TeV} & \multicolumn{2}{c}{2.2 fb$^{-1}$ at 13 TeV}  \\
 \hline
Signal & Expected & Observed & Expected & Observed  \\
%[+0.05in]
\hline
\rule{0pt}{3ex}
$\PWprime \to e\nu$ & 3.18 & 3.22 & 3.7 & 3.8 \\
\hline
$\PWprime \to \mu\nu$ & 3.09 & 2.99 & 3.8 & 4.0 \\
\hline
$\PWprime \to \Plepton\nu$ & 3.26 & 3.28 & 4.2 & 4.4 \\
\end{tabular}
\end{center}
 \caption{Upper limit at 95\% C.L. on mass of SSM $\PWprime$ for two CMS analyses using $\sqrt{s} =$ 8 and 13 TeV \pp collision data~\cite{wprime_cms_8TeV,wprime_cms_13TeV}.}
\label{tab:Wprime_limit_vs_years_CMS}
\end{table*}

% TODO rewrite it. write it as outlook not as a summary
One of the possibilities to extend the analysis is to consider BSM models which predict an associated pair production of the DM particles with SM $W$ boson. Such models have been investigated in a previous ATLAS search~\cite{wprime_8TeV}.
A sensitivity study of the signal selection for different DM models has been performed (see \AppendixRef{app:monoW}).
The study concluded that the sensitivity of the DM models, recommended by the ATLAS/CMS Dark Matter Forum~\cite{DM_forum_2015} for Run-2 searches, is small.
Even with more luminosity, the DM interpretation in this channel is not worthwhile.


% recovery excluded phase space and mT-binned samples
There are still ways to improve the analysis.
One of the possibilities is to include phase-space that has been excluded due
to the problem with the alignment in the muon spectrometer as was described in \SectionRef{subsec:lepton_selection}.
Another improvement mentioned previously is to use $m_T$ binned diboson and top MC background samples, which will provide enough statistics to populate the high-$m_T$ region and no extrapolations will be needed (which is the dominant contribution to the systematic uncertainty).

% muon identification cut
One can investigate the possibility to apply less strict muon identification criteria in a way such that the muon efficiency is increased without affecting significantly neither the muon resolution nor the background rejection power.

% reduce the efficiency of muon reconstruction for approximately 20$\%$ in comparison with ``tight'' one. However one need to be careful that the muon momentum resolution doesn't worsen significantly.

% new methods from SM W measurements
A better understanding of the missing transverse energy could result in a significant reduction of the systematic uncertainty in the low-$m_T$ region, which is used to validate the background estimations.




