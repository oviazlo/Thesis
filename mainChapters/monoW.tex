

\chapter{Search for new charged bosons in final states with one muon and missing transverse energy}
\label{chap:Wprime}

% ***Explanation of my personal contribution to the analysis.***
% TODO this is not a real text to be used - just a plan of what I want to write,,,
% \begin{itemize}
%  \item MC simultaion
%  \item second analysis of muon channel
%  \begin{itemize}
%   \item comparison of results with Magnar
%  \end{itemize}
%  \item mono-W investigation
% \end{itemize}


% PLAN
% <What this chapter about - short intro to Wprime analysis>
% <What was my personal contribution to the analysis> 


This chapter describe search for new charged boson (namely $\PWprime$) in final state with one lepton and missing transverse energy ($E_T^{miss}$).
Search was done with first $\sqrt{s}$ = 13 TeV data collected in 2015 by ATLAS with corresponded luminosity of 3.2 fb$^{-1}$.
% TODO describe more about the analysis.
% Analysis focuses on high-p$_T$ 

% Key features of the analysis.
One can highlight three key features of the analysis:
\begin{itemize}
 \item Modeling of the background prediction. 
 Largest part of the background originates from the charged-current Drell-Yan process and the analysis selection tests it up to the few TeV.
 Thus it is crucially important to use latest and most precise high-order calculations and corrections available at a time.
 \item Lepton selection. Due to topologically simple event selection (one isolated lepton and missing energy) 
 this analysis use most energetic lepton candidates available at 13 TeV center-of-mass energy collisions.
 This is why reconstruction efficiency and momentum resolution of high-$p_T$ leptons are factors of special interest for the analysis.
 \item Missing transverse energy, $E_T^{miss}$. Along with the lepton momentum, $E_T^{miss}$ is used for transverse invariant mass, $m_T$, calculations,
 which is the signal discriminant and search variable in the analysis. In order to $m_T$ be most precisely reconstructed and modelled one need
 to have not only good lepton momentum resolution but also good understanding of the events missing $E_T$.
\end{itemize}




This analysis have been made public as a conference note~\cite{ATLAS-CONF-2015-063} with early reasults and later as a paper~\cite{Aaboud:2016zkn} where final results were presented.

My personal contribution to the analysis can be concluded in following three parts.
I performed complete analysis in muon channel. Signal selection, control and validation plots, estimation of systematics and other.
This part was done in parrallel with other collaborator to make sure that results are robust and are the same from two independent analyses and there are no mistaken done.

% TODO discuss with Monika, should I include it or not. If yes - how should I argument about why it was not use and how to explain differnce comparing to inclusive sample?
Secondly I was investigating ways to decrease systematics caused by limited statistic of inclusive diboson and ttbar backgound samples at high-$m_{T}$ region.
One option was to generate lepton-neutrino mass-binned samples. After investigation it was found out that these samples doesn't populate high-$m_{T}$ region
good enough, so it was decided to develop unique (first ever?) samples in bins of lepton-$E_T^{miss} $ $m_T$.

Third part was related to investigation of use another models which can be potentially sensitive in signal selection.
Main focus of this investigation was mono-W Dark Matter (DM) models, where pair of DM particles candidates are produced in final state in association with SM W boson.
Two sets of model were considered: simplified and Effective Field Theory models [links???]. 

% TODO discuss with Monika - how should I position analysis - as a dedicated search of Wprime or as more model-independent search?

\section{Motivation}
\label{sec:wprimeIntro}
% TODO model-independent search???
% TODO experimental signature - lepton + MET
% TODO a signal discriminant: mT
???

\section{$\PWprime$ signal}
\label{sec:wprimeSignal} 
% TODO rewrite it a little bit, cause it is directly copied from the note...

% MC generator
Samples for the signal process $\PWprime \to \mu \nu$ are produced with the leading-order (LO) 
{\scshape pythia-8.183}~\cite{pythia8} generator for a series of $\PWprime$ masses. 
The {\scshape pythia} default settings are used, i.e. the 
$\PWprime$ has the same $V-A$ couplings as the Standard Model $W$ boson but
interference between the $\PWprime$ and $W$ is not included. 
The couplings to the SM $W$ and $Z$ boson are turned off. 
Therefore, only decays in quarks and leptons is possible.

% generated masses; flat sample with possibility to reweight to any given mass
One large sample is generated which is flat in $\PWprime$ mass and therefore has high 
statistics from low to high $m_T$.  The flat sample can then be used for any 
$\PWprime$ mass by reweighting with the correct line shape for the desired $\PWprime$ mass.
Several additional samples are generated with fixed $\PWprime$ mass which
allows the reweighting procedure to be verified.

% comparison with W boson background (show plot of Wprime overlaid on top of W)
Invaraint mass and transverse mass distributions for the samples with fixed $\PWprime$ mass on top of $W$ background are shown in \FigureRef{fig:signal_with_W}.


\begin{figure}
\begin{subfigure}{.5\textwidth}
  \centering
  \includegraphics[width=\textwidth]{Wprime/Signal_onTopOf_W_invMass.eps}
\end{subfigure}%
\begin{subfigure}{.5\textwidth}
  \centering
  \includegraphics[width=\textwidth]{Wprime/Signal_onTopOf_W_mT.eps}
\end{subfigure}
\caption{Invariant mass (left) and transverse mass (right) spectrum of the $\PWprime$ signal 
on top of the $W$ background on generated MC level.}
  \label{fig:signal_with_W}
\end{figure}


\section{Background processes}
\label{sec:wprimeBackgrounds}

Analysis is focused in a search of the new charged, heavy boson which produce 
an excess in the final signature with one lepton and missing transverse energy.
Thus it is important to examine all other SM processes which contribute 
to such final state as well.

% All background predictions are obtained with MC simulation, except for non-prompt
% lepton contribution, which arising due to jets and photon being misreconstructed
% as leptons.

Since this chapter is focused on muon decay channel, processes which produce muon and
missing transverse energy will be discuss, however in general the same processes are 
relevant for the electron decay channel as well.

Dominant expected background in the analysis is the SM W boson production,
The SM W decays to lepton and neutrino which reconstructed as missing transverse energy,
Contribution of this process to the $m_T$ spectrum is appearing like a Jacobian peak 
with maximum around 80 GeV and slowly falling tail above 80 GeV.
% TODO probably I need to move it somewhere else, no?
Since $\PWprime$ conceptually is a heavier version of the SM W, it also appears in the 
transerse mass distribution as a Jacobian peak around the mass of the $\PWprime$ boson
(as shown in \FigureRef{fig:signal_with_W}). 
% W boson bkg
W boson production is simulated with \powhegbox\ v2~\cite{Alioli:2010xd}and {\scshape pythia-8.186} at next-to-leading-order (NLO) using the CT10~\cite{CT10} NLO PDFs. 
Cross sections are corrected to NNLO with using the CT14NNLO PDF set by applying QCD and Electroweak (EW) mass dependent $K$-factors to the MC generator cross sections.
In order to get sufficient statistics at high transverse mass several samples binned in invariant mass of the lepton-neutrino pair are used.
% Z boson bkg
The neutral current Drell-Yan process $Z/\gamma^* \to \mu \mu$ can contribute to the muon plus $E_T^{miss}$ final state, when of the muons are not properly reconstructed 
in the detector. In that case it will be not used in the $E_T^{miss}$ calculations as well, which will contribute to the $E_T^{miss}$ itself, as will be described in the \SectionRef{subsec:etmiss}.
Thus $Z$ boson production have to be considered as well. The process is simulated with the same MC generators and in the same way as W boson production process.
% Contribution from taus
Contribution of the processes $W \to \tau \nu$ and $Z \to \tau \tau$, which can contribute to the muon channel 
when tau decays to muon and neutrinos as $\tau^{-} \to \mu^{-} \overline{\nu_{\mu}} \nu_{\tau} $ are considered and simulated in the same way as well.

% Top
Another background which contribute to the final state of interest is the $t\bar{t}$ and single top production.
Top quark decays immediately to the $W$ boson and $b$ quark. Further leptonic decay of $W$ provides an isolated muon and $E_T^{miss}$ from escaping neutrino.
% TODO make feynman diagrams the same as in magnars thesis p.91-92
Some Feynman diagrams of the top production processes are shown in \FigureRef{???} and \FigureRef{???}.
This background is simulated with \powhegbox\ and {\scshape pythia-6.428}~\cite{Pythia} at next-to-leading-order (NLO) using the CT10~\cite{CT10} NLO PDFs.
All these processes are considered as a ``Top'' background in the text above.

% Diboson
More than one SM gauge boson can be produced in a single hard interaction, thus processes where $WW$, $WZ$ and $ZZ$ boson pairs are considered as well.
Some Feynman diagrams of the diboson processes are shown in \FigureRef{???}.
Contribution to the final state of interest can be provided via decays as $WZ \to \Plepton \nu \nu \nu$ or $WZ \to \Plepton \nu q \overline{q}$.
These processes are simulated with {\scshape sherpa-2.1.1}~\cite{Sherpa} using the CT10 NLO PDFs.
Such processes are grouped as a ``diboson'' background. 

% mT-binned samples
Only inclusive samples for both diboton and top backgrounds were available. 
This is why it was considered to produce samples binned in the transverse mass of lepton plus $E_T^{miss}$.
But due to technical complications these samples were not produced for 2015 analysis 
and a dedicated fitting procedure for high-$m_T$ region have been used for estiamation of this backgrounds.
However it is planned to use these samples for 2016 analysis.

% Table with MC
List of all processes with used MC generatrs are summarized in \TableRef{tab:MC_cross}.

\begin{table}[ht]
  \begin{center}
    \begin{tabular}{l|c|c|c}

      \hline
Process &  Generator&  PDF set & Normalization \\
&  + fragmentation/ &  & based on \\
&  hadronization & &\\
\hline\hline
&   &   \multirow{4}{*}{CT14NNLO~\cite{Dulat:2015mca}} & NNLO QCD \\
$W +$ jets, & \powhegbox\ v2~\cite{Alioli:2010xd} & &  with \vrap~\cite{vrap}, \\
$Z/\gamma^* +$ jets & + {\scshape pythia-8.186}~\cite{pythia8}  & &  NLO QED \\
 & & &  with \mcsanc~\cite{Bardin:2012jk,Bondarenko:2013nu} \\
\hline
\ttbar, t-channel $t$, & \powhegbox\ & \multirow{2}{*}{CT10} & \multirow{2}{*}{NLO QCD} \\
s-channel $Wt$ & + {\scshape pythia-6.428}~\cite{Pythia} & &  \\
\hline
\multirow{2}{*}{$WW, WZ, ZZ$} & \multirow{2}{*}{{\scshape sherpa-2.1.1}~\cite{Sherpa}} & \multirow{2}{*}{CT10} & \multirow{2}{*}{???} \\
 & & &  \\
\hline
\hline
\multirow{2}{*}{$W^\prime \rightarrow \Plepton \nu$} & \multirow{2}{*}{{\scshape pythia-8.183}} &   \multirow{2}{*}{NNPDF2.3 LO} & NNLO QCD \\
& & &  with \vrap, \\
\hline
\end{tabular}
\end{center}
  \caption{List of MC generated samples used for background prediction. 
  The used MC generator, PDF set and order of cross section calculations used for the normalization are listed for each sample.
  }
\label{tab:MC_cross}
\end{table}

\section{Event and lepton selection}
\label{sec:wprimeSelection}

The analysis is based on the pp collision data collected in 2015 by the ATLAS detector with 13 TeV center of mass energy.
The integrated luminosity of the sample corresponds to 3.2 fb$^{-1}$, the mean number of interactions per bunch crossing was 14.

An event selected for the analysis have at least one reconstructed vertex with at least two tracks matched to it. 
If there are several vertices, the one with the highest
$\sum p^2_T$, where $p_T$ are transverse momenta of the matched tracks, is chosen.
Events should have at least one muon candidates, and fire the single muon trigger, 
which requires the presence of one muon with $p_T > 50$ GeV.

\subsection{Lepton selection}
% description of muon reconstruction in the ATLAS detector
Muon reconstruction in ATLAS is performed independently in ID and MS detectors. 
Later the information from the detectors are combined to form a muon track.
There are different possibilities how to combine information from both detectors, 
thus four different muon types are defined in ATLAS~\cite{muon_performance_2015}.
In this analysis so-called combined muons are used, for which track reconstruction is done 
separately in the ID and MS and than global refit is done to form a combined track.
% muon momentum reconstruction
Muon $p_T$ is measured from the track curvature.

% As well as for electrons, there are different sets of identification criteria which provide different background supression, reconstruction efficiency and momentum measurement resolution: Loose, Medium, Tight and High$-pT$.

Muons of interest are high-$p_T$ isolated muons, tracks of which originates
from the primary vertex. They have to satisfy set of following criteria:
\begin{itemize}
 \item $p_T > 55$ GeV: to ensure a high and flat trigger efficiency.
 \item $|\eta|<2.5$, excluding $1.01 < |\eta| < 1.1$: muons have to be within ID acceptance. 
 Exclusion region is applied in order to reject muons whose tracks in the muon spectrometer fall into poorly equationed chambers (relative barrel-endcap equationment).
%  TODO see comment from Monika to SS chapter
 \item Transverse and longitudinal impact parameters $|d_0|/\sigma(d_0) < 3$ and $|z_0 \times sin \theta| < 10$ mm: 
 to verify that the muon was produced close to the primary vertex and to reject muon originating from decays of long-lived particles. Recommendation by the muon combined performance (MCP) working group has
 more strict requirement $|z_0 \times sin \theta| < 0.5$ mm, however after dedicated studies it 
 was decided to relax this requirement to 10 mm as will be described in \SectionRef{subsec:wprime_cut_optimization}.
  \item Pass ``high-$p_T$'' muon set of identification criteria defined in~\cite{muon_performance_2015}. \\ 
 This set aims to maximize the momentum resolution for tracks with $p_T > 100$ GeV.
 It includes tight requirements on the MS part of the track which reduce reconstruction
 efficiency of the muons up to 20$\%$ with respect to other identification sets, however, 
 improvement of the $p_T$ resolution is reaching approximately 30$\%$.
%  TODO describe why LooseTrackOnly isolation is enough
 \item Pass LooseTrackOnly isolation requirement~\cite{muon_performance_2015}. 
 This requirement provide 99$\%$ constant efficiency over complete ($\eta$,$p_T$) phase space.
 The discriminating variable is the ratio of the sum of $p_T$ of all tracks (excluding the muon itself) with $p_T > 1.0$ GeV within  $\Delta R = min(10$ GeV $/p_T^{\mu}, 0.3)$ 
 cone around the the muon track to muon track $p_T^{\mu}$.
\end{itemize}  
 
\subsection{Signal selection optimization}
\label{subsec:wprime_cut_optimization}
% TODO additional lepton veto cut

In order to suppress contribution from neutral current Drell-Yan $Z/\gamma^*$ and \ttbar processes
where two isolated leptons are expected in the final states an additional lepton veto requirement
is applied. If second muon, which pass passing a loosened version of the above selection with the ``high-$p_T$'' identification working point replaced by the requirement to pass either ``medium'' criteria or
``high-$p_T$`` identification criteria and and with a lower $p_T$ cut of $20$~GeV is found in the event, it is vetoed.
Events are also vetoed if additional electron passing the selection below is found.
\begin{itemize}
\item $|\eta| < 2.47$, excluding barrel-endcap calorimeter transition region $1.37 < |\eta| < 1.52$.
\item $p_T > 20$~GeV
\item Transverse impact parameters $|d_0|/\sigma(d_0) < 5$
\item Pass the likelihood ``medium'' identification criteria~\cite{ATL-PHYS-PUB-2015-041}.
\item Pass the ``loose'' isolation criteria~\cite{ATLAS-CONF-2016-024}.
\item Has to not overlap with muon, $\Delta R(e,\mu)>0.1$. If it is overlapped ($\Delta R(e,\mu)>0.1$)
it is assumed that the electron candidate arose from photon radiation from the muon and the event is kept.
\end{itemize}
Veto requirements leads to a significant reduction of the dimuon ($Z$) background
at high transverse mass as well as some reduction of the $t\bar{t}$ background, this leads to a reduction of the total background level of approximately $10$--$15\%$ at high transverse mass. 
The signal efficiency is found to be essentially unaffected.
The possibility of using the ``loose'' identification working point for the additional muon veto was also considered. But it was found to provide tiny improvement ($1$--$3\%$ additional reduction of the total background level) with respect to using ``medium'' working point, thus latter was chosen to be used.

% TODO z sin_theta
The default recommendation by the muon combined performance working group is to apply requirement $|z_0 \times sin \theta| < 0.5$ mm. The main purpose of this requirement is to veto events with cosmic muons.
However by making requirement on absence of the second muon in the event we ``automatically'' discard most of the events with cosmic muons. After dedicated study by the MCP group it was found that on 
$|z_0 \times sin \theta|$ can be loose to 10 mm, without significant decrease on cosmic muon rejection.
Figure \ref{fig:Muon_LepVtxEff} shows the $d_0$ significance and $|z_0 \times sin \theta|$ cut efficiency. The efficiency is shown on the left side for the recommended $|z_0 \times sin \theta|$ cut value of $0.5$ mm and on the right side for a looser
cut value of $10$~mm. The nominal cut value leads to a reduction of the selection efficiency by about $1$\% due to picking the wrong vertex as primary vertex. This efficiency
is partially restored by increasing the cut value to $10$~mm. 

\begin{figure}[]
  \centering
  \includegraphics[width=0.45\textwidth]{Wprime/IPplot05mm.eps}
  \includegraphics[width=0.45\textwidth]{Wprime/IPplot10mm.eps}
  \caption{$d_{0}$ and $|z_0 \times sin \theta|$ cut efficiencies. The $d_{0}$ efficiency is shown for the cut recommended by the tracking group. In the $|z_0 \times sin \theta|$ case the recommended cut of $0.5$ mm (left) and an alternative cut of $10$ mm (right) are shown. The efficiencies are calculated for combined muons in the
$\PWprime$ flat sample passing the medium or high-$p_T$ working point requirements.}
  \label{fig:Muon_LepVtxEff}
\end{figure}


\subsection{Transverse mass and missing transverse energy}
\label{subsec:etmiss}
The missing transverse energy, $E_T^{miss}$, is calculated following the ATLAS recommendation and recommendations described in ref.~\cite{met2015_1,met2015_2}.
It is evaluated as a vector sum of the $p_T$ of selected objects:
\begin{itemize}
 \item muons which satisfy analysis signal selection.
 \item electrons which satisfy requirements described previously in
 \SectionRef{subsec:wprime_cut_optimization} with stronger transverse momentum requirements $p_T > 55$ GeV
 and likelihood ``tight'' identification criteria.
 \item tau leptons which satisfy ``medium'' identification criteria~\cite{tau_id_8TeV}, $|\eta| < 2.5$, excluding  $1.37 < |\eta| < 1.52$ and $p_T > 20$~GeV.
 \item photons which satisfy ``tight'' identification criteria~\cite{photon_id_2011}, $|\eta| < 2.37$, excluding  $1.37 < |\eta| < 1.52$ and $p_T > 25$~GeV.
 \item jets reconstructed with anti-$k_t$ algorithm~\cite{jet_anti_kt} with radius parameter of 0.4.
 The jets are calibrated using the method described in ref.~\cite{jet_calib_syst_13TeV}.
 Only jets with $p_T > 20$~GeV and $|\eta| < 2.4$ are used. \toAsk[is $|\eta| < 2.4$ correct? in paper it's 5?]
 \item tracks originating from the primary vertex with $p_T > 0.5$ GeV and $|\eta| < 2.5$ but not belonging to any of the reconstructed physics objects listed above.
\end{itemize}
The missing transverse energy is required to be $E_T^{miss} > 55$ GeV in order 
to balances the lepton transverse momentum cut.
Such high cut allows to significantly suppress 
the multi-jet background which will be described below.

% TODO describe why if muon is not reconstructed it will contribute to Etmiss (see Magnars thesis, page 89)

The main variable of interest which is used as a signal discriminant is transverse mass
\begin{equation}
 m_\mathrm{T} = \sqrt{2 p_\mathrm{T} p_T^{miss} (1-\cos\varphi_{\Plepton\nu})}
\end{equation}
where $\varphi_{\Plepton\nu}$ is the angle between the muon and $E_T^{miss}$ in the transverse plane.
The transverse mass has to be $m_\mathrm{T} > 110$~GeV.

\section{Background estimation}
\label{sec:wprime_backgroundEstimation}
% TODO describe sources of fakes:

The backgrounds which have final state with a real muon originating from the primary vertex and 
which can contribute to the signal selection are modelled with MC simulation.
List of all considered processes is shown in \TableRef{tab:MC_cross}.
Such muons are called ``real'' (prompt) muons.

However, processes with multijet final states (with one or more jets) will also contribute to
the signal selection due to probability of wrong reconstruction of the jet activity as a muon.
Such muon candidates are called ``fake'' (non-prompt) muons.
The ``fake'' muon can be a real muon which originate from a heavy flavor hadron decay within a jet
or from pion or kaon decays. But because they are not originate from the primary vertex they are not
desired to be selected in the signal region and thus are called ``fake'' muons.
The ``fake'' muon background is estimated using a data driven technique. 
``Fake'' muons are expected to be in general non-isolated, although some
fraction does pass the isolation cut and ends up in the selected event sample. Isolation variables
hence provide a strong separation of ``fake'' and ``real'' muons, and are essential to the data-driven
estimation of the ``fake'' background.
The ``fake'' background is estimated using the Matrix Method, which is presented below.

\subsection{The Matrix Method}
\label{subsec:matrix_method}
% TODO what is the difference between atrix and fake factor methods?
% TODO some motivation why we use exactly his method in this analysis.

% The cut used to distinguish between loose and tight muons is the isolation cut, and the loose muons
% are thus defined as passing all muon selection cuts except the isolation cut. Tight muons correspond
% to the baseline selection.

% TODO text taken directly from support note --> rewrite it!!!
The goal of the method is to get an estimation of fake muon which
pass the selection. The idea on how to obtain this number is to loosen
some of the identification cuts for the signal selection and measure
the efficiency for these looser objects to pass the signal selection. This
efficiency gives a handle for the fake contribution of the signal.

The matrix method provides a connection between the true number of
``real'' muons ($N_R$) and the true number of ``fakes'' ($N_F$) and
the measurable quantities from a sample in which the muons have are passing loose selection ($N_L$) criteria and at the same time failing a tight selection and a sample in which the muons pass the tight
selection ($N_T$) criteria via Eq.~\ref{eq:mm1}.
\begin{equation}
  \left(\begin{array}{c} N_T \\ N_L \end{array}\right)&=
  \begin{pmatrix}
    \epsilon_R & \epsilon_F \\
    1-\epsilon_R & 1- \epsilon_F \\
  \end{pmatrix}
  \left(\begin{array}{c} N_R \\ N_F \end{array}\right)
  \label{eq:mm1}
\end{equation} 
The vector on the right hand side of the equation describes the
inaccessible truth quantities. Since these truth quantities are
independent from each other, which means $N_R$ has no component from
$N_F$, also the vector on the left side of the equation has to fulfill
this criteria. Because of this $N_L$ should not contain $N_T$ and in
the following $L$ indicates ``pass loose but not tight'' or short
``loose fail tight''.

The entries in the matrix are called real efficiency ($\epsilon_R$)
and fake efficiency ($\epsilon_F$) and denote the probability for a real
(fake) electron to pass from loose to tight and is given by:
\begin{equation}
 \epsilon_F = \frac{N_{\textrm tight}^{\textrm fake}}{N_{\textrm loose}^{\textrm fake}} ,\qquad & \epsilon_R = \frac{N_{\textrm tight}^{\textrm real}}{N_{\textrm loose}^{\textrm real}}.
  \label{eq:mm2}
\end{equation}
The total number of muons passed signal selection is given in the first line of the matrix:
\begin{equation*}
 N_T&=N_{\textrm tight}^{\textrm real} + N_{\textrm tight}^{\textrm fake}=\epsilon_R N_R + \epsilon_F N_F\, ,
\end{equation*}
By inverting the matrix one get an equation for the truth variables:
\begin{equation*}
\left(\begin{array}{c} N_R \\ N_F \end{array}\right)&=
\frac{1}{\epsilon_R(1-\epsilon_F)-\epsilon_F(1-\epsilon_R)}
\begin{pmatrix}
 1- \epsilon_F & -\epsilon_F \\
\epsilon_R-1 & \epsilon_R  \\
\end{pmatrix}
\left(\begin{array}{c} N_T \\ N_L \end{array}\right)
\end{equation*} 
For the number of fake electrons which pass the selection it follows
then by insertion:
\begin{equation}
N_{\textrm tight}^{\textrm fake} = \epsilon_F N_F=\frac{\epsilon_{F}}{\epsilon_{R}-\epsilon_{F}}\left(\epsilon_{R}(N_{L}+N_{T})-N_{T}\right) \,
  \label{eq:mm5}
\end{equation}
which only contains measurable quantities.

The cut used to distinguish between loose and tight muons is the isolation cut, and the loose muons
are thus defined as passing all muon selection cuts except the isolation cut. Tight muons correspond
to the baseline selection.

``Real'' muon efficiency is found from MC simulation, because MC reproduce efficiency of the isolation cut in data well. ``Fake'' muon efficiency is found from a control region designed to have a high purity of ``fake'' muons. The region is defined in the same way as signal selection without cuts in $E_T^{miss}$ and $m_T$ and following additional requirements:
\begin{itemize}
\item At least one jet with $p_T > 40$~GeV which does not overlap ($\Delta R > 0.2$)
with the selected muon.
\item Opening angle in the transverse plane between the muon and the $E_T^{miss}$, $\Delta\phi_{\mu,E_T^{miss}} < 0.5$.
\item No $Z$ candidate (any two muons with $80 < m_{\mu\mu} < 100$~GeV).
\item $d_0$ significance greater than $1.5$.
\item $E_T^{miss} < 55$~GeV, ensuring that the control region does not overlap with the signal region.
\end{itemize}
This region is enhanced with ``fake'' muons, however, significant ``real'' muon contamination is present in the region as well. This is why a ``real'' contribution predicted with MC modeling is subtracted.

Obtained efficiencies are shown in \FigureRef{fig:matrix_method_efficiencies}.
\begin{figure}[]
  \centering
  \includegraphics[width=0.45\textwidth]{Wprime/realEffNominal.eps}
  \includegraphics[width=0.45\textwidth]{Wprime/fakeEffNominal.eps}
  \caption{\toDo}
  \label{fig:matrix_method_efficiencies}
\end{figure}

Systematic uncertainty were estimated for both ``real'' and ``fake'' muon efficiencies, 
but due to really small contribution of the ``fake'' background to the signal region (at most approximately 1$\%$) the impact of its systematic uncertainties on the total background was found to be negligible.

\subsection{Background extrapolation}
% TODO explain why it hurts to have huge statistical fluctuation for limit settings 
% TODO (it was somewhere in the note?)
% % from app_muonSmooth.tex:
% In a single-bin statistical analysis,
% it is straight forward to take into account the statistical uncertainty of the MC background
% estimates, as one can simply add it in quadrature to the systematic uncertainties on the
% background level in the search region. However, in a multi-bin analysis, results depend on
% the shape of the background distribution, and one should avoid propagating clearly unphysical features
% in this shape to the final results.

The top and diboson backgrounds suffer currently from lack of MC statistics at high mass.
% TODO say something about mT-binned samples HERE!!!
Therefore
these backgrounds are fitted and extrapolated to obtain a smooth description in the high mass region.

% TODO this is from Marcus --> change phrasing!!!
The fit functions are based on functions which are commonly used to extrapolate the background
 in the search for di-jet resonances~\cite{???} and also have been used in the $8~\TeV$ dilepton resonance search~\cite{???}.
 The first function used to extrapolate the background is defined as follows:
\begin{equation}
 f(\mt) = e^{-a} m_\mathrm{T}^{b} m_\mathrm{T}^{c \log(m_\mathrm{T})}.
  \label{eq:dijetfunc}
\end{equation}
The second function is a modified power law function:
\begin{equation}
 f(\mt) = \frac{a}{(m_\mathrm{T}+b)^{c}}.
  \label{eq:powerlaw}
\end{equation}
Several fits are performed with both functions and varying start and end point of the fit range. The fit with the best $\chi^{2}/N.d.o.f$ of all fits is taken
as central value for the extrapolation and the envelope of all fits as systematic uncertainty for the extrapolation. The statistical uncertainty of the fit parameters
has found to be negligible.\\

For the top background, the
starting point of the fit range was varied from $140$~GeV to $260$~GeV in steps of $20$~GeV. The end point
was varied from $600$~GeV to $900$~GeV in steps of $25$~GeV. For the diboson background,
the starting point of the fit range was varied from $120$~GeV to $240$~GeV in steps of $20$~GeV. The end point
was varied from $500$~GeV to $700$~GeV in steps of $25$~GeV. The extrapolated background was stitched to the background
estimated with Monte Carlo at $\mt=600$~GeV in both the top and diboson cases. 

The results are shown for the top background in
Fig.~\ref{fig:mu_extrapolate_top} and for the diboson background in Fig.~\ref{fig:mu_extrapolate_diboson}.
Both figures show the full set of different fits (left) as well as the assigned central value
and corresponding systematic uncertainty (right). The central value is taken to be the fit with the best $\chi^2$
and the systematic uncertainty is taken as the envelope.
\begin{figure}[!htb]
  \centering
  \includegraphics[width=0.49\textwidth]{Wprime/top_extrapolate_fits.eps}
  \includegraphics[width=0.49\textwidth]{Wprime/top_extrapolate.eps}
  \caption{Results of fitting and extrapolation for the top background. Both the full set of individual
fits (left) and the resulting central value and uncertainty (right) are shown.}
  \label{fig:mu_extrapolate_top}
\end{figure}
\begin{figure}[!htb]
  \centering
  \includegraphics[width=0.49\textwidth]{Wprime/diboson_extrapolate_fits.eps}
  \includegraphics[width=0.49\textwidth]{Wprime/diboson_extrapolate.eps}
  \caption{Results of fitting and extrapolation for the diboson background. Both the full set of individual
fits (left) and the resulting central value and uncertainty (right) are shown.}
  \label{fig:mu_extrapolate_diboson}
\end{figure}

% TODO add extrapolation of the multijet background!!!
In order to extend the multijet background estimates beyond the region where there is sufficient statistics, a simple power law fit is performed:
\begin{equation}
\frac{dN}{d m_T} = a\, m_T^{-b}
\end{equation}
with $a$ and $b$ as the fitted parameters. Fits are done in the ranges
$150$--$300~\GeV$ and $200$--$300~\GeV$.

\section{Validation Region}

To validate the ``fake'' muon background estimate and MC description of the ``real'' muon backgrounds
% , and also to ensure that
% all cut variables used in the definition of the fake muon control region are reasonably
% well described in the real muon MC,
a validation region is used, which is defined in the same way as signal selection but without the $E_T^{miss}$ and $m_T$ requirements. Validation region with and without using isolation requirements
are shown which correspond to tight and loose muon definitions used in the matrix method.

% For the purpose of validation of the fake muon background estimate, and also to ensure that
% all cut variables used in the definition of the fake muon control region are reasonably
% well described in the real muon MC, we consider several relevant distributions
% before the \mettext\ and \mt\ cuts. The distributions are shown both in the inclusive
% loose muon sample, i.e. without the isolation cut applied, and in the tight muon sample.
% The virtue of considering the inclusive loose muon sample, is that the fake muon background
% is larger, making for more convincing validation of the background estimate. Also, the fake
% muon background in this sample has a different, much weaker, dependence on the fake-muon
% efficiency, as we are considering $N_F$ rather than $\epsilon_F N_F$ (see Eq.~\eqref{eq:mm5}).
% This means that the distributions in the loose muon sample serve more as validation
% of the real-muon efficiency measurement and the MC description, and less as validation of the
% fake-muon efficiency measurement.

% TODO make similar explanation of the plot discrepancies!!!
% The validation distributions are shown in Figs.~\ref{fig:muMMval1} and~\ref{fig:muMMval2}.
% In general, reasonable agreement is observed. The most obvious discrepancy is seen in the
% distribution of the number of jets. The disagreement is likely mostly due to the modeling
% of jet emission in the $W\to\mu\,\nu$ MC, where only one jet emission is included at the
% matrix element level. This affects in principle the MC subtraction in the ``fake muon'' control region,
% but the efficiency of the baseline cut $N_\mathrm{jet}\geq1$ is not strongly affected by the
% discrepancy, which is prominent only for $N_\mathrm{jet}\geq2$. Furthermore, the systematics
% variations with no $N_\mathrm{jet}$ cut are clearly unaffected by the MC description of this variable.

\begin{figure}[!htb]
  \centering
  \includegraphics[width=0.49\textwidth]{Wprime/Njet40loose.eps}
  \includegraphics[width=0.49\textwidth]{Wprime/Njet40.eps}
  \includegraphics[width=0.49\textwidth]{Wprime/deltaPhiLoose.eps}
  \includegraphics[width=0.49\textwidth]{Wprime/deltaPhi.eps}
  \includegraphics[width=0.49\textwidth]{Wprime/d0sigLoose.eps}
  \includegraphics[width=0.49\textwidth]{Wprime/d0sig.eps}
  \caption{
%   Distributions of the number of jets (top), $\Delta\phi_{\mu,\met}$ (middle), and $d_0$ significance (bottom)
% in the inclusive loose (left) and tight (right) muon samples. The distributions are considered
% before the \mettext\ and \mt\ cuts.
}
  \label{fig:muMMval1}
\end{figure}
\begin{figure}[!htb]
  \centering
  \includegraphics[width=0.49\textwidth]{Wprime/METloose.eps}
  \includegraphics[width=0.49\textwidth]{Wprime/MET.eps}
  \includegraphics[width=0.49\textwidth]{Wprime/pTloose.eps}
  \includegraphics[width=0.49\textwidth]{Wprime/pT.eps}
  \includegraphics[width=0.49\textwidth]{Wprime/mTloose.eps}
  \includegraphics[width=0.49\textwidth]{Wprime/mT.eps}
  \caption{
%   Distributions of the \mettext\ (top), \pt\ (middle), and \mt\ (bottom)
% in the inclusive loose (left) and tight (right) muon samples. The distributions are considered
% before the \mettext\ and \mt\ cuts.
}
  \label{fig:muMMval2}
\end{figure}


\section{Signal Region}
\label{sec:wprimeSignalRegion}
???

\begin{figure}[]
  \centering
  \includegraphics[width=0.49\textwidth]{Wprime/muon_eta.eps}
  \includegraphics[width=0.49\textwidth]{Wprime/muon_phi.eps}
  \caption{
%   Muon \eta\ (top) and $\phi$ (bottom) distributions after final selection. The uncertainty band in the ratio plot
% includes all systematic uncertainties which are included in the statistical analysis except the integrated
% luminosity uncertainty ($5\%$).
}
  \label{fig:mu_results_etaphi}
\end{figure}

\begin{figure}[]
  \centering
  \includegraphics[width=0.49\textwidth]{Wprime/muon_pT.eps}
  \includegraphics[width=0.49\textwidth]{Wprime/muon_MET.eps}
 \caption{
%  Muon \pt and \mettext\ distributions after final selection. The uncertainty band in the ratio plot
% includes all systematic uncertainties which are included in the statistical analysis except the integrated
% luminosity uncertainty ($5\%$).
}
  \label{fig:mu_results_ptmet}
\end{figure}


\begin{figure}[]
  \centering
  \includegraphics[width=0.65\textwidth]{Wprime/muon_mT.eps}
  \caption{
%   Muon \mt\ distribution after final selection. Shown is the total background estimate with 
% resonant \wp\ signal overlaid for various pole masses. The uncertainty band in the ratio plot
% includes all systematic uncertainties which are included in the statistical analysis except the integrated
% luminosity uncertainty ($5\%$).
}
  \label{fig:MT_mu_Wprime}
\end{figure}

\begin{table}[]
  \centering
  \scriptsize
  \begin{tabular}{|c|c|c|c|c|c|c|c|}
    
    \multirow{2}{*}{Process} & \multicolumn{7}{c|}{$m_T$ [\GeV]} \\
& $110$--$150$ & $150$--$200$ & $200$--$400$ & $400$--$600$ & $600$--$1000$ & $1000$--$3000$ & $3000$--$7000$ \\ \hline 
$W$ & $98100\pm10000$ & $21000\pm2000$ & $7700\pm400$ & $476\pm30$ & $110\pm9$ & $13.0\pm1.2$ & $0.051\pm0.010$ \\ 
Top & $9900\pm700$ & $5410\pm340$ & $3090\pm140$ & $120\pm6$ & $13\pm5$ & $0.44\pm0.32$ & $0.00005\pm0.00030$ \\ 
$Z/\gamma^*$ & $7700\pm1000$ & $2130\pm250$ & $840\pm70$ & $37\pm4$ & $7.6\pm1.8$ & $0.64\pm0.06$ & $0.0037\pm0.0007$ \\ 
Diboson & $1140\pm80$ & $588\pm33$ & $326\pm14$ & $20.6\pm1.2$ & $3.8\pm2.1$ & $0.4\pm0.4$ & $0.002\pm0.008$ \\ 
Multi-jet & $1350\pm40$ & $551\pm23$ & $180\pm10$ & $5.6\pm1.0$ & $0.85\pm0.21$ & $0.078\pm0.028$ & $0.00038\pm0.00022$ \\ \hline 
Total SM & $118000\pm12000$ & $29700\pm2600$ & $12100\pm600$ & $660\pm40$ & $135\pm11$ & $14.6\pm1.4$ & $0.058\pm0.013$ \\ \hline 
Data & $131672$ & $31980$ & $12393$ & $631$ & $121$ & $15$ & $0$ \\ 
\end{tabular}
\caption{Contributions of individual backgrounds with uncertainties for different $m_T$ regions.
The uncertainties include both statistical and systematic uncertainty, and all weights are included
so that the total background level can be compared to data. The systematic uncertainty includes all systematic 
uncertainties which are included in the statistical analysis except the uncertainty
on the integrated luminosity ($5\%$). For the multi-jet background, only statistical uncertainty is shown,
since the multi-jet systematics are not included in the statistical analysis.}
\label{tab:muBkgData}
\end{table}


\section{Systematic Uncertainties}
\label{sec:wprimeSystematics}
???
\subsection{Muon efficiency, resolution and scale}
???
\subsection{Jet energy scale and resolution}
???
\subsection{Transverse missing energy scale and resolution}
???
\subsection{Background estimate uncertainty}
???
\subsection{Systematic uncertainty on the signal}
???

\subsection{Summary}

\begin{table}
\begin{center}
\centering
\small
\begin{tabular}{l|cc}
\toprule
Source &  Background  &  Signal  \\
\midrule
Trigger &\syspair{3}{4} & \syspair{4}{4}\\
Lepton reconstruction  &\multirow{2}{*}{\syspair{5}{8}} & \multirow{2}{*}{\syspair{5}{7}}\\
and identification & & \\
Lepton isolation &\syspair{5}{5} & \syspair{5}{5}\\
Lepton momentum &\multirow{2}{*}{\syspair{3}{11}} & \multirow{2}{*}{\syspair{1}{4}}\\
scale and resolution & & \\
$E_T^{miss}$ resolution and scale &\syspair{<0.5}{<0.5} &\syspair{<0.5}{<0.5}\\
Jet energy resolution &\syspair{1}{2} &\syspair{<0.5}{<0.5}\\
\midrule
Multijet background & \syspair{1}{1} & {\sc n/a} ({\sc n/a})\\
Diboson \& top-quark bkg. &\syspair{5}{15} & {\sc n/a} ({\sc n/a})\\
PDF choice for DY &\syspair{<0.5}{1} & {\sc n/a} ({\sc n/a})\\
PDF variation for DY &\syspair{8}{12} & {\sc n/a} ({\sc n/a})\\
Electroweak corrections &\syspair{4}{6} & {\sc n/a} ({\sc n/a})\\
\midrule
Luminosity &\syspair{5}{5} &\syspair{5}{5}\\
\midrule
Total &\syspair{14}{25} & \syspair{9}{12}\\
\bottomrule
\end{tabular}
\end{center}
\caption{Systematic uncertainties on the expected number of events as evaluated at $m_T = $ 2 (4)~\TeV, both for signal events 
with a \wpssm\ mass of 2~(4)~\TeV\ and for background. Uncertainties estimated to have an impact
$< 3\%$ on the expected number of events in both channels and for all values of $m_T$ are not listed.
Uncertainties that are not applicable are denoted ``n/a''. \label{tab:syst}}
\end{table}

\section{Cross section and mass limits}
% TODO say that Bayesian approach is used...
???

\begin{figure}[]
  \centering
  \includegraphics[width=0.65\textwidth]{Wprime/acceptance.eps}
  \caption{
%   Total signal acceptance times efficiency versus SMM \wp\ pole mass for the SSM \wp\ model in the muon channel.
  }
  \label{fig:AccEff_mu}
\end{figure}

\subsection{$\PWprime$ cross section and mass limits}
???

\begin{figure}[]
  \centering
\includegraphics[width=0.49\textwidth]{Wprime/Limit_xsec_wprime_m_Sys.eps}
\includegraphics[width=0.49\textwidth]{Wprime/Limit_xsec_wprime_e_Sys.eps}
\caption{$\PWprime$ cross section limit results for the muon (left) and electron (right) channels.}
\label{fig:wprime_limits}
\end{figure}


\begin{figure}[]
  \centering
\includegraphics[width=0.65\textwidth]{Wprime/Limit_xsec_wprime_comb_Sys.eps}
\caption{$\PWprime$ cross section limit results for combined both muon and electron channels.}
\label{fig:wprime_limits}
\end{figure}


\begin{table}[]
  \centering
  \begin{tabular}{c|cc}
    \hline
    \hline
    &  \multicolumn{2}{c}{$m_{\PWprime}$ lower limit [\TeV]} \\
    Decay     &  Expected & Observed \\
    \hline
    \wpe  & 3.99 & 3.96 \\
    \wpmu & 3.72 & 3.56 \\
    \wpl  & 4.18 & 4.07 \\
    \hline
    \hline
  \end{tabular}
  \caption{Expected and observed 95\% CL lower limit on the \wpssm\ mass in the electron and muon channels and their combination.}
  \label{tab:limits_mass_wp}
\end{table}

% Sources:
% 
% Dark Matter Benchmark Models for Early LHC Run-2 Searches:
% Report of the ATLAS/CMS Dark Matter Forum (July 6, 2015)
% http://arxiv.org/pdf/1507.00966.pdf
% 


\section{Search for dark matter pair-production with a leptonically decaying W boson}
\label{chap:monoW}

\subsection{Introduction}

Beside $\PWprime$ model discussed above there are other BSM models 
with lepton plus $E_{T}^{miss}$ final state signature.
One of such model is associated 
production of pair of weakly interacting massive particles (WIMP)
with SM W boson. WIMPs are one of possible dark matter candidates (DM).

% Another possible BSM signature which can be easily tested in signal region is
% associated production of pair of weakly interacting massive particles (WIMP), 
% which are candidates for dark matter (DM) particles, with SM W. 
Because WIMPs don't interact strongly or electromagnetic they will most probably escape from detector the same way as neutrino
from leptonic W decay and will contribute to $E_{T}^{miss}$ of the event.

In this section qualitative study based on using models recommended by Dark Matter Forum (link?) are presented 
which address question about sensitivity of signal selection presented above to such kind of models.


% While searching for $\PWprime$ boson one can naivly expect that distributions of 
% transferred energy of decay products will look quite similar, which
% means that distributions of transfer lepton momentum and missing energy will 
% look the same as well. But if we consider pair-production of dark matter 
% particles associated with W we expect 

\subsection{Theoretical models}

% TODO rephrase first setence
There are plethora of models which try to introduce and explain DM as a possible particles which can be produced at LHC. 
But all these models can be classified to the three distinct classes: DM Effective Field Theories (EFT), Simplified DM models 
and Complete DM models. EFT approach allows to describe the DM-SM interactions mediated by all 
kinematically inaccesible particles in a universal way. 
It allows to derive stringent bounds on the ``new physics'' scale $\Lambda$. 
Simplified models are charachterized by the most important state mediating the DM interaction with SM. Unlike EFT approach,
simplified models are able to describe correctly the full kinematics of DM production, because they resolve the EFT contact interaction in single-particle 
s- or t-channel exchange. Complete DM models allows to describe correlation between observables~\cite{arXiv:1506.03116}.

Main focus of current study will stay on EFT and Simplified DM models with W boson and DM particles produced in the final state. 
In simplfied model W is produced as initial state radiation from one of incoming quarks as shown at \FigureRef{fig:feynMonoWSimple}. 
In EFT approach W can be produces as final state particle (\FigureRef{fig:feynMonoWEFT}) or initial state radiation (\FigureRef{fig:feynMonoWEFT}).

% TODO: explanation of simplified and EFT models (take from DM forum report or fro Bell's papers).


% and mediator connect SM interaction from one side and DM 

% TODO find reference from DM forum report which explain idea of simlified models 
% (that we introduce additionla mediator which on one side interact with SM particles and on other side - with dark matter particles).
% same for EFT models. But for now skip this part.

% TODO describe possible type of interection for mono-W models.

% TODO figure out what does it mean constructive and destructive intereference for DX models?

% TODO explanation for contact interaction is needed as well, ha-ha-ha...

% TODO say that our main focus is on simplified models because they are recommended by Dark Matter forum.

\begin{figure}[]

\centering
\begin{subfigure}{.5\textwidth}
  \centering
  \includegraphics[width=0.3\textheight]{monoW/simplifiedDM_diagram.pdf}
\end{subfigure}%
\begin{subfigure}{.5\textwidth}
  \centering
  \includegraphics[width=0.3\textheight]{monoW/simplified_tChannel_model_diag.pdf}
\end{subfigure}
  \caption{Feynman diagrams of production of dark matter pairs ($\chi\overline{\chi}$) associated with a W boson in simplified model 
	   in s-channel (left) and t-channel (right) scenarious.}
  \label{fig:feynMonoWSimple}
\end{figure}

% TODO change y-axis title. because it is normalized distribution!!!

\begin{figure}[]

\centering
\begin{subfigure}{.5\textwidth}
  \centering
  \includegraphics[width=0.3\textheight]{monoW/WWxx_diagramm_v2.pdf}
\end{subfigure}%
\begin{subfigure}{.5\textwidth}
  \centering
  \includegraphics[width=0.3\textheight]{monoW/EFT_D5_model_diag_v3.pdf}
\end{subfigure}
  \caption{Representative diagrams for production of dark matter pairs ($\chi\overline{\chi}$) associated with a W boson in models where
dark matter interacts directly with W boson (left) or with quarks (right).}
  \label{fig:feynMonoWEFT}
\end{figure}

Current study cover both s- and t-channel simplified models with different mass of $Z'$ mediator and with different mass of DM particles.
EFT approach are represented by two models which correspond to diagrams which are shown at \FigureRef{fig:feynMonoWEFT}. 
They are charachterized by energy scale of ``new physics'' $\Lambda$ (or effective mass $M^{*}$) as well as mass of produced DM particles.
Both models assume vector type of interaction between DM and SM sectors. For D52 model (which correspond to the right diagram in  \FigureRef{fig:feynMonoWEFT}) 
we consider only constructive interferance with SM.

\subsection{Sensitivity studies}

Main idea of the study is to understand kinematics of DM models in final state with one lepton and missing energy and to estimate sensitivity 
of signal selection to such kind of models.
Signal selection are designed with focus of high-$m_{T}$ region in order to get rid of dominant SM W background.
Kinematic distribution of DM models were studied to evaluate contribution to the signal selection and to make comparison
with signal from $\PWprime$ model.
It's worth to mention that due to the fact that mass of DM particles doesn't affect kinematic distribution of the event 
all simulated DM samples used are with mass of DM particle was set to 1 GeV.
(should some plots to be shown to proof this???).

In \FigureRef{fig:kinematicsSChannel} normalized kinematic distributions of transverse mass of lepton and $E_{T}^{miss}$ of the event
and $E_{T}^{miss}$ of the event for s-channel simplified model as well as EFT models with comparison with $\PWprime$ model are shown.
First distinguishable charachteristic of all DM models is that there is no clear peak structure in any kinematic distribution, 
which is expected because transverse missing energy is formed by
neutrino from W boson decay and two DM particles which are independent between each other. 
Second feature of DM models that main contribution are tends to be in low-$m_{T}$ region.
Especially it concern simplified models, for which dominant contribution is outside of signal region for any parameter of mediator mass.
But with increasing of mediator's mass $m_{T}$ distribution tends to become more flat and moves towards high-$m_{T}$ region.
Also with increasing mediator mass cross section of process significantly decreasing (see \TableRef{tab:TriggerDetails}) 
and for mass equal 10 TeV cross section is a few order of magnitude lower than background (need proof, find plot ???) which 
makes signal selection not sensitive for s-channel simplified DM model.

% TODO reprhase it. It not visible from table - only from non-normalized plot. Plot should be mono-W with W backgound only? Or I should also include other backgounds?

\begin{figure}[]

\begin{subfigure}{.5\textwidth}
  \centering
  \includegraphics[width=\textwidth]{monoW/mT_kinemComparison_simplS_EFT_Wprime.png}
\end{subfigure}%
% \begin{subfigure}{.333\textwidth}
%   \centering
%   \includegraphics[width=0.25\textheight]{monoW/pT_kinemComparison_simplS_EFT_Wprime.png}
% \end{subfigure}
\begin{subfigure}{.5\textwidth}
  \centering
  \includegraphics[width=\textwidth]{monoW/EtMiss_kinemComparison_simplS_EFT_Wprime.png}
\end{subfigure}

% \includegraphics[width=0.75\textheight]{monoW/kinematics_simplifiedSChannel_EFT_Wprime.png}
\caption{Normalized kinematic distributions of transverse mass (left) and transverse missing energy in the event (right) of simplified model in s-channel as well as EFT models compared to $\PWprime$ distribution.}
  \label{fig:kinematicsSChannel}
\end{figure}

In \FigureRef{fig:scaledKin} transverse mass distribution are show for s- and t-channels simplified models as well as EFT and $\PWprime$ signals with comparison to SM W background 
scaled to according cross process section. Distribution for t-channel model looks almost identical to one for s-channel. 
But cross sections for t-channel processes are for one-two order of magnitude lower compared to s-channel (see \TableRef{tab:TriggerDetails}) which leads to conclution
that signal selection even less sensitive for t-channel simplified model. Aslo SM W backround are for a few orders higher in all range of $m_{T}$ than any sample of simplified model.
For D52 EFT model excess over SM W backgound can be seen in high-$m_{T}$ region. While for WW$\chi\chi$ EFT model it is not the case, because cross section is very low.

% TODO us or D5 or D52. harmonize stuff!!!

\begin{figure}[]
\begin{subfigure}{.5\textwidth}
  \centering
  \includegraphics[width=\textwidth]{monoW/dm_final_S_vs_T_channel_pad1.png} 
\end{subfigure}%
% \begin{subfigure}{.33\textwidth}
%   \centering
%   \includegraphics[width=0.95\textwidth]{monoW/pT_kinemComparison_simplT_EFT_Wprime.png}
% \end{subfigure}
\begin{subfigure}{.5\textwidth}
  \centering
  \includegraphics[width=\textwidth]{monoW/dm_final_EFT_vs_SMW_pad1.png}
\end{subfigure}
% \includegraphics[width=0.75\textheight]{monoW/kinematics_simplifiedTChannel_EFT_Wprime.png}
\caption{Kinematic distributions of transverse mass of simplified model in s- and t-channels (left) 
and EFT models with $\PWprime$ signal (right) in comparison with SM W background scaled to according process cross section.}
  \label{fig:scaledKin}
\end{figure}

Transverse mass distribution for both EFT models looks similar and majority (???) of the signal lays in the signal region. 
% Cross section for both processes strongly depends on energy scale of new physics $\Lambda$ (or effective mediator mass for D52 model).
% Dependence of cross section for WW$\chi\chi$ model versus energy scale is shown in \FigureRef{fig:lambdaScan}. 

% TODO describe what is D52 model. That it is vector interaction constructive interference

\begin{table}[]
  \begin{tabular}{r|c|c|c}
    Model 	& Channel 	  & Parameters	    & Cross section, [nb] \\
    \midrule
    Simplified  & s-channel	  & $M_{Z'}$=1TeV    & .05181 \\
		&		  & $M_{Z'}$=100GeV  & .001989 \\
		&		  & $M_{Z'}$=10TeV   & 7.5E-11 \\
		& t-channel	  & $M_{Z'}$=10GeV   & .0018895 \\
		&		  & $M_{Z'}$=100GeV  & 9.2E-5 \\
		&		  & $M_{Z'}$=2TeV    & 4.874E-8 \\
    \midrule
%     EFT 	& WW$\chi\chi$	  & $m_{\chi}$=1GeV; $\Lambda$=3TeV    & 3.6E-10 \\
% 		& D52		  & $m_{\chi}$=1GeV; $M^{*}$=1TeV	& 4.4E-4 \\
EFT 	& WW$\chi\chi$	  & $\Lambda$=3TeV    & 3.6E-10 \\
	& D52		  & $M^{*}$=1TeV	& 4.4E-4 \\
    \midrule	
$\PWprime$ 	& 	  & $m_{\PWprime} =$ 2 GeV   & 1.1E-4 \\    
		%     Wprime ($m_{\PWprime}$=2TeV) & 1.1E-4 \\
  \end{tabular}
  \caption{Mono-W cross section for different theoretical models.}
  \label{tab:TriggerDetails}
\end{table}


\begin{figure}[]
 \includegraphics[width=0.6\textheight]{monoW/WWxx_Wlv_DM1_LambdaScan.pdf}
  \caption{Cross section of WW$\chi\chi$ EFT model as a function of energy scale of new physics, $\Lambda$.}
  \label{fig:lambdaScan}
\end{figure}

\subsection{Validity of EFT approach and summary}

Cross section for both EFT processes strongly depends on energy scale of new physics $\Lambda$ (or effective mediator mass $M^{*}$ for D52 model).
Dependence of cross section for WW$\chi\chi$ model versus energy scale is shown in \FigureRef{fig:lambdaScan}.
In order for WW$\chi\chi$ process have sizeable cross section compared with $\PWprime$ ($M_{\PWprime}$=2TeV) model which is used as a reference in this study, 
energy scale of new physics $\Lambda$ has to be of order 200-300 GeV. But according to ~\cite{arXiv:1512.00476}: 
``The EFT approximation is valid when the momentum transfer in a given
process of interest is much smaller than the mass of the mediating
particle. For momentum transfer larger than or comparable to
$\Lambda$, the EFT description will break down.''
Moment transfer for 13 TeV collisions at LHC correspond to scale of few TeV, 
which mean that $\Lambda$ has to be of order of TeV in order for model to be valid.

% TODO cross check this statement with Dark matter report...
% http://arxiv.org/pdf/1507.00966.pdf
Also for WW$\chi\chi$ model there is no straightforward way to compare the
results with non-collider searches for DM
which make this model less appealing comparing with other models [add non-collider DM production diagram].

D5 constructive (D52) model violate weak gauge invariance as described at ~\cite{arXiv:1503.07874}.
It leads to spurious cross section enhancements at LHC energies. It is not recommended to be used anymore by Dark Matter Forum (some link???).

\subsection{Conclusion}

Transverse mass and $E_{T}^{miss}$ for all presented DM models with comparison with $\PWprime$ ($M_{W'}$ = 2TeV) signal are shown at \FigureRef{fig:scaledKin}.
All distribution are scaled according to the cross section of the process. It's clearly seen that simplified models tends to contribute to low-$m_{T}$ region
outside of the signal selection. With increasing of mediator mass $Z'$ cross section drops significantly and become indistinguishable with SM W background.
On other hand EFT DM samples contribute to high-$m_{T}$ region but as desribed above D52 model has physicaly unmotivated high cross section and WW$\chi\chi$ 
significantly smaller comparing with $\PWprime$ signal for physicaly motivated values of scale of new physics $\Lambda$. Which leads to the conclusion that
lepton and $E_{T}^{miss}$ signal selection is not sensitive for DM searches.



Similar studies was done by Bell and collaborators~\cite{arXiv:1512.00476}, where authors estimated approximate upper limit on for 3000 fb$^{-1}$. 
At \FigureRef{fig:bellExclLim} exclusion limit
as a function of mass of DM particle and mass of DM-SM mediator $Z'$ is shown. One can notice that obtained limits for mono-lepton channel are significantly 
worse than from all other channel, especially than from di-jet analysis. Conclusion of authors are identical to the conclusion of this study that mono-lepton channel
are not sensitive enough for DM searches and is significantly worse comparing to all other hadronic channels.

\begin{figure}[]
 \includegraphics[width=0.8\textwidth]{monoW/schan1.pdf}
  \caption{Exclusion limit for the s-channel $Z'$ model as a function of mass of dark matter particle, $m_{\chi}$, 
  and mass of DM-SM mediator, $m_{Z'}$, reported in~\cite{arXiv:1512.00476}.
  Exclusions are shown as shaded regions for LUX and for mono-jet and di-jets at 8 TeV, 
  and the reaches are shown for the mono lepton and mono fat jet searches at 14 TeV 3000 fb$^{-1}$.}
  \label{fig:bellExclLim}
\end{figure}

\section{Outlook}
\label{sec:wprimeConclusion}
