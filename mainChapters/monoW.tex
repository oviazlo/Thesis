\chapter{Search for new charged bosons in final states with one muon and missing transverse momentum}
\label{chap:Wprime}

% PLAN
% <What this chapter about - short intro to Wprime analysis>
% <What was my personal contribution to the analysis> 


This chapter describes search for new spin-1 heavy charged boson (namely $\PWprime$) in the final state with one lepton and missing transverse momentum ($E_T^{miss}$).
The search was done with first $\sqrt{s}$=13~TeV data collected in 2015 by ATLAS with a corresponding luminosity of 3.2~fb$^{-1}$. The search has been performed with muon and electron channels, however, this chapter is focused on the muon channel and only final results for the electron channel are shown, as well as the combination of both channels.

% TODO describe more about the analysis.
% Analysis focuses on high-p$_T$ 

% Key features of the analysis.

% TODO move it to preface!!!
% This analysis have been made public as a conference note~\cite{ATLAS-CONF-2015-063} with early reasults and later as a paper~\cite{Aaboud:2016zkn} where final results were presented.
% 
% \toDo[carefully revisit and rewrite this]
% 
% My personal contribution to the analysis can be concluded in following three parts.
% I was involved in the analysis of muon channel. Signal selection, muon and missing transverse momentum performance, validation plots and estimation of some systematic uncertainties.
% This part was done in parrallel with other collaborator to make sure that results are robust and are the same from two independent analyses codes and there are no mistaken done.
% 
% Secondly I was investigating ways to decrease systematic uncertainty caused by the limited available statistic of inclusive diboson and top backgound samples at high-$m_{T}$ region.
% One option was to generate lepton-neutrino mass-binned samples. However, it was found out that these samples doesn't populate enough high-$m_{T}$ region
% so it was decided to make samples in bins of lepton-$E_T^{miss}$ $m_T$.
% This is still ongoing project and it is planned to finish these samples for paper of the $\PZprime$ and $\PWprime$ results from 2015-2016.
% 
% The third part was related to the investigation sensitivity of $\PWprime$ signal selection for Dark Matter (DM) search.
% Main focus of this investigation was mono-W DM models, where pair of DM particles candidates are produced in final state in association with SM W boson.
% Two type of DM of models were considered: simplified and Effective Field Theory models. 


% TODO discuss with Monika - how should I position analysis - as a dedicated search of Wprime or as more model-independent search?

%*******************************************************************************
% MOTIVATION
%*******************************************************************************


\section{Search strategy}
\label{sec:wprimeIntro}
% TODO model-independent search???
% TODO experimental signature - lepton + MET
% TODO a signal discriminant: mT

As was described in \SectionRef{subsec:bsm_models}, there is a large number of models
which predict a new spin-1 gauge charged boson $\PWprime$. Thus it is not practical to perform a number of dedicated searches for all of them. This is why the so-called ``sequential'' Standard Model (SSM) is often used, being a reference benchmark model.
This model aims to provide a clear interpretation of the experimental results and
is used for results comparison between different experiments.
It assumes $\PWprime$ to be a heavy ``copy'' of the SM $W$ with the same couplings to leptons, quarks and gauge bosons. 
Given such couplings a new decay channel (with respect to the SM $W$)
should be present: $\PWprime \to WZ$. It would be the dominant decay mode 
for high $\PWprime$ masses and will lead to the $\PWprime$ width larger
than its mass at  $m_{\PWprime} > 500$~GeV. 
Some models predict such channel to be heavily suppressed, 
such as the Left-Right Symmetric model, described previously, in a case of $m_{W_R} \gg m_{W_L}$.
Thus, branching ratio for this channel is set to be zero for any $\PWprime$ mass.
The branching ratio of $\PWprime \to \mu\nu$ or $e\nu$ as a function of $\PWprime$ mass
is shown in \FigureRef{fig:wprimeBR}. Rapid decrease at approximately 200 GeV corresponds to the decay channel $\PWprime \to tb$.
\begin{figure}[!htb]
  \centering
  \includegraphics[width=7.5cm]{Wprime/WprimeBR.eps}
  \caption{Branching ratio of $\PWprime \rightarrow e\nu$ or $\mu\nu$ as a function of the $\PWprime$ mass.}
  \label{fig:wprimeBR}
\end{figure}

Considering the used model one can highlight three key features essential for this search:
\begin{itemize}
 \item Precise modeling of the background prediction. 
 The dominant part of the background originates from the charged-current Drell-Yan process and the analysis selection tests it up to few TeV.
 Thus, it is crucially important to use the latest and most precise high-order calculations and corrections available at the time.
 \item High-$p_T$ lepton selection. Due to a topologically simple event selection (one isolated lepton and missing energy) 
 this analysis uses most energetic lepton candidates available at 13 TeV center-of-mass energy collisions.
 This is why reconstruction efficiency and momentum resolution of high-$p_T$ leptons are factors of particular interest for the analysis.
 \item Missing transverse energy, $E_T^{miss}$. Along with the lepton momentum, $E_T^{miss}$ is used for transverse invariant mass $m_T$ (which will be defined in \SectionRef{subsec:etmiss}) calculations,
 which is the signal discriminant and search variable in the analysis. For $m_T$ to be precisely reconstructed and modelled one has to have a good understanding of the missing $E_T$ reconstruction as well as good lepton momentum resolution.
\end{itemize}

\section{$\PWprime$ signal in Monte Carlo}
\label{sec:wprimeSignal} 
% TODO rewrite it a little bit, cause it is directly copied from the note!!!

% TODO description of the flat sample:
% Wprime flat template sample which has a flat lepton-neutrino invariant mass spectrum via the removal of the breit-wigner term from pythia generation
% 
% the sample produced will have a mass spectrum which is essentially flat, with statistics at low to high mass, with no resonance shape. 
% This spectrum can then be almost exactly re-weighted to any Wprime resonance mass, given there are enough statistics in the flat spectrum

% MC generator
Samples for the signal process $\PWprime \to \mu \nu$ are produced with the leading-order (LO) 
{\scshape pythia-8.183}~\cite{pythia8} generator for a series of $\PWprime$ masses. 
Additionally, the so-called flat sample is produced. This sample has a flat lepton-neutrino invariant mass spectrum.
Thus, it can be reweighted with the correct line shape to any desired $\PWprime$ mass by an appropriate reweighting function.
To verify the validity of this sample and validity of the reweighting procedure, comparisons with fixed mass samples were done.
The flat sample was generated with large statistics to cover
a wide range of the transverse invariant mass $m_T$.

% TODO explain what is ``Wprime signal''... see comment from Oxana by Dec7
% comparison with W boson background (show plot of Wprime overlaid on top of W)
The invariant mass and transverse mass distributions for the $\PWprime$ samples with pole mass of 2,3,4 and 5 TeV are shown superimposed on top of SM $W$ background in \FigureRef{fig:signal_with_W}.

\begin{figure}
\begin{subfigure}{.5\textwidth}
  \centering
  \includegraphics[width=\textwidth]{Wprime/Signal_onTopOf_W_invMass.eps}
\end{subfigure}%
\begin{subfigure}{.5\textwidth}
  \centering
  \includegraphics[width=\textwidth]{Wprime/Signal_onTopOf_W_mT.eps}
\end{subfigure}
\caption{Invariant mass (left) and transverse mass (right) spectrum of the $\PWprime$ signal on top of the $W$ background on generated MC level.}
  \label{fig:signal_with_W}
\end{figure}

%*******************************************************************************
% BKG PROCESSES AND MC
%*******************************************************************************


\section{Background processes}
\label{sec:wprimeBackgrounds}

In order to look for a potential signal from the new physics, one has first to examine all other SM processes which contribute to the final state of interest.

% All background predictions are obtained with MC simulation, except for non-prompt
% lepton contribution, which arising due to jets and photon being misreconstructed
% as leptons.

Since this chapter is focused on the muon decay channel, the processes which produce a muon and missing transverse momentum will be discussed, 
however, in general, the same processes are relevant for the electron decay channel as well.

The dominant expected background in the analysis comes from the charged current Drell-Yan process.
The SM W decays to a lepton and a neutrino, which will be reconstructed as missing transverse momentum in the detector.
The contribution of this process appears like a Jacobian peak in the $m_T$ spectrum
with maximum around 80 GeV and slowly falling tail above 80 GeV.
% TODO probably I need to move it somewhere else, no?
Since $\PWprime$ conceptually is a heavier version of the SM W, it also appears in the transverse mass distribution as a Jacobian peak around the pole mass of the $\PWprime$ boson (as shown in \FigureRef{fig:signal_with_W}). 
% W boson bkg
The charged current Drell-Yan process is simulated with \powhegbox\ v2~\cite{Alioli:2010xd}and {\scshape pythia-8.186} generators at next-to-leading-order (NLO) using the CT10~\cite{CT10} NLO PDFs. 
The cross section is corrected to next-to-next-to leading order (NNLO) using the CT14NNLO PDF set by applying QCD and Electroweak (EW) mass dependent $K$-factors to the MC generator cross sections.
To get sufficient statistics at high transverse mass, several samples, binned in invariant mass of the lepton-neutrino pair, are used if available.

% Z boson bkg
The neutral current Drell-Yan process $Z/\gamma^* \to \mu \mu$ can contribute to the muon plus $E_T^{miss}$ final state when one of the muons is not properly reconstructed in the detector. 
In that case, it will be not used in the $E_T^{miss}$ calculations, but it will contribute to the $E_T^{miss}$, as will be described in \SectionRef{subsec:etmiss}.
Thus $Z$ boson production has to be considered as well. The process is simulated with the same MC generators and in the same way as the W boson production process.
% Contribution from taus
Contribution of the processes $W \to \tau \nu$ and $Z \to \tau \tau$, which can contribute to the muon channel 
when tau decays to muon and neutrinos as $\tau^{-} \to \mu^{-} \overline{\nu_{\mu}} \nu_{\tau}$, are considered and simulated in the same way as well.

% Top
Another background which contributes to the final state of interest is the $t\bar{t}$ and single top production.
Top quark decays immediately to the $W$ boson and $b$ quark. The further leptonic decay of $W$ provides an isolated muon and $E_T^{miss}$ from escaping neutrino.
% TODO make Feynman diagrams the same as in Magnars thesis p.91-92
Some Feynman diagrams of the top production processes are shown in \FigureRef{???} and \FigureRef{???}.
This background is simulated with \powhegbox\ and {\scshape pythia-6.428}~\cite{Pythia} at NLO using the CT10~\cite{CT10} NLO PDFs.
All these processes are considered as a ``Top'' background in the text below.

% Diboson
More than one SM gauge boson can be produced in a single hard interaction, thus processes with $WW$, $WZ$ and $ZZ$ boson pairs produced in the final state are considered as well.
Some Feynman diagrams of such processes are shown in \FigureRef{???}.
Contribution to the muon plus $E_T^{miss}$ final state can come from decays like $WZ \to \Plepton \nu \nu \nu$ or $WZ \to \Plepton \nu q \overline{q}$.
These processes are simulated with {\scshape sherpa-2.1.1}~\cite{Sherpa} using the CT10 NLO PDFs.
Such processes are grouped as a ``diboson'' background. 

% mT-binned samples
Only inclusive samples for both diboson and top backgrounds were available from official ATLAS production. 
This is why it was decided to produce samples binned in the transverse mass of lepton plus $E_T^{miss}$.
However, due to technical complications these samples were not finished for the 2015 analysis 
so a dedicated fitting procedure to the high-$m_T$ region has been used to estimate these backgrounds.
However, it is planned to use these samples for the combined 2015 and 2016 study.

% Table with MC
List of all background processes with used MC generators is shown in \TableRef{tab:MC_cross}.

\begin{table}[ht]
  \begin{center}
    \begin{tabular}{l|c|c|c}

      \hline
Process &  Generator&  PDF set & Normalization \\
&  + fragmentation/ &  & based on \\
&  hadronization & &\\
\hline\hline
&   &   \multirow{4}{*}{CT14NNLO~\cite{Dulat:2015mca}} & NNLO QCD \\
$W +$ jets, & \powhegbox\ v2~\cite{Alioli:2010xd} & &  with \vrap~\cite{vrap}, \\
$Z/\gamma^* +$ jets & + {\scshape pythia-8.186}~\cite{pythia8}  & &  NLO QED \\
 & & &  with \mcsanc~\cite{Bardin:2012jk,Bondarenko:2013nu} \\
\hline
\ttbar, t-channel $t$, & \powhegbox\ & \multirow{2}{*}{CT10} & \multirow{2}{*}{NLO QCD} \\
s-channel $Wt$ & + {\scshape pythia-6.428}~\cite{Pythia} & &  \\
\hline
\multirow{2}{*}{$WW, WZ, ZZ$} & \multirow{2}{*}{{\scshape sherpa-2.1.1}~\cite{Sherpa}} & \multirow{2}{*}{CT10} & \multirow{2}{*}{???} \\
 & & &  \\
\hline
\hline
\multirow{2}{*}{$W^\prime \rightarrow \Plepton \nu$} & \multirow{2}{*}{{\scshape pythia-8.183}} &   \multirow{2}{*}{NNPDF2.3 LO} & NNLO QCD \\
& & &  with \vrap, \\
\hline
\end{tabular}
\end{center}
  \caption{List of MC generated samples used for background prediction. 
  The used MC generator, PDF set and order of cross section calculations used for the normalization are listed for each sample.
  }
\label{tab:MC_cross}
\end{table}

%*******************************************************************************
% SELECTION
%*******************************************************************************


\section{Event and lepton selection}
\label{sec:wprimeSelection}

The analysis is based on pp collision data collected in 2015 by the ATLAS detector with 13 TeV center of mass energy.
The integrated luminosity of the data sample corresponds to 3.2 fb$^{-1}$, the mean number of interactions per bunch crossing was 14.

An event selected for the analysis has to have at least one reconstructed vertex with at least two tracks matched to it. 
If there are several vertices, the one with the highest
$\sum p^2_T$, where $p_T$ are transverse momenta of the matched tracks, is chosen.
Events should have at least one muon candidate, and fire the single muon trigger, 
which requires the presence of one muon with $p_T > 50$ GeV.

\subsection{Lepton selection}
% description of muon reconstruction in the ATLAS detector
Muon reconstruction in ATLAS is performed independently in ID and MS detectors. 
Information from the detectors is then combined to form a muon track.
There are different ways of combining information from both detectors, 
thus four different muon types are used in ATLAS~\cite{muon_performance_2015}.
In this analysis the so-called combined muons are used, for which track reconstruction is done 
separately in the ID and MS, after which a global refit is done to form a combined track.
% TODO what is measured: $p_T$ or momentum? check muon paper
Muon $p_T$ is measured from the track curvature.

% As well as for electrons, there are different sets of identification criteria which provide different background supression, reconstruction efficiency and momentum measurement resolution: Loose, Medium, Tight and High$-pT$.

Muons of interest are high-$p_T$ isolated muons, tracks of which originate
from the primary vertex. Candidates have to satisfy the following set of criteria:
\begin{itemize}
 \item $p_T > 55$ GeV: to ensure a high and uniform trigger efficiency.
 \item $|\eta|<2.5$, excluding $1.01 < |\eta| < 1.1$: muons have to be within ID acceptance. 
 Exclusion region is applied in order to reject muons whose tracks in the muon spectrometer fall into poorly aligned chambers (relative barrel-endcap alignment).
%  TODO see comment from Monika to SS chapter.
 \item Transverse and longitudinal impact parameters $|d_0|/\sigma(d_0)<3$ and $|z_0\times sin\theta|<10$~mm: 
 to verify that the muon was produced close to the primary vertex and to reject muons originating from decays of long-lived particles. Recommendation by the muon combined performance (MCP) working group has
 more strict requirement $|z_0 \times sin \theta| < 0.5$ mm, however after dedicated studies it 
 was decided to relax this requirement to 10 mm as will be described in \SectionRef{subsec:wprime_cut_optimization}.
 \item Pass ``high-$p_T$'' set of muon identification criteria defined in~\cite{muon_performance_2015}. \\ 
 This set aims to maximize the momentum resolution for tracks with $p_T > 100$~GeV.
 It includes tight requirements on the MS part of the track, which reduces reconstruction
 efficiency of the muons up to 20$\%$ with respect to other identification sets, however, 
 improvement of the $p_T$ resolution is reaching approximately 30$\%$.
%  TODO describe why LooseTrackOnly isolation is enough
 \item Pass ``LooseTrackOnly'' isolation requirement~\cite{muon_performance_2015}. 
 This requirement provides 99$\%$ constant efficiency over complete ($\eta$,$p_T$) phase space.
 The discriminating variable is the ratio of the sum of $p_T$ of all tracks (excluding the muon itself) with $p_T > 1.0$~GeV within $\Delta R = min(10$ GeV $/p_T^{\mu}, 0.3)$ 
 cone around the muon track, to the muon track $p_T^{\mu}$.
\end{itemize}  
 
\subsection{Optimization of the signal selection}
\label{subsec:wprime_cut_optimization}
% TODO additional lepton veto cut

To suppress contribution from the neutral current Drell-Yan $Z/\gamma^*$ and $t\bar{t}$ processes,
where two isolated leptons are expected in the final state, an additional lepton veto requirement
is applied. If one muon passes the signal selection and the second muon fails the signal but passes a loosened version of the above selection with the ``high-$p_T$'' identification working point replaced by the requirement to pass either ``medium'' or
``high-$p_T$`` identification criteria, and passes a lower $p_T$ cut of $20$~GeV, such event is vetoed.
Events are also vetoed if an additional electron passing the selection below is found.
\begin{itemize}
\item $|\eta| < 2.47$, excluding barrel-endcap calorimeter transition region $1.37 < |\eta| < 1.52$
\item $p_T > 20$~GeV
\item Transverse impact parameters $|d_0|/\sigma(d_0) < 5$
\item Pass the likelihood ``medium'' identification criteria~\cite{ATL-PHYS-PUB-2015-041}
\item Pass the ``loose'' isolation criteria~\cite{ATLAS-CONF-2016-024}
\item Electron must not overlap with the muon: $\Delta R(e,\mu)>0.1$. If it overlaps ($\Delta R(e,\mu)>0.1$)
it is assumed that the electron candidate arises from photon radiation from the muon and the event is kept.
\end{itemize}
Veto requirements lead to a significant reduction of the dimuon ($Z$) background
at high transverse mass as well as some reduction of the $t\bar{t}$ background.
The reduction of the total background level is approximately $10$--$15\%$ at high transverse mass. 
The signal efficiency is found to be essentially unaffected.
The possibility of using the ``loose'' identification working point for the additional muon veto was also considered. However, it was found to provide only a tiny improvement ($1$--$3\%$ additional reduction of the total background level) with respect to using ``medium'' working point, thus the latter was chosen to be used.

% TODO z sin_theta
The default recommendation by the muon combined performance (MCP) working group is to apply requirement $|z_0 \times sin \theta| < 0.5$ mm. The main purpose of this requirement is to veto events with cosmic muons.
However, by making the requirement of absence of the second muon in the event we ``automatically'' discard most of the events with cosmic muons. After a dedicated study by the MCP group, it was found that the 
$|z_0 \times sin \theta|$ requirement can be loosened to 10 mm, without significant decrease of the cosmic muon rejection power.
Figure \ref{fig:Muon_LepVtxEff} shows $d_0$ significance ($|d_0|/\sigma(d_0)$) and $|z_0 \times sin \theta|$ cut efficiencies. The efficiency shown in the left panel is for the recommended $|z_0 \times sin \theta|$ cut value of $0.5$ mm, while in the right panel - for a looser
cut value of $10$~mm. The nominal cut value leads to a reduction of the selection efficiency by about $1$\% due to picking a wrong vertex as the primary vertex. This efficiency is partially restored by the loosened cut.

\begin{figure}[]
  \centering
  \includegraphics[width=0.45\textwidth]{Wprime/IPplot05mm.eps}
  \includegraphics[width=0.45\textwidth]{Wprime/IPplot10mm.eps}
  \caption{$d_{0}$ and $|z_0 \times sin \theta|$ cut efficiencies. The $d_{0}$ efficiency is shown for the cut recommended by the tracking group. In the $|z_0 \times sin \theta|$ case the recommended cut of $0.5$ mm (left) and an alternative cut of $10$ mm (right) are shown. The efficiencies are calculated for combined muons in the
$\PWprime$ flat sample passing the medium or high-$p_T$ working point requirements.}
  \label{fig:Muon_LepVtxEff}
\end{figure}


\subsection{Transverse mass and missing transverse momentum}
\label{subsec:etmiss}
The missing transverse momentum, $E_T^{miss}$, is calculated following the ATLAS recommendation and recommendations described in ref.~\cite{met2015_1,met2015_2}.
$E_T^{miss}$ is calculated as as a vector sum of the $p_T$ of selected objects:
\begin{itemize}
 \item muons which satisfy analysis signal selection;
 \item electrons which satisfy requirements described previously in
 \SectionRef{subsec:wprime_cut_optimization} with stronger transverse momentum requirements $p_T > 55$ GeV
 and likelihood ``tight'' identification criteria (which is the electron signal selection);
 \item tau leptons which satisfy ``medium'' identification criteria~\cite{tau_id_8TeV} and $|\eta| < 2.5$, excluding  $1.37 < |\eta| < 1.52$ and $p_T > 20$~GeV requirements;
 \item photons which satisfy ``tight'' identification criteria~\cite{photon_id_2011}, $|\eta| < 2.37$, excluding  $1.37 < |\eta| < 1.52$ and $p_T > 25$~GeV requirements;
 \item jets reconstructed with anti-$k_t$ algorithm~\cite{jet_anti_kt} with radius parameter of 0.4.
 Jets are calibrated using the method described in ref.~\cite{jet_calib_syst_13TeV}.
 Only jets with $p_T > 20$~GeV and $|\eta| < 5$ are used;
 \item tracks originating from the primary vertex with $p_T > 0.5$ GeV and $|\eta| < 2.5$ but not belonging to any of the reconstructed physics objects listed above;
 \item \toAsk[unused calo contribution?]
\end{itemize}
The missing transverse momentum is required to be larger than 55~GeV in order 
to balance the lepton transverse momentum cut.
Such high cut allows to significantly suppress 
the multi-jet background, which will be described below.

% TODO describe why if muon is not reconstructed it will contribute to Etmiss (see Magnars thesis, page 89)

The main variable of interest, $m_T$, which is used for statistical discovery analysis is defined as:
\begin{equation}
 m_\mathrm{T} = \sqrt{2 p_\mathrm{T} E_T^{miss} (1-\cos\varphi_{\Plepton\nu})}
\end{equation}
where $\varphi_{\Plepton\nu}$ is the angle between the muon momentum and the missing transverse momentum in the transverse plane.

The transverse mass has to be $m_\mathrm{T} > 110$~GeV.

%*******************************************************************************
% BACKGROUND ESTIMATION
%*******************************************************************************

\section{Background estimation}
\label{sec:wprime_backgroundEstimation}
% TODO describe sources of fakes:

The background which arises from neutral and charged current Drell-Yan processes ($W$, $Z$), diboson
production ($WW$,$WZ$,$ZZ$) and top background (in general, all processes listed in top part of \TableRef{tab:MC_cross}) is estimated with MC simulation.

Processes with multijet final state have a small chance to be reconstructed as those which contain
an isolated muon. There are two cases when jets lead to a reconstructed muon. One of the possibilities is when a hadron is not stopped in the calorimeter and go the whole way through up to the muon spectrometer and is misidentified as a muon. Another possibility is when a real muon originates from the jet activity. However, such muons are not produced in the primary vertex, because they originate from decays of heavy flavor hadrons, which have long lifetime and travel for few hundreds $\mu m$ before they decay.
Thus, reconstructed muons from jets are called ``fake'' muons.

Due to the huge cross section of the processes which lead to the multijet final state it is very difficult and time consuming to simulate their production with MC simulation, this is why a data-driven method,
the so-called Matrix Method, is used to estimate contribution of the multijet processes to the signal muon selection.

% The backgrounds which have final state with a real muon originating from the primary vertex and 
% which can contribute to the signal selection are modelled with MC simulation.
% List of all considered processes is shown in \TableRef{tab:MC_cross}.
% Such muons are called ``real'' (prompt) muons.
% 
% However, processes with multijet final states (with one or more jets) will also contribute to
% the signal selection due to probability of wrong reconstruction of the jet activity as a muon.
% Such muon candidates are called ``fake'' (non-prompt) muons.
% The ``fake'' muon can be a real muon which originate from a heavy flavor hadron decay within a jet
% or from pion or kaon decays. But because they are not originate from the primary vertex they are not
% desired to be selected in the signal region and thus are called ``fake'' muons.
% The ``fake'' muon background is estimated using a data driven technique. 
% ``Fake'' muons are expected to be in general non-isolated, although some
% fraction does pass the isolation cut and ends up in the selected event sample. Isolation variables
% hence provide a strong separation of ``fake'' and ``real'' muons, and are essential to the data-driven
% estimation of the ``fake'' background.
% The ``fake'' background is estimated using the Matrix Method, which is presented below.

\subsection{The Matrix Method}
\label{subsec:matrix_method}
The goal of the method is to get an estimation of ``fake'' muon contribution to the signal muon selection.
The ``fake'' muons in general are expected to be non-isolated, however, some fraction does pass the isolation cut. 
Thus, the idea of the method is to measure the probability (efficiency) of the ``fake'' muons
which pass the loosened signal selection without isolation requirement to pass the nominal muon signal selection.
Also one wants to measure the same efficiency for the ``real'' muons, which originate from the processes with real prompt muons from the primary vertex.

The matrix method provides a connection between the number of true ``fakes'' ($N_F$) and the number of true ``real'' muons ($N_R$) and the measurable quantities from a sample in which the muons are passing loose selection ($N_L$) criteria and at the same time failing a tight selection and a sample in which the muons pass the tight selection ($N_T$) criteria, via \EquationRef{eq:mm1}.
\begin{equation}
  \left(\begin{array}{c} N_T \\ N_L \end{array}\right)&=
  \begin{pmatrix}
    \epsilon_R & \epsilon_F \\
    1-\epsilon_R & 1- \epsilon_F \\
  \end{pmatrix}
  \left(\begin{array}{c} N_R \\ N_F \end{array}\right)
  \label{eq:mm1}
\end{equation} 
The vector on the right hand side of the equation corresponds to the
truth quantities which are independent from each other.
It means that quantities in the vector on the left hand side
have to be independent as well, thus $N_L$ should not contain any events
from $N_T$, and this is why the former value is defined as passing the loose selection but failing the tight one.

Matrix consists of efficiencies of ``fake'' and ``real'' muons passing signal selection which are denoted as $\epsilon_F$ and $\epsilon_R$, respectively.
Efficiency is defined as the ratio of the number of ``fake''(``real'') muons which pass the tight selection,
$N_{tight}^{fake}$ ($N_{tight}^{real}$), to the number of muons which fail the tight but pass the loose selection,
$N_{loose}^{fake}$ ($N_{loose}^{real}$):
\begin{equation}
 \epsilon_F = \frac{N_{\textrm tight}^{\textrm fake}}{N_{\textrm loose}^{\textrm fake}} ,\qquad & \epsilon_R = \frac{N_{\textrm tight}^{\textrm real}}{N_{\textrm loose}^{\textrm real}}.
  \label{eq:mm2}
\end{equation}
The total number of muons passing the signal selection is given in the first line of the matrix:
\begin{equation}
 N_T&=N_{\textrm tight}^{\textrm real} + N_{\textrm tight}^{\textrm fake}=\epsilon_R N_R + \epsilon_F N_F\, ,
\label{eq:number_of_tight_muons}
\end{equation}
and consists of the fraction of ``fake'' muons which pass the signal selection and the fraction of ``real'' muons.
The value of interest is the true number of ``fake'' muons which pass the signal selection. One can express it by inverting the matrix in \EquationRef{eq:mm1} and using the relation in \EquationRef{eq:number_of_tight_muons}:
\begin{equation}
\left(\begin{array}{c} N_R \\ N_F \end{array}\right)&=
\frac{1}{\epsilon_R(1-\epsilon_F)-\epsilon_F(1-\epsilon_R)}
\begin{pmatrix}
 1- \epsilon_F & -\epsilon_F \\
\epsilon_R-1 & \epsilon_R  \\
\end{pmatrix}
\left(\begin{array}{c} N_T \\ N_L \end{array}\right)
\end{equation} 
Thus, the true number of ``fake'' muons which pass the signal selection is:
\begin{equation}
N_{\textrm tight}^{\textrm fake} = \epsilon_F N_F=\frac{\epsilon_{F}}{\epsilon_{R}-\epsilon_{F}}\left(\epsilon_{R}(N_{L}+N_{T})-N_{T}\right) \,
  \label{eq:mm5}
\end{equation}
which is expressed from measurable quantities only and can be calculated from the data.

The cut used to distinguish between loose and tight muons is the isolation cut, and the loose muons
are thus defined as passing the signal muon selection cuts, except the isolation cut. Tight muons correspond
to the baseline selection.

The ``real'' muon efficiency is found from MC simulation, because MC reproduces well efficiency of the isolation cut in data. The ``fake'' muon efficiency is found from a control region designed to have a high purity of ``fake'' muons. The region is defined in the same way as the signal selection without cuts on $E_T^{miss}$ and $m_T$ and following additional requirements:
\begin{itemize}
\item At least one jet with $p_T > 40$~GeV which does not overlap ($\Delta R > 0.2$)
with the selected muon.
\item Opening angle in the transverse plane between the muon and the $E_T^{miss}$, $\Delta\phi_{\mu,E_T^{miss}} < 0.5$.
\item No $Z$ candidate (any two muons with $80 < m_{\mu\mu} < 100$~GeV).
\item $d_0$ significance, $|d_0|/\sigma(d_0)$, greater than $1.5$.
\item $E_T^{miss} < 55$~GeV, ensuring that the control region does not overlap with the signal region.
\end{itemize}
This region is enhanced with ``fake'' muons, however, a significant ``real'' muon contamination is present in the region as well. This is why a ``real'' contribution predicted with MC modeling is subtracted.

The obtained efficiencies are shown in \FigureRef{fig:matrix_method_efficiencies}.
\begin{figure}[]
  \centering
  \includegraphics[width=0.45\textwidth]{Wprime/realEffNominal.eps}
  \includegraphics[width=0.45\textwidth]{Wprime/fakeEffNominal.eps}
  \caption{Efficiency of the ``real'' (left) and ``fake'' (right) muons as a function of muon $p_T$ used in the data-driven matrix method to estimate contribution of the multijet background to the signal selection.}
  \label{fig:matrix_method_efficiencies}
\end{figure}

Systematic uncertainty was estimated for both ``real'' and ``fake'' muon efficiencies.
The ``fake'' muon efficiency uncertainty was estimated by variation of the requirements for ``fake'' muon control region. These variations were:
% TODO rewrite list!!!
\begin{itemize}
\item Removing the $Z$ veto and $\Delta\phi_{\mu,\met}$ cuts.
\item Removing the $Z$ veto and $\Delta\phi_{\mu,\met}$ cuts, but tightening the $d_0$ significance cut to $2$.
\item Removing the $d_0$ significance cut. 
\item Using a tighter $d_0$ significance cut of $2$.
\item Removing the jet requirement.
\item Removing the jet requirement, but tightening the $d_0$ significance
cut to $2$.
\item Requiring $\met < 20~\GeV$.
\item Requiring $20 < \met < 55~\GeV$.
\end{itemize}
The ``fake'' efficiency was recalculated by using each if this requirement separately.
Effect on the ``fake'' muon efficiency is shown in \FigureRef{fig:matrix_method_systematics} (right).

Since the ``real'' muon efficiency is obtained from MC simulation one can use efficiency obtained with tag-and-probe method using muon pairs in the invariant mass window $80 < m_{\mu\mu} < 102~\GeV$ from $Z\to\mu\mu$ decays as a systematic uncertainty variation. 
Comparison of the efficiencies obtained with MC simulation and with tag-and-probe method are shown in \FigureRef{fig:matrix_method_systematics} (left).

\begin{figure}[]
  \centering
  \includegraphics[width=0.45\textwidth]{Wprime/realEffVariations.eps}
  \includegraphics[width=0.45\textwidth]{Wprime/fakeEffVariations.eps}
  \caption{Systematic variations of the ``real'' (left) and ``fake'' (right) muon efficiencies as a function of muon $p_T$.}
  \label{fig:matrix_method_systematics}
\end{figure}

An impact of the systematic variations of the efficiencies on the final $m_T$ spectrum of the multijet background will be discussed in \SectionRef{subsec:multijet_systematcs}.

%*******************************************************************************
% VALIDATION REGIONS
%*******************************************************************************

\subsection{Multijet validation region}

\toAsk[this subsection has to be carefully reviewed]
% definition of the validation region
To test data-driven prediction of the multijet background one want to find a region
where its contribution will be enhanced. Also region has to be kinematically 
close to the signal selection.
A validation region is defined in the same way as the signal selection but without the $E_T^{miss}$ and $m_T$ requirements. Also distribution with and without using isolation requirements are considered, 
which corresponds to tight and loose muon definitions used in the matrix method described above.

% plots used for multijet CR
The validation distributions of variables, used in the definition of the enhanced ``fake'' muon control region, are shown in \FigureRef{fig:muMMval1}.
Reasonable agreement within 10$\%$ for both tight and loose distributions is observed for $\Delta\phi_{\mu,E_T^{miss}}$ and muon $d_0$ significance.
The most obvious discrepancy is seen in the
distribution of the number of jets. The disagreement is likely mostly due to the modeling
of jet emission in the $W\to\mu\,\nu$ MC, where only one jet emission is included at the
matrix element level calculations.
This affects in principle the MC subtraction in the ``fake'' muon control region,
but discrepancy is present only for $N_\mathrm{jet}\geq2$, while control region is 
defined with requirement $N_\mathrm{jet}\geq1$ and thus this discrepancy is not 
strongly affected the ``fake'' muon efficiency calculations.
Furthermore, as a systematic variation of the ``fake'' muon efficiency was estimated with no $N_\mathrm{jet}$ cut at all and was found to be negligible which prove that discrepancy in
the $N_\mathrm{jet}$ is negligible as well.

% TODO think about some additional description here!
% TODO check in the Magnars thesis
Distribution of the $E_T^{miss}$, muon momentum and $m_T$ are shown in \FigureRef{fig:muMMval2}.
As one can note in the loose muon sample contribution of the multijet background are significantly enhanced with respect to the tight sample. 
Taking into account size of systematic uncertainties, discussed in \SectionRef{sec:wprimeSystematics}, data and background prediction agrees reasonably which proves validity of the data-driven jet background method.
The tight muon sample includes isolation requirement and thus contribution from multijet background is significantly reduced. Also because no $E_T^{miss}$ and $m_T$ are applied 
there are significantly more statistics available than in the signal region which allows 
to test overall background prediction. In general, reasonable agreement is observed. 

\toAsk[Is explanation above good enough?]

\toAsk[is ``reasonable agreement'' sounds strange for such plots?]

\toAsk[should I focus more of the shape comparison of tight and loose plots?]

\toDo[more discussion here?]

\begin{figure}[]
  \centering
  \includegraphics[width=0.49\textwidth]{Wprime/Njet40loose.eps}
  \includegraphics[width=0.49\textwidth]{Wprime/Njet40.eps}
  \includegraphics[width=0.49\textwidth]{Wprime/deltaPhiLoose.eps}
  \includegraphics[width=0.49\textwidth]{Wprime/deltaPhi.eps}
  \includegraphics[width=0.49\textwidth]{Wprime/d0sigLoose.eps}
  \includegraphics[width=0.49\textwidth]{Wprime/d0sig.eps}
  \caption{
  Distributions of the number of jets (top), $\Delta\phi_{\mu,\met}$ (middle), 
  and $d_0$ significance (bottom) in the inclusive loose (left) and tight (right) muon samples. 
  The distributions are considered before the $E_T^{miss}$ and $m_T$ cuts.
}
  \label{fig:muMMval1}
\end{figure}
\begin{figure}[]
  \centering
  \includegraphics[width=0.49\textwidth]{Wprime/METloose.eps}
  \includegraphics[width=0.49\textwidth]{Wprime/MET.eps}
  \includegraphics[width=0.49\textwidth]{Wprime/pTloose.eps}
  \includegraphics[width=0.49\textwidth]{Wprime/pT.eps}
  \includegraphics[width=0.49\textwidth]{Wprime/mTloose.eps}
  \includegraphics[width=0.49\textwidth]{Wprime/mT.eps}
  \caption{
  Distributions of the $E_T^{miss}$ (top), $p_T$ (middle), and $m_T$ (bottom)
  in the inclusive loose (left) and tight (right) muon samples. 
  The distributions are considered before the $E_T^{miss}$ and $m_T$ cuts.
}
  \label{fig:muMMval2}
\end{figure}


\subsection{Background extrapolation}
% TODO explain why it hurts to have huge statistical fluctuation for limit settings 
% TODO (it was somewhere in the note?)
% % from app_muonSmooth.tex:
% In a single-bin statistical analysis,
% it is straight forward to take into account the statistical uncertainty of the MC background
% estimates, as one can simply add it in quadrature to the systematic uncertainties on the
% background level in the search region. However, in a multi-bin analysis, results depend on
% the shape of the background distribution, and one should avoid propagating clearly unphysical features
% in this shape to the final results.

MC simulation of the top and diboson background processes were available from official ATLAS production only as inclusive samples,
thus they didn't provide enough statistics in the high-$m_T$ region.
Therefore these backgrounds were fitted in the low-$m_T$ region and extrapolated to obtain smooth description in the high-$m_T$ region.

% TODO this is from Marcus --> change phrasing!!!
The fit was done with functions that are commonly used to extrapolate the background, for example as in the $8~\TeV$ dilepton resonance search~\cite{Aad:2014cka}.
The first considered function is defined as:
\begin{equation}
 f(\mt) = e^{-a} m_\mathrm{T}^{b} m_\mathrm{T}^{c \log(m_\mathrm{T})}
  \label{eq:dijetfunc}
\end{equation}
The second one is:
\begin{equation}
 f(\mt) = \frac{a}{(m_\mathrm{T}+b)^{c}}
  \label{eq:powerlaw}
\end{equation}
These two functions were used to preform fits of the backgrounds in different ranges. 
The best fit according to the $\chi^{2}/N.d.o.f$ value has been used as a central value.
The systematic uncertainty is estimated as the envelope of all fits.
Statistical uncertainty of the fit was found to be negligible.

The starting range for the top background was in the range from $140$~GeV to $260$~GeV in steps of $20$~GeV. The end fit point varied from $600$~GeV to $900$~GeV in steps of $25$~GeV.
For diboson background these values were from $120$~GeV to $240$~GeV and from $500$~GeV to $700$~GeV, respectively, with the same step widths as for top.
The extrapolation was stitched to the background estimated by Monte Carlo at $\mt=600$~GeV in both cases.

Fits and appropriate systematic uncertainty estimates are shown in 
\FigureRef{fig:mu_extrapolate_top} for the top and in \FigureRef{fig:mu_extrapolate_diboson} for the diboson backgrounds.
\begin{figure}[!htb]
  \centering
  \includegraphics[width=0.49\textwidth]{Wprime/top_extrapolate_fits.eps}
  \includegraphics[width=0.49\textwidth]{Wprime/top_extrapolate.eps}
  \caption{Fit and extrapolation of the top background. Both the full set of individual
fits (left) and the resulting central value and uncertainty (right) are shown.}
  \label{fig:mu_extrapolate_top}
\end{figure}
\begin{figure}[!htb]
  \centering
  \includegraphics[width=0.49\textwidth]{Wprime/diboson_extrapolate_fits.eps}
  \includegraphics[width=0.49\textwidth]{Wprime/diboson_extrapolate.eps}
  \caption{Fit and extrapolation of the diboson background. Both the full set of individual
fits (left) and the resulting central value and uncertainty (right) are shown.}
  \label{fig:mu_extrapolate_diboson}
\end{figure}

The multijet data-driven background estimation suffers from large statistical fluctuation in the high-$m_T$ region, thus, the same fitting and extrapolation procedures are done as well.
% A simple power law fit is performed, which was found to be the most appropriate way according to the 8 TeV analysis~\cite{wprime_8TeV}:
% \begin{equation}
% \frac{dN}{d m_T} = a\, m_T^{-b}
% \end{equation}
Fits are done in the ranges of $150$--$300~\GeV$ and $200$--$300~\GeV$.
The extrapolation was stitched to the multijet background estimate $\mt=300$~GeV.

% TODO where it was stitched?

%*******************************************************************************
% SYSTEMATICS
%*******************************************************************************

\section{Systematic Uncertainties}
\label{sec:wprimeSystematics}
\subsection{Muon efficiency, resolution and scale}
The muon efficiency corrections from the MCP group are obtained using the tag-and-probe method
on $Z\to\mu\mu$ and $J/\psi\to\mu\mu$ decays in data~\cite{MCP13TeV}. Systematic uncertainties
are derived from variations of the tag-and-probe selection, background subtraction etc. following
the methodology documented in ref.~\cite{MCPrun1}.

The muon momentum corrections from the MCP are obtained by fitting certain correction constants 
to match the invariant mass distribution in $Z\to\mu\mu$ and $J/\psi\to\mu\mu$ decays in MC
to that observed in data~\cite{MCP13TeV}. The dependence of the muon momentum on the fit parameters
is given by a model where each parameter is associated to a certain source of potential data/MC disagreement.
Systematic uncertainties are derived from variations of the fit procedure, alignment studies etc. 
following the methodology documented in ref~\cite{MCPrun1}.

\subsection{Jet energy scale and resolution}
% TODO write something more (about jet energy scale)
The jet energy scale and resolution uncertainties enter the analysis through 
the $\met$ calculation, since the $\met$ is calculated using calibrated jets. 
The uncertainties for the jet energy scale and resolution are provided 
by the ATLAS JetEtMiss working group ~\cite{jet_calib_syst_13TeV, JESUncer13TeV}. 
% A reduced set of uncertainties with three nuisance parameters is chosen for the jet energy scale. This reduced set of
% nuisance parameters simplifies the correlations between the different sources of the jet energy scale uncertainty (JET\_GroupedNP\_1, JET\_GroupedNP\_2, JET\_GroupedNP\_3).
% Four scenarios of correlation models are provided by the JetEtMiss group. The final result of an analysis using the reduced set must not
% depend on a specific choice of correlation model. 
The jet energy scale uncertainty has been tested for different recommended scenarios 
and was found to be negligible for all of them.\\
No resolution smearing is applied in the default scenario. 
According to the working group recommendation, effect of the smearing has to be used as a systematic uncertainty.\\ The jet uncertainties are fully correlated between the electron and muon channel.

\subsection{Transverse missing energy scale and resolution}
The uncertainties for the $\met$ scale and resolution are provided by the JetEtMiss group~\cite{met2015_1}. They enter
the analysis through the soft term in the $\met$ calculation, 
which corresponds to the energy deposits in the calorimeter not associated with
any reconstructed physics objects (leptons, photons, jets).
The uncertainties cover differences between data and MC and are only applied to MC. 
The $\met$ uncertainties are fully correlated between the electron and muon channel. 
The jet, electron and muon energy/momentum uncertainties are affecting the $\met$
calculation. These uncertainties are propagated to the $\met$ calculation in the same way.
% TODO see comment from Oxana about sentence above!

\subsection{Background estimate uncertainty}

Uncertainty of the charged and neutral current Drell-Yan processes were estimated by variations of $\alpha_s$ and electroweak corrections as well as by using PDF error set and estimation of difference between alternative PDF sets.
The $\alpha_s$ influence was estimated by varying $\alpha_s$ by $\pm 0.0003$. This corresponds to the 90\% C.L uncertainty. The effect on the $W$ background when doing this variation
was $3\%$ at maximum and the effect is therefore neglected. 
The variation of the electroweak corrections was estimated to be larger than $3\%$ and was
taken into account during limit settings.
The PDF uncertainty of the CT14NNLO PDF is one of the main theory uncertainties
and it was calculated by using 90$\%$ CL PDF error set.
Uncertainty related to the choice of the PDF set to use was estimated by comparing
results with NNPDF3.0~\cite{Ball:2014uwa}.
The difference between CT14 and HERAPDF2.0 is not considered as the PDF does not include high Bjorken x data. 
% TODO fix above about x. comment from Oxana

Uncertainty of ``Top'' and ``Diboson'' backgrounds consists of the theoretical uncertainty on cross section and the high-$m_T$ extrapolation uncertainty. The former uncertainty affects the total background
prediction less than 3$\%$ and thus is neglected; the latter one becomes considerable at the high-$m_T$ region and is taken into account during limit settings.

The detailed description of the theoretical uncertainties on the MC cross section can be found in~\cite{Aaboud:2016zkn}.

\subsection{Trigger and luminosity}
Trigger systematic uncertainty is evaluated by the ATLAS trigger group and it is related to the trigger
efficiency of muons with different $\eta$ and $p_T$.
Luminosity uncertainty was estimated in the same way as described in \SectionRef{sec:lucid_performance}, however, the current analysis is using a preliminary luminosity
uncertainty equal to 5$\%$, which was obtained from the preliminary calibration of the luminosity scale done using data from vdM scans in August 2015.

\subsection{Multijet background}
\label{subsec:multijet_systematcs}

Systematic variations on the ``real'' and ``fake'' efficiency used in the matrix method were
described previously in \SectionRef{subsec:matrix_method}.
Multijet background have a very small contribution to the muon signal selection, as can be seen in 
\FigureRef{fig:muMMfinal} (left), where fraction of the multijet background to the total background
after the final selection is shown as function of $m_T$. Thus affect from the systematic
variations of the matrix method efficiencies on the total background is small as well, as shown in
\FigureRef{fig:muMMfinal} (right). 
Systematic uncertainty on the total background is at the level of 1 $\%$ for $m_T<4$~TeV and less than 2$\%$ for $m_T>4$~TeV.

\begin{figure}[!htb]
  \centering
  \includegraphics[width=0.49\textwidth]{Wprime/totalBGfrac.eps}
  \includegraphics[width=0.49\textwidth]{Wprime/totalBGsys.eps}
  \caption{The fraction that the fake muon background constitutes of the total background
as function of \mt\ (left) and the effect of systematics variations on the total background
level as function of \mt\ (right). The power law fits are used for the fake muon background above
$\mt=300~\GeV$. In the left plot, the fit range $150$--$300~\GeV$ is used, and in the right plot, the
range $150$--$300~\GeV$ corresponds to solid lines and $200$--$300~\GeV$ to dashed lines.}
  \label{fig:muMMfinal}
\end{figure}

\subsection{Summary}
\TableRef{tab:syst} lists various systematic uncertainty sources
and their size for the background and for the 2 or 4~TeV $\PWprime$ signal at a transverse mass equal to 2 and 4~TeV.
Uncertainties which do not have a value are either neglected or do not apply. 
All uncertainties below $3$\% have been neglected so far
since they do not affect the final result of the statistical analysis. 
The remaining experimental and theoretical systematics are applied to the background.
Only the experimental uncertainties are applied to the signal. 

\begin{table}
\begin{center}
\centering
\small
\begin{tabular}{l|cc}
\toprule
Source &  Background  &  Signal  \\
\midrule
Trigger &\syspair{3}{4} & \syspair{4}{4}\\
Lepton reconstruction  &\multirow{2}{*}{\syspair{5}{8}} & \multirow{2}{*}{\syspair{5}{7}}\\
and identification & & \\
Lepton isolation &\syspair{5}{5} & \syspair{5}{5}\\
Lepton momentum &\multirow{2}{*}{\syspair{3}{11}} & \multirow{2}{*}{\syspair{1}{4}}\\
scale and resolution & & \\
$E_T^{miss}$ resolution and scale &\syspair{<0.5}{<0.5} &\syspair{<0.5}{<0.5}\\
Jet energy resolution &\syspair{1}{2} &\syspair{<0.5}{<0.5}\\
\midrule
Multijet background & \syspair{1}{1} & {\sc n/a} ({\sc n/a})\\
Diboson \& top-quark bkg. &\syspair{5}{15} & {\sc n/a} ({\sc n/a})\\
PDF choice for DY &\syspair{<0.5}{1} & {\sc n/a} ({\sc n/a})\\
PDF variation for DY &\syspair{8}{12} & {\sc n/a} ({\sc n/a})\\
Electroweak corrections &\syspair{4}{6} & {\sc n/a} ({\sc n/a})\\
\midrule
Luminosity &\syspair{5}{5} &\syspair{5}{5}\\
\midrule
Total &\syspair{14}{25} & \syspair{9}{12}\\
\bottomrule
\end{tabular}
\end{center}
\caption{Systematic uncertainties on the expected number of events as evaluated at $m_T = $ 2 (4)~\TeV, both for signal events 
with a \wpssm\ mass of 2~(4)~\TeV\ and for background. Uncertainties estimated to have an impact
$< 3\%$ on the expected number of events in both channels and for all values of $m_T$ are not listed.
Uncertainties that are not applicable are denoted ``n/a''. \label{tab:syst}}
\end{table}

%*******************************************************************************
% SIGNAL REGION
%*******************************************************************************
\section{Signal Region}
\label{sec:wprimeSignalRegion}
The muon $\eta$, $\phi$, $p_T$, and $E_T^{miss}$ distributions in the signal region are shown in \FigureRef{fig:mu_results_etaphi} and \FigureRef{fig:mu_results_ptmet}. 
The dominant contribution to the signal region originates from W boson background.
No visible excess is observed, and good agreement between data and background estimate is observed.

The basis for the statistical analysis and the main distribution of interest 
is the transverse mass distributions which is shown in \FigureRef{fig:MT_mu_Wprime}.
The resonant $\PWprime$ signal overlaid on background prediction is shown as well.
As one can see from the ratio plot data is systematically above the total background prediction
in low-$m_T$ region but are within $\pm 1 \sigma$ uncertainty band, which is dominated by the $E_T^{miss}$ systematic uncertainty at low $m_T$.

\begin{figure}[]
  \centering
  \includegraphics[width=0.49\textwidth]{Wprime/muon_eta.eps}
  \includegraphics[width=0.49\textwidth]{Wprime/muon_phi.eps}
  \caption{
  Muon \eta\ (left) and $\phi$ (right) distributions after final selection. The uncertainty band in the ratio plot includes all systematic uncertainties which are included in the statistical analysis except the integrated luminosity uncertainty ($5\%$).
}
  \label{fig:mu_results_etaphi}
\end{figure}

\begin{figure}[]
  \centering
  \includegraphics[width=0.49\textwidth]{Wprime/muon_pT.eps}
  \includegraphics[width=0.49\textwidth]{Wprime/muon_MET.eps}
 \caption{
 Muon $p_T$ (left) and $E_T^{miss}$ (right) distributions after final selection. The uncertainty band in the ratio plot includes all systematic uncertainties which are included in the statistical analysis except the integrated luminosity uncertainty ($5\%$).
}
  \label{fig:mu_results_ptmet}
\end{figure}


\begin{figure}[]
  \centering
  \includegraphics[width=0.65\textwidth]{Wprime/muon_mT.eps}
  \caption{
  Muon $m_T$ distribution after final selection. 
  Shown is the total background estimate with resonant $\PWprime$ signal overlaid for various pole masses. 
  The uncertainty band in the ratio plot includes all systematic uncertainties which are included in the statistical analysis except the integrated luminosity uncertainty (5$\%$).}
  \label{fig:MT_mu_Wprime}
\end{figure}

\TableRef{tab:muBkgData}
shows the contributions of individual backgrounds as well as the total background
and the data in different $\mt$ regions. The quoted uncertainties include both systematic and
statistical uncertainties except the uncertainty on the integrated luminosity ($5\%$).
One can observe good agreement between data and total background prediction in all $\mt$ regions. Charged-current DY provides the dominant contribution to the high-$m_T$ region which is above 90$\%$ of the total background for $m_T>1$ TeV region. No events with $m_T > 3$ TeV are observed in data.

\begin{table}[]
  \centering
  \scriptsize
  \begin{tabular}{|c|c|c|c|c|c|c|c|}
    
    \multirow{2}{*}{Process} & \multicolumn{7}{c|}{$m_T$ [\GeV]} \\
& $110$--$150$ & $150$--$200$ & $200$--$400$ & $400$--$600$ & $600$--$1000$ & $1000$--$3000$ & $3000$--$7000$ \\ \hline 
$W$ & $98100\pm10000$ & $21000\pm2000$ & $7700\pm400$ & $476\pm30$ & $110\pm9$ & $13.0\pm1.2$ & $0.051\pm0.010$ \\ 
Top & $9900\pm700$ & $5410\pm340$ & $3090\pm140$ & $120\pm6$ & $13\pm5$ & $0.44\pm0.32$ & $0.00005\pm0.00030$ \\ 
$Z/\gamma^*$ & $7700\pm1000$ & $2130\pm250$ & $840\pm70$ & $37\pm4$ & $7.6\pm1.8$ & $0.64\pm0.06$ & $0.0037\pm0.0007$ \\ 
Diboson & $1140\pm80$ & $588\pm33$ & $326\pm14$ & $20.6\pm1.2$ & $3.8\pm2.1$ & $0.4\pm0.4$ & $0.002\pm0.008$ \\ 
Multi-jet & $1350\pm40$ & $551\pm23$ & $180\pm10$ & $5.6\pm1.0$ & $0.85\pm0.21$ & $0.078\pm0.028$ & $0.00038\pm0.00022$ \\ \hline 
Total SM & $118000\pm12000$ & $29700\pm2600$ & $12100\pm600$ & $660\pm40$ & $135\pm11$ & $14.6\pm1.4$ & $0.058\pm0.013$ \\ \hline 
Data & $131672$ & $31980$ & $12393$ & $631$ & $121$ & $15$ & $0$ \\ 
\end{tabular}
\caption{Contributions of individual backgrounds with uncertainties for different $m_T$ regions.
The uncertainties include both statistical and systematic uncertainty, and all weights are included
so that the total background level can be compared to data. The systematic uncertainty includes all systematic 
uncertainties which are included in the statistical analysis except the uncertainty
on the integrated luminosity ($5\%$). For the multi-jet background, only statistical uncertainty is shown,
since the multi-jet systematics are not included in the statistical analysis.}
\label{tab:muBkgData}
\end{table}

\toAsk[something more here?]

%*******************************************************************************
% XSEC and MASS LIMITS
%*******************************************************************************

\section{Cross section and mass limits}
% TODO say that Bayesian approach is used...
\toDo[some text here]


% TODO describe look for deviation
To search for a $\PWprime$ signal-like excess in the data a log likelihood ratio test is performed using RooStat~\cite{RooStat_project} framework.
The likelihood function is constructed as the product of Poisson probabilities of over all $m_T$ bins in the search region.
The effect of systematic uncertainties are described by nuisance parameters in the likelihood function.

One of the needed component for the statistical analysis to calculate the number of expected events is total efficiency (product of acceptance and reconstruction efficiency) 
of the signal selection to reconstruct $\PWprime$ decay, which is shown in \FigureRef{fig:AccEff_mu}.

\toDo[explain shape of the total efficiency curve!]

\begin{figure}[]
  \centering
  \includegraphics[width=0.65\textwidth]{Wprime/acceptance.eps}
  \caption{
%   Total signal acceptance times efficiency versus SMM \wp\ pole mass for the SSM \wp\ model in the muon channel.
  }
  \label{fig:AccEff_mu}
\end{figure}

Statistical analysis demonstrates that no excess larger than 2$\sigma$ is observed.
This is why upper limit on the cross section for the production of $\PWprime$ times branching ratio has been set. 
A Bayesian approach have been used for the limit settings and limits were calculated with help of the Bayesian Analysis Toolkit~\cite{BAT}.

\toAsk[is it OK to not explain in the detail about limit setting procedure?]

\toDo[add something here???]

Upper limits are set on the cross-section times branching ratio, $\PWprime \rightarrow \Plepton\nu$, at 95\% C.L. and mass 
limits are extracted using the relationship between the theoretical $\PWprime$ cross section and pole mass. Limits are presented for the electron, muon, 
and combined lepton channels in \TableRef{tab:limits_mass_wp}, \FigureRef{fig:wprime_limits} and \FigureRef{fig:wprime_limits_combined}.

\toAsk[which conclution whuols I made? comparison with 8 TeV?]

\begin{figure}[]
  \centering
\includegraphics[width=0.49\textwidth]{Wprime/Limit_xsec_wprime_m_Sys.eps}
\includegraphics[width=0.49\textwidth]{Wprime/Limit_xsec_wprime_e_Sys.eps}
\caption{$\PWprime$ cross section limit results for the muon (left) and electron (right) channels.}
\label{fig:wprime_limits}
\end{figure}


\begin{figure}[]
  \centering
\includegraphics[width=0.65\textwidth]{Wprime/Limit_xsec_wprime_comb_Sys.eps}
\caption{$\PWprime$ cross section limit results for combined both muon and electron channels.}
\label{fig:wprime_limits_combined}
\end{figure}


\begin{table}[]
  \centering
  \begin{tabular}{c|cc}
    \hline
    \hline
    &  \multicolumn{2}{c}{$m_{\PWprime}$ lower limit [\TeV]} \\
    Decay     &  Expected & Observed \\
    \hline
    \wpe  & 3.99 & 3.96 \\
    \wpmu & 3.72 & 3.56 \\
    \wpl  & 4.18 & 4.07 \\
    \hline
    \hline
  \end{tabular}
  \caption{Expected and observed 95\% CL lower limit on the \wpssm\ mass in the electron and muon channels and their combination.}
  \label{tab:limits_mass_wp}
\end{table}



%*******************************************************************************
%*******************************************************************************
% MONO-W STUDY
%*******************************************************************************
%*******************************************************************************

% Sources:
% 
% Dark Matter Benchmark Models for Early LHC Run-2 Searches:
% Report of the ATLAS/CMS Dark Matter Forum (July 6, 2015)
% http://arxiv.org/pdf/1507.00966.pdf
% 

\section{Search for dark matter pair-production with a leptonically decaying W boson}
\label{chap:monoW}

\subsection{Introduction}

Beside $\PWprime$ model discussed above there are other BSM models 
with lepton plus $E_{T}^{miss}$ final state signature.
One of such model is associated 
production of pair of weakly interacting massive particles (WIMP)
with SM W boson. WIMPs are one of possible dark matter candidates (DM).

% Another possible BSM signature which can be easily tested in signal region is
% associated production of pair of weakly interacting massive particles (WIMP), 
% which are candidates for dark matter (DM) particles, with SM W. 
Because WIMPs don't interact strongly or electromagnetic they will most probably escape from detector the same way as neutrino
from leptonic W decay and will contribute to $E_{T}^{miss}$ of the event.

In this section qualitative study based on using models recommended by Dark Matter Forum (link?) are presented 
which address question about sensitivity of signal selection presented above to such kind of models.


% While searching for $\PWprime$ boson one can naivly expect that distributions of 
% transferred energy of decay products will look quite similar, which
% means that distributions of transfer lepton momentum and missing energy will 
% look the same as well. But if we consider pair-production of dark matter 
% particles associated with W we expect 

\subsection{Theoretical models}

% TODO rephrase first setence
There are plethora of models which try to introduce and explain DM as a possible particles which can be produced at LHC. 
But all these models can be classified to the three distinct classes: DM Effective Field Theories (EFT), Simplified DM models 
and Complete DM models. EFT approach allows to describe the DM-SM interactions mediated by all 
kinematically inaccesible particles in a universal way. 
It allows to derive stringent bounds on the ``new physics'' scale $\Lambda$. 
Simplified models are charachterized by the most important state mediating the DM interaction with SM. Unlike EFT approach,
simplified models are able to describe correctly the full kinematics of DM production, because they resolve the EFT contact interaction in single-particle 
s- or t-channel exchange. Complete DM models allows to describe correlation between observables~\cite{arXiv:1506.03116}.

Main focus of current study will stay on EFT and Simplified DM models with W boson and DM particles produced in the final state. 
In simplfied model W is produced as initial state radiation from one of incoming quarks as shown at \FigureRef{fig:feynMonoWSimple}. 
In EFT approach W can be produces as final state particle (\FigureRef{fig:feynMonoWEFT}) or initial state radiation (\FigureRef{fig:feynMonoWEFT}).

% TODO: explanation of simplified and EFT models (take from DM forum report or fro Bell's papers).


% and mediator connect SM interaction from one side and DM 

% TODO find reference from DM forum report which explain idea of simlified models 
% (that we introduce additionla mediator which on one side interact with SM particles and on other side - with dark matter particles).
% same for EFT models. But for now skip this part.

% TODO describe possible type of interection for mono-W models.

% TODO figure out what does it mean constructive and destructive intereference for DX models?

% TODO explanation for contact interaction is needed as well, ha-ha-ha...

% TODO say that our main focus is on simplified models because they are recommended by Dark Matter forum.

\begin{figure}[]

\centering
\begin{subfigure}{.5\textwidth}
  \centering
  \includegraphics[width=0.3\textheight]{monoW/simplifiedDM_diagram.pdf}
\end{subfigure}%
\begin{subfigure}{.5\textwidth}
  \centering
  \includegraphics[width=0.3\textheight]{monoW/simplified_tChannel_model_diag.pdf}
\end{subfigure}
  \caption{Feynman diagrams of production of dark matter pairs ($\chi\overline{\chi}$) associated with a W boson in simplified model 
	   in s-channel (left) and t-channel (right) scenarious.}
  \label{fig:feynMonoWSimple}
\end{figure}

% TODO change y-axis title. because it is normalized distribution!!!

\begin{figure}[]

\centering
\begin{subfigure}{.5\textwidth}
  \centering
  \includegraphics[width=0.3\textheight]{monoW/WWxx_diagramm_v2.pdf}
\end{subfigure}%
\begin{subfigure}{.5\textwidth}
  \centering
  \includegraphics[width=0.3\textheight]{monoW/EFT_D5_model_diag_v3.pdf}
\end{subfigure}
  \caption{Representative diagrams for production of dark matter pairs ($\chi\overline{\chi}$) associated with a W boson in models where
dark matter interacts directly with W boson (left) or with quarks (right).}
  \label{fig:feynMonoWEFT}
\end{figure}

Current study cover both s- and t-channel simplified models with different mass of $Z'$ mediator and with different mass of DM particles.
EFT approach are represented by two models which correspond to diagrams which are shown at \FigureRef{fig:feynMonoWEFT}. 
They are charachterized by energy scale of ``new physics'' $\Lambda$ (or effective mass $M^{*}$) as well as mass of produced DM particles.
Both models assume vector type of interaction between DM and SM sectors. For D52 model (which correspond to the right diagram in  \FigureRef{fig:feynMonoWEFT}) 
we consider only constructive interferance with SM.

\subsection{Sensitivity studies}

Main idea of the study is to understand kinematics of DM models in final state with one lepton and missing energy and to estimate sensitivity 
of signal selection to such kind of models.
Signal selection are designed with focus of high-$m_{T}$ region in order to get rid of dominant SM W background.
Kinematic distribution of DM models were studied to evaluate contribution to the signal selection and to make comparison
with signal from $\PWprime$ model.
It's worth to mention that due to the fact that mass of DM particles doesn't affect kinematic distribution of the event 
all simulated DM samples used are with mass of DM particle was set to 1 GeV.
(should some plots to be shown to proof this???).

In \FigureRef{fig:kinematicsSChannel} normalized kinematic distributions of transverse mass of lepton and $E_{T}^{miss}$ of the event
and $E_{T}^{miss}$ of the event for s-channel simplified model as well as EFT models with comparison with $\PWprime$ model are shown.
First distinguishable charachteristic of all DM models is that there is no clear peak structure in any kinematic distribution, 
which is expected because transverse missing energy is formed by
neutrino from W boson decay and two DM particles which are independent between each other. 
Second feature of DM models that main contribution are tends to be in low-$m_{T}$ region.
Especially it concern simplified models, for which dominant contribution is outside of signal region for any parameter of mediator mass.
But with increasing of mediator's mass $m_{T}$ distribution tends to become more flat and moves towards high-$m_{T}$ region.
Also with increasing mediator mass cross section of process significantly decreasing (see \TableRef{tab:TriggerDetails}) 
and for mass equal 10 TeV cross section is a few order of magnitude lower than background (need proof, find plot ???) which 
makes signal selection not sensitive for s-channel simplified DM model.

% TODO reprhase it. It not visible from table - only from non-normalized plot. Plot should be mono-W with W backgound only? Or I should also include other backgounds?

\begin{figure}[]

\begin{subfigure}{.5\textwidth}
  \centering
  \includegraphics[width=\textwidth]{monoW/mT_kinemComparison_simplS_EFT_Wprime.png}
\end{subfigure}%
% \begin{subfigure}{.333\textwidth}
%   \centering
%   \includegraphics[width=0.25\textheight]{monoW/pT_kinemComparison_simplS_EFT_Wprime.png}
% \end{subfigure}
\begin{subfigure}{.5\textwidth}
  \centering
  \includegraphics[width=\textwidth]{monoW/EtMiss_kinemComparison_simplS_EFT_Wprime.png}
\end{subfigure}

% \includegraphics[width=0.75\textheight]{monoW/kinematics_simplifiedSChannel_EFT_Wprime.png}
\caption{Normalized kinematic distributions of transverse mass (left) and transverse missing energy in the event (right) of simplified model in s-channel as well as EFT models compared to $\PWprime$ distribution.}
  \label{fig:kinematicsSChannel}
\end{figure}

In \FigureRef{fig:scaledKin} transverse mass distribution are show for s- and t-channels simplified models as well as EFT and $\PWprime$ signals with comparison to SM W background 
scaled to according cross process section. Distribution for t-channel model looks almost identical to one for s-channel. 
But cross sections for t-channel processes are for one-two order of magnitude lower compared to s-channel (see \TableRef{tab:TriggerDetails}) which leads to conclution
that signal selection even less sensitive for t-channel simplified model. Aslo SM W backround are for a few orders higher in all range of $m_{T}$ than any sample of simplified model.
For D52 EFT model excess over SM W backgound can be seen in high-$m_{T}$ region. While for WW$\chi\chi$ EFT model it is not the case, because cross section is very low.

% TODO us or D5 or D52. harmonize stuff!!!

\begin{figure}[]
\begin{subfigure}{.5\textwidth}
  \centering
  \includegraphics[width=\textwidth]{monoW/dm_final_S_vs_T_channel_pad1.png} 
\end{subfigure}%
% \begin{subfigure}{.33\textwidth}
%   \centering
%   \includegraphics[width=0.95\textwidth]{monoW/pT_kinemComparison_simplT_EFT_Wprime.png}
% \end{subfigure}
\begin{subfigure}{.5\textwidth}
  \centering
  \includegraphics[width=\textwidth]{monoW/dm_final_EFT_vs_SMW_pad1.png}
\end{subfigure}
% \includegraphics[width=0.75\textheight]{monoW/kinematics_simplifiedTChannel_EFT_Wprime.png}
\caption{Kinematic distributions of transverse mass of simplified model in s- and t-channels (left) 
and EFT models with $\PWprime$ signal (right) in comparison with SM W background scaled to according process cross section.}
  \label{fig:scaledKin}
\end{figure}

Transverse mass distribution for both EFT models looks similar and majority (???) of the signal lays in the signal region. 
% Cross section for both processes strongly depends on energy scale of new physics $\Lambda$ (or effective mediator mass for D52 model).
% Dependence of cross section for WW$\chi\chi$ model versus energy scale is shown in \FigureRef{fig:lambdaScan}. 

% TODO describe what is D52 model. That it is vector interaction constructive interference

\begin{table}[]
  \begin{tabular}{r|c|c|c}
    Model 	& Channel 	  & Parameters	    & Cross section, [nb] \\
    \midrule
    Simplified  & s-channel	  & $M_{Z'}$=1TeV    & .05181 \\
		&		  & $M_{Z'}$=100GeV  & .001989 \\
		&		  & $M_{Z'}$=10TeV   & 7.5E-11 \\
		& t-channel	  & $M_{Z'}$=10GeV   & .0018895 \\
		&		  & $M_{Z'}$=100GeV  & 9.2E-5 \\
		&		  & $M_{Z'}$=2TeV    & 4.874E-8 \\
    \midrule
%     EFT 	& WW$\chi\chi$	  & $m_{\chi}$=1GeV; $\Lambda$=3TeV    & 3.6E-10 \\
% 		& D52		  & $m_{\chi}$=1GeV; $M^{*}$=1TeV	& 4.4E-4 \\
EFT 	& WW$\chi\chi$	  & $\Lambda$=3TeV    & 3.6E-10 \\
	& D52		  & $M^{*}$=1TeV	& 4.4E-4 \\
    \midrule	
$\PWprime$ 	& 	  & $m_{\PWprime} =$ 2 GeV   & 1.1E-4 \\    
		%     Wprime ($m_{\PWprime}$=2TeV) & 1.1E-4 \\
  \end{tabular}
  \caption{Mono-W cross section for different theoretical models.}
  \label{tab:TriggerDetails}
\end{table}


\begin{figure}[]
 \includegraphics[width=0.6\textheight]{monoW/WWxx_Wlv_DM1_LambdaScan.pdf}
  \caption{Cross section of WW$\chi\chi$ EFT model as a function of energy scale of new physics, $\Lambda$.}
  \label{fig:lambdaScan}
\end{figure}

\subsection{Validity of EFT approach and summary}

Cross section for both EFT processes strongly depends on energy scale of new physics $\Lambda$ (or effective mediator mass $M^{*}$ for D52 model).
Dependence of cross section for WW$\chi\chi$ model versus energy scale is shown in \FigureRef{fig:lambdaScan}.
In order for WW$\chi\chi$ process have sizeable cross section compared with $\PWprime$ ($M_{\PWprime}$=2TeV) model which is used as a reference in this study, 
energy scale of new physics $\Lambda$ has to be of order 200-300 GeV. But according to ~\cite{arXiv:1512.00476}: 
``The EFT approximation is valid when the momentum transfer in a given
process of interest is much smaller than the mass of the mediating
particle. For momentum transfer larger than or comparable to
$\Lambda$, the EFT description will break down.''
Moment transfer for 13 TeV collisions at LHC correspond to scale of few TeV, 
which mean that $\Lambda$ has to be of order of TeV in order for model to be valid.

% TODO cross check this statement with Dark matter report...
% http://arxiv.org/pdf/1507.00966.pdf
Also for WW$\chi\chi$ model there is no straightforward way to compare the
results with non-collider searches for DM
which make this model less appealing comparing with other models [add non-collider DM production diagram].

D5 constructive (D52) model violate weak gauge invariance as described at ~\cite{arXiv:1503.07874}.
It leads to spurious cross section enhancements at LHC energies. It is not recommended to be used anymore by Dark Matter Forum (some link???).

\subsection{Conclusion}

Transverse mass and $E_{T}^{miss}$ for all presented DM models with comparison with $\PWprime$ ($M_{W'}$ = 2TeV) signal are shown at \FigureRef{fig:scaledKin}.
All distribution are scaled according to the cross section of the process. It's clearly seen that simplified models tends to contribute to low-$m_{T}$ region
outside of the signal selection. With increasing of mediator mass $Z'$ cross section drops significantly and become indistinguishable with SM W background.
On other hand EFT DM samples contribute to high-$m_{T}$ region but as desribed above D52 model has physicaly unmotivated high cross section and WW$\chi\chi$ 
significantly smaller comparing with $\PWprime$ signal for physicaly motivated values of scale of new physics $\Lambda$. Which leads to the conclusion that
lepton and $E_{T}^{miss}$ signal selection is not sensitive for DM searches.



Similar studies was done by Bell and collaborators~\cite{arXiv:1512.00476}, where authors estimated approximate upper limit on for 3000 fb$^{-1}$. 
At \FigureRef{fig:bellExclLim} exclusion limit
as a function of mass of DM particle and mass of DM-SM mediator $Z'$ is shown. One can notice that obtained limits for mono-lepton channel are significantly 
worse than from all other channel, especially than from di-jet analysis. Conclusion of authors are identical to the conclusion of this study that mono-lepton channel
are not sensitive enough for DM searches and is significantly worse comparing to all other hadronic channels.

\begin{figure}[]
 \includegraphics[width=0.8\textwidth]{monoW/schan1.pdf}
  \caption{Exclusion limit for the s-channel $Z'$ model as a function of mass of dark matter particle, $m_{\chi}$, 
  and mass of DM-SM mediator, $m_{Z'}$, reported in~\cite{arXiv:1512.00476}.
  Exclusions are shown as shaded regions for LUX and for mono-jet and di-jets at 8 TeV, 
  and the reaches are shown for the mono lepton and mono fat jet searches at 14 TeV 3000 fb$^{-1}$.}
  \label{fig:bellExclLim}
\end{figure}

\section{Outlook}
\label{sec:wprimeConclusion}
