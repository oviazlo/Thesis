

\chapter{Search for new charged bosons in final states with one muon and missing transverse energy}
\label{chap:Wprime}

% ***Explanation of my personal contribution to the analysis.***
% TODO this is not a real text to be used - just a plan of what I want to write,,,
% \begin{itemize}
%  \item MC simultaion
%  \item second analysis of muon channel
%  \begin{itemize}
%   \item comparison of results with Magnar
%  \end{itemize}
%  \item mono-W investigation
% \end{itemize}


% PLAN
% <What this chapter about - short intro to Wprime analysis>
% <What was my personal contribution to the analysis> 


This chapter describe search for new charged boson (namely $\PWprime$) in final state with one lepton and missing transverse energy ($E_T^{miss}$).
Search was done with first $\sqrt{s}$ = 13 TeV data collected in 2015 by ATLAS with corresponded luminosity of 3.2 fb$^{-1}$.
% TODO describe more about the analysis.
% Analysis focuses on high-p$_T$ 

% Key features of the analysis.
One can highlight three key features of the analysis:
\begin{itemize}
 \item Modeling of the background prediction. 
 Largest part of the background originates from the charged-current Drell-Yan process and the analysis selection tests it up to the few TeV.
 Thus it is crucially important to use latest and most precise high-order calculations and corrections available at a time.
 \item Lepton selection. Due to topologically simple event selection (one isolated lepton and missing energy) 
 this analysis use most energetic lepton candidates available at 13 TeV center-of-mass energy collisions.
 This is why reconstruction efficiency and momentum resolution of high-$p_T$ leptons are factors of special interest for the analysis.
 \item Missing transverse energy, $E_T^{miss}$. Along with the lepton momentum, $E_T^{miss}$ is used for transverse invariant mass, $m_T$, calculations,
 which is the signal discriminant and search variable in the analysis. In order to $m_T$ be most precisely reconstructed and modelled one need
 to have not only good lepton momentum resolution but also good understanding of the events missing $E_T$.
\end{itemize}




This analysis have been made public as a conference note~\cite{ATLAS-CONF-2015-063} with early reasults and later as a paper~\cite{Aaboud:2016zkn} where final results were presented.

My personal contribution to the analysis can be concluded in following three parts.
I performed complete analysis in muon channel. Signal selection, control and validation plots, estimation of systematics and other.
This part was done in parrallel with other collaborator to make sure that results are robust and are the same from two independent analyses and there are no mistaken done.

% TODO discuss with Monika, should I include it or not. If yes - how should I argument about why it was not use and how to explain differnce comparing to inclusive sample?
Secondly I was investigating ways to decrease systematics caused by limited statistic of inclusive diboson and ttbar backgound samples at high-$m_{T}$ region.
One option was to generate lepton-neutrino mass-binned samples. After investigation it was found out that these samples doesn't populate high-$m_{T}$ region
good enough, so it was decided to develop unique (first ever?) samples in bins of lepton-$E_T^{miss} $ $m_T$.

Third part was related to investigation of use another models which can be potentially sensitive in signal selection.
Main focus of this investigation was mono-W Dark Matter (DM) models, where pair of DM particles candidates are produced in final state in association with SM W boson.
Two sets of model were considered: simplified and Effective Field Theory models [links???]. 

% TODO discuss with Monika - how should I position analysis - as a dedicated search of Wprime or as more model-independent search?

\section{Motivation}
\label{sec:wprimeIntro}



\section{$\PWprime$ signal}
\label{sec:wprimeSignal} 

\section{Background processes}
\label{sec:wprimeBackgrounds}

\begin{table}[ht]
  \begin{center}
    \begin{tabular}{l|c|c|c}

      \hline
Process &  Generator&  PDF set & Normalization \\
&  + fragmentation/ &  & based on \\
&  hadronization & &\\
\hline\hline
&   &   \multirow{4}{*}{CT14NNLO~\cite{Dulat:2015mca}} & NNLO QCD \\
$W +$ jets, & \powhegbox\ v2~\cite{Alioli:2010xd} & &  with \vrap~\cite{vrap}, \\
$Z/\gamma^* +$ jets & + {\scshape pythia-8.186}~\cite{pythia8}  & &  NLO QED \\
 & & &  with \mcsanc~\cite{Bardin:2012jk,Bondarenko:2013nu} \\
\hline
\ttbar, t-channel $t$, & \powhegbox\ & \multirow{2}{*}{CT10} & \multirow{2}{*}{NLO QCD} \\
s-channel $Wt$ & + {\scshape pythia-6.428}~\cite{Pythia} & &  \\
\hline
\multirow{2}{*}{$WW, WZ, ZZ$} & \multirow{2}{*}{{\scshape sherpa-2.1.1}~\cite{Sherpa}} & \multirow{2}{*}{CT10} & \multirow{2}{*}{???} \\
 & & &  \\
\hline
\hline
\multirow{2}{*}{$W^\prime \rightarrow \Plepton \nu$} & \multirow{2}{*}{{\scshape pythia-8.183}} &   \multirow{2}{*}{NNPDF2.3 LO} & NNLO QCD \\
& & &  with \vrap, \\
\hline
\end{tabular}
\end{center}
  \caption{List of MC generated samples used for background prediction. 
  The used MC generator, PDF set and order of cross section calculations used for the normalization are listed for each sample.
  }
\label{tab:MC_cross}
\end{table}

\section{Event and lepton selection}
\label{sec:wprimeSelection}

\subsection{Event selection}
\subsection{Lepton selection}
\subsection{Definition of the missing transverse energy}

\section{Data driven multijet background estimate}
\label{sec:wprimeMultijetBackground}
\subsection{The matrix method}
\subsection{The real muon efficiency}
\subsection{The fake muon efficiency}
\subsection{Results}

\section{Signal Region}
\label{sec:wprimeSignalRegion}

\section{Systematic Uncertainties}
\label{sec:wprimeSystematics}
\subsection{Muon efficiency, resolution and scale}
\subsection{Jet energy scale and resolution}
\subsection{Transverse missing energy scale and resolution}
\subsection{Background estimate uncertainty}
\subsection{Systematic uncertainty on the signal}

\section{Cross section and mass limits}
\subsection{Statistical analysis for discovery and exclusion}
\subsection{$\PWprime$ cross section and mass limits}



% Sources:
% 
% Dark Matter Benchmark Models for Early LHC Run-2 Searches:
% Report of the ATLAS/CMS Dark Matter Forum (July 6, 2015)
% http://arxiv.org/pdf/1507.00966.pdf
% 


\section{Search for dark matter pair-production with a leptonically decaying W boson}
\label{chap:monoW}

\subsection{Introduction}

Beside $\PWprime$ model discussed above there are other BSM models 
with lepton plus $E_{T}^{miss}$ final state signature.
One of such model is associated 
production of pair of weakly interacting massive particles (WIMP)
with SM W boson. WIMPs are one of possible dark matter candidates (DM).

% Another possible BSM signature which can be easily tested in signal region is
% associated production of pair of weakly interacting massive particles (WIMP), 
% which are candidates for dark matter (DM) particles, with SM W. 
Because WIMPs don't interact strongly or electromagnetic they will most probably escape from detector the same way as neutrino
from leptonic W decay and will contribute to $E_{T}^{miss}$ of the event.

In this section qualitative study based on using models recommended by Dark Matter Forum (link?) are presented 
which address question about sensitivity of signal selection presented above to such kind of models.


% While searching for $\PWprime$ boson one can naivly expect that distributions of 
% transferred energy of decay products will look quite similar, which
% means that distributions of transfer lepton momentum and missing energy will 
% look the same as well. But if we consider pair-production of dark matter 
% particles associated with W we expect 

\subsection{Theoretical models}

% TODO rephrase first setence
There are plethora of models which try to introduce and explain DM as a possible particles which can be produced at LHC. 
But all these models can be classified to the three distinct classes: DM Effective Field Theories (EFT), Simplified DM models 
and Complete DM models. EFT approach allows to describe the DM-SM interactions mediated by all 
kinematically inaccesible particles in a universal way. 
It allows to derive stringent bounds on the ``new physics'' scale $\Lambda$. 
Simplified models are charachterized by the most important state mediating the DM interaction with SM. Unlike EFT approach,
simplified models are able to describe correctly the full kinematics of DM production, because they resolve the EFT contact interaction in single-particle 
s- or t-channel exchange. Complete DM models allows to describe correlation between observables~\cite{arXiv:1506.03116}.

Main focus of current study will stay on EFT and Simplified DM models with W boson and DM particles produced in the final state. 
In simplfied model W is produced as initial state radiation from one of incoming quarks as shown at \FigureRef{fig:feynMonoWSimple}. 
In EFT approach W can be produces as final state particle (\FigureRef{fig:feynMonoWEFT}) or initial state radiation (\FigureRef{fig:feynMonoWEFT}).

% TODO: explanation of simplified and EFT models (take from DM forum report or fro Bell's papers).


% and mediator connect SM interaction from one side and DM 

% TODO find reference from DM forum report which explain idea of simlified models 
% (that we introduce additionla mediator which on one side interact with SM particles and on other side - with dark matter particles).
% same for EFT models. But for now skip this part.

% TODO describe possible type of interection for mono-W models.

% TODO figure out what does it mean constructive and destructive intereference for DX models?

% TODO explanation for contact interaction is needed as well, ha-ha-ha...

% TODO say that our main focus is on simplified models because they are recommended by Dark Matter forum.

\begin{figure}[hb]

\centering
\begin{subfigure}{.5\textwidth}
  \centering
  \includegraphics[width=0.3\textheight]{monoW/simplifiedDM_diagram.pdf}
\end{subfigure}%
\begin{subfigure}{.5\textwidth}
  \centering
  \includegraphics[width=0.3\textheight]{monoW/simplified_tChannel_model_diag.pdf}
\end{subfigure}
  \caption{Feynman diagrams of production of dark matter pairs ($\chi\overline{\chi}$) associated with a W boson in simplified model 
	   in s-channel (left) and t-channel (right) scenarious.}
  \label{fig:feynMonoWSimple}
\end{figure}

% TODO change y-axis title. because it is normalized distribution!!!

\begin{figure}[hb]

\centering
\begin{subfigure}{.5\textwidth}
  \centering
  \includegraphics[width=0.3\textheight]{monoW/WWxx_diagramm_v2.pdf}
\end{subfigure}%
\begin{subfigure}{.5\textwidth}
  \centering
  \includegraphics[width=0.3\textheight]{monoW/EFT_D5_model_diag_v3.pdf}
\end{subfigure}
  \caption{Representative diagrams for production of dark matter pairs ($\chi\overline{\chi}$) associated with a W boson in models where
dark matter interacts directly with W boson (left) or with quarks (right).}
  \label{fig:feynMonoWEFT}
\end{figure}

Current study cover both s- and t-channel simplified models with different mass of $Z'$ mediator and with different mass of DM particles.
EFT approach are represented by two models which correspond to diagrams which are shown at \FigureRef{fig:feynMonoWEFT}. 
They are charachterized by energy scale of ``new physics'' $\Lambda$ (or effective mass $M^{*}$) as well as mass of produced DM particles.
Both models assume vector type of interaction between DM and SM sectors. For D52 model (which correspond to the right diagram in  \FigureRef{fig:feynMonoWEFT}) 
we consider only constructive interferance with SM.

\subsection{Sensitivity studies}

Main idea of the study is to understand kinematics of DM models in final state with one lepton and missing energy and to estimate sensitivity 
of signal selection to such kind of models.
Signal selection are designed with focus of high-$m_{T}$ region in order to get rid of dominant SM W background.
Kinematic distribution of DM models were studied to evaluate contribution to the signal selection and to make comparison
with signal from $\PWprime$ model.
It's worth to mention that due to the fact that mass of DM particles doesn't affect kinematic distribution of the event 
all simulated DM samples used are with mass of DM particle was set to 1 GeV.
(should some plots to be shown to proof this???).

In \FigureRef{fig:kinematicsSChannel} normalized kinematic distributions of transverse mass of lepton and $E_{T}^{miss}$ of the event
and $E_{T}^{miss}$ of the event for s-channel simplified model as well as EFT models with comparison with $\PWprime$ model are shown.
First distinguishable charachteristic of all DM models is that there is no clear peak structure in any kinematic distribution, 
which is expected because transverse missing energy is formed by
neutrino from W boson decay and two DM particles which are independent between each other. 
Second feature of DM models that main contribution are tends to be in low-$m_{T}$ region.
Especially it concern simplified models, for which dominant contribution is outside of signal region for any parameter of mediator mass.
But with increasing of mediator's mass $m_{T}$ distribution tends to become more flat and moves towards high-$m_{T}$ region.
Also with increasing mediator mass cross section of process significantly decreasing (see \TableRef{tab:TriggerDetails}) 
and for mass equal 10 TeV cross section is a few order of magnitude lower than background (need proof, find plot ???) which 
makes signal selection not sensitive for s-channel simplified DM model.

% TODO reprhase it. It not visible from table - only from non-normalized plot. Plot should be mono-W with W backgound only? Or I should also include other backgounds?

\begin{figure}[hb]

\begin{subfigure}{.5\textwidth}
  \centering
  \includegraphics[width=\textwidth]{monoW/mT_kinemComparison_simplS_EFT_Wprime.png}
\end{subfigure}%
% \begin{subfigure}{.333\textwidth}
%   \centering
%   \includegraphics[width=0.25\textheight]{monoW/pT_kinemComparison_simplS_EFT_Wprime.png}
% \end{subfigure}
\begin{subfigure}{.5\textwidth}
  \centering
  \includegraphics[width=\textwidth]{monoW/EtMiss_kinemComparison_simplS_EFT_Wprime.png}
\end{subfigure}

% \includegraphics[width=0.75\textheight]{monoW/kinematics_simplifiedSChannel_EFT_Wprime.png}
\caption{Normalized kinematic distributions of transverse mass (left) and transverse missing energy in the event (right) of simplified model in s-channel as well as EFT models compared to $\PWprime$ distribution.}
  \label{fig:kinematicsSChannel}
\end{figure}

In \FigureRef{fig:scaledKin} transverse mass distribution are show for s- and t-channels simplified models as well as EFT and $\PWprime$ signals with comparison to SM W background 
scaled to according cross process section. Distribution for t-channel model looks almost identical to one for s-channel. 
But cross sections for t-channel processes are for one-two order of magnitude lower compared to s-channel (see \TableRef{tab:TriggerDetails}) which leads to conclution
that signal selection even less sensitive for t-channel simplified model. Aslo SM W backround are for a few orders higher in all range of $m_{T}$ than any sample of simplified model.
For D52 EFT model excess over SM W backgound can be seen in high-$m_{T}$ region. While for WW$\chi\chi$ EFT model it is not the case, because cross section is very low.

% TODO us or D5 or D52. harmonize stuff!!!

\begin{figure}[hb]
\begin{subfigure}{.5\textwidth}
  \centering
  \includegraphics[width=\textwidth]{monoW/dm_final_S_vs_T_channel_pad1.png} 
\end{subfigure}%
% \begin{subfigure}{.33\textwidth}
%   \centering
%   \includegraphics[width=0.95\textwidth]{monoW/pT_kinemComparison_simplT_EFT_Wprime.png}
% \end{subfigure}
\begin{subfigure}{.5\textwidth}
  \centering
  \includegraphics[width=\textwidth]{monoW/dm_final_EFT_vs_SMW_pad1.png}
\end{subfigure}
% \includegraphics[width=0.75\textheight]{monoW/kinematics_simplifiedTChannel_EFT_Wprime.png}
\caption{Kinematic distributions of transverse mass of simplified model in s- and t-channels (left) 
and EFT models with $\PWprime$ signal (right) in comparison with SM W background scaled to according process cross section.}
  \label{fig:scaledKin}
\end{figure}

Transverse mass distribution for both EFT models looks similar and majority (???) of the signal lays in the signal region. 
% Cross section for both processes strongly depends on energy scale of new physics $\Lambda$ (or effective mediator mass for D52 model).
% Dependence of cross section for WW$\chi\chi$ model versus energy scale is shown in \FigureRef{fig:lambdaScan}. 

% TODO describe what is D52 model. That it is vector interaction constructive interference

\begin{table}[tp]
  \begin{tabular}{r|c|c|c}
    Model 	& Channel 	  & Parameters	    & Cross section, [nb] \\
    \midrule
    Simplified  & s-channel	  & $M_{Z'}$=1TeV    & .05181 \\
		&		  & $M_{Z'}$=100GeV  & .001989 \\
		&		  & $M_{Z'}$=10TeV   & 7.5E-11 \\
		& t-channel	  & $M_{Z'}$=10GeV   & .0018895 \\
		&		  & $M_{Z'}$=100GeV  & 9.2E-5 \\
		&		  & $M_{Z'}$=2TeV    & 4.874E-8 \\
    \midrule
%     EFT 	& WW$\chi\chi$	  & $m_{\chi}$=1GeV; $\Lambda$=3TeV    & 3.6E-10 \\
% 		& D52		  & $m_{\chi}$=1GeV; $M^{*}$=1TeV	& 4.4E-4 \\
EFT 	& WW$\chi\chi$	  & $\Lambda$=3TeV    & 3.6E-10 \\
	& D52		  & $M^{*}$=1TeV	& 4.4E-4 \\
    \midrule	
$\PWprime$ 	& 	  & $m_{\PWprime} =$ 2 GeV   & 1.1E-4 \\    
		%     Wprime ($m_{\PWprime}$=2TeV) & 1.1E-4 \\
  \end{tabular}
  \caption{Mono-W cross section for different theoretical models.}
  \label{tab:TriggerDetails}
\end{table}


\begin{figure}[hb]
 \includegraphics[width=0.6\textheight]{monoW/WWxx_Wlv_DM1_LambdaScan.pdf}
  \caption{Cross section of WW$\chi\chi$ EFT model as a function of energy scale of new physics, $\Lambda$.}
  \label{fig:lambdaScan}
\end{figure}

\subsection{Validity of EFT approach and summary}

Cross section for both EFT processes strongly depends on energy scale of new physics $\Lambda$ (or effective mediator mass $M^{*}$ for D52 model).
Dependence of cross section for WW$\chi\chi$ model versus energy scale is shown in \FigureRef{fig:lambdaScan}.
In order for WW$\chi\chi$ process have sizeable cross section compared with $\PWprime$ ($M_{\PWprime}$=2TeV) model which is used as a reference in this study, 
energy scale of new physics $\Lambda$ has to be of order 200-300 GeV. But according to ~\cite{arXiv:1512.00476}: 
``The EFT approximation is valid when the momentum transfer in a given
process of interest is much smaller than the mass of the mediating
particle. For momentum transfer larger than or comparable to
$\Lambda$, the EFT description will break down.''
Moment transfer for 13 TeV collisions at LHC correspond to scale of few TeV, 
which mean that $\Lambda$ has to be of order of TeV in order for model to be valid.

% TODO cross check this statement with Dark matter report...
% http://arxiv.org/pdf/1507.00966.pdf
Also for WW$\chi\chi$ model there is no straightforward way to compare the
results with non-collider searches for DM
which make this model less appealing comparing with other models [add non-collider DM production diagram].

D5 constructive (D52) model violate weak gauge invariance as described at ~\cite{arXiv:1503.07874}.
It leads to spurious cross section enhancements at LHC energies. It is not recommended to be used anymore by Dark Matter Forum (some link???).

\subsection{Conclusion}

Transverse mass and $E_{T}^{miss}$ for all presented DM models with comparison with $\PWprime$ ($M_{W'}$ = 2TeV) signal are shown at \FigureRef{fig:scaledKin}.
All distribution are scaled according to the cross section of the process. It's clearly seen that simplified models tends to contribute to low-$m_{T}$ region
outside of the signal selection. With increasing of mediator mass $Z'$ cross section drops significantly and become indistinguishable with SM W background.
On other hand EFT DM samples contribute to high-$m_{T}$ region but as desribed above D52 model has physicaly unmotivated high cross section and WW$\chi\chi$ 
significantly smaller comparing with $\PWprime$ signal for physicaly motivated values of scale of new physics $\Lambda$. Which leads to the conclusion that
lepton and $E_{T}^{miss}$ signal selection is not sensitive for DM searches.



Similar studies was done by Bell and collaborators~\cite{arXiv:1512.00476}, where authors estimated approximate upper limit on for 3000 fb$^{-1}$. 
At \FigureRef{fig:bellExclLim} exclusion limit
as a function of mass of DM particle and mass of DM-SM mediator $Z'$ is shown. One can notice that obtained limits for mono-lepton channel are significantly 
worse than from all other channel, especially than from di-jet analysis. Conclusion of authors are identical to the conclusion of this study that mono-lepton channel
are not sensitive enough for DM searches and is significantly worse comparing to all other hadronic channels.

\begin{figure}[hb]
 \includegraphics[width=0.8\textwidth]{monoW/schan1.pdf}
  \caption{Exclusion limit for the s-channel $Z'$ model as a function of mass of dark matter particle, $m_{\chi}$, 
  and mass of DM-SM mediator, $m_{Z'}$, reported in~\cite{arXiv:1512.00476}.
  Exclusions are shown as shaded regions for LUX and for mono-jet and di-jets at 8 TeV, 
  and the reaches are shown for the mono lepton and mono fat jet searches at 14 TeV 3000 fb$^{-1}$.}
  \label{fig:bellExclLim}
\end{figure}

\section{Outlook}
\label{sec:wprimeConclusion}

\chapter{Conclusion}