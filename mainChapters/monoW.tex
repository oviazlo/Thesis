\chapter{Dark Matetr Mono-W model}
\label{chap:monoW}


\section{Intoduction}

Another possible BSM signature which can be easyly tested in signal region is
associated production of pair of weakly interacting massive particles (WIMP), 
which are candidates for dark natter particles, with SM W. 
In this scenario MET will consist not only from neutrino from W decay but as 
well from pair of WIMPs.

In this chapter study based on using different popular models are presented 
which address question is our signal selection could be sensitive to such 
kind of models.


% While searching for $\PWprime$ boson one can naivly expect that distributions of 
% transferred energy of decay products will look quite similar, which
% means that distributions of transfer lepton momentum and missing energy will 
% look the same as well. But if we consider pair-production of dark matter 
% particles associated with W we expect 

\section{Theoretical models}

% TODO: explanation of simplified and EFT models (take from DM forum report or fro Bell's papers).

% TODO find reference from DM forum report which explain idea of simlified models 
% (that we introduce additionla mediator which on one side interact with SM particles and on other side - with dark matter particles).
% same for EFT models. But for now skip this part.

% TODO describe possible type of interection for mono-W models.

% TODO figure out what does it mean constructive and destructive intereference for DX models?

% TODO explanation for contact interaction is needed as well, ha-ha-ha...


For simplfied model case W is produced as initial state radiation from one of incoming quarks \FigureRef{fig:feynMonoWSimple}. 
For EFT models W can be produces as final state particle \FigureRef{fig:cc1a} as well as initial state radiation \FigureRef{fig:cc1a}.

\begin{figure}[hb]
  \subfloat[][Example 1a]{\label{fig:cc1a}\includegraphics[width=0.3\textheight]{monoW/simplifiedDM_diagram.pdf}}\quad
  \subfloat[][Example 1d]{\label{fig:cc1d}\includegraphics[width=0.3\textheight]{monoW/simplified_tChannel_model_diag.pdf}}
  \caption{Feynman diagrams of Simplified dark matter mono-W models.}
  \label{fig:feynMonoWSimple}
\end{figure}

\begin{figure}[hb]
  \subfloat[][Example 1a]{\label{fig:cc1a}\includegraphics[width=0.3\textheight]{monoW/WWxx_diagramm.png}}\quad
  \subfloat[][Example 1d]{\label{fig:cc1d}\includegraphics[width=0.3\textheight]{monoW/EFT_D5_model_diag.pdf}}
  \caption{Feynman diagrams of EFT dark matter mono-W models.}
  \label{fig:feynMonoWEFT}
\end{figure}


\begin{table}[tp]
  \begin{tabular}{r|c|c|c}
    Model 	& Channel 	  & Parameters	    & Cross section, [nb] \\
    \midrule
    Simplified  & s-channel	  & $M^{*}$=1TeV    & .05181 \\
		&		  & $M^{*}$=100GeV  & .001989 \\
		&		  & $M^{*}$=10TeV   & 7.5E-11 \\
		& t-channel	  & $M^{*}$=10GeV   & .0018895 \\
		&		  & $M^{*}$=100GeV  & 9.2E-5 \\
		&		  & $M^{*}$=2TeV    & 4.874E-8 \\
    \midrule
    EFT 	& WW$\chi\chi$	  & $m_{\chi}$=1GeV; $\Lambda$=3TeV    & 3.6E-10 \\
		& D52		  & $m_{\chi}$=1GeV; $M^{*}$=1TeV	& 4.4E-4 \\
		%     Wprime ($m_{\PWprime}$=2TeV) & 1.1E-4 \\
  \end{tabular}
  \caption{Mono-W cross section for different theoretical models.}
  \label{tab:TriggerDetails}
\end{table}



\begin{figure}[hb]
 \includegraphics[width=0.6\textheight]{monoW/WWxx_Wlv_DM1_LambdaScan.pdf}
  \caption{Cross section of WW$\chi\chi$ EFT model as a function of energy scale of new physics, $\Lambda$.}
  \label{fig:lambdaScan}
\end{figure}