\chapter{Searches for beyond Standard Model physics with same-sign dileptons}
\label{chap:SS}
\section{Motivation}
% ******************************************************************************
% TODO
% - write something about lepton signature... why it has benefits wrt to jets and photons...
% - write which models we are looking for. how signal looks like with comparison with background
% - write about benefits of the general search (or model-independent? or it's dangerous to say like this?)... theorists can use these results for their models?
% - motivation flow: why leptons -> why same-sign dilepton -> why model-independent search?
% ******************************************************************************

The idea of the current analysis is to make a general search for a new physics in a same-sign dilepton channel. \toFix
% Why do we use leptons as final objects?
Leptons as a physics signature are very clean and straighforward observalbles. 
ATLAS detector capable to detect leptons with amazing momentum resolution precision... \toFix[show some resolution plot]

In resonant searches for new physics using leptons are beneficial due to fact that resonance peak for potential BSM signals will be narrower with compariosn to e.g. using jets.
Because energy of leptons is reconstructed very well resonance peaks 
BSM bosons whose decay would manifest itself as a narrow resonance in the dilepton mass spectrum

% Why do we use same-sign dilepton channel
Looking for a same-sign dilepton channel is motivated by the fact that cross section for SM processes with such signature are really small, which make 
it very sensitive possible signals of new physics.  \toFix
% Why model-independent search
There are other searches in ATLAS which taget the same final signature but with requiring additional cuts on number of jets \toFix[link] of missing transverse energy \toFix[link].
In such way these analyses target limited number of specific models.
Analyses presented in this chapter aims to be more general and model-independent. 
\toFix[Explain which models does this analysis cover?]

\section{Background processes}
\label{sec:wprimeBackgrounds}
% ******************************************************************************
% TODO
% - describe that there are hree possibilities to reconstruct same-sign lepton pair
% - 
% ******************************************************************************


There are three different components which contribute to the signal selection.
The first component is so called prompt contribution when same-sign dilepton originates from the SM processes.
Just a few SM processes contribute to the same-sign dilepton final state. In such processes same-sign leptons originates from semi-weakly decay of top quark
and/or leptonic decays of $W, Z$ bosons. Some example of the Feynman diagrams are shown in \FigureRef{fig:Diboson_feynman}.
Overall SM processes with same-sign dilepton final state have relatively small cross section.

The second component comes from the wrong reconstruction of the lepton itself, when pion or jet is reconstructed as a lepton or from leptons which were born 
not in the pp collision but from decays of secondary particles, e.g. kaons. This component is called non-prompt or fake component.

The third component comes from misidentification of the electrical charge of the lepton, which make processes, where opposing lepton pair is produced, 
contribute to the signal selection as well. The charge misidentification effect become visible for high-momentum leptons, when curvature of the track 
is hard to reconstruct.

% TODO describe connection of written above with table with all used MC samples...
% TODO normalization is just checked for it is not applied as a scale factor. Cross check with Monika!!!
Prompt contribution is described by the MC simulation. 
% with applying the overall normalization of the MC samples to the data in the control region.
Non-prompt contribution is derived with the data-driven method, so called fake factor method.
Charge flip rate is estimated from Z boson lepton decays and is applied to the opposite-sign MC samples.

List of all used MC samples are shown in \TableRef{tab:MC_cross}. All other processes which are not listed in the table don't contribute
significantly to the same-sign selection and are considered negligable.

\begin{table}[ht]
  \begin{center}
    \begin{tabular}{l|c|c|c}

      \hline
Process &  Generator&  PDF set & Normalisation \\
&  + fragmentation/ &  & based on \\
&  hadronisation & &\\
\hline\hline
\multirow{2}{*}{$WZ$ } &  \multirow{2}{*}{{\scshape sherpa-1.4.1} \cite{Sherpa}} &   \multirow{2}{*}{CT10 \cite{ct10}} & NLO QCD \\
 & & &  with {\scshape mcfm-6.2}\cite{mcfm} \\
\hline
\multirow{2}{*}{$ZZ$}  &  \multirow{2}{*}{{\scshape sherpa-1.4.1}} & \multirow{2}{*}{CT10} & NLO QCD  \\
& & &  with {\scshape mcfm}-6.2 \\
\hline
\multirow{2}{*}{\Wpm\Wpm}  & M{\scshape ad}G{\scshape raph}-5.1.4.8 \cite{madgraph4}  &   \multirow{2}{*}{CTEQ6L1 \cite{cteq}}  &  \multirow{2}{*}{LO QCD} \\
&  {\scshape pythia-8.165} \cite{pythia8}& &\\
\hline
\ttbar $V$, & M{\scshape ad}G{\scshape raph}-5.1.4.8  & \multirow{2}{*}{CTEQ6L1} & \multirow{2}{*}{NLO QCD \cite{top9,ttbarW}} \\
$V=W,Z$ &  + {\scshape pythia-6.426} & & \\
\hline
 MPI $VV$ &  \multirow{2}{*}{{\scshape pythia-8.165}\cite{pythia8}}  &  \multirow{2}{*}{CTEQ6L1} &  \multirow{2}{*}{LO QCD} \\
 $V=W,Z$ &  & & \\[+0.025in]
\hline
\hline
\multirow{2}{*}{$Z/\gamma^* +$ jets} & {\scshape alpgen-2.14} \cite{Alpgen}&\multirow{2}{*}{CTEQ6L1}& {\scshape dynnlo-1.1} \cite{dynnlo} with \\
 & + {\scshape herwig-6.520} \cite{Herwig1, Herwig2}& & MSTW2008 NNLO \cite{mstw} \\
\hline
\multirow{2}{*}{\ttbar} & {\scshape mc@nlo}-4.06 \cite{Mcnlo, Mcnlo2} & \multirow{2}{*}{CT10}&{NNLO+NNLL } \\
& + {\scshape herwig-6.520} & & QCD \cite{top1,top2,top3,top4,top5,top6} \\
\hline
\multirow{2}{*}{$Wt$} & {\scshape mc@nlo-4.06}  & \multirow{2}{*}{CT10}& {NNLO+NNLL } \\
   & + {\scshape herwig-6.520} & & QCD \cite{top7,top8}\\
\hline
\multirow{2}{*}{$W^{\pm}W^{\mp}$} & \multirow{2}{*}{{\scshape sherpa-1.4.1}} & \multirow{2}{*}{CT10}& NLO QCD \\
& &  & with {\scshape mcfm-6.2}\\
\hline
\multirow{2}{*}{$W\gamma$} & \multirow{2}{*}{{\scshape sherpa}-1.4.1} & \multirow{2}{*}{CT10}& NLO QCD\\
& &  & with {\scshape mcfm-6.3}\\
\hline
\end{tabular}
\end{center}
  \caption{Generated samples used for background estimates. The generator, PDF set and order of cross-section calculations used for the normalisation
  are shown for each sample.
  The upper part of the table shows the MC samples used for the SM background coming from leptons with the same charge (MPI stands for multiple parton interactions), the lower part gives the background sources arising in the \ee\ or \emu\ channel due to electron charge misidentification.}
\label{tab:MC_cross}
\end{table}

\begin{figure}

\begin{subfigure}{.5\textwidth}
  \centering
  \includegraphics[width=\textwidth]{SS/feynman/WZ_electron.eps}
\end{subfigure}%
\begin{subfigure}{.5\textwidth}
  \centering
  \includegraphics[width=\textwidth]{SS/feynman/ttbar_electron_v2.eps}
\end{subfigure}

\caption{Possible production diagrams for diboson (left) and $\ttbar W$ leading to same-sign dileptons. \toAsk[check diagrams]}
  \label{fig:precHitFracPerTrack}
\end{figure}


\section{Event selection}
% TODO mention here about size of the dataset and that it was 8 TeV 50 ns data.
% TODO description of the used triggers here!

% TODO ToTHINK MC cross section and sample sizes\toFix

Events of interest are events which have two high-$p_T$ isolated leptons \toFix

20.3 fb$^{-1}$ of data collected at 8 TeV center of mass energy pp collisions, with mean number of interactions per crossing equal to 21, 
with 50 ns bunch distance were used for the analysis.
Firstly events are required to have at least one reconstructed primary vertex with at least three tracks matched to it. If there are few vertices, the one with the highest
$\sum p^2_T$, where $p_T$ is transverse momenta of the tracks, is chosen.
Events which pass \toFix trigger, which selects events with two isolated \toFix leptons with $p_{T}>25$\toFix and $p_{T}>20$\toFix, were selected.
% TODO Cleaning and all other event-related cuts to describe here...

\subsection{Electron selection}

Electron candidates are reconstructed by matching of the track from ID with energy deposits in the EM calorimeter.

\toDo[explain how electron energy is measured]

Electrons used for the signal selection are required to satisfy critereas:
\begin{itemize}
 \item $p_T > 20$ GeV. To ensure a high and flat trigger efficiency and to harmonize $p_T$ requirement between three analysis channels ($ee$, $\mu\mu$ and $e\mu$) 
 \toAsk[is it OK to use ``channel''?].
 \item $|\eta|<2.47$ to be within high-granularity acceptance of EM calorimeter, but excluding barrel-end-cap transition region $1.37<|\eta|<1.52$.
 \item Transverse and longitudinal impact parameters $|d_0|/\sigma(d_0) < 3$ and $|z_0 \times sin \theta| < 1$ mm. 
 To verify that electron was born close to the primary vertex and to reject electrons which were originated from the decay of long-lived particles.
 \item To pass ``tight'' electron set of identification criterea defined in~\cite{electron_tight}. \\ 
 \toFix[explain shortly purpose of the cut]
 \item To pass isolation requirement. To distinguish prompt electrons from those associated with jet activity.
 \item No reconstructed jet within $\Delta R < 0.4$ \toFix[define R]. To remove ambiguous clasification of the track, when track is clasified as electron and as jet simultaneously. 
 Or when jet is too close to the electron which will lead to poor reconstruction of latter. 
\end{itemize}

Isolation requirements were chosen in order to reach pile-up independent efficiency of more than 99$\%$ for electrons 
with $p_T >$ 40 GeV. The requirements consist from two parts. \toAsk[two parts? firstly? secondly?]
Firstly, the sum of the transverse energies in the EM and hadronic calorimeters around the electron within 
$\Delta R < 0.2$, with core electron energy subtracted from the sum, has to be less than 
3 GeV $+ (p_T^e - 20$ GeV$) \times 0.037$, where $p_T^e$ is electron transverse momentum.
Secondly, the sum of the $p_T$ of all tracks with $p_T > 0.4$ GeV within $\Delta R < 0.3$ around the electron track has
to be less than 10$\%$ of electron $p_T$.

Electrons used in control regions are required to fail one or few requirements above but pass looser one.

\subsection{Definition of same-sign electron pair}
% TODO inv. mass cut; treatment of event with three leptons;
All electrons which passed selection are considered while forming a lepton pair.
All combinations are checked and it is allowed to have more than one pair selected in one event.
Electrons in the pair are chatechorized by the $p_T$: electror with higher $p_T$ is called leading lepton, while another one with lower $p_T$ subleading lepton.
Leading lepton have to pass more strict cut $p_T>25$ GeV cut while subleading lepton have to satisfy $p_T>20$ GeV as described above.

If event contains an opposite-sign same flavour lepton pair which satisfies condition $( |m_{\Plepton\Plepton} - m_{Z}| < 10$ GeV$)$,
where $m_{\Plepton\Plepton}$ is lepton pair invariant mass and $m_{Z}$ is mass of Z boson, event is discarded.
This significantly suppress prompt background.

To avoid low-mass hadronic resonances like $J/\psi$ or $\varUpsilon$ to show up in the mass spectrum only same-sign pair with $m_{\Plepton\Plepton}>15$ GeV are selected.
Aditionaly, if same-sign lepton pair satisfies $70 < m_{\Plepton\Plepton}< 110$ GeV, which corresponds to the Z peak region, event is removed.
This region is used for estimation of the electron charge misidentification background.

\section{Estimation of the non-prompt and charge flip backgrounds}
\subsection{Non-prompt background}
\toDo[short description of the fake factor method. Put details on which regions were used.]
\toDo[reference to Anthonys thesis]

\subsection{Prompt opposite-sign dilepton with charge flip}
% TODO
% subsection flow:
% CF sources --> method and used regions --> validation of the method --> deriving scale factors --> appling to specific MC samples
\toDo[description how charge flip rate was obtained, and plots if CF rate vs. pT and eta.]
\toAsk[why we define CF rate in 80..100 GeV region, while we cut 70..110 GeV from signal region?]
\toAsk[CF at high pT. extrapolation? of one large bin with huge error? look it up!]

As was mentioned before charge of one reconstructed electron from pair can be missidentified, which lead to that
\toAsk[lead to that?]
processes with opposite-sign dilepton final state will contribute as a background for same-sign dilepton signal selection.
Because background is relatevily small in the signal selection a charge flip cannot be neglected and have to be properly estimated. Two cases are considered as a charge flip. When electron charge is actually misreconstructed due to matching of the wrong track to the EM cluster or by too small curvature of the high-momentum track. Or when electron emit a photon by bremsstrahlung, which decays to electron-positron pair. 

In order to estimate how many events with opposite-sign dileptons will ``leak'' to same-sign dilepton signal region one
need to know charge flip rate for electrons. It is expected that charge flip will depends from electron momentum (due to track curvature) and from $\eta$ (due to different geometry of the detector versus $\eta$).

Electrons from Z boson decay provide a great possibility to estimate charge flip rate. By applying signal selection
and requiring invariant mass of the lepton pair to be around Z mass, $80 < m_{ee} < 100$ GeV, one obtain a pure
sample of electron pairs, with one electron with missidentified charge. 


Charge flip can be measured both with Monte Carlo simulation as well as with data-driven techniques.


In \FigureRef{fig:chargeFlip_structure} MC simulation of the charge flip \toDo 


To estimate charge flip \toDo.

\begin{figure}
\begin{center}
 \includegraphics[width=0.7\columnwidth]{SS/support_note/chargeflip/misidratept_ZDY.eps}
\caption{\toDo[caption]}
\label{fig:chargeFlip_structure}
\end{center}
\end{figure}







% \begin{figure}
% \begin{center}
%  \includegraphics[width=0.7\columnwidth]{SS/paper_draft/misidrate_datamc.eps}
% \caption{Electron charge misidentification rate as a function of $\eta$. Data/MC comparison [data is likelihood of T_and_P \toFix]}
% \label{fig:track_ext_fraction}
% \end{center}
% \end{figure}




\section{Control regions}

In order to make sure that all backgrounds are modelled/estimated correctly in the signal region one can define and use special control (or validation) regions.
These regions have to have kinematics close to the one in the signal region but should not overlap with it. Each region is designed to test one given background type at a time,
which means that in each control region only one background type has to give dominant contribution with comparison to all other backgrounds.


\subsection{Prompt opposite-sign dilepton}
\subsection{Prompt same-sign enhanced control region}
% TODO table with final numbers to show how good MC normalization is wrt to data.
\subsection{Electron charge flip enhanced region}
\subsection{Fake-enriched region}

\section{Signal Region}
\label{sec:wprimeSignalRegion}

\section{Systematic Uncertainties}
\label{sec:wprimeSystematics}
\subsection{Electron ID and momentum scale}
\subsection{Event-level and modelling systematics}
\subsection{Background uncertainty estimation}
\subsection{Systematic uncertainty on the signal}

\section{Cross section and mass limits}
\subsection{Statistical analysis for discovery and exclusion}
\subsection{Fiducial cross section limits on new physics}
\subsection{Mass limits on double charged Higgs}

\section{Outlook}
\label{sec:wprimeConclusion}

