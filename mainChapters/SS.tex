\chapter{Searches for new physics with same-sign dileptons}
\label{chap:SS}
\section{Motivation}

Despite the great success of the Standard Model in describing the interactions of the elementary particles, there are a plethora of phenomena
which cannot be explained by the SM. Gravity, Dark Energy and Matter, neutrino masses and other observations are not predicted by the
SM, which indicates its incompleteness.
This is why many studies at LHC experiments are focused on probing the high energy regime to search for new physics, 
the observation or non-observation of which can support or disprove different extensions of the SM.

Many BSM models predict same-sign dilepton in the final state.
It is a very clean signature as few SM processes lead to such a final state.
This leads to high sensitivity for new physics.

There are a number of searches in ATLAS which target same-sign dilepton final state signatures, both inclusive ones and searches with additional requirements such as the number of jets or amount of missing transverse energy~\cite{heavy_majorana_neutrino_paper,floderus_paper,Aad:2014pda}.
% There is number of inclusive searches which target same-sign dilepton final state as well as searches with additional requirements on number of jets or missing transverse energy in ATLAS~\cite{heavy_majorana_neutrino_paper,floderus_paper,Aad:2014pda}.
The analysis presented here aims to be as model independent as possible and to probe exclusively the same-sign dilepton signature and all the BSM models which predict it.

The signal selection is based on minimal kinematic requirements on leptons.
Electrons are clean objects which can be reconstructed with high efficiency in ATLAS.
The search observable for the analysis is the same-sign dilepton invariant mass distribution.

If no signal from new physics is found, an upper limit on fiducial cross section for new physics can be set.
As an example of a BSM model with the same-sign dilepton final state, the pair production of doubly charged Higgs bosons is studied in detail within the search presented in this thesis.

The analysis searches in three channels: $e^{\pm}e^{\pm}$, $\mu^{\pm}\mu^{\pm}$ and $e^{\pm}\mu^{\pm}$.
This chapter will describe the search only for the $e^{\pm}e^{\pm}$ channel.

\section{Background processes}
\label{sec:wprimeBackgrounds}

There are four different sources of background which contribute to the signal region.
The first source is the so-called prompt background which corresponds to prompt (originating from the primary vertex) same-sign dileptons arising from SM processes.
Just a few SM processes have same-sign dileptons in the final state. 
% For example, they can originate from semileptonic decays of top quarks and/or leptonic decays of $W$ or $Z$ bosons.
For example, they can originate from an associate production of \ttbar~quark pair with vector boson or from production of two vector bosons,
where top quarks decay semileptonically and vector bosons decay leptonically.
The corresponding Feynman diagrams are shown in \FigureRef{fig:prompt_bkg_feynman_diag}.
In general, SM processes with the same-sign dilepton final state have relatively small cross sections at high dilepton invariant mass regions of interest.

The second source comes from wrongful identification of the lepton itself, when a pion or a jet is reconstructed as a lepton, or from leptons which were produced 
not in the pp collision but from decays of secondary particles, e.g. kaons. This source is called non-prompt or fake background.

The third source comes from the misidentification of the electric charge of the lepton, which makes processes in which an opposite-sign prompt lepton pair is produced 
contribute to the signal selection as well. The charge misidentification effect becomes significant for high-momentum leptons, when the curvature of the track in the ID is difficult to reconstruct. 
This source also includes events in which a prompt electrons emit a photon due to hard bremsstrahlung. Subsequently the photon creates an electron-positron pair in which one of the two leptons receives most of the energy.
One lepton from the pair can form a same-sign electron pair with the original prompt lepton. These two processes are referred to as charge misidentification 
background later in the text.

The last source arises from $W\gamma$ processes, where an electron-positron pair is created from photon conversion, and by combining with an electron from $W$ decay
a same-sign lepton pair can be formed. 

Prompt backgrounds are modelled by MC simulations. 
The contribution from non-prompt backgrounds is derived using a data-driven method, the so-called fake factor method.
Charge misidentification and $W\gamma$ backgrounds are estimated from MC simulation as well. However, as will be shown in \SectionRef{subsec:CF_definition}, the electron charge misidentification probability is not properly described by the MC simulation, meaning that a special correction, the so-called charge misidentification scale factor, obtained from the data, is used to correct for this.

The list of all the MC samples used in the analysis is shown in \TableRef{tab:MC_cross}. 
All listed samples were centrally produced by the ATLAS simulation group.
Processes which are not listed in the table do not contribute
significantly to the same-sign signal region and are considered negligible.
This table summarizes which MC generators and parton distribution function (PDF) sets were used, and shows the perturbative QCD order the cross section calculations were performed to.


\begin{table}[ht]
  \begin{center}
    \begin{tabular}{l|c|c|c}

      \hline
Process &  Generator&  PDF set & Normalization \\
&  + fragmentation/ &  & based on \\
&  hadronization & &\\
\hline\hline
\multirow{2}{*}{$WZ$ } &  \multirow{2}{*}{{\scshape sherpa-1.4.1} \cite{Sherpa}} &   \multirow{2}{*}{CT10 \cite{CT10}} & NLO QCD \\
 & & &  with {\scshape mcfm-6.2}\cite{mcfm} \\
\hline
\multirow{2}{*}{$ZZ$}  &  \multirow{2}{*}{{\scshape sherpa-1.4.1}} & \multirow{2}{*}{CT10} & NLO QCD  \\
& & &  with {\scshape mcfm}-6.2 \\
\hline
\multirow{2}{*}{\Wpm\Wpm}  & M{\scshape ad}G{\scshape raph}-5.1.4.8 \cite{madgraph4}  &   \multirow{2}{*}{CTEQ6L1 \cite{cteq}}  &  \multirow{2}{*}{LO QCD} \\
&  {\scshape pythia-8.165} \cite{pythia8}& &\\
\hline
\ttbar $V$, & M{\scshape ad}G{\scshape raph}-5.1.4.8  & \multirow{2}{*}{CTEQ6L1} & \multirow{2}{*}{NLO QCD \cite{top9,ttbarW}} \\
$V=W,Z$ &  + {\scshape pythia-6.426} & & \\
\hline
 MPI $VV$ &  \multirow{2}{*}{{\scshape pythia-8.165}\cite{pythia8}}  &  \multirow{2}{*}{CTEQ6L1} &  \multirow{2}{*}{LO QCD} \\
 $V=W,Z$ &  & & \\[+0.025in]
\hline
\hline
\multirow{2}{*}{$Z/\gamma^* +$ jets} & {\scshape alpgen-2.14} \cite{Alpgen}&\multirow{2}{*}{CTEQ6L1}& {\scshape dynnlo-1.1} \cite{dynnlo} with \\
 & + {\scshape herwig-6.520} \cite{Herwig1, Herwig2}& & MSTW2008 NNLO \cite{mstw} \\
\hline
\multirow{2}{*}{\ttbar} & {\scshape mc@nlo}-4.06 \cite{Mcnlo, Mcnlo2} & \multirow{2}{*}{CT10}&{NNLO+NNLL } \\
& + {\scshape herwig-6.520} & & QCD \cite{top1,top2,top3,top4,top5,top6} \\
\hline
\multirow{2}{*}{$Wt$} & {\scshape mc@nlo-4.06}  & \multirow{2}{*}{CT10}& {NNLO+NNLL } \\
   & + {\scshape herwig-6.520} & & QCD \cite{top7,top8}\\
\hline
\multirow{2}{*}{$W^{\pm}W^{\mp}$} & \multirow{2}{*}{{\scshape sherpa-1.4.1}} & \multirow{2}{*}{CT10}& NLO QCD \\
& &  & with {\scshape mcfm-6.2}\\
\hline
\multirow{2}{*}{$W\gamma$} & \multirow{2}{*}{{\scshape sherpa}-1.4.1} & \multirow{2}{*}{CT10}& NLO QCD\\
& &  & with {\scshape mcfm-6.3}\\
\hline
\end{tabular}
\end{center}
  \caption{List of MC generated samples used for background prediction. 
  The used MC generator, the PDF set and the order of the cross section calculations used for the normalization are listed for each sample.
  The upper part of the table contains MC samples which provide same-sign dilepton in the final state
  (MPI stands for multiple parton interactions), 
  while the lower part contains samples which contribute to the signal selection due to electron charge misidentification.}
\label{tab:MC_cross}
\end{table}

\begin{figure}

\begin{subfigure}{.5\textwidth}
  \centering
  \includegraphics[width=\textwidth]{SS/feynman/WZ_electron_v2.eps}
\end{subfigure}%
\begin{subfigure}{.5\textwidth}
  \centering
  \includegraphics[width=\textwidth]{SS/feynman/ttbar_electron_v2.eps}
\end{subfigure}

\caption{Production diagrams for diboson (left) and $\ttbar W$ processes leading to the same-sign dileptons.}
  \label{fig:prompt_bkg_feynman_diag}
\end{figure}
% TODO Peter: check arrows 

\section{Event selection}
% TODO mention here about size of the dataset and that it was 8 TeV 50 ns data.
% TODO description of the used triggers here!
% TODO [no jet selection is described]

The analysis is based on the pp collision data collected in 2012 by the ATLAS detector with 8 TeV center of mass energy.
The integrated luminosity of the sample corresponds to 20.3 fb$^{-1}$ and the mean number of interactions per bunch crossing was 21.

An event selected for the analysis must have at least one reconstructed vertex with at least three tracks matched to it. 
If there are several vertices, the one with the highest
$\sum p^2_T$, where $p_T$ are transverse momenta of the matched tracks, is chosen.
Events should have at least two electron candidates and fire the dilepton trigger, that requires the presence of two electrons with $p_T > 12$ GeV.
In order to exclude ambiguities between electron and jet reconstruction, the electrons are required to
be isolated. Isolation is calculated by summing up the particle momenta around the electron candidate within a defined cone size, 
$\Delta R = \sqrt{ (\eta_e-\eta_i)^2 + (\phi_e-\phi_i)^2 }$,
where $\eta_e$ and $\phi_e$ are rapidity and azimuthal angle of the electron candidate while $\eta_i$ and $\phi_i$ are rapidity and azimuthal angle of track or calorimeter cluster, not related to the electron candidate. Detailed explanations of the isolation and other requirements on electron candidates are given later in this thesis.

\subsection{Electron selection}
\label{subsec:electron_selection}
The next step is the selection of isolated high-$p_T$ electrons present in the event.
Electron reconstruction in the ATLAS detector central region ($\eta<2.5$) is done by matching tracks from the inner detector with energy deposits in the EM calorimeter.
The spatial resolution of the electron candidate in ($\eta$,$\phi$) plane is taken from the parameters of the matched track,
while energy is calculated from energy deposits in the EM calorimeter.

% ATLAS use three sets of electron identification criteria, which are ordered by background-rejection power and identification efficiency~\cite{electron_tight}.
% These sets are labelled as loose, medium and tight. Loose set provide has poor background-rejection power while tight - has the best.
% But background-rejection power comes at cost of identification efficiency

% so as to provide increasing
% background-rejection power at some cost to the identi-fication efficiency. 

The electrons of interest are well-reconstructed candidates which satisfy
the following requirements:
\begin{itemize}
 \item $p_T > 20$ GeV: to ensure a high and constant trigger efficiency as a function of $p_T$ and to harmonize $p_T$ requirement between the three analysis channels ($ee$, $\mu\mu$ and $e\mu$) 
 \item $|\eta|<1.37$ or $1.52<|\eta|<2.47$: to be within high-granularity acceptance of the EM calorimeter, but excluding the barrel-end-cap transition region.
 \item Electron tracks have to originate from the primary vertex.
 The transverse impact parameter significance which is defined as the ratio of the absolute transverse impact parameter ($d_0$) to its uncertainty ($\sigma(d_0)$) has to be below three. The distance between the $z$-coordinate of the primary vertex and $z$-position of the point of closest approach of the electron track in the ID to the beamline is required to be less than 1 mm.
 These requirements also reject electrons originating from decays of long-lived particles.
 \item Pass ``tight'' electron set of identification criteria defined in. \\ 
 ATLAS defined three set of reference electron identification criteria designed for use in analyses: ``loose'', ``medium'' and ``tight''.
 These criteria are designed in a hierarchical way. The background-rejection power is increasing from ``loose'' to ``tight'' criteria at some cost to the identification efficiency.
 The ``tight'' set provides a factor of two background rejection power with respect to the ``medium'' criteria, at the cost of less than 10$\%$ electron efficiency~\cite{electron_tight}.
 \item No reconstructed jet within $\Delta R < 0.4$: 
 to ensure that electron is not part of a jet.
%  to remove ambiguous classification of the track, 
%  when track is classified as an electron and as a jet simultaneously,
%  or when a jet is too close to the electron, leading to poor reconstruction of the latter. `
 \item Pass isolation requirement: to distinguish prompt electrons from those associated with jet activity.
\end{itemize}

The isolation requirement was chosen in order to reach pile-up independent efficiency of more than 99$\%$ for electrons 
with $p_T >$ 40 GeV. The requirement has two parts.
Firstly, the sum of the transverse energies in the EM and hadronic calorimeters around the electron within 
$\Delta R < 0.2$, with core electron energy subtracted from the sum, has to be less than 
3 GeV $+ (p_T - 20$ GeV$) \times 0.037$
\footnote{
Core electron energy corresponds to the electron energy deposit in the calorimeter in the core cluster, to which electron candidate is assigned. However, not all electron energy is contained in the core cluster, part of it leaks to the neighbor clusters which are used for electron isolation requirement. This effect is taken into account in the isolation requirement formula.}, 
where $p_T$ is electron transverse momentum.
Secondly, the sum of the $p_T$ of all tracks with $p_T > 0.4$ GeV within $\Delta R < 0.3$ around the electron track has
to be less than 10$\%$ of the electron $p_T$.

Electrons used in the so-called validation regions (VR) defined in \SectionRef{sec:bkg_validation}, used for validation of the background estimation, 
are required to fail one or several requirements above but pass looser one, as will be described further in the text.

\subsection{Electron pair selection}
% TODO inv. mass cut; treatment of event with three leptons;
All the electrons which passed the selection are used in the lepton pair formation step.
All combinations are considered, and it is allowed to have more than one pair selected in one event.
Electrons in the pair are classified by $p_T$: the electron with a higher $p_T$ is called leading lepton, while the other one is the subleading lepton.
The leading lepton has to pass a stricter cut of $p_T>25$ GeV, while the subleading lepton has to satisfy $p_T>20$ GeV as described above.

To avoid low-mass hadronic resonances like $J/\psi$ or $\varUpsilon$ showing up in the invariant mass spectrum, 
only same-sign pairs with $m_{\Plepton\Plepton}>15$ GeV are selected.
Additionally, if same-sign lepton pairs with the invariant mass $70 < m_{\Plepton\Plepton}< 110$ GeV 
which corresponds to the Z peak region, are found in the event, the event is discarded.
This region is used for estimation of the electron charge misidentification background as described further.

Since more than one pair is allowed in one event, an additional requirement which significantly suppresses the prompt background contribution is required.
If an event contains an opposite-sign same-flavour lepton pair which satisfies the condition ($|m_{\Plepton\Plepton} - m_{Z}| < 10$ GeV,
where $m_{\Plepton\Plepton}$ is the lepton pair invariant mass and $m_{Z}$ is the mass of Z boson), the event is discarded.

\section{Estimation of the charge misidentification and non-prompt backgrounds}

\subsection{Prompt opposite-sign dilepton with charge misidentification}
\label{subsec:CF_definition}
% TODO
% \toAsk[CF at high pT. extrapolation? of one large bin with huge error? look it up!]

Since the charge of one of the reconstructed electrons from the pair can be misidentified, 
processes with the $e^{+}e^{-}$ final state can contribute to the signal region.
As the prompt background is relatively small in the signal region, the possibility of a charge misidentification cannot be neglected and has to be precisely estimated. 
Two cases are considered as a charge misidentification background. 
The first one is when the electron charge is truly misidentified due to matching of the wrong track to the EM cluster 
or due to a small curvature of the high-momentum track. 
The second one arises when an electron emits a photon by bremsstrahlung, which in turn decays to an electron-positron pair. 
Instead of the original electron, an electron from the photon conversion can be attributed to the pair and can have an opposite charge.

In order to estimate how many events with opposite-sign dileptons contribute to the same-sign dilepton signal region, 
one has to know the electron charge misidentification rate. It is expected that the charge misidentification depends on the electron momentum (due to the track curvature) 
and on $\eta$ (due to the variation of the detector material with $\eta$).

Electrons from Z boson decays provide a huge sample of opposite-sign lepton pairs which can be used to estimate the charge misidentification rate. 
By applying the signal selection criteria
and requiring the invariant mass of the lepton pair to be around the Z mass, $80 < m_{ee} < 100$ GeV, one obtains a pure
sample of electron pairs, where the charge of one electron is misidentified. Knowing the number of opposite-sign pairs which pass the signal selection, 
one can extract the charge misidentification rate.

In \FigureRef{fig:chargeFlip_structure} the charge misidentification rate obtained from MC simulation as a function of $p_T$ is shown.
By using generator level information one can distinguish charge misidentification due to photon conversion from other effects as shown in the figure.
The contribution from photon conversions from bremsstrahlung dominates 
the total charge misidentification rate over the entire $p_T$ range except in the high-$p_T$ region, where misidentification 
of charge due to the very small curvature of the tracks becomes the dominant effect.

\begin{figure}
\begin{center}
 \includegraphics[width=0.7\columnwidth]{SS/support_note/chargeflip/misidratept_ZDY_v2.eps}
\caption{Electron charge misidentification rate, obtained from MC simulation using electrons from Z boson decay. 
Contribution from charge misidentification due to photon conversion is shown separately on the plot and as a fraction of the total rate in the ratio plot below.}
\label{fig:chargeFlip_structure}
\end{center}
\end{figure}

To verify the MC estimation of the charge misidentification, a special data-driven technique, the so-called likelihood method, 
is used in the same way as in ref.~\cite{same_sign_paper_7tev}.

% This method is based on using a maximum likelihood fit to extract the charge misidentification rates in different kinematic regions simultaneously~\cite{cf_top_paper,alonso_thesis}.

This method is based on the assumption that charge misidentification rates for different $\eta$ ($p_T$) regions are independent.
Then one can express the number of same-sign pairs in $Z$ peak region, where one electron is in the region $i$ and another one in the region $j$, as:
\begin{equation}
 N_{SS}^{ij} = N^{ij}(\epsilon_i+\epsilon_j)
\end{equation}
where $N^{ij}$ is the number of all electron pairs (where one electron is in the region $i$ and another in the region $j$) and $\epsilon_i (\epsilon_j)$ is the probability of the electron charge misidentification in the region $i (j)$. The number of same-sign pairs, $N_{SS}^{ij}$, is described by the Poisson distribution if one assumes that all same-sign pairs from $Z$ peak region arise from charge misidentification. 
By constructing a likelihood function as a product of Poisson probabilities for all possible combinations of $i$ and $j$ regions and making a maximum logarithmic likelihood fit, with $\epsilon_i$ and $\epsilon_i$ as free parameters, one can obtain misidentification rates for each region.

This method is found to give the most precise estimate as it uses the most of the available statistics and provides kinematically unbiased results
for same-sign dilepton analysis with 7 TeV data ~\cite{anthony_thesis}.
Two alternative methods (direct extraction and tag-and-probe methods)
use a subset of the available statistics for the rate measurements~\cite{cf_top_paper}.
The direct extraction method uses only pairs, in which both electrons are from the same $\eta$ ($p_T$) bin. The tag-and-probe method requires one electron to pass very strict cuts while another one has to pass analysis selection cuts. Meanwhile, the likelihood method uses all pairs which satisfy the signal selection requirements~\cite{alonso_thesis}.
% There are two alternative methods: direct extraction and tag-and-probe methods. Direct extraction method uses only pairs, in which both electrons are from the same $\eta$ ($p_T$) bin. Tag-and-probe method requires one electron to pass very strict cuts while other one has to pass analysis selection cut. Both these methods use a subset of the available statistics for the rate measurements, while likelihood method uses all pairs which satisfy the signal selection requirements.

To cross-check the applicability of the method, it was first applied to reconstructed MC events, and then the charge misidentification rates were compared with those obtained from the generator level information. 
The comparison demonstrates that the likelihood method provides a very reliable result as shown in
\FigureRef{fig:likelihood_cross_check}.

\begin{figure}
\begin{subfigure}{.5\textwidth}
  \centering
  \includegraphics[width=\textwidth]{SS/support_note/chargeflip/eta_mctruth.eps}
\end{subfigure}%
\begin{subfigure}{.5\textwidth}
  \centering
  \includegraphics[width=\textwidth]{SS/support_note/chargeflip/pt_mctruth.eps}
\end{subfigure}
\caption{Electron charge misidentification rate, obtained from MC simulation using generator level information and the likelihood method. 
Ratio between the rate obtained with the likelihood method to the rate obtained with truth MC information is shown below in the ratio plot. }
\label{fig:likelihood_cross_check}
\end{figure}

The charge misidentification rate as a function of $p_T$ and $\eta$ is shown in \FigureRef{fig:charge_flip_data_vs_mc}
for the MC simulation and the likelihood data-driven prediction.
The dependence of the rate on $p_T$ is well described by MC simulation, while some difference is observed
in high-$\eta$ region. This is why $\eta$-dependent correction factors were calculated as a ratio of the charge misidentification rate extracted from collision data to the rate predicted by the MC simulation.

\begin{figure}
\begin{subfigure}{.5\textwidth}
  \centering
  \includegraphics[width=\textwidth]{SS/support_note/chargeflip/misidrate_datamc.eps}
\end{subfigure}%
\begin{subfigure}{.5\textwidth}
  \centering
  \includegraphics[width=\textwidth]{SS/support_note/chargeflip/misidratept_datamc.eps}
\end{subfigure}
\caption{Electron charge misidentification rate, obtained from MC simulation and from real collision data with the likelihood method as a function of $|\eta|$ (lett) and $p_T$ (right).
Ratio plot shows charge misidentification scale factor (SF), which corresponds to the ratio between rates from data and MC simulation.}
\label{fig:charge_flip_data_vs_mc}
\end{figure}

The charge misidentification background consists of opposite-sign lepton pairs which were reconstructed as same-sign pairs.
The background is estimated from MC simulation (all the considered processes are shown in the lower part of the \TableRef{tab:MC_cross}) 
and is corrected by a charge misidentification scale factor to properly reproduce the $\eta$-dependence of the charge misidentification rate.

To estimate the systematic error of the derived scale factors, the following uncertainty sources were considered:
\begin{itemize}
 \item The width of the invariant mass window, used to select electrons from the $Z$ peak, was varied by 10 GeV in both directions.
 The largest difference between the nominal scale factors and the scale factor obtained from the $Z$ peak window width variation was taken as a systematic error.
 \item The electron isolation requirement was loosened by 4 GeV in both track- and calorimeter-based isolation criteria.
\end{itemize}
Electrons from Z decays have a limited $p_T$-range of up to around 100 GeV. As can be seen in~\FigureRef{fig:charge_flip_data_vs_mc} (right), 
the MC simulation describes the $p_T$ dependence of the charge misidentification rate very well, thus it was decided to rely on MC simulation 
for high-$p_T$ leptons. Additional studies were done to estimate systematic uncertainty in this case.
Special MC samples with varied detector alignment and amount of detector material (with variations of 5-20$\%$ depending on the sub-detector) 
were used in order to estimate the effect on the charge misidentification rate.
This resulted in an estimate of a 20$\%$ error, which was assigned to the high-$p_T$ lepton charge misidentification rate.

The charge misidentification scale factor is also applied for the MC prediction of the background from $W\gamma$ process
as the origin of creating a same-sign electron pair is very similar.
In this case one electron in the pair originates from the $W$ decay and the other from $\gamma$ conversion.

The total systematic uncertainty on opposite-sign backgrounds from charge misidentification is 9$\%$, and that on photon misidentificaiton for $W\gamma$ is 13$\%$, making them among the largest single sources of uncertainty as can be seen in \TableRef{tab:syst} at the end of this subsection.

\subsection{Non-prompt background}
\label{subsec:fakes_description}

% TODO comment from Monika
Another type of background present after the signal selection is the so-called non-prompt background.
Main sources of this background are jets misidentified as electrons and electrons which do not originate from 
the primary vertex, e.g. electrons from semi-leptonic decays of heavy flavor quarks ($b$, $c$).
To estimate this background a data-driven method, the so-called fake factor method, is used.

The first step of the methods is to define a background region which does not overlap with the signal region, in which the
contribution of non-prompt electrons is dominant, while the contribution from prompt electrons is minimal.
Events in this region must contain exactly one reconstructed electron (probably a jet misreconstructed as electron) with $p_T > 20$ GeV.
To counterbalance the electron, a jet in the opposite azimuthal direction ($\Delta \phi (e,jet) > 2.4$) is required.
By requiring strictly one electron one can make sure that the background region does not overlap with the signal selection and processes 
like Drell-Yan and \ttbar~will be suppressed.
To make sure that the jet and the non-prompt electron are well balanced in terms of energy, the jet is
required to have $p_T > 30$ GeV.
To suppress contributions from W boson production, a requirement on the transverse invariant mass of 
$m_\mathrm{T}$
\footnote{$m_\mathrm{T} = \sqrt{2 p_\mathrm{T} p_T^{miss} (1-\cos\varphi_{\Plepton\nu})}$, where
$p_\mathrm{T}$ is the transverse momentum of the electron, $p_T^{miss}$ is the missing transverse momentum of the event
and $\varphi_{\Plepton\nu}$ is the angle in transverse plane between electron direction and direction of the missing momentum.}
$ < 40$ GeV is applied.

In this background region the fake factor $f$ is defined as:
\begin{equation}
f = \frac{N_{\mathrm{P}} - N_{\mathrm{P}}^{\mathrm{prompt}}}{N_{\mathrm{F}}  - N_{\mathrm{F}}^{\mathrm{prompt}}}
\label{eq:fakefactor}
\end{equation}
where $N_{\mathrm{P}}$ is the number of reconstructed electrons in the background region which pass the electron signal selection
described above in \SectionRef{subsec:electron_selection}, and $N_{\mathrm{F}}$ is the number of electrons which do not fulfill the 
signal electron selection requirements but satisfy a looser selection. This looser selection is identical to that for the signal selection, except 
that the electron only has to satisfy medium electron identification criteria instead of the tight ones,
and it has to fail the calorimeter- or track-based isolation criterion.
$N_{\mathrm{P}}^{\mathrm{prompt}}$ and $N_{\mathrm{F}}^{\mathrm{prompt}}$ are the numbers (obtained from MC simulation) of real prompt
leptons which pass the signal and looser selections respectively.
The contribution of prompt leptons has to be subtracted in order to make sure that 
the fake factor is evaluated as a ratio of non-prompt electrons that passed the signal selection to
the number of non-prompt electrons that passed the looser selection and there is no contamination from real prompt leptons.
The fake factor measured in this way can then be used to predict the non-prompt background in the signal region.
In order to take into account possible different kinematics of a lepton in the region where the fake factor was derived 
and the region were it will be applied, the fake factor is measured as a function of electron $p_T$ and $\eta$.
The total number of non-prompt same-sign pairs, $N_{\mathrm{NP}}$, in the signal region can be estimated as:
\begin{equation}
N_{\mathrm{NP}} = \sum_{i}^{\mathrm{N_{P_l F_s}}} f_{\mathrm{s}}(p_{\mathrm{Ti}},|\eta_{i}|) + \sum_{i}^{\mathrm{N_{F_l P_s}}} f_{\mathrm{l}}(p_{\mathrm{T}i},|\eta_{i}|) - \sum_{i}^{\mathrm{N_{F_l F_s}}} f_{\mathrm{l}}(p_{\mathrm{T}i},|\eta_{i}|) \times f_{\mathrm{s}}(p_{\mathrm{T}i},|\eta_{i}|)
\label{eq:fake_pred}.
\end{equation}
The first term corresponds to the number of electron pairs ($N_{P_l F_s}$) 
where the leading electron (denoted by ``l'') passes selection criteria ($P_l$) and the subleading electron (denoted by ``s'')
fails to fulfill it ($F_s$) but passes the looser selection used for the fake factor calculation. 
The contribution of every such pair to the signal region is scaled by the fake factor 
$f_{\mathrm{s}}(p_{\mathrm{Ti}},|\eta_{i}|)$, where $p_\mathrm{Ti}$ and $\eta_{i}$ are the transverse momentum and pseudorapidity
of the subleading electron of a pair which fails the signal selection. 
Similarly, the second term represents the number of pairs where
the leading electron fails the signal selection while the subleading one passes it. 
The last term corresponds to the case in which both the leading
and subleading electrons fail the signal selection. This term has to be subtracted to correct for the double
counting of non-prompt electron pairs.

To estimate the systematic errors of the method one can test all the assumptions made and take into account possible differences between 
the region used to derive fake factors and the region in which they are applied. The following sources were considered:
\begin{itemize}
 \item statistical uncertainty of the data sample used to derive fake factors;
 \item prompt MC subtraction, which is done to verify that there is no contamination from prompt leptons when deriving/applying fake factors.
 Due to the luminosity uncertainty of 2.8$\%$ and the uncertainty of MC cross section (7$\%$ on major prompt processes), the prompt MC subtraction
 was varied by a conservative value of 10$\%$;
 \item requirement on the $p_T$ of the jet balancing the lepton. Results were recomputed with it raised to 50 GeV in order to test the dependence of the fake factor on kinematics of the jets faking electrons;
 \item difference in the non-prompt background composition in the region used to derive the fake factors and the region where they were applied.
 Non-prompt background can originate from jets which were created by gluons or light quarks as well as from heavy flavour jets.
 The fake factor depends on the proportion of these two categories of jets in the region. Therefore the fake factors were derived separately
 for heavy and light flavour jets, and the difference between the results was taken as systematic error.
\end{itemize}
A more detailed description of the evaluated systematic uncertainties and the fake factor method can be found in~\cite{anthony_thesis}.

The electron fake factor as a function of electron $p_T$ together with the statistical and systematic errors is
shown in \FigureRef{fig:ff_e_errs}.
\begin{figure}[h]
\begin{center}
%\includegraphics[width=0.48\textwidth]{figures/electrons/electron-syst}
\includegraphics[width=0.65\textwidth]{SS/support_note/electrons/electron_fake_plot.eps}
\caption{Electron fake factor $f$ as a function of electron $p_{T}$. Combined statistical and systematic errors are shown as shaded areas.}
\label{fig:ff_e_errs}
\end{center}
\end{figure} 
% TODO[say that fake factors were also measured not only for signal selection, but also for less strong selections which are used for fake validation regions].

Verification of the fake factor method is done with the help of validation regions (VR) which will be described in \SectionRef{subsec:fake_validation}.

% The fake background includes all background processes where at least one of the leptons is fake. The dominant sources are $W$+jets and QCD multijets with smaller contributions arising from $Z$+jets and $t\bar{t}$. To assess this background, a data-driven method, known as the fake factor method, is employed. This method is used in several ATLAS analyses, particularly diboson measurements and searches\footnotemark. The method employed was covered in great depth in the previous 7 TeV incarnation of this analysis.

% \footnotetext{The following analyses, among others, use the fake factor method: SM $WW \ra \ell\nu\ell\nu$ and $WZ \ra \ell\nu\ell\ell$, $H \ra WW \ra \ell\nu\ell\nu$, and exotics $WZ \ra \ell\nu\ell\ell$ search.}

\section{Background validation regions}
\label{sec:bkg_validation}

In order to make sure that all backgrounds are modelled/predicted correctly in the signal region, one can define and use special validation regions.
These regions should be kinematically close to the signal region but should not overlap with it. 
The first validation region tests the overall normalization of the background prediction to data.
Other described regions are designed to test one given background type at a time, which means that one background type is dominant over all others.

\subsection{Prompt opposite-sign dileptons}

The first validation region is defined by using exactly the same event selection as for the signal region,
but requiring the two leptons to have opposite charges. 
This validation region does not test any specific background.
% The MC-based background estimation is normalized to the luminosity of the data sample.
% This is why overall normalization of the simulation to the data in the opposite-sign region verifies trigger and lepton reconstruction efficiencies.
The MC-based background estimation is normalized only to the luminosity of the data sample, and hence verifies that trigger and lepton reconstruction efficiencies are well modelled.
The correct description of the Z peak shape in data by the MC simulation tests electron energy scale and resolution.
In \FigureRef{fig:OS_CR} the invariant mass of opposite-sign electron pairs is shown.
\TableRef{tab:dilep_isoOS} gives the observed and the expected number of the electron pairs.
Good agreement between data and simulation is observed.

\begin{table}[htbp]
\begin{center}
\begin{tabular}{l|c}

Process & Number of electron pairs \\\hline\hline
%         Drell-Yan	& $ 4701110 \pm 329108 $	\\[+0.05in]
% 	$t\bar{t}$	& $ 14580.8 \pm 874.92 $	\\[+0.05in]
% 	Dibosons	& $ 12210.9 \pm 545.6 $	\\[+0.05in]
% 	Non-prompt	& $ 8321.28 \pm 244.4 $	\\[+0.05in]
% 	$W\gamma$	& $ 243.03 \pm 35.2 $	\\[+0.05in]
% 	MPI	& $ 32.74 \pm 32.74 $	\\[+0.05in]
% 	\hline
% 	Total expectation	& $ 4736500 \pm 329109 $	\\[+0.05in]
% 	\hline
% 	Observation in data	& $ 4895830 $	\\[+0.05in]
        Drell-Yan	& $ 4700000 \pm 330000 $	\\[+0.05in]
	$t\bar{t}$	& $ 14580 \pm 870 $	\\[+0.05in]
	Dibosons	& $ 12210 \pm 540 $	\\[+0.05in]
	Non-prompt	& $ 8320 \pm 240 $	\\[+0.05in]
	$W\gamma$	& $ 243 \pm 35 $	\\[+0.05in]
	MPI	& $ 33 \pm 33 $	\\[+0.05in]
	\hline
	Total expectation	& $ 4730000 \pm 330000 $	\\[+0.05in]
	\hline
	Observation in data	& $ 4895830 $	\\[+0.05in]
	\hline
% 	Agreement ($(N_{exp} - N_{obs})/\sigma_{exp}$) & -0.48 \\[+0.05in]
	Agreement & -0.48 $\sigma$ \\[+0.05in]
% TODO to ask somone about agreement
\hline  
\end{tabular}
\end{center}
\caption{Observed and expected number of lepton pairs for the control region with opposite-sign, isolated leptons.
Agreement between the observed and the expected number of pairs is quoted in the bottom of the table as a fraction of the total uncertainty on the prediction.
} %The significance of the difference between the number of data events observed and that predicted is calculated considering the statistical error on the data and the systematic uncertainty on the prediction.}
\label{tab:dilep_isoOS}
\end{table}

\begin{figure}[h]
\begin{center}
\includegraphics[width=0.65\textwidth]{SS/support_note/dielectrons/crs/OS_mod_v4.pdf}
\caption{Invariant mass of the opposite-sign electron pairs that passed signal selection.
The data is shown as closed circles. The stacked histograms represent the background estimations. 
The last bin is an overflow bin.
}
\label{fig:OS_CR}
\end{center}
\end{figure} 

\subsection{Prompt same-sign dileptons}
% TODO table with final numbers to show how good MC normalization is wrt to data.

The same-sign dilepton prompt background originates predominantly from $WZ$ and $ZZ$ processes, where both Z and W bosons decay leptonically.
In order to check the normalization of these processes, a dedicated validation region is used.
In a fully reconstructed event in which one of these processes took place, one can find at least one same-sign and one opposite-sign lepton pair.
In order to enhance the $WZ$ and $ZZ$ contributions in the validation region, at least three leptons are required in the event, where one lepton pair
has to be a same-sign electron pair and the other one - an opposite-sign same-flavour pair (from $Z$ boson decay). 
The invariant mass of the opposite-sign pair has to be within 10 GeV of the Z boson mass.

The expected and observed numbers of same-sign pairs in this region are listed in \TableRef{tab:promptCR_yields}, 
and the ratios between them are shown in \TableRef{tab:prompt_ratios}. 
The expectations are in good agreement with the observation.

\begin{table*}[htbp]
\begin{center}
\resizebox{\textwidth}{!}{
\begin{tabular}{l|c|c|c|c|c|c|c}
\hline 
Sample & \multicolumn{6}{|c}{Number of electron-electron pairs with  $m(e^{\pm}e^{\pm})$} \\
 & $>15$~GeV & $>100$~GeV & $>200$~GeV & $>300$~GeV & $>400$~GeV & $>500$~GeV & $>600$~GeV \\
\hline \hline
Non-prompt & $49 \pm 14$ & $31.1 \pm 8.1$ & $11.1 \pm 3.0$ & $3.4 \pm 1.3$ & $1.22 \pm 0.72$ & $0.81 \pm 0.63$ & $0.41 \pm 0.44$ \\
\hline
Prompt total & $226 \pm 18$ & $133.8 \pm 9.2$ & $36.7 \pm 3.0$ & $11.6 \pm 1.3$ & $3.44 \pm 0.63$ & $1.15 \pm 0.34$ & $0.38 \pm 0.18$ \\
\hline
$W/ \gamma$ & $0.0 \pm 0.0$ & $0.0 \pm 0.0$ & $0.0 \pm 0.0$ & $0.0 \pm 0.0$ & $0.0 \pm 0.0$ & $0.0 \pm 0.0$ & $0.0 \pm 0.0$ \\
\hline
Charge Flip total & $0.00036 \pm 0.00068$ & $0.0 \pm 0.0$ & $0.0 \pm 0.0$ & $0.0 \pm 0.0$ & $0.0 \pm 0.0$ & $0.0 \pm 0.0$ & $0.0 \pm 0.0$ \\
\hline \hline
Sum of Backgrounds & $275 \pm 23$ & $165 \pm 12$ & $47.9 \pm 4.2$ & $15.0 \pm 1.9$ & $4.65 \pm 0.95$ & $1.96 \pm 0.71$ & $0.78 \pm 0.47$ \\
\hline \hline
Data  & $268 \pm 16$ & $156 \pm 12$ & $46.0 \pm 6.8$ & $14.0 \pm 3.7$ & $6.0 \pm 2.4$ & $3.0 \pm 1.7$ & $1.0 \pm 1.3$ \\
\hline 
\end{tabular}
}
\end{center}
\caption{Expected and observed numbers of pairs for various cuts on the dilepton invariant mass. The uncertainties shown are quadratic sums of the statistical and systematic uncertainties.}
\label{tab:promptCR_yields}
\end{table*}

\begin{table*}[htbp]
\begin{center}
\resizebox{\textwidth}{!}{
\begin{tabular}{c|c|c|c|c|c|c}
\multicolumn{6}{c}{Ratio between observed and expected for $m(e^{\pm}e^{\pm})$} \\
$>15$~GeV & $>100$~GeV & $>200$~GeV & $>300$~GeV & $>400$~GeV & $>500$~GeV & $>600$~GeV \\
\hline \hline
$0.97 \pm 0.09$ & $0.95 \pm 0.10$ & $0.96 \pm 0.17$ & $0.93 \pm 0.27$ & $1.3 \pm 0.6$ & $1.5 \pm 1.0$ & $1.3 \pm 1.9$ \\
\hline \hline
\end{tabular}
}
\end{center}
\caption{Ratio between observed and expected same-sign pairs in the $WZ$ and $ZZ$ control region for various cuts on the dielectron invariant mass. 
The uncertainties account for both statistical and systematic errors.}
\label{tab:prompt_ratios}
\end{table*}

\FigureRef{fig:prompt_CR} shows the invariant mass distribution in the prompt validation region. As was mentioned above, 
the invariant mass $70 < m_{\Plepton\Plepton}< 110$ GeV region has been excluded
to be used for the estimation of the electron charge misidentification background.
The simulation agrees well with data.

\begin{figure}[h]
\begin{center}
%\includegraphics[width=0.48\textwidth]{figures/electrons/electron-syst}
\includegraphics[width=0.65\textwidth]{SS/support_note/PromptCR/ElEl/2isoSS_ee_mll_pr.eps}
\caption{Invariant mass of reconstructed same-sign electron pairs in the prompt background validation region. The last bin is an overflow bin.}
\label{fig:prompt_CR}
\end{center}
\end{figure} 


\subsection{Electron charge misidentification}

As was described above, events with opposite-sign lepton pairs in the final state can be reconstructed as same-sign lepton pairs, 
if the charge of one of the leptons was wrongly identified.
Misidentification probability is well modelled as a function of $p_T$ by MC simulation, 
but $\eta$-dependence has to be corrected by scale factors obtained with a data-driven method.
One can make a sanity check, comparing data from a $Z$ peak window (same-sign pairs with invariant mass $80 < m_{ee} < 100$~GeV)
with MC simulation corrected by the charge misidentification scale factors.

The invariant mass of the same-sign pairs within the $Z$ peak window is shown in \FigureRef{fig:charge_flip_CR_inv_mass}. 
The $p_T$ and $\eta$ distributions for the leading electron are shown in \FigureRef{fig:charge_flip_CR_kinematics}.
A good agreement is observed between the data and MC expectations, which demonstrates correctness of the derived charge misidentification scale factor.

The observed and the expected numbers of electron pairs are also shown in \TableRef{tab:ee_isoSS_Z} for all same-sign electron pairs and separately for positively and negatively charged pairs.

\begin{figure}[h]
\begin{center}
%\includegraphics[width=0.48\textwidth]{figures/electrons/electron-syst}
\includegraphics[width=0.65\textwidth]{SS/paper_draft/2isoSS_ee_mll_ssz.eps}
\caption{Invariant mass of reconstructed same-sign electron pairs in the validation region for the charge misidentification background prediction.
The dominant background contribution arises from electron charge misidentification.
}
\label{fig:charge_flip_CR_inv_mass}
\end{center}
\end{figure} 

\begin{figure}[h]
\begin{subfigure}{.5\textwidth}
  \centering
  \includegraphics[width=\textwidth]{SS/paper_draft/2isoSS_ee_pt1_ssz.eps}
\end{subfigure}%
\begin{subfigure}{.5\textwidth}
  \centering
  \includegraphics[width=\textwidth]{SS/paper_draft/2isoSS_ee_eta1_ssz.eps}
\end{subfigure}
\caption{Leading electron $p_T$ (left) and $\eta$ (right) distributions in the charge misidentification validation region. The last bin is an overflow bin in the left figure.}
  \label{fig:charge_flip_CR_kinematics}
\end{figure}


\begin{table}[htbp]
\begin{center}
\begin{tabular}{l|c}
\hline
Process & Number of $ee$ pairs \\\hline\hline
%
\multicolumn{2}{c}{\textbf{Same-sign $ee$ $Z$ mass window.}} \\\hline 
        Non-prompt      & $200 \pm 110$ \\[+0.05in]
        Charge Flips & $12400 \pm 1300$ \\[+0.05in]
        Prompt Electrons & $143.4 \pm 8.1$ \\[+0.05in]
        $W\gamma$  & $26.8 \pm 5.6$ \\[+0.05in]
            \hline
        Total Prediction & $12700 \pm 1300$ \\[+0.05in]
            \hline
        Data       &       $11793 \pm 110$ \\[+0.05in]
            \hline
        Agreement  &      0.8 $\sigma$ \\[+0.05in]
\hline \hline
\multicolumn{2}{c}{\textbf{Same-sign $e^{+}e^{+}$ $Z$ mass window.}} \\\hline 
        Fakes      & $66 \pm 60$ \\[+0.05in]
        Charge Flips & $6380 \pm 670$ \\[+0.05in]
        Prompt Electrons & $82.0 \pm 5.0$ \\[+0.05in]
        $W\gamma$  & $17.5 \pm 4.0$ \\[+0.05in]
            \hline
        Total Prediction & $6540 \pm 680$ \\[+0.05in]
            \hline
        Data       &        $5908 \pm 77$ \\[+0.05in]
            \hline
        Agreement  &     1.0 $\sigma$ \\[+0.05in]
%
\hline \hline
\multicolumn{2}{c}{\textbf{Same-sign $e^{-}e^{-}$ $Z$ mass window.}} \\\hline 
        Fakes      & $131 \pm 63$ \\[+0.05in]
        Charge Flips & $5990 \pm 630$ \\[+0.05in]
        Prompt Electrons & $61.4 \pm 3.9$ \\[+0.05in]
        $W\gamma$  & $9.4 \pm 2.3$ \\[+0.05in]
            \hline
        Total Prediction & $6190 \pm 630$ \\[+0.05in]
            \hline
        Data       &        $5885 \pm 77$ \\[+0.05in]
            \hline
        Agreement  &     0.5 $\sigma$ \\[+0.05in]
\hline 
\end{tabular}
\end{center}
\caption{Observed and expected numbers of lepton pairs for the control region with same-sign, isolated electrons falling inside the $Z$ mass window. 
The uncertainties of the predictions are combined statistical and systematic ones.
Agreement between observed and expected number of pairs is quoted as a fraction of the total uncertainty on the prediction.
}
\label{tab:ee_isoSS_Z}
\end{table}


\subsection{Non-prompt background validation region}
\label{subsec:fake_validation}
% TODO [explain here how fake factors for validation regions were calculated]

To verify the fake factor method, a set of validation regions is checked. 
These regions have to be as close as possible kinematically to
the nominal signal selection as well as to the looser signal selection, 
which is used in non-prompt background estimation with \EquationRef{eq:fake_pred}.

% TODO make it more clear. Ask Oxana!
Two validation regions were defined, shown in dark blue in a schematic representation in \FigureRef{fig:fake_validation_regions}. 
Both regions are identical to the nominal signal selection (shown in red) 
except with either a weaker identification requirement (denoted as ``VR1''), or a weaker isolation cut (``VR2'').
Looser selections for the validation regions which were used for fake factor calculation are shown as well in lighter blue. Such a design of the validation region provides similar kinematics to the one in the signal region.

\begin{figure}[h]
\begin{center}
%\includegraphics[width=0.48\textwidth]{figures/electrons/electron-syst}
\includegraphics[width=0.7\textwidth]{SS/Fake_region_block_diagram.pdf}
\caption{Schematic representation of the kinematic phase space of non-prompt validation regions with respect to signal region.}
\label{fig:fake_validation_regions}
\end{center}
\end{figure}

% TODO make it more clear. Ask Oxana!
% In total, four different validation regions are defined.
% One validation region consists of same-sign electron pairs, where both electrons pass the signal region isolation requirement, 
% but fail the tight electron identification criteria while passing medium one.
% The second validation region consists of same-sign electron pairs, where both electrons fail the signal isolation requirement
% but pass a looser intermediate isolation cut (loosened by 4 GeV) instead.
% The third and fourth validation regions are identical to the second one but require only the leading (third region) or the subleading (fourth region)
% electron to fail the signal isolation requirement but pass a looser intermediate isolation cut instead.

However, requirements of the VR1 and VR2 regions can be applied either on both electrons in a pair or only on one from a pair.
Thus, in total four different validation regions are used:
\begin{itemize}
 \item ``Medium electron identification'' VR. Corresponds to VR1, when both electrons pass medium identification criteria.
 \item ``Weak isolation on both leptons'' VR. Corresponds to VR2, when both electrons pass weaker isolation criteria compared to the signal one 
 \footnote{``weaker isolation criteria'' corresponds to requirement of the sum of the transverse energies in the EM and hadronic calorimeters around the electron within $\Delta R < 0.2$ to be less than 7 GeV $+ (p_T - 20$ GeV$) \times 0.037$.
 While signal requirement is 3 GeV $+ (p_T - 20$ GeV$) \times 0.037$}.
 \item ``Weak isolation on subleading electron'' VR. Corresponds to VR2, when only subleading electron from a pair pass medium identification criteria, while leading pass signal isolation requirement.
 \item ``Weak isolation on leading electron'' VR. Corresponds to VR2, when leading electron from a pair pass medium identification criteria, while subleading pass signal isolation requirement.
\end{itemize}

\FigureRef{fig:fakeCR_part1} and \FigureRef{fig:fakeCR_part2} show the invariant mass distributions for the validation regions described above. 
The agreement between observation and prediction is generally good.  
\TableRef{tab:ee_fakeCR} shows the expected and the observed numbers of electron pairs. 
The uncertainties quoted are statistical only. 

\begin{table}[htbp]
    \centering
    \resizebox{\textwidth}{!}{
    \begin{tabular}{ l | r r r r r r r }
        \hline
        Validation region 		& Fakes 			& Prompt 	& Charge Flip 			& W$\gamma$ 		& Total Pred 			& Data		&  Agreement($\sigma$) \\
            \hline

        Medium electron identification 	& $ 111.04 \pm 27.4 $	&	$ 2.9 \pm 0.5 $	& $ 72.46 \pm 16.75 $		& $ 8.78 \pm 2.3 $	& $ 195.18 \pm 32.2 $		& $ 217 \pm 15 $&  -0.62\\
        Weak isolation on both electrons 	& $ 252.9 \pm 133.64 $	& $ 1.23 \pm 0.3 $	& $ 29.07 \pm 10.1 $		& $ 0.27 \pm 0.28 $	& $ 283.47 \pm 134.02 $		& $285 \pm 17 $	&  -0.01\\
        Weak isolation on subleading electron 	& $ 519.21 \pm 120.72 $ & $ 32.88 \pm 2.14 $	& $ 52.69 \pm 14.87 $		& $ 17.64 \pm 4.32 $	& $ 622.42 \pm 121.72 $		& $574 \pm 24 $ &  0.39\\
        Weak isolation on leading electron 	& $ 154.97 \pm 58.67 $  & $ 13.28 \pm 1.21 $	& $ 15.96 \pm 7.5 $		& $ 5.12 \pm 1.72 $	& $ 189.33 \pm 59.19 $		& $ 224 \pm 15 $&  -0.57\\
        
        \hline
    \end{tabular}
    }
\caption{Expected and observed numbers of electron pairs for the different same-sign $ee$ fake control regions. 
The uncertainties on the predictions include the statistical and systematic uncertainties (fake factor and charge misidentification 
uncertainties have been included; other systematic uncertainties are negligible in these regions).
Agreement between observed and expected number of pairs is quoted in the last column of the table as a fraction of the total uncertainty on the prediction.
% For the fake predictions, a systematic uncertainty derived for the signal region is assumed.
}
\label{tab:ee_fakeCR}
\end{table}



\begin{figure}
\begin{subfigure}{.5\textwidth}
  \centering
  \includegraphics[width=\textwidth]{SS/support_note/dielectrons/crs/dec13_fake_medium_v2.eps}
\end{subfigure}%
\begin{subfigure}{.5\textwidth}
  \centering
  \includegraphics[width=\textwidth]{SS/support_note/dielectrons/crs/dec13_fake_bothInter_v2.eps}
\end{subfigure}
\caption{
% Invariant $e^{\pm}e^{\pm}$ mass distributions for non-prompt background prediction with ``medium'' electron identification (left) and with weak isolation on both electrons (right).
% The hatched areas show the statistical uncertainty of the background prediction.
Invariant $e^{\pm}e^{\pm}$ mass distribution
plots of prediction and data in the VR with both electron passing medium electron identification criteria (left) and in the VR passing weaker isolation (right).
The hatched areas show the statistical uncertainty of the background prediction.
}
  \label{fig:fakeCR_part1}
\end{figure}

\begin{figure}
\begin{subfigure}{.5\textwidth}
  \centering
  \includegraphics[width=\textwidth]{SS/support_note/dielectrons/crs/dec13_fake_leadNom_sublInter_v2.eps}
\end{subfigure}%
\begin{subfigure}{.5\textwidth}
  \centering
  \includegraphics[width=\textwidth]{SS/support_note/dielectrons/crs/dec13_fake_leadInter_sublNom_v2.eps}
\end{subfigure}
\caption{
% Invariant $e^{\pm}e^{\pm}$ mass distributions for non-prompt background prediction with weak isolation on the subleading electron (left) and on the leading electron (right).
% The hatched areas show the statistical uncertainty of the background prediction.
Invariant $e^{\pm}e^{\pm}$ mass distribution
plots of prediction and data in the VR with subleading electron passing weaker isolation (left) and in the VR with leading electron passing weaker isolation (right).
The hatched areas show the statistical uncertainty of the background prediction.
}
  \label{fig:fakeCR_part2}
\end{figure}



\section{Systematic Uncertainties}
\label{sec:ss_Systematics}
% TODO [add Fast/Full sim. systematics]

A set of possible systematic sources which can affect background predictions were studied.
These sources are presented below.
Systematic uncertainties related to the data-driven methods for non-prompt and charge misidentification background estimations
were described already in \SectionRef{subsec:fakes_description} and \SectionRef{subsec:CF_definition}, respectively.

\subsection{Electron reconstruction}
\label{subsec:elec_reco_system}
Several systematic uncertainties are related to the electron reconstruction procedure.
These uncertainties are provided by the ATLAS e/gamma working group which studies the electron and photon identification performance of the ATLAS detector.
They provide recommendations and uncertainty estimations for all physics analyses which use electron or photon final states.

Electron reconstruction and tight identification criteria efficiencies are obtained with so-called tag-and-probe data-driven method.
This method allows one to measure from data the efficiency of a studied electron selection using $Z \to e e$ and $J/\psi \to e e$ resonance decays. 
% TODO check following question (was corrected by Peter)
One electron from a pair is selected by requiring very strict criteria, 
while the second one is required to pass the selection in the analysis.
By counting the number of pairs from the resonance decays (by fitting the invariant mass resonance peak) which were selected or 
were rejected due to the cut one can extract the efficiency. 
Detailed information on the method can be found, for example, in ref.~\cite{tag-and-probe}.
The reconstruction efficiency uncertainty range is between 1.3-2.4\% depending on $\eta$, while the 
tight identification criteria efficiency uncertainty range is between 2.0-2.8\% depending on both $p_T$ and $\eta$~\cite{electron_reco_id_2011}.

Reconstruction of the electron energy is optimized using multivariate algorithms.
The electron energy scale and energy resolution are obtained using electrons from Z boson decays.
Their uncertainties are provided as a function of $p_T$ and $\eta$~\cite{electron_energy_errors_Run1} 
by the working group as well.

The total effect of these uncertainties on the total background prediction 
is shown under the name ``Electron reconstruction and identification'' in \TableRef{tab:syst}.

\subsection{Trigger and luminosity}
% TODO I just put trigger stuff (why it is >2%?) under the carpet. Figure it out for the defence!!!
The electron trigger efficiency varies with $p_T$ and $\eta$ and is measured with respect to the offline identification. This uncertainty is estimated by the ATLAS trigger group.
% This uncertainty is estimated to be at the $\sim1\%$ level by the ATLAS trigger groups.
% The resulting uncertainty on the yield in the signal region is different, since there are two leptons that can pass the trigger requirement. 

To scale the background prediction obtained with MC simulation to the data 
one has to know the integrated luminosity of the collected data sample. 
Therefore the luminosity uncertainty propagates to all the backgrounds measured using MC simulation.
The uncertainty on the total integrated luminosity in 2012 is equal to 2.8$\%$~\cite{Aad:2013ucp} 
and it was obtained in a similar way to that described in \SectionRef{sec:lucid_performance}.

\subsection{Statistics and theoretical cross section}
% TODO describe how much statistics (wrt data lumi) samples had!
The limited number of simulated events in the Monte Carlo samples leads to additional uncertainty which is listed as ``MC statistics'' in \TableRef{tab:syst}.
This uncertainty also includes the effect of the limited number of events in data sets used in data-driven methods to measure the charge misidentification rate and the fake factor.
The statistical uncertainty is significant in the high-mass region.

% TODO describe who did all this tests and variations! Who it was???
As one can see from \TableRef{tab:MC_cross}, different processes were simulated using different MC generators, PDF sets and level of perturbative higher order calculations.
Additional uncertainty arises from  their choice.
To estimate these uncertainties, different MC generators, parton showers and hadronization models are tested.
Uncertainties resulting from the choice of PDF and the value of the strong coupling constant $\alpha_{\mathrm{s}}$ are
estimated by using different PDF sets following recommendations from~\cite{pdf4lhc}.
Also renormalization and factorization scales are varied by a factor of two to estimate the effect on the cross section and the selection efficiency.
The summary list of uncertainties used in the analysis is shown in \TableRef{tab:systematics_common}.
Detailed information about cross section calculations and their errors for some processes 
are reported in~\cite{diboson_cross_section,ttW_cross_section,ttV_cross_section}.

\begin{table}[ht]
\begin{center}
\begin{tabular}{l|l|c}
Source & Process & Uncertainty \\
\hline
\multirow{2}{*}{Trigger} & Signal and background & \multirow{2}{*}{2.1-2.6\%}  \\
& from MC simulations &\\
\hline
Electron reconstruction & Signal, prompt &\multirow{2}{*}{1.9--2.7\%}\\
and identification & background&\\\hline
Electron charge  & Opposite-sign& \multirow{2}{*}{9\%} \\
misidentification& backgrounds&\\\hline
Determination of & Non-prompt &\multirow{2}{*}{22\%}\\
fake factor $f$& backgrounds&  \\\hline
\multirow{2}{*}{Luminosity} & Signal and background& \multirow{2}{*}{2.8\%}\\
& from MC simulations&\\\hline
\multirow{2}{*}{MC statistics} & Backgrounds from &  \multirow{2}{*}{5\%}\\
& MC simulations &\\\hline
%Differences between fast & \multirow{2}{*}{Signal}& \multirow{2}{*}{1.8\%}& \multirow{2}{*}{5\%}& \multirow{2}{*}{0.7\%}\\
%and normal simulation &&&&\\\hline
Photon misidentification & \multirow{2}{*}{$W\gamma$} & \multirow{2}{*}{13\%}\\
as electron&&\\\hline
\multirow{2}{*}{MC cross sections} & Prompt, opposite-& \multirow{2}{*}{4\%}\\
& sign backgrounds & \\
\end{tabular}
\end{center}
\caption{Sources of systematic uncertainty (in \%) of the signal yield and the expected background predictions, described in the second column, for the mass range $m_{ee} > 15$ GeV.
}
\label{tab:syst}
\end{table}


\begin{table}[ht]
\begin{center}
\begin{tabular}{l|c}
Processes affected & Uncertainty \\
\hline
 Drell-Yan (Charge flips) & $\pm$7\% \\
 $WZ$ & $\pm$7\% \\
 $ZZ$ & $\pm$5\% \\
 $t\bar{t}W$, $t\bar{t}Z$  & $\pm$22\% \\
 $W^{\pm}W^{\pm}$ & $\pm$50\% \\
 MPI $WW$, $WZ$, $ZZ$ & $\pm$100\% \\
 $t\bar{t}$ & $\pm$5\% \\
 $W\gamma$ & $\pm$14\% \\
\end{tabular}
\end{center}
\caption{Theoretical uncertainties of the production cross section of SM processes modelled by MC.}
\label{tab:systematics_common}
\end{table}



\section{Signal Region}
\label{sec:ss_signalRegion}

The same-sign electron pair invariant mass in the signal region is shown in~\FigureRef{fig:signal_mass}.
% The last bin in the histogram is an overflow bin, which includes pairs with invariant mass higher than 600 GeV.
The observed number of pairs is compatible with the predicted background.
As one can see, the dominant background arises from the charge misidentification component.
The predicted contributions from each background process with different invariant mass cuts are shown in \TableRef{tab:2iso_ee_SS}.
In \FigureRef{fig:signal_kinematics} the kinematics of the leading lepton are shown. 
In \FigureRef{fig:delta_phi} the angle between same-sign electrons in the pair is shown as well.
The background prediction describes the observed numbers of all of these distributions reasonably well within the uncertainty bands.

\begin{figure}[h]
\begin{center}
\includegraphics[width=0.7\textwidth]{SS/paper_draft/2isoSS_ee_mll.eps}
\caption{Invariant mass distribution for $e^{\pm}e^{\pm}$ pairs in the signal region. 
The shaded band in the lower plot corresponds to the combination of the statistical and systematic uncertainties of the background prediction.
The last bin is an overflow bin.}
\label{fig:signal_mass}
\end{center}
\end{figure}

\begin{table*}[htbp]
\begin{center}
\resizebox{\textwidth}{!}{
\begin{tabular}{l|c|c|c|c|c|c|c}
\hline
Sample & \multicolumn{5}{|c}{Number of electron pairs with  $m(e^{\pm}e^{\pm})$} \\
 & $>15$~GeV & $>100$~GeV & $>200$~GeV & $>300$~GeV & $>400$~GeV & $>500$~GeV & $>600$~GeV \\
\hline \hline
Non-prompt	& $ 518.57 \pm 120.17 $	& $ 247.49 \pm 49.5 $	& $ 71.67 \pm 13.15 $	& $ 22.66 \pm 4.8 $	& $ 8.13 \pm 2.42 $	& $ 3.12 \pm 1.49 $	& $ 0.78 \pm 1.01 $	\\[+0.05in]
\hline\hline
$W\gamma$	& $ 175.25 \pm 36.28 $	& $ 74.89 \pm 15.62 $	& $ 22.42 \pm 5.15 $	& $ 8.04 \pm 2.26 $	& $ 3.84 \pm 1.31 $	& $ 2.69 \pm 1.05 $	& $ 1.02 \pm 0.57 $	\\[+0.05in]
\hline\hline
Drell-Yan	& $ 968.61 \pm 145.63 $	& $ 513.53 \pm 77.7 $	& $ 130.91 \pm 26.99 $	& $ 36.1 \pm 12.17 $	& $ 12.8 \pm 7.89 $	& $ 4.79 \pm 4.86 $	& $ 4.79 \pm 4.86 $	\\[+0.05in]
$t\bar{t}$	& $ 36.92 \pm 6.01 $	& $ 30.1 \pm 4.99 $	& $ 14.55 \pm 2.8 $	& $ 5.05 \pm 1.32 $	& $ 2.15 \pm 0.78 $	& $ 1.05 \pm 0.58 $	& $ 1.18 \pm 0.56 $	\\[+0.05in]
$WW$	& $ 13.01 \pm 2.34 $	& $ 10.74 \pm 1.96 $	& $ 4.85 \pm 0.97 $	& $ 1.86 \pm 0.45 $	& $ 0.68 \pm 0.22 $	& $ 0.43 \pm 0.16 $	& $ 0.28 \pm 0.13 $	\\[+0.05in]
\hline
Charge Flip total	& $ 1018.54 \pm 145.78 $	& $ 554.37 \pm 77.89 $	& $ 150.31 \pm 27.16 $	& $ 43.01 \pm 12.25 $	& $ 15.62 \pm 7.93 $	& $ 6.27 \pm 4.89 $	& $ 6.25 \pm 4.89 $	\\[+0.05in]
\hline\hline
$ZZ$	& $ 86.05 \pm 7.21 $	& $ 22.42 \pm 2.11 $	& $ 6.75 \pm 0.84 $	& $ 1.78 \pm 0.37 $	& $ 0.61 \pm 0.2 $	& $ 0.34 \pm 0.16 $	& $ 0.21 \pm 0.12 $	\\[+0.05in]
$WZ$	& $ 234.36 \pm 22.24 $	& $ 132.79 \pm 12.76 $	& $ 37.12 \pm 3.9 $	& $ 10.95 \pm 1.43 $	& $ 3.23 \pm 0.61 $	& $ 1.5 \pm 0.4 $	& $ 0.5 \pm 0.22 $	\\[+0.05in]
$t\bar{t}W$	& $ 5.33 \pm 1.23 $	& $ 3.83 \pm 0.89 $	& $ 1.32 \pm 0.32 $	& $ 0.44 \pm 0.11 $	& $ 0.14 \pm 0.04 $	& $ 0.08 \pm 0.03 $	& $ 0.03 \pm 0.01 $	\\[+0.05in]
$t\bar{t}Z$	& $ 1.73 \pm 0.41 $	& $ 1.2 \pm 0.29 $	& $ 0.4 \pm 0.1 $	& $ 0.11 \pm 0.04 $	& $ 0.03 \pm 0.01 $	& $ 0.02 \pm 0.01 $	& $ 0.01 \pm 0.01 $	\\[+0.05in]
$WWjj$	& $ 14.99 \pm 7.59 $	& $ 12.1 \pm 6.14 $	& $ 5.55 \pm 2.84 $	& $ 2.35 \pm 1.22 $	& $ 1.22 \pm 0.66 $	& $ 0.4 \pm 0.24 $	& $ 0.16 \pm 0.11 $	\\[+0.05in]
MPI	& $ 4.04 \pm 4.06 $	& $ 1.6 \pm 1.61 $	& $ 0.38 \pm 0.39 $	& $ 0.06 \pm 0.07 $	& $ 0.02 \pm 0.02 $	& $ 0 \pm 0 $	& $ 0 \pm 0 $	\\[+0.05in]
\hline
Prompt total	& $ 346.51 \pm 24.95 $	& $ 173.94 \pm 14.44 $	& $ 51.52 \pm 4.93 $	& $ 15.7 \pm 1.92 $	& $ 5.25 \pm 0.92 $	& $ 2.34 \pm 0.49 $	& $ 0.91 \pm 0.28 $	\\[+0.05in]
\hline\hline
Total Background	& $ 2058.86 \pm 193.92 $	& $ 1050.69 \pm 94.67 $	& $ 295.92 \pm 30.99 $	& $ 89.41 \pm 13.49 $	& $ 32.83 \pm 8.44 $	& $ 14.41 \pm 5.25 $	& $ 8.96 \pm 5.04 $	\\[+0.05in]
\hline\hline
Data	& $ 1976 $	& $ 987 $	& $ 265 $	& $ 83 $	& $ 30 $	& $ 13 $	& $ 7 $	\\[+0.05in]

\hline
\end{tabular}
}
\end{center}
\caption{Expected and observed numbers of pairs of isolated same-sign electrons for various cuts on the dielectron invariant mass, \mee. The uncertainties shown include statistical and systematic contributions.}
\label{tab:2iso_ee_SS}
\end{table*}


\begin{figure}
\begin{subfigure}{.5\textwidth}
  \centering
  \includegraphics[width=\textwidth]{SS/paper_draft/2isoSS_ee_pt1.eps}
\end{subfigure}%
\begin{subfigure}{.5\textwidth}
  \centering
  \includegraphics[width=\textwidth]{SS/paper_draft/2isoSS_ee_eta1.eps}
\end{subfigure}
\caption{$p_T$ and $\eta$ distributions of the leading electron in the signal region. The last bin is an overflow bin in the left figure.}
  \label{fig:signal_kinematics}
\end{figure}

\begin{figure}[h]
\begin{center}
\includegraphics[width=0.7\textwidth]{SS/support_note/dielectrons/crs/signal_deltaPhi_v2.eps}
\caption{Azimuthal angle difference between two same-sign electrons from the pair in the signal region. The uncertainties shown include statistical and systematic contributions.}
\label{fig:delta_phi}
\end{center}
\end{figure}



\subsection{Limit setting}

The background prediction describes data very well and there are no significant visible deviations which could indicate a potential presence of the BSM signal.
The idea of this analysis is to be as general as possible and to perform a search for new physics without favoring any BSM model.
Thus, the next step is to calculate exclusion limits on any type of new physics with prompt same-sign lepton pairs.

Due to the limited acceptance of the detector the signal region was designed to use the phase space which is used in ATLAS for precision measurements.
This restricted acceptance results in the following relation between total, $\sigma$, and measured, or fiducial, $\sigma^{fid}$, cross-sections:
% The total cross section, $\sigma$, of the process relates with the measured cross section in the signal region, the so-called fiducial cross section $\sigma^{fid}$, as:
\begin{equation}
 \sigma = \dfrac{\sigma^{fid}}{\langle N_{pair} \rangle A}
 \label{eq:cross_section}
\end{equation}
where $\langle N_{pair} \rangle$ is the average number of same-sign pairs produced per event and $A$ is the fiducial acceptance (or volume).
The definition of the fiducial volume is discussed in \SectionRef{subsec:fid_volume_eff}.

The cross section limits are derived using a $CL_s$~\cite{CLs_tecnique,CLs_2} prescription with the help of the RooStat~\cite{RooStat_project} framework 
provided by the ATLAS Statistics Committee. The $CL_s$ method states that the signal hypothesis is excluded at the confidence level $CL$ when
\begin{equation}
 1 - CL_s \leq CL
\end{equation}
where $CL_s$ is defined as
\begin{equation}
 CL_s \equiv \dfrac{CL_{s+b}}{CL_b}
\end{equation}
where $CL_b$ is a confidence level observed for the background-only hypothesis and $CL_{s+b}$ for the background plus signal hypothesis.
In practise, $CL_b$ ($CL_{s+b}$) is a probability to find the observed data given an expected background (background plus signal).
This probability is Poisson-distributed and is calculated based on the number of observed and expected same-sign electron pairs in the signal region.
The test statistic used for the limit setting is a log-likelihood-ratio test.
Systematic uncertainties and their correlations are taken into account with this method.
For example, the charge misidentification scale factor uncertainty is correlated across Drell-Yan, $t\bar{t}$, $WW$ and $W\gamma$ background samples.
Experimental errors such as electron reconstruction, identification, energy scale and trigger are each treated as a single shared parameter across all the expected backgrounds and signal.
The MC cross section errors are independent except for the diboson ($ZZ$, $WZ$ and $WW$) samples.
The luminosity is common to all the background samples. The statistical errors are independent.

Following this prescription and using the number of expected and observed same-sign lepton pairs one can compute 
upper limits at a given confidence level (typically at 95$\%$ level) on the number of same-sign lepton pairs ($N_{95}$)
arising from new physics beyond the SM. Limits can be set for different invariant mass thresholds, 
because the dilepton invariant mass is the main observable in the analysis.
Limits on the number of pairs can be translated into upper limits on the fiducial cross section as:
\begin{equation}
 \sigma_{95}^{fid} = \dfrac{N_{95}}{\epsilon_{fid} \times \int \mathscr{L} dt}
 \label{eq:fid_cross_section}
\end{equation}
where $\int \mathscr{L} dt$ is the integrated luminosity of the data and $\epsilon_{fid}$ is a fiducial efficiency for finding a same-sign electron pair from
a possible signal from new physics in the fiducial volume, which is described in the next section.

% TODO explain fiducial efficiency

\subsection{Fiducial volume and fiducial efficiency}
\label{subsec:fid_volume_eff}

As can be seen from \EquationRef{eq:cross_section} and~(\ref{eq:fid_cross_section}), in order to translate the number of measured and expected lepton pairs 
to the cross section limit of a signal from new physics, one has to know the fiducial volume and efficiency.
The reason is that the detector does not reconstruct leptons with a 100$\%$ efficiency, and it does not cover the whole solid angle around the interaction point.
The fiducial volume represents the phase space region which is truncated so that it mimics the detector acceptance.
It is defined by a set of cuts on the truth (generator) level. 
Kinematic cuts on the electrons are identical to the one used in the signal region definition on reconstruction level:
\begin{itemize}
 \item Leading electron $p_T > 25$ GeV
 \item Subleading electron $p_T > 20$ GeV
 \item $|\eta|<1.37$ or $1.52<|\eta|<2.47$
\end{itemize}
Requirements on the electron pair are the same as well:
\begin{itemize}
 \item Same-sign pair with $m_{ee} > 15$ GeV
 \item Veto pairs with $70 < m_{ee} < 110$ GeV
 \item No opposite-sign same-flavour pairs with $|m_{ee} - m_{Z}| < 10$ GeV
\end{itemize}
Since electrons are required to be isolated on reconstruction level, isolation has to be applied on the truth level as well.
Track-based isolation on truth level is identical to that on reconstruction level:
all charged particles within the cone $\Delta R < 0.3$ around the electron with $p_T > 0.4$ GeV are considered and the
sum of their $p_T$ has to be smaller than 10$\%$ of the electron $p_T$.
Calorimeter-based isolation was not applied on the truth level due to a significantly different behaviour at reconstruction and truth levels.

In the case of an ideal detector, fiducial volume would correspond to the geometrical detector acceptance.
Since the real detector does not provide a 100$\%$ reconstruction and identification efficiency, a fiducial efficiency is used in order to relate the fiducial volume with the real geometrical detector acceptance, which is defined as:
\begin{equation}
 \epsilon_{fid} = \dfrac{N_r}{N_f}
\end{equation}
where $N_f$ is the number of electron pairs which pass the fiducial volume cuts on the truth level and $N_r$ - which pass the fiducial volume cuts on the truth level 
as well as all the signal selection cuts on the reconstruction level.

In order to perform a search for new physics in a model-independent way, the fiducial efficiency has to be constant and to not depend on the type of the BSM model.
However, different models provide different $p_T$ and $\eta$ spectra, and the electron reconstruction efficiency depends on both $p_T$ and $\eta$.
As reported in ref.~\cite{electron_tight}, the efficiency can vary up to 15$\%$ for the tight identification criteria with respect to the electron $p_T$.
Also, the presence and number of jets in the final state is model dependent, and affects the electron isolation efficiency, which will have an effect on 
the fiducial efficiency.

In order to estimate the value of the fiducial efficiency, efficiencies for four different BSM models were calculated:
\begin{itemize}
 \item Doubly charged Higgs. This model assumes production of two doubly charged scalar bosons which, decaying leptonically, will provide two pairs of same-sign leptons.
 No jets are produced in the final state of the hard process. The Higgs mass was varied between 100 GeV and 1000 GeV.
 \item Colored Zee-Babu model. Diquarks are produced, which decay to two leptoquarks with the same charge, 
 which subsequently decay to a lepton and a quark. The final state consists of one same-sign lepton pair and two jets.
 The masses considered in the model are 2.5-3.5 TeV for the diquark and 1-1.4 TeV for leptoquarks.
 \item Production of a heavy right-handed $W_R$ boson and a heavy Majorana neutrino. $W_R$ decays to a lepton and a Majorana neutrino, 
 which decays to a $W$ boson and another lepton. The final state consists of one same-sign pair and products from the W boson decay.
 The mass of $W_R$ was varied between 1 TeV and 2 TeV, while the mass of the Majorana neutrino was in the range of 0.25-1.5 TeV
 \item Pair production of a fourth generation down-type quark. Both quarks decay semi-leptonically to a $t$ quark and subsequently to a $b$ quark.
 The final state consists of two jets and four W bosons. At least two same-sign bosons have to decay leptonically in order to provide a same-sign lepton pair.
 This model is characterized by large hadronic activity due to the high jet multiplicity.
 The mass of the fourth generation quark was varied from 400 GeV to 1 TeV.
\end{itemize}
The fiducial efficiencies for these models were calculated with different dilepton mass thresholds.
The obtained efficiencies are in the range of 48-74$\%$. The lowest efficiency was observed for the fourth generation down-type quark model, while the highest one is found for the heavy right-handed $W_R$ boson and heavy Majorana neutrino process. The latter model has larger efficiency with respect to the doubly charged Higgs model
because the final state of the doubly charged Higgs model contains two same-sign pairs compared to one pair in the Majorana model.
The efficiencies were measured separately for positive and negative same-sign pairs, and no significant differences were observed.

To provide a conservative cross section limit setting for new physics beyond the SM, the lowest obtained efficiency, which was 48.3$\%$, was used.

\subsection{Fiducial cross section limits}

Computed upper limits at the 95$\%$ confidence level on the fiducial cross section ($\sigma_{95}^{fid}$)
of new physics beyond the SM for the invariant mass thresholds used in \TableRef{tab:2iso_ee_SS}
are shown in \FigureRef{fig:inclusive_fid_limit}. Separate limits for positive and negative same-sign pairs are shown in \FigureRef{fig:signal_kinematics_v2}.
The expected limits are shown together with the 2$\sigma$ uncertainty bands. The limits are summarized in \TableRef{tab:limits}.

\begin{figure}[h]
\begin{center}
\includegraphics[width=0.7\textwidth]{SS/paper/limit_ee_all.eps}
\caption{Fiducial cross section upper limits at 95\% C.L. for new physics contributing to the signal region for events with $e^{\pm}e^{\pm}$ pairs.
Green and yellow bands correspond to the 1$\sigma$ and 2$\sigma$ uncertainty bands on the expected limits respectively.}
\label{fig:inclusive_fid_limit}
\end{center}
\end{figure}


\begin{figure}
\begin{subfigure}{.5\textwidth}
  \centering
  \includegraphics[width=\textwidth]{SS/support_note/limits/limit_ee_neg.eps}
\end{subfigure}%
\begin{subfigure}{.5\textwidth}
  \centering
  \includegraphics[width=\textwidth]{SS/support_note/limits/limit_ee_pos.eps}
\end{subfigure}
\caption{Fiducial cross section upper limits at 95\% C.L. for new physics contributing to the signal region
for events with $e^{+}e^{+}$ (left) and $e^{+}e^{+}$ (right) pairs. Green and yellow bands correspond to the 1$\sigma$ and 2$\sigma$ uncertainty bands on the expected limits respectively.}
  \label{fig:signal_kinematics_v2}
\end{figure}


\begin{table*}[!ht]
\begin{center}
\begin{tabular}{c||c|c||c|c||c|c}

 & \multicolumn{6}{c}{95\%  CL upper limit [fb]} \\
 & \multicolumn{2}{c||}{$e^{\pm}e^{\pm}$} & \multicolumn{2}{c||}{$e^{+}e^{+}$} & \multicolumn{2}{c}{$e^{-}e^{-}$} \\
Mass range & Expected & Observed & Expected & Observed & Expected & Observed \\
%[+0.05in]
\hline
\rule{0pt}{3ex}
  $>15$~GeV   &  $39^{+10}_{-13}$        &  32    &    $27^{+11}_{-6}$         &  28    &    $23^{+8}_{-5}$          &  19\\
  $>100$~GeV  &  $19^{+6}_{-6}$          &  14    &    $14.3^{+5.4}_{-2.8}$    &  13.5  &    $10.8^{+4.4}_{-2.4}$    &  9.0\\
  $>200$~GeV  &  $6.8^{+2.6}_{-1.7}$     &  5.3   &    $5.4^{+2.0}_{-1.4}$     &  4.6   &    $3.9^{+1.4}_{-1.2}$     &  3.5\\
  $>300$~GeV  &  $3.3^{+1.3}_{-0.4}$     &  3.3   &    $2.5^{+0.9}_{-0.6}$     &  2.0   &    $2.1^{+0.7}_{-0.5}$     &  2.6\\
  $>400$~GeV  &  $2.02^{+0.74}_{-0.21}$  &  2.03  &    $1.59^{+0.47}_{-0.34}$  &  1.64  &    $1.56^{+0.41}_{-0.31}$  &  1.35\\
  $>500$~GeV  &  $1.25^{+0.36}_{-0.26}$  &  1.10  &    $1.44^{+0.34}_{-0.36}$  &  1.55  &    $0.69^{+0.27}_{-0.17}$  &  0.64\\
  $>600$~GeV  &  $0.99^{+0.34}_{-0.20}$  &  1.02  &    $1.27^{+0.37}_{-0.26}$  &  1.10  &    $0.58^{+0.21}_{-0.08}$  &  0.61\\

\end{tabular}
\end{center}
 \caption{Upper limit at 95\% CL on the fiducial cross-section for $e^{\pm} e^{\pm}$ pairs from non-SM signals. 
 The expected limits and their $1 \sigma$ uncertainties are given together with the observed limits derived from the data. 
 Limits are given inclusively and separated by charge.}
\label{tab:limits}
\end{table*}


\section{Mass limits of doubly charged Higgs}
As an example of a BSM model which produces same-sign lepton pairs in the final state, the pair production of doubly charged Higgs bosons was studied.
The search strategy is the same as described above.
Since the final state of this model has two same-sign pairs of leptons, no jet activity and no missing transverse energy is present in the event.
Therefore, there is no need to optimize the signal selection.
Doubly charged Higgs decays should be visible as a sharp peak in the dilepton invariant mass.
Using a fiducial efficiency calculated in bins of 100 GeV (as it was done for the fiducial limit calculations) is not optimal from the point of view of the signal sensitivity.
Thus, the search is performed in mass bins with a mass-dependent width.

\subsection{MC simulation}
Signal samples were generated with Pythia8~\cite{pythia8} with different masses of the left-handed and right-handed doubly charged Higgs bosons.
The simulated masses were produced in the range of 50-600 GeV in steps of 50 GeV with one additional mass point at 1 TeV.
The kinematics of left- and right-handed Higgs bosons is identical, but the production rate is different due to different coupling to the Z boson mediator~\cite{dch_note}. The cross sections were calculated with NLO precision. 
% TODO [link] for NLO cross section

\subsection{Model acceptance and efficiency}
The width of the doubly charged Higgs decay is dominated by the detector momentum resolution of the electrons. Since the decay width depends on the doubly charged Higgs mass, the search was performed using mass bins with variable width.
% The idea behind the mass bin widths optimization is based on two facts. 
For an optimal bin width, two competing factors have to be considered.
On one hand, a mass bin has to cover as much signal as possible,
while on the other hand, the background contribution in the mass bin is desired to be as small as possible.
To satisfy both conditions, the signal significance, $S$, was chosen as an optimization criterion:
% TODO Peter: explaine ued significane formula
\begin{equation}
 S = \sqrt{ 2((s+B)ln(1+s/B)-s) } 
\end{equation}
where $s$ is the expected signal and $B=b+\delta b^2$ is the predicted background plus background systematic uncertainty squared.
A bin width is parameterized as a second degree polynomial of the Higgs mass.
Coefficients of the polynomial were the parameters to optimize.
During the optimization procedure it became clear that there is a third effect which has to be taken into account.
Due to limited statistics of the predicted background, one cannot have too many mass bins, otherwise the cross section
exclusion limit will fluctuate significantly from bin to bin. 
% The optimal bin width parameterization was found to be $\pm(0.04 \times m(\dch) + 0.2 \cdot 10^{-4} \times m(\dch)^{2})$,
% where $m(\dch)$ is the Higgs mass.

The next step is to define how many lepton pairs produced in Higgs boson decays are reconstructed, selected and fall into the mass bin.
The number of generated Higgs bosons is known, 
thus one only needs to count number of reconstructed same-sign pairs which pass the signal selection in a given mass bin.
The ratio of reconstructed to the total number of generated pairs corresponds to the total efficiency, which includes the effect of the acceptance and efficiency of the signal selection.
The efficiencies for each mass point were calculated, but in order to interpolate between the simulated mass points, the total efficiency $\varepsilon_{tot}$ is fitted by
an empirical piecewise function:
\begin{equation}
\varepsilon_{tot}(m) = \begin{cases} p_{0} (1-e^{-(m-p_{1})/p_{2}}) & ,\mbox{if } m < 450\mbox{ GeV} \\ 
p_{3} + p_{4} m & ,\mbox{if } m \geq 450\mbox{ GeV} \end{cases}
\label{eq:ee_eff}
\end{equation}
where $m$ is the Higgs mass and $p_{0}, p_{1}, p_{2}, p_{3}, p_{4}$ are fit parameters shown in \TableRef{tab:ee_eff_params}.
% TODO Peter: see comment in printouts
\begin{table}[htbp]
    \begin{center}
    \begin{tabular}{ l | l }
        \hline
        Parameter & Value \\
        \hline
        $p_{0}$    & $4.76 \times 10^{-1}$ \\[+0.05in]
        $p_{1}$    & $2.94 \times 10^{+1}$ \\[+0.05in]
        $p_{2}$    & $1.05 \times 10^{+2}$ \\[+0.05in]
        $p_{3}$    & $4.51 \times 10^{-1}$ \\[+0.05in]
        $p_{4}$    & set by requiring continuity \\[+0.05in]
        \hline
    \end{tabular}
    \end{center}
    \caption{Fitted parameter values for Equation~\ref{eq:ee_eff}, which gives $\varepsilon_{tot}(m)$.}
    \label{tab:ee_eff_params}
\end{table}

The computed efficiencies for available mass points and their fit are shown in \FigureRef{fig:signal_efficiency}.

\begin{figure}[h]
\begin{center}
\includegraphics[width=0.6\textwidth]{SS/support_note/DCHLimits/effic_DCH_ee_pairSF_v2.eps}
\caption{Total reconstruction efficiency ($\varepsilon_{tot}$) of the doubly charged Higgs boson decay to same-sign electron pair as a function of simulated $H^{\pm\pm}$ mass, fitted with piecewise empirical function.}
\label{fig:signal_efficiency}
\end{center}
\end{figure}

% TODO \toDo[describe systematics for the DCH signal]
% TODO [CLs = CLs+b/CLb. What is the signal in case of limit on new physics? Read Ingas thesis]

\subsection{Cross section and mass limits}

The invariant mass distribution of the same-sign electron pairs together with the signal from a left-handed Higgs 
with masses between 300 GeV and 500 GeV are shown in \FigureRef{fig:signal_mass_v2}.
The branching ratio of the $H^{\pm\pm} \to e^{\pm}e^{\pm}$ decay is assumed to be 100$\%$.

\begin{figure}[h]
\begin{center}
\includegraphics[width=0.7\textwidth]{SS/support_note/DCHPlots/ee/2isoSS_ee_mll_dch.eps}
\caption{Invariant mass distributions for $\ee$ pairs passing the full event selection. 
Open histograms show the expected signal from simulated $H^{\pm\pm}$ samples,
assuming a 100\% branching ratio to the same-sign electron pair. The last bin is an overflow bin. Only statistical uncertainties for data are shown.}
\label{fig:signal_mass_v2}
\end{center}
\end{figure}

The upper cross section limit on pair production of the doubly charged Higgs is set in the same way as the limits on the new physics described above.
The cross section is determined as:
\begin{equation}
 \sigma_{HH}\times BR =\frac{N_{H}^{rec}}{2\times A\times \epsilon \times \int \mathscr{L} dt}
\end{equation}
where BR is the branching ratio of the $H^{\pm\pm} \to e^{\pm}e^{\pm}$ decay, 
$N_{H}^{rec}$ is the number of reconstructed $H^{\pm\pm}$, $A\times \epsilon$ is the total efficiency described earlier and
$\int \mathscr{L} dt$ is the integrated luminosity.
The factor 2 in the denominator is needed to take into account the presence of two same-sign pairs from $H^{++}$ and $H^{--}$ in the event.

The upper cross section limit times branching ratio (which is assumed to be 100$\%$) at 95$\%$ CL is shown in \FigureRef{fig:dch_limits_mass}.
The variations between the mass bins is caused by fluctuations of the predicted background due to the low statistics per bin.
A good agreement between expected and observed limit lines can be seen, and all deviations are within $2\sigma$.
The theoretical cross section curves as a function of Higgs mass for left- and right-handed doubly charged Higgs bosons 
are shown as well.
Lower mass limits of the model correspond to the intersection of the theoretical curve with the expected cross section limit. The obtained
mass limits are summarized in \TableRef{tab:limits_mass}.

\begin{figure}[h]
\begin{center}
\includegraphics[width=0.6\textwidth]{SS/paper/limitDCH_ee_all.eps}
\caption{Upper limits at 95\% C.L. on the cross section as a function of $e^{\pm}e^{\pm}$ invariant mass for the production of a double charged Higgs boson 
assuming a 100\% branching ratio to $e^{\pm}e^{\pm}$. The green and yellow bands correspond to the 1$\sigma$ and 2$\sigma$ bands on the expected limits respectively.
Pair production cross sections for left and right-handed $H^{\pm\pm}$ are overlaid.}
\label{fig:dch_limits_mass}
\end{center}
\end{figure}

\begin{table}[htbp]
\begin{center}
\begin{tabular}{c||c|c}
& \multicolumn{2}{c}{95\%  C.L. upper limit [GeV]}\\
Signal & expected & observed \\
\hline
$H^{\pm\pm}_L$ & $552.6^{+11.1}_{-49.9}$ & $551.2 \pm 3.1$ \\
\hline
$H^{\pm\pm}_R$ & $424.8^{+1.0}_{-59.7}$ & $374.0 \pm 6.2$ \\
\end{tabular}
\end{center}
\caption{Upper limit at 95\% C.L. on mass of \dch, assuming 100\% branching ratio to $e^{\pm}e^{\pm}$.}
\label{tab:limits_mass}
\end{table}

These limits can also be interpreted as mass limits as a function of the branching ratio for $H^{\pm\pm}_L$ and $H^{\pm\pm}_R$ decays, which are shown in \FigureRef{fig:dch_limits_BR}.

\begin{figure}
\begin{subfigure}{.5\textwidth}
  \centering
  \includegraphics[width=\textwidth]{SS/paper/limitDCH_ee_allvsBR_LH.eps}
\end{subfigure}%
\begin{subfigure}{.5\textwidth}
  \centering
  \includegraphics[width=\textwidth]{SS/paper/limitDCH_ee_allvsBR_RH.eps}
\end{subfigure}
\caption{95\% C.L. limits on the doubly charged Higgs mass vs 
branching ratio of $H^{\pm\pm}_L$ (left) and $H^{\pm\pm}_R$ (right) for events with $e^{\pm}e^{\pm}$ pairs.}
  \label{fig:dch_limits_BR}
\end{figure}

\section{Summary and outlook}
\label{sec:ssOutlook}
% \begin{itemize}
%  \item direct comparison of 7 and 8 TeV analyses are impossible because different efficiencies have been used
%  \item two additional bins in the inclusive limit plot wrt to 7 TeV analysis. what about 13 TeV analysis?
%  \item comparison of the DCH results (7 TeV and 13 TeV)?
%  \item overview of the systematics wrt to 7 and 13 TeV analyses.
%  \item comparison of DCH mass limits as a reference?
%  \item systematization of the charge flip approach for Run 2 --> better understanding and smaller systematics.
% \end{itemize}

% fiducial limits
An inclusive search for a new physics has been performed in the final state of a same-sign electron pair using 20.3 fb$^{-1}$ of 8 TeV center of mass pp collision data~\cite{ss_8TeV}. Limits on the fiducial cross sections has been set as a function of electron pair invariant mass with cuts ranging from $>15$ to $>600$ GeV.
With respect to the previous ATLAS analysis performed with 7 TeV center of mass data (reported in ref.~\cite{ss_7TeV}) cross section limits have been 
extended with two additional invariant mass bins ($>500$ and $>600$ GeV).

% DCH limits
The analysis result has also been used for a narrow bin search for \dch.
Upper limit on the cross section for pair production of right- and left-handed doubly charged Higgs have been set as a function of \dch mass.
Based on predicted cross section limits a mass limit has been derived as well.

Comparison of the obtained 95\% C.L. upper \dch mass limit, assuming 100\% branching ratio to $e^{\pm}e^{\pm}$ with results from 7 TeV~\cite{dch_7TeV_paper} and 13 TeV~\cite{dch_13TeV_conf} analyses performed by ATLAS are shown in \TableRef{tab:DCH_limit_vs_years}.
As one can see from \TableRef{tab:DCH_limit_vs_years} mass limits benefit from the larger center of mass pp collision data though using a dataset with larger integrated luminosity will improve limits further.

\begin{table*}[]
\begin{center}
\begin{tabular}{c||c|c||c|c||c|c}

 & \multicolumn{6}{c}{95\%  C.L. upper limit [GeV]} \\
 & \multicolumn{2}{c||}{4.7 fb$^{-1}$ at 7 TeV} & \multicolumn{2}{c||}{20.3 fb$^{-1}$ at 8 TeV} & \multicolumn{2}{c}{13.9 fb$^{-1}$ at 13 TeV} \\
 \hline
Signal & Expected & Observed & Expected & Observed & Expected & Observed \\
%[+0.05in]
\hline
\rule{0pt}{3ex}
% WITH errors in 8 TeV values:
% $H^{\pm\pm}_L$ & 407 & 409 & $552.6^{+11.1}_{-49.9}$ & $551.2 \pm 3.1$ & 580 & 570 \\
% \hline
% $H^{\pm\pm}_R$ & 329 & 322 & $424.8^{+1.0}_{-59.7}$ & $374.0 \pm 6.2$  & 460 & 420 \\

% WITHOU errors in 8 TeV values:
$H^{\pm\pm}_L$ & 407 & 409 & 553 & 551 & 580 & 570 \\
\hline
$H^{\pm\pm}_R$ & 329 & 322 & 425 & 374 & 460 & 420 \\
\end{tabular}
\end{center}
 \caption{Upper limit at 95\% C.L. on mass of \dch, assuming 100\% branching ratio to $e^{\pm}e^{\pm}$ for 7, 8 and 13 TeV analyses results.}
\label{tab:DCH_limit_vs_years}
\end{table*}

% TODO check $E/p$ stuff
% improvement of signal selection
There are different ways to improve and develop the analysis selection.
One possibility to significantly increase the sensitivity of the search is to optimize the signal region to more efficiently reject the charge misidentification background, which is the dominant background source. In the Run-2 LHC period, many studies have been undertaken in the ATLAS collaboration to investigate electron charge misidentification effect and to suppress it by using new variables e.g. electron $E/p$ (where $E$ is the energy measured in the calorimeter and $p$ is measured by the inner detector).

% improvement of the systematics and methods
Another possibility is to investigate sources of systematic uncertainty more deeply.
Comparing the three analyses, the dominant sources of systematic uncertainties arise from modelling of the charge misidentification and non-prompt backgrounds for all three. Thus, improving the method used for estimation of these backgrounds would reduce the systematic uncertainty. This is especially important for high-$p_T$ electrons where the charge misidentification rate is poorly understood.
Also, increased statistics available in Run-2 can be used for better understanding of the non-prompt background source.

% Other analysis improvements possible developments
As was mentioned before, the analyses have been performed with three channels: $e^{\pm}e^{\pm}$, $\mu^{\pm}\mu^{\pm}$, $e^{\pm}\mu^{\pm}$. However, it might be possible to add a tau lepton channel, which would complement results from the other three. 

% Also, one can use a different definition of the fiducial region where the fiducial efficiency is counted separately per lepton, as was done in the search with three charged leptons final state reported in ref.~\cite{TheATLAScollaboration:2013cia}.



