\chapter{Searches for beyond Standard Model physics with same-sign dileptons}
\label{chap:SS}
\section{Motivation}
% ******************************************************************************
% TODO
% - write something about lepton signature... why it has benefits wrt to jets and photons...
% - write which models we are looking for. how signal looks like with comparison with background
% - write about benefits of the general search (or model-independent? or it's dangerous to say like this?)... theorists can use these results for their models?
% - motivation flow: why leptons -> why same-sign dilepton -> why model-independent search?
% ******************************************************************************

\toAsk[should I use ``same-sign'' or ``same-sign''? Can I use both simultaneously?]

\toDo[no jet selection]
\toDo[add Fast/Full sim systematics]

\toDo[list of BSM models with same-sign dilepton final state]

\toDo[outlook; more explanation for fid. and DCH limits]

The idea of the current analysis is to make a general search for a new physics in a same-sign dilepton channel. \toFix
% Why do we use leptons as final objects?
Leptons as a physics signature are very clean and straightforward observables. 
ATLAS detector capable to detect leptons with amazing momentum resolution precision... \toFix[show some resolution plot]

In resonant searches for new physics using leptons are beneficial due to fact that resonance peak for potential BSM signals will be narrower with comparison to e.g. using jets.
Because energy of leptons is reconstructed very well resonance peaks 
BSM bosons whose decay would manifest itself as a narrow resonance in the dilepton mass spectrum

% Why do we use same-sign dilepton channel
Looking for a same-sign dilepton channel is motivated by the fact that cross section for SM processes with such signature are really small, which make 
it very sensitive possible signals of new physics.  \toFix
% Why model-independent search
There are other searches in ATLAS which target the same final signature but with requiring additional cuts on number of jets \toFix[link] of missing transverse energy \toFix[link].
In such way these analyses target limited number of specific models.
Analyses presented in this chapter aims to be more general and model-independent. 
\toFix[Explain which models does this analysis cover?]

\section{Background processes}
\label{sec:wprimeBackgrounds}
% ******************************************************************************
% TODO
% - describe that there are hree possibilities to reconstruct same-sign lepton pair
% - 
% ******************************************************************************


There are four different components which contribute to the signal region.
The first component is the so called prompt background which correspond to prompt same-sign dileptons which originate from the SM processes.
Just a few SM processes have same-sign dileptons in the final state. In such processes same-sign leptons originates from semi-weakly decays of top quarks
and/or leptonic decays of $W$ or $Z$ bosons. Some example of the Feynman diagrams are shown in \FigureRef{fig:prompt_bkg_feynman_diag}.
In general SM processes with same-sign dilepton final state have relatively small cross sections.

The second component comes from the wrong reconstruction of the lepton itself, when pion or jet is reconstructed as a lepton or from leptons which were born 
not in the pp collision but from decays of secondary particles, e.g. kaons. This component is called non-prompt or fake background.

% TODO add trident event Feynman diagram
The third component comes from the misidentification of the electrical charge of the lepton, which make processes, where opposing lepton pair is produced, 
to contribute to the signal selection as well. The charge misidentification effect become significant for high-momentum leptons, when curvature of the track 
is hard to reconstruct. Also events where prompt electron emit a photon due to bremsstrahlung, which in its turn create an electron-positron pair.
One lepton from the pair can form a same-sign electron pair with original prompt lepton. These two processes are referred as charge misidentification or
charge flip background later in the text.

The last component originates from $W\gamma$ process, where an electron-positron pair is created from photon conversion and by combination with electron from $W$ decay
same-sign lepton pair can be formed. It is considered separately from charge flip background because it does not contain prompt opposite-sign electron pair.

% TODO describe connection of written above with table with all used MC samples...
% TODO normalization is just checked for it is not applied as a scale factor. Cross check with Monika!!!
Prompt background is described by the MC simulation. 
% with applying the overall normalization of the MC samples to the data in the control region.
Non-prompt background is derived with the data-driven method, the so-called fake factor method.
Charge flip and $W\gamma$ backgrounds are estimated from the MC simulation plus are corrected be the charge flip scale factor, derived in data-driven way.

List of all used MC samples used in the analysis are shown in \TableRef{tab:MC_cross}. All other processes which are not listed in the table don't contribute
significantly to the same-sign signal region and are considered negligible.
This table summarize which MC generators and PDF sets were used and order of cross section calculations.

\begin{table}[ht]
  \begin{center}
    \begin{tabular}{l|c|c|c}

      \hline
Process &  Generator&  PDF set & Normalization \\
&  + fragmentation/ &  & based on \\
&  hadronization & &\\
\hline\hline
\multirow{2}{*}{$WZ$ } &  \multirow{2}{*}{{\scshape sherpa-1.4.1} \cite{Sherpa}} &   \multirow{2}{*}{CT10 \cite{CT10}} & NLO QCD \\
 & & &  with {\scshape mcfm-6.2}\cite{mcfm} \\
\hline
\multirow{2}{*}{$ZZ$}  &  \multirow{2}{*}{{\scshape sherpa-1.4.1}} & \multirow{2}{*}{CT10} & NLO QCD  \\
& & &  with {\scshape mcfm}-6.2 \\
\hline
\multirow{2}{*}{\Wpm\Wpm}  & M{\scshape ad}G{\scshape raph}-5.1.4.8 \cite{madgraph4}  &   \multirow{2}{*}{CTEQ6L1 \cite{cteq}}  &  \multirow{2}{*}{LO QCD} \\
&  {\scshape pythia-8.165} \cite{pythia8}& &\\
\hline
\ttbar $V$, & M{\scshape ad}G{\scshape raph}-5.1.4.8  & \multirow{2}{*}{CTEQ6L1} & \multirow{2}{*}{NLO QCD \cite{top9,ttbarW}} \\
$V=W,Z$ &  + {\scshape pythia-6.426} & & \\
\hline
 MPI $VV$ &  \multirow{2}{*}{{\scshape pythia-8.165}\cite{pythia8}}  &  \multirow{2}{*}{CTEQ6L1} &  \multirow{2}{*}{LO QCD} \\
 $V=W,Z$ &  & & \\[+0.025in]
\hline
\hline
\multirow{2}{*}{$Z/\gamma^* +$ jets} & {\scshape alpgen-2.14} \cite{Alpgen}&\multirow{2}{*}{CTEQ6L1}& {\scshape dynnlo-1.1} \cite{dynnlo} with \\
 & + {\scshape herwig-6.520} \cite{Herwig1, Herwig2}& & MSTW2008 NNLO \cite{mstw} \\
\hline
\multirow{2}{*}{\ttbar} & {\scshape mc@nlo}-4.06 \cite{Mcnlo, Mcnlo2} & \multirow{2}{*}{CT10}&{NNLO+NNLL } \\
& + {\scshape herwig-6.520} & & QCD \cite{top1,top2,top3,top4,top5,top6} \\
\hline
\multirow{2}{*}{$Wt$} & {\scshape mc@nlo-4.06}  & \multirow{2}{*}{CT10}& {NNLO+NNLL } \\
   & + {\scshape herwig-6.520} & & QCD \cite{top7,top8}\\
\hline
\multirow{2}{*}{$W^{\pm}W^{\mp}$} & \multirow{2}{*}{{\scshape sherpa-1.4.1}} & \multirow{2}{*}{CT10}& NLO QCD \\
& &  & with {\scshape mcfm-6.2}\\
\hline
\multirow{2}{*}{$W\gamma$} & \multirow{2}{*}{{\scshape sherpa}-1.4.1} & \multirow{2}{*}{CT10}& NLO QCD\\
& &  & with {\scshape mcfm-6.3}\\
\hline
\end{tabular}
\end{center}
  \caption{Generated samples used for background estimates. The generator, PDF set and order of cross-section calculations used for the normalization
  are shown for each sample.
  The upper part of the table shows the MC samples used for the SM background coming from leptons with the same charge (MPI stands for multiple parton interactions), the lower part gives the background sources arising due to electron charge misidentification.}
\label{tab:MC_cross}
\end{table}

\begin{figure}

\begin{subfigure}{.5\textwidth}
  \centering
  \includegraphics[width=\textwidth]{SS/feynman/WZ_electron_v2.eps}
\end{subfigure}%
\begin{subfigure}{.5\textwidth}
  \centering
  \includegraphics[width=\textwidth]{SS/feynman/ttbar_electron_v2.eps}
\end{subfigure}

\caption{Production diagrams for diboson (left) and $\ttbar W$ processes leading to the same-sign dileptons.}
  \label{fig:prompt_bkg_feynman_diag}
\end{figure}


\section{Event selection}
% TODO mention here about size of the dataset and that it was 8 TeV 50 ns data.
% TODO description of the used triggers here!

% TODO ToTHINK MC cross section and sample sizes\toFix
First step in the selection is to select events with fired dilepton trigger ???
Events of interest are events which have two high-$p_T$ isolated leptons \toFix

20.3 fb$^{-1}$ of data collected at 8 TeV center of mass energy pp collisions, with mean number of interactions per crossing equal to 21, 
with 50 ns distance between bunches were used for the analysis.
Events are required to have at least one reconstructed primary vertex with at least three tracks matched to it. If there are few vertices, the one with the highest
$\sum p^2_T$, where $p_T$ is transverse momenta of the matched tracks, is chosen.
Events which pass \toFix trigger, which selects events with two isolated \toFix leptons with $p_{T}>25$\toFix and $p_{T}>20$\toFix, were selected.
% TODO Cleaning and all other event-related cuts to describe here...

\subsection{Electron selection}
\label{subsec:electron_selection}
Next step in the selection is to select all isolated high-$p_T$ electrons present in the event.
Electron reconstruction is done by matching track from inner detector with an energy deposits in the EM calorimeter.

\toDo[add how electrons are reconstructed in the ATLAS from electron ID note]

\toDo[explain how electron energy is measured]

\toDo[describe three channels: ee, emu, mumu]

Electrons of interest are good reconstructed candidates, track of which originates from the primary vertex. To select such candidates following requirements have to be satisfied:
\begin{itemize}
 \item $p_T > 20$ GeV. To ensure a high and flat trigger efficiency and to harmonize $p_T$ requirement between three analysis channels ($ee$, $\mu\mu$ and $e\mu$) 
 \item $|\eta|<2.47$ to be within high-granularity acceptance of EM calorimeter, but excluding barrel-end-cap transition region $1.37<|\eta|<1.52$.
 \item Transverse and longitudinal impact parameters $|d_0|/\sigma(d_0) < 3$ and $|z_0 \times sin \theta| < 1$ mm. 
 To verify that electron was born close to the primary vertex and to reject electrons which were originated from the decay of long-lived particles.
 \item To pass ``tight'' electron set of identification criteria defined in~\cite{electron_tight}. \\ 
 \toFix[explain shortly purpose of the cut]
 \item To pass isolation requirement. To distinguish prompt electrons from those associated with jet activity.
 \item No reconstructed jet within $\Delta R < 0.4$ \toFix[define R]. To remove ambiguous classification of the track, when track is classified as electron and as jet simultaneously. 
 Or when jet is too close to the electron which will lead to poor reconstruction of latter. 
\end{itemize}

Isolation requirement were chosen in order to reach pile-up independent efficiency of more than 99$\%$ for electrons 
with $p_T >$ 40 GeV. The requirement has two parts.
Firstly, the sum of the transverse energies in the EM and hadronic calorimeters around the electron within 
$\Delta R < 0.2$, with core electron energy subtracted from the sum, has to be less than 
3 GeV $+ (p_T - 20$ GeV$) \times 0.037$, where $p_T$ is electron transverse momentum.
Secondly, the sum of the $p_T$ of all tracks with $p_T > 0.4$ GeV within $\Delta R < 0.3$ around the electron track has
to be less than 10$\%$ of electron $p_T$.

Electrons used in validation regions are required to fail one or few requirements above but pass looser one.

\subsection{Definition of same-sign electron pair}
% TODO inv. mass cut; treatment of event with three leptons;
All electrons which passed selection are considered while forming a lepton pairs.
All combinations are checked and it is allowed to have more than one pair selected in one event.
Electrons in the pair are classified by the $p_T$: electron with higher $p_T$ is called leading lepton, while another one with lower $p_T$ subleading lepton.
Leading lepton have to pass more strict cut $p_T>25$ GeV cut while subleading lepton have to satisfy $p_T>20$ GeV as described above.

If an event contains an opposite-sign same flavour lepton pair which satisfies condition $( |m_{\Plepton\Plepton} - m_{Z}| < 10$ GeV$)$,
where $m_{\Plepton\Plepton}$ is lepton pair invariant mass and $m_{Z}$ is mass of Z boson, event is discarded.
This significantly suppress prompt background.

To avoid low-mass hadronic resonances like $J/\psi$ or $\varUpsilon$ to show up in the invariant mass spectrum only same-sign pair with $m_{\Plepton\Plepton}>15$ GeV are selected.
Additionally, same-sign lepton pairs with invariant mass $70 < m_{\Plepton\Plepton}< 110$ GeV, which corresponds to the Z peak region, is found in the event, than such event is removed.
This region is used for estimation of the electron charge misidentification background.

\section{Estimation of the charge flip and non-prompt backgrounds}

\subsection{Prompt opposite-sign dilepton with charge flip}
\label{subsec:CF_definition}
% TODO
% subsection flow:
% CF sources --> method and used regions --> validation of the method --> deriving scale factors --> appling to specific MC samples
\toDo[description how charge flip rate was obtained, and plots if CF rate vs. pT and eta.]
\toAsk[why we define CF rate in 80..100 GeV region, while we cut 70..110 GeV from signal region?]
\toAsk[CF at high pT. extrapolation? of one large bin with huge error? look it up!]

Since the charge of one of the reconstructed electrons from the pair can be misidentified, 
processes with opposite-sign dilepton final state can contribute to signal region.
Because prompt background is relatively small in the signal region possibility of a charge flip cannot be neglected and have to be properly estimated. 
Two cases are considered as a charge flip background. 
First one is when electron charge is actually misidentified due to matching of the wrong track to the EM cluster or due to small curvature of the high-momentum track. 
Second one is when electron emits a photon by bremsstrahlung, which in its turn decays to electron-positron pair. 
Instead of the original electron, electron from photon conversion ca be chosen to the pair and can have opposite charge.

In order to estimate how many events with opposite-sign dileptons will ``leak'' to same-sign dilepton signal region one
need to know electron charge flip rate. It is expected that charge flip will depends from electron momentum (due to track curvature) 
and from $\eta$ (due to different amount of material of the detector versus $\eta$).

Electrons from Z boson decay provide a huge sample of opposite-sign lepton pairs which can be used to estimate charge flip rate. 
By applying signal selection
and requiring invariant mass of the lepton pair around Z mass, $80 < m_{ee} < 100$ GeV, one obtain a pure
sample of electron pairs, where charge of one electron is flipped. Knowing number of opposite-sign pairs which pass signal selection, 
one can extract charge flip rate.

In \FigureRef{fig:chargeFlip_structure} charge flip rate obtained from MC simulation as a function of $p_T$ is shown.
By using information on truth (generated) level one can distinguish charge flip due to photon conversion from other effects which and it is shown in the figure.
Contribution from photon conversion from bremsstrahlung process is dominant 
to total charge flip rate over all $p_T$ range except high-$p_T$ region, where dominant effect become misidentification 
of charge due to very small curvature of tracks.

\begin{figure}
\begin{center}
 \includegraphics[width=0.7\columnwidth]{SS/support_note/chargeflip/misidratept_ZDY.eps}
\caption{\toDo[caption]}
\label{fig:chargeFlip_structure}
\end{center}
\end{figure}

To verify MC estimation of charge flip a special data-driven technique, the so-called likelihood method is used in the same way as in ref.~\cite{same_sign_paper_7tev}.
This method is based on using a maximum likelihood fit to extract the charge flip rates in different kinematic regions simultaneously~\cite{cf_top_paper}.
This method found to be the most precise due to using most statistics available and to provide kinematically unbiased results~\cite{anthony_thesis} 
for same-sign dilepton analysis with 7 TeV data.

To cross-check applicability of the method it was used on 
MC simulation first and charge flip rates were compared with the one from generator level (``truth'') information. 

Comparison demonstrates that likelihood method provide very reliable results as shown in
\FigureRef{fig:likelihood_cross_check}.

\begin{figure}
\begin{subfigure}{.5\textwidth}
  \centering
  \includegraphics[width=\textwidth]{SS/support_note/chargeflip/eta_mctruth.eps}
\end{subfigure}%
\begin{subfigure}{.5\textwidth}
  \centering
  \includegraphics[width=\textwidth]{SS/support_note/chargeflip/pt_mctruth.eps}
\end{subfigure}
\caption{\toDo}
\label{fig:likelihood_cross_check}
\end{figure}

Charge flip rate as a function of $p_T$ and $\eta$ is shown in \FigureRef{fig:charge_flip_data_vs_mc}.
Comparison between MC simulation and likelihood data-driven predictions are shown. 
Dependence of the rate as a function of $p_T$ is well described by MC simulation, while some difference is observed
in high-$\eta$ region. This is why $\eta$ dependent scale factors were calculated.

\begin{figure}
\begin{subfigure}{.5\textwidth}
  \centering
  \includegraphics[width=\textwidth]{SS/support_note/chargeflip/misidrate_datamc.eps}
\end{subfigure}%
\begin{subfigure}{.5\textwidth}
  \centering
  \includegraphics[width=\textwidth]{SS/support_note/chargeflip/misidratept_datamc.eps}
\end{subfigure}
\caption{\toDo}
\label{fig:charge_flip_data_vs_mc}
\end{figure}

Charge flip background consists from opposite-sign lepton pairs which were reconstructed as same-sign pairs.
Background is estimated from MC simulation (all considered processes are shown in lower part of the \TableRef{tab:MC_cross}) 
plus it is corrected by charge flip scale factor to properly reproduce $\eta$-dependence 
of charge flip rate.


% TODO systematics on CF scale factor
To estimate systematic error of the derived scale factors following uncertainty sources were considered:
\begin{itemize}
 \item The width of the invariant mass window, used to preselect electrons from $Z$ peak, was varied by 10 GeV.
 The largest difference between central value scale factors and scale factor obtained from variation was taken as systematic error.
 \item The electron isolation requirement were loosed by 4 GeV in both track- and calorimeter-based isolation criteria.
\end{itemize}
Electrons from Z decays have limited $p_T$-range up to around 100 GeV. As can be seen in~\FigureRef{fig:charge_flip_data_vs_mc} (right), 
MC simulation describe $p_T$ dependence of the charge flip rate very good, this is why it was decided to rely on MC simulation 
for the high-$p_T$ leptons. Additional studies were done to estimate systematic uncertainty in this case.
A detector alignment and amount of detector material was varied in order to get an effect on charge flip rate.
Conservative estimation of 20$\%$ error were assigned to high-$p_T$ lepton charge flip rate.

Total systematic uncertainty on the background prediction is shown in~\TableRef{tab:syst}.

Charge flip scale factor and uncertainty are also applied for the MC prediction of the background from $W\gamma$ process as well.
\toAsk[should I say smth. more about $W\gamma$ process]

\subsection{Non-prompt background}
\label{subsec:fakes_description}
\toDo[short description of the fake factor method. Put details on which regions were used.]

Another type of background present after the signal selection is the so-called non-prompt background.
Main sources of this background are jets misidentified as electrons and electrons which originate not from 
primary vertex, e.g. electrons from semi-leptonic decays of heavy flavor quarks ($b$, $c$).
To estimate this background a data-driven method, the so-called fake factor method, is used.

The first of the methods step is to define a background region, which does not overlap with the signal region, where 
contribution of non-prompt electrons are dominant, while contribution from prompt leptons is as small as possible.
It must contain strictly one reconstructed electron (probably jet misreconstructed as electron) with $p_T > 20$ GeV.
To counterbalance the lepton a side away jet in the opposite azimuthal direction ($\Delta \phi (e,jet) > 2.4$) is required.
By requiring strictly one electron one can make sure that background region will not overlap with the signal selection and processes 
like Drell-Yan and \ttbar~will be suppressed.
To make sure that away side jet and non-prompt electron are well balanced in terms of energy, away side jet is
requested to have $p_T > 30$ GeV.
To suppress contribution from W boson production processes requirement on transverse invariant mass 
$m_\mathrm{T} > 40$ GeV is applied.
\footnote{$m_\mathrm{T} = \sqrt{2 p_\mathrm{T} E_\mathrm{T}^{miss} (1-\cos\varphi_{\Plepton\nu})}$, where
$p_\mathrm{T}$ is transverse momentum of the electron, $E_\mathrm{T}^{miss}$ is missing transverse energy of the event
and $\varphi_{\Plepton\nu}$ is the angle between of them.}

In this background region one determine a fake factor $f$ as:
\begin{equation}
f = \frac{N_{\mathrm{P}} - N_{\mathrm{P}}^{\mathrm{prompt}}}{N_{\mathrm{F}}  - N_{\mathrm{F}}^{\mathrm{prompt}}}.
\label{eq:fakefactor}
\end{equation}
where $N_{\mathrm{P}}$ is number of reconstructed electrons in the background region which pass electron signal selection,
described above in \SectionRef{subsec:electron_selection} and $N_{\mathrm{F}}$ is number of electrons which do not meet 
signal electron selection requirements but satisfy a looser selection. Looser selection is identical for the signal selection except 
requirements that electron have to satisfy ``medium'' electron set of identification criteria~\cite{electron_tight} (instead of ``tight'') 
and it have to fail the calorimeter- or track-based isolation cut, or both.
$N_{\mathrm{P}}^{\mathrm{prompt}}$ and $N_{\mathrm{F}}^{\mathrm{prompt}}$ are numbers of real prompt
leptons, obtained from MC simulation, which pass signal and looser selections accordingly.
Contribution of prompt leptons have to be subtracted in order to make sure that 
fake factor is defined as a ratio of non-prompt electrons passed signal selection over 
the number of non-prompt electrons passed looser selection and there is no contamination from real prompt leptons.
When fake factor have been measured in the background region it can be used to predict non-prompt background in the signal region.
In order to take into account possible different kinematics of a lepton in the region were fake factor was derived 
and region were it will be applied, fake factor is measured as a function of electron $E_\mathrm{T}$ and $\eta$.
The total number of the non-prompt same-sign pairs, $N_{\mathrm{NP}}$, in the signal region can be estimated as:
\toAsk[is phrasing above clear enough?]
\begin{equation}
N_{\mathrm{NP}} = \sum_{i}^{\mathrm{N_{P_l F_s}}} f_{\mathrm{s}}(p_{\mathrm{Ti}},|\eta_{i}|) + \sum_{i}^{\mathrm{N_{F_l P_s}}} f_{\mathrm{l}}(p_{\mathrm{T}i},|\eta_{i}|) - \sum_{i}^{\mathrm{N_{F_l F_s}}} f_{\mathrm{l}}(p_{\mathrm{T}i},|\eta_{i}|) \times f_{\mathrm{s}}(p_{\mathrm{T}i},|\eta_{i}|).
\label{eq:fake_pred}
\end{equation}
where the first term corresponds to the number of electron pairs ($N_{P_l F_s}$) 
where the leading electron (denoted by ``l'') passes selection criteria ($P_l$) and subleading electron 
(denoted by ``s'')
fails to fulfill it ($F_s$), but passes the looser selection used for fake factor calculation. 
Contribution of every such pair to the signal region is scaled by the fake factor 
$f_{\mathrm{s}}(p_{\mathrm{Ti}},|\eta_{i}|)$, where $p_\mathrm{Ti}$ and $\eta_{i}$ are transverse momentum and $\eta$
of the subleading electron of a pair which fails the signal selection. 
Similarly, the second term represents number of pairs where
leading electron fails signal selection while the subleading passes it. 
The last term corresponds to the case when both leading
and subleading electrons fail signal selection. This term has to be subtracted from the sum to correct for the double
counting of non-prompt electron pairs.

% TODO talk about systematics here:
To estimate systematic errors of the method one can test all assumptions made in the method and take into account possible differences between the region used to derive 
fake factors, and region where they will be applied. The following sources were considered:
\begin{itemize}
 \item the statistical uncertainty of the used data sample to derive fake factors.
 \item prompt MC subtraction, which is done to verify that there is no contamination from prompt leptons when deriving/applying fake factors.
 Due to the luminosity uncertainty of 2.8$\%$ and the uncertainty on MC cross section (7$\%$ on major prompt processes), prompt MC subtraction
 was varied by 10$\%$.
 \item requirement on away side jet $p_T$. It was varied up to 50 GeV in order to test dependence of the fake factor from faking jet kinematics.
 \item difference in the non-prompt background composition in the region used to derive fake factor and the region where they were applied.
 Non-prompt background can originate from jets, which were created by gluons or light quarks, as well as from heavy flavour jets.
 Fake factor depends from the proportion of this two categories of jets in the region. This is why fake factors were derived separately
 for heavy and light flavour jets and difference between the results was taken as systematic error.
\end{itemize}
More detailed description of the evaluated systematic uncertainties and the fake factor method overall can be found in~\cite{anthony_thesis}.

The electron fake factor as a function of electrons $E_\mathrm{T}$ with statistical and systematic errors are
shown in \FigureRef{fig:ff_e_errs}.
\begin{figure}[h]
\begin{center}
%\includegraphics[width=0.48\textwidth]{figures/electrons/electron-syst}
\includegraphics[width=0.65\textwidth]{SS/support_note/electrons/electron_fake_plot.eps}
\caption{Electron fake factor $f$ as a function of electron $E_\mathrm{T}$. Combined statistical and systematic errors shown with error bars.}
\label{fig:ff_e_errs}
\end{center}
\end{figure} 
\toDo[say that fake factors were also measured not only for signal selection, but also for less strong selections
which are used for fake validation regions].
\toAsk[what fake CR says us? we use different fake factors for them then for signal, so there is no direct relation?]

Verification of the fake factor method is done with help of validation regions which are described further.




% The fake background includes all background processes where at least one of the leptons is fake. The dominant sources are $W$+jets and QCD multijets with smaller contributions arising from $Z$+jets and $t\bar{t}$. To assess this background, a data-driven method, known as the fake factor method, is employed. This method is used in several ATLAS analyses, particularly diboson measurements and searches\footnotemark. The method employed was covered in great depth in the previous 7 TeV incarnation of this analysis.

% \footnotetext{The following analyses, among others, use the fake factor method: SM $WW \ra \ell\nu\ell\nu$ and $WZ \ra \ell\nu\ell\ell$, $H \ra WW \ra \ell\nu\ell\nu$, and exotics $WZ \ra \ell\nu\ell\ell$ search.}

\section{Background validation regions}

In order to make sure that all backgrounds are modelled/predicted correctly in the signal region one can define and use special validation regions.
These regions should be kinematically close to the signal region but do not overlap with it. 
Each region is designed to test one given background type at a time,
which means that in each validation region only one background type has to give a dominant contribution with respect to all other backgrounds.

\subsection{Prompt opposite-sign dileptons}

First validation region is defined by using exactly the same event selection as for the signal region,
but requiring the two leptons to have opposite charge. 
This validation region does not test any background specifically.
Overall normalization of the simulation to the data verify trigger and lepton reconstruction efficiencies.
While correct description of the Z peak shape in data by MC simulation tests electron scale momentum and resolution.
In \FigureRef{fig:OS_CR} invariant mass of opposite-sign electron pairs is shown.
In \TableRef{tab:dilep_isoOS} observed and expected number of the electron pair are shown as well.
Good agreement between data and simulation can be seen.

\begin{table}[htbp]
\caption{Observed and expected number of lepton pairs for the control region with opposite-sign, isolated leptons.} %The significance of the difference between the number of data events observed and that predicted is calculated considering the statistical error on the data and the systematic uncertainty on the prediction.}
\begin{center}
\begin{tabular}{l|c}

Process & Number of lepton pairs \\\hline\hline
        Drell-Yan	& $ 4701110 \pm 329108 $	\\[+0.05in]
	$t\bar{t}$	& $ 14580.8 \pm 874.92 $	\\[+0.05in]
	Dibosons	& $ 12210.9 \pm 545.6 $	\\[+0.05in]
	Non-prompt	& $ 8321.28 \pm 244.4 $	\\[+0.05in]
	$W\gamma$	& $ 243.03 \pm 35.2 $	\\[+0.05in]
	MPI	& $ 32.74 \pm 32.74 $	\\[+0.05in]
	\hline
	Total Prediction	& $ 4736500 \pm 329109 $	\\[+0.05in]
	\hline
	Observation in data	& $ 4895830 $	\\[+0.05in]
	\hline
	Agreement ($\sigma$) & -0.48 \\[+0.05in]

\hline  
\end{tabular}
\end{center}
\label{tab:dilep_isoOS}
\end{table}

\begin{figure}[h]
\begin{center}
\includegraphics[width=0.65\textwidth]{SS/support_note/dielectrons/crs/OS_mod_v4.pdf}
\caption{\toDo}
\label{fig:OS_CR}
\end{center}
\end{figure} 



\subsection{Prompt same-sign dileptons}
% TODO table with final numbers to show how good MC normalization is wrt to data.

Same-sign dilepton prompt background originates dominantly from $WZ$ and $ZZ$ processes, where both Z and W bosons decay leptonicaly.
In order to check normalization of these processes prompt validation region is used.
In fully reconstructed event, where one of these processes took place, one can find at least one same-sign and one opposite-sign lepton pairs.
In order to enhance $WZ$ and $ZZ$ contributions in the validation region at least three leptons are required in the event, where one lepton pair
has to be same-sign electron pair and other one opposite-sign same-flavour pair (from $Z$ boson decay). 
Invariant mass of opposite-sign pair has to be close to $Z$ boson mass $(|m_{\Plepton\Plepton} - m_{Z}| < 10$ GeV$)$.

Expected and observed number of same-sign pairs in this region are listed in \TableRef{tab:promptCR_yields}, 
and the ratios between observed and expected number of pairs are summarized in \TableRef{tab:prompt_ratios}. 
The expectations are in good agreement with observation.

\begin{table*}[htbp]
\caption{Expected and observed numbers of pairs for various cuts on the dilepton invariant mass. The uncertainties shown are the quadratic sum of the statistical and systematic uncertainties.}
\begin{center}
\resizebox{\textwidth}{!}{
\begin{tabular}{l|c|c|c|c|c|c|c}
\hline 
Sample & \multicolumn{6}{|c}{Number of electron-electron pairs with  $m(e^{\pm}e^{\pm})$} \\
 & $>15$~GeV & $>100$~GeV & $>200$~GeV & $>300$~GeV & $>400$~GeV & $>500$~GeV & $>600$~GeV \\
\hline \hline
Non-prompt & $49 \pm 14$ & $31.1 \pm 8.1$ & $11.1 \pm 3.0$ & $3.4 \pm 1.3$ & $1.22 \pm 0.72$ & $0.81 \pm 0.63$ & $0.41 \pm 0.44$ \\
\hline
Prompt total & $226 \pm 18$ & $133.8 \pm 9.2$ & $36.7 \pm 3.0$ & $11.6 \pm 1.3$ & $3.44 \pm 0.63$ & $1.15 \pm 0.34$ & $0.38 \pm 0.18$ \\
\hline
$W/ \gamma$ & $0.0 \pm 0.0$ & $0.0 \pm 0.0$ & $0.0 \pm 0.0$ & $0.0 \pm 0.0$ & $0.0 \pm 0.0$ & $0.0 \pm 0.0$ & $0.0 \pm 0.0$ \\
\hline
Charge Flip total & $0.00036 \pm 0.00068$ & $0.0 \pm 0.0$ & $0.0 \pm 0.0$ & $0.0 \pm 0.0$ & $0.0 \pm 0.0$ & $0.0 \pm 0.0$ & $0.0 \pm 0.0$ \\
\hline \hline
Sum of Backgrounds & $275 \pm 23$ & $165 \pm 12$ & $47.9 \pm 4.2$ & $15.0 \pm 1.9$ & $4.65 \pm 0.95$ & $1.96 \pm 0.71$ & $0.78 \pm 0.47$ \\
\hline \hline
Data  & $268 \pm 16$ & $156 \pm 12$ & $46.0 \pm 6.8$ & $14.0 \pm 3.7$ & $6.0 \pm 2.4$ & $3.0 \pm 1.7$ & $1.0 \pm 1.3$ \\
\hline 
\end{tabular}
}
\end{center}
\label{tab:promptCR_yields}
\end{table*}


\begin{table*}[htbp]
\caption{Ratio between observed and expected same-sign pairs in the $WZ$ and $ZZ$ control region for various cuts on the dielectron invariant mass. 
The uncertainties account for both statistical and systematic errors.}
\begin{center}
\resizebox{\textwidth}{!}{
\begin{tabular}{c|c|c|c|c|c|c}
\multicolumn{6}{c}{Ratio between observed and expected for $m(e^{\pm}e^{\pm})$} \\
$>15$~GeV & $>100$~GeV & $>200$~GeV & $>300$~GeV & $>400$~GeV & $>500$~GeV & $>600$~GeV \\
\hline \hline
$0.97 \pm 0.09$ & $0.95 \pm 0.10$ & $0.96 \pm 0.17$ & $0.93 \pm 0.27$ & $1.3 \pm 0.6$ & $1.5 \pm 1.0$ & $1.3 \pm 1.9$ \\
\hline \hline
\end{tabular}
}
\end{center}
\label{tab:prompt_ratios}
\end{table*}

\FigureRef{fig:prompt_CR} shows an invariant mass distribution in the prompt validation region.
Simulation agrees well with data.

\begin{figure}[h]
\begin{center}
%\includegraphics[width=0.48\textwidth]{figures/electrons/electron-syst}
\includegraphics[width=0.65\textwidth]{SS/support_note/PromptCR/ElEl/2isoSS_ee_mll_pr.eps}
\caption{\toDo}
\label{fig:prompt_CR}
\end{center}
\end{figure} 


\subsection{Electron charge flip}

As was described before events with opposite-sign lepton pairs in the final state can be reconstructed as same-sign lepton pairs, if charge for one of leptons was wrongly identified.
Misidentification probability is well modelled as function of $p_T$ by MC simulation 
but $\eta$-dependence have to be corrected by scale factors obtained with data-driven method.
One can make a sanity check, to cross check data from $Z$ peak region (same-sign pairs with invariant mass $80 < m_{ee} < 100$~GeV)
with the MC simulation corrected by the charge flip scale factors.

Invariant mass of the same-sign $Z$ peak is shown in \FigureRef{fig:charge_flip_CR_inv_mass}. 
Also $p_T$ and $\eta$ distribution for the leading electron are shown in \FigureRef{fig:charge_flip_CR_kinematics}.
A good agreement is observed which demonstrates correctness of derived charge flip scale factor.

Observed and expected number of electron pairs are also shown in \TableRef{tab:ee_isoSS_Z} for combined $e^{\pm}e^{\pm}$ and separately for $e^{+}e^{+}$ and $e^{-}e^{-}$ pairs.

\begin{figure}[h]
\begin{center}
%\includegraphics[width=0.48\textwidth]{figures/electrons/electron-syst}
\includegraphics[width=0.65\textwidth]{SS/paper_draft/2isoSS_ee_mll_ssz.eps}
\caption{\toDo}
\label{fig:charge_flip_CR_inv_mass}
\end{center}
\end{figure} 

\begin{figure}
\begin{subfigure}{.5\textwidth}
  \centering
  \includegraphics[width=\textwidth]{SS/paper_draft/2isoSS_ee_pt1_ssz.eps}
\end{subfigure}%
\begin{subfigure}{.5\textwidth}
  \centering
  \includegraphics[width=\textwidth]{SS/paper_draft/2isoSS_ee_eta1_ssz.eps}
\end{subfigure}
\caption{\toDo Leading electron $E_\mathrm{T}$ and $\eta$ distributions in the same-sign $Z$ peak control region with two isolated electrons.}
  \label{fig:charge_flip_CR_kinematics}
\end{figure}


\begin{table}[htbp]
\caption{Observed and expected number of lepton pairs for the control region with same-sign, isolated electrons falling inside the $Z$ mass window. 
The uncertainties on the predictions are combined statistical then systematic \toDo[is it the case? or it is only statistical errors?].}
\begin{center}
\begin{tabular}{l|c}
\hline
Process & Number of $ee$ pairs \\\hline\hline
%
\multicolumn{2}{c}{\textbf{Same-sign $ee$ $Z$ mass window.}} \\\hline 
        Non-prompt      & $200 \pm 110$ \\[+0.05in]
        Charge Flips & $12400 \pm 1300$ \\[+0.05in]
        Prompt Electrons & $143.4 \pm 8.1$ \\[+0.05in]
        $W\gamma$  & $26.8 \pm 5.6$ \\[+0.05in]
            \hline
        Total Prediction & $12700 \pm 1300$ \\[+0.05in]
            \hline
        Data       &       $11793 \pm 110$ \\[+0.05in]
            \hline
        Agreement  &      0.8$\sigma$ \\[+0.05in]
\hline \hline
\multicolumn{2}{c}{\textbf{Same-sign $e^{+}e^{+}$ $Z$ mass window.}} \\\hline 
        Fakes      & $66 \pm 60$ \\[+0.05in]
        Charge Flips & $6380 \pm 670$ \\[+0.05in]
        Prompt Electrons & $82.0 \pm 5.0$ \\[+0.05in]
        $W\gamma$  & $17.5 \pm 4.0$ \\[+0.05in]
            \hline
        Total Prediction & $6540 \pm 680$ \\[+0.05in]
            \hline
        Data       &        $5908 \pm 77$ \\[+0.05in]
            \hline
        Agreement  &     1.0$\sigma$ \\[+0.05in]
%
\hline \hline
\multicolumn{2}{c}{\textbf{Same-sign $e^{-}e^{-}$ $Z$ mass window.}} \\\hline 
        Fakes      & $131 \pm 63$ \\[+0.05in]
        Charge Flips & $5990 \pm 630$ \\[+0.05in]
        Prompt Electrons & $61.4 \pm 3.9$ \\[+0.05in]
        $W\gamma$  & $9.4 \pm 2.3$ \\[+0.05in]
            \hline
        Total Prediction & $6190 \pm 630$ \\[+0.05in]
            \hline
        Data       &        $5885 \pm 77$ \\[+0.05in]
            \hline
        Agreement  &     0.5$\sigma$ \\[+0.05in]
\hline 
\end{tabular}
\end{center}
\label{tab:ee_isoSS_Z}
\end{table}


\subsection{Non-prompt background validation region}
To verify the fake factor method, a set of validation regions is checked. 
These regions have to be as much as possible kinematically close to
the nominal signal selection as well as looser signal selection, 
which is used in non-prompt background estimation with \EquationRef{eq:fake_pred}.

Looser signal selection have weaker isolation requirements and weaker 
electron identification requirement compared in the nominal selection.
This is why two validation regions are used, one which is identical
to the nominal signal selection, except weaker isolation cut and another
one - except weaker identification requirement. 
Schematic representation of the signal and validation regions is shown
in \FigureRef{fig:fake_validation_regions}. 
% TODO
\toDo[explain here how fake factors for validation regions were calculated]
Looser selections for the
validation regions which were used for fake factor calculation are shown
as well. As can be seen from the picture, such design of the validation
region allows to test non-prompt estimation for signal region in the 
best way possible.

\begin{figure}[h]
\begin{center}
%\includegraphics[width=0.48\textwidth]{figures/electrons/electron-syst}
\includegraphics[width=0.7\textwidth]{SS/fake_regions_black_diagram.pdf}
\caption{\toDo}
\label{fig:fake_validation_regions}
\end{center}
\end{figure}

In total, four different validation regions were defined.
One validation region consists from same-sign electron pairs, where both electrons pass the signal region isolation requirement, 
but fail ``tight'' electron identification requirement, while passing ``medium''.
Second validation region consists from same-sign electron pairs, where both electrons fails the signal isolation requirement
but pass a looser \textit{intermediate} isolation cut, which is loosened by 4 GeV, instead.
Third and fourth validation regions are identical for the second one but requires only leading (third region) or subleading (fourth region)
electron to fails the signal isolation requirement but pass a looser \textit{intermediate} isolation cut instead.

\FigureRef{fig:fakeCR_part1} and \FigureRef{fig:fakeCR_part2} show the invariant mass distributions for the validation regions descried above. 
The agreement between observation and prediction is generally good.  
\TableRef{tab:ee_fakeCR} shows the expected and observed numbers of electron pairs. 
The uncertainties quoted are statistical only. 

\toDo[say about uncertainties]
% For the uncertainty on the non-prompt background predictions, 
% this includes the uncertainty associated with limited statistics where the fakes are estimated together 
% with the resulting fractional systematic uncertainty on the fake factor as propagated through for the signal region. 



\begin{table}[htbp]
\caption{Expected and observed numbers of electron pairs for the different same-sign $ee$ fake control regions. 
The uncertainties on the predictions include the statistical and systematic uncertainties (fake factor and charge flip 
uncertainties have been included; other systematic uncertainties are negligible in these regions). 
% For the fake predictions, a systematic uncertainty derived for the signal region is assumed.
}
    \centering
    \resizebox{\textwidth}{!}{
    \begin{tabular}{ l | r r r r r r r }
        \hline
        Validation region 		& Fakes 			& Prompt 	& Charge Flip 			& W$\gamma$ 		& Total Pred 			& Data		&  Agreement($\sigma$) \\
            \hline

        Medium electron identification 	& $ 111.04 \pm 27.4 $	&	$ 2.9 \pm 0.5 $	& $ 72.46 \pm 16.75 $		& $ 8.78 \pm 2.3 $	& $ 195.18 \pm 32.2 $		& $ 217 \pm 15 $&  -0.62\\
        Weak isolation on both electrons 	& $ 252.9 \pm 133.64 $	& $ 1.23 \pm 0.3 $	& $ 29.07 \pm 10.1 $		& $ 0.27 \pm 0.28 $	& $ 283.47 \pm 134.02 $		& $285 \pm 17 $	&  -0.01\\
        Weak isolation on subleading electron 	& $ 519.21 \pm 120.72 $ & $ 32.88 \pm 2.14 $	& $ 52.69 \pm 14.87 $		& $ 17.64 \pm 4.32 $	& $ 622.42 \pm 121.72 $		& $574 \pm 24 $ &  0.39\\
        Weak isolation on leading electron 	& $ 154.97 \pm 58.67 $  & $ 13.28 \pm 1.21 $	& $ 15.96 \pm 7.5 $		& $ 5.12 \pm 1.72 $	& $ 189.33 \pm 59.19 $		& $ 224 \pm 15 $&  -0.57\\
        
        \hline
    \end{tabular}
    }
\label{tab:ee_fakeCR}
\end{table}



\begin{figure}
\begin{subfigure}{.5\textwidth}
  \centering
  \includegraphics[width=\textwidth]{SS/support_note/dielectrons/crs/dec13_fake_medium_v2.eps}
\end{subfigure}%
\begin{subfigure}{.5\textwidth}
  \centering
  \includegraphics[width=\textwidth]{SS/support_note/dielectrons/crs/dec13_fake_bothInter_v2.eps}
\end{subfigure}
\caption{\toDo Invariant mass distributions for different $e^{\pm}e^{\pm}$ control regions enhanced in fake background. 
The dashed lines show the statistical uncertainty on the background prediction.}
  \label{fig:fakeCR_part1}
\end{figure}

\begin{figure}
\begin{subfigure}{.5\textwidth}
  \centering
  \includegraphics[width=\textwidth]{SS/support_note/dielectrons/crs/dec13_fake_leadNom_sublInter_v2.eps}
\end{subfigure}%
\begin{subfigure}{.5\textwidth}
  \centering
  \includegraphics[width=\textwidth]{SS/support_note/dielectrons/crs/dec13_fake_leadInter_sublNom_v2.eps}
\end{subfigure}
\caption{\toDo Invariant mass distributions for different $e^{\pm}e^{\pm}$ control regions enhanced in fake background. 
The dashed lines show the statistical uncertainty on the background prediction.}
  \label{fig:fakeCR_part2}
\end{figure}



\section{Systematic Uncertainties}
\label{sec:ss_Systematics}

\toDo[some general words about importance of the systematics...]

A set of possible systematic sources which can affect background predictions were studied.
Description of the considered sources are presented below.
Systematic uncertainties related to the data-driven methods for non-prompt and charge flip background estimations
were described in \SectionRef{subsec:fakes_description} and \SectionRef{subsec:CF_definition} accordingly.

\subsection{Electron reconstruction}
Several systematic uncertainties are related to the electron reconstruction procedure.
These uncertainties are provided by the ATLAS e/gamma working group which works on electron and photon identification performance of the ATLAS detector.
They provide recommendations and uncertainty estimations for all physics analyses which use electron or photon final states.

Electron reconstruction and ``tight'' identification efficiencies are obtained with tag-and-probe methods using
events with leptonic decays of $W$ and $Z$ bosons as well as $J/\psi$ mesons.
Reconstruction efficiency uncertainty range from 1.3-2.4\% depending on $\eta$ while
``tight'' identification efficiency uncertainty range from 2.0-2.8\% depending on both $E_\mathrm{T}$ and $\eta$~\cite{electron_reco_id_2011}.

The reconstruction of the electron energy is optimized using multivariate algorithms.
Electron energy scale and energy resolution are obtained using electrons from Z boson decays.
Their uncertainties are provided as function of $E_\mathrm{T}$ and $\eta$~\cite{electron_energy_errors_Run1} 
by working group as well.

All these uncertainties are grouped together under the name ``Electron reconstruction ans identification'' in \TableRef{tab:syst}.

\subsection{Trigger and luminosity}
Efficiency of trigger for electrons with different $p_T$ and $\eta$ have some uncertainty as well.
This uncertainty is estimated to be at $\sim1\%$ level by the ATLAS trigger groups.
The resulting uncertainty on the yield in the signal region is different, since there are two leptons that can pass the trigger requirement. 

To scale background prediction obtained with MC simulation to the data 
one need to know an integrated luminosity of the collected data sample. 
This is why a luminosity uncertainty propagates to all backgrounds measured from MC simulation.
The integrated luminosity uncertainty in 2012 is equal to 2.8$\%$~\cite{Aad:2013ucp} 
and it was obtained in similar way as described in \SectionRef{sec:lucid_performance}.

\subsection{Statistics and theoretical cross section}
Limited number of simulated events in the Monte Carlo samples lead to an additional uncertainty which is listed as ``MC statistics'' in \TableRef{tab:syst}.
This uncertainty also include effect of the limited number of events in data sets used in data-driven methods to measure charge flip rate and fake factor.
Statistical uncertainty is significant for high-mass bins.

As one can see from \TableRef{tab:MC_cross} different processes were simulated with using different MC generators and PDF sets.
Additional uncertainty arises from  their choice.
To estimate these uncertainties different MC generators, parton shower and hadronization models are tested.
Uncertainty on PDF and strong coupling constant $\alpha_{\mathrm{s}}$ are estimated by using different PDF sets following recommendations from~\cite{pdf4lhc}.
Also renormalization and factorization scales are varied by factor of two to estimate an effect on total cross section of the processes.
Summary list of uncertainties used in the analysis is shown in \TableRef{tab:systematics_common}.
Detailed information about cross section calculations and their errors for some processes 
are reported in~\cite{diboson_cross_section,ttW_cross_section,ttV_cross_section}.

% \subsection{Photon misidentification as electron}
% TODO


\toDo[describe about $W\gamma$ background process somewhere]

\toFix[define ``MPI'' somewhere before using in the table]

\begin{table}[ht]
\begin{center}
\begin{tabular}{l|l|c}
Source & Process & Uncertainty \\
\hline
\multirow{2}{*}{Trigger} & Signal and background & \multirow{2}{*}{2.1-2.6\%}  \\
& from MC simulations &\\
\hline
Electron reconstruction & Signal, prompt &\multirow{2}{*}{1.9--2.7\%}\\
and identification & background&\\\hline
Electron charge  & Opposite-sign& \multirow{2}{*}{9\%} \\
misidentification& backgrounds&\\\hline
Determination of & Non-prompt &\multirow{2}{*}{22\%}\\
fake factor $f$& backgrounds&  \\\hline
\multirow{2}{*}{Luminosity} & Signal and background& \multirow{2}{*}{2.8\%}\\
& from MC simulations&\\\hline
\multirow{2}{*}{MC statistics} & Backgrounds from &  \multirow{2}{*}{5\%}\\
& MC simulations &\\\hline
%Differences between fast & \multirow{2}{*}{Signal}& \multirow{2}{*}{1.8\%}& \multirow{2}{*}{5\%}& \multirow{2}{*}{0.7\%}\\
%and normal simulation &&&&\\\hline
Photon misidentification & \multirow{2}{*}{$W\gamma$} & \multirow{2}{*}{13\%}\\
as electron&&\\\hline
\multirow{2}{*}{MC cross sections} & Prompt, opposite-& \multirow{2}{*}{4\%}\\
& sign backgrounds & \\
\end{tabular}
\end{center}
\caption{Sources of systematic uncertainty (in \%) on the signal yield and the expected background predictions, described in the second column, for the mass range $m_{ee} > 15$ GeV.
}
\label{tab:syst}
\end{table}


\begin{table}[ht]
\caption{Theoretical uncertainties on the production cross section.}
\begin{center}
\begin{tabular}{l|c}
Processes affected & Uncertainty \\
\hline
 Drell-Yan (Charge flips) & $\pm$7\% \\
 $WZ$ & $\pm$7\% \\
 $ZZ$ & $\pm$5\% \\
 $t\bar{t}W$, $t\bar{t}Z$  & $\pm$22\% \\
 $W^{\pm}W^{\pm}$ & $\pm$50\% \\
 MPI $WW$, $WZ$, $ZZ$ & $\pm$100\% \\
 $t\bar{t}$ & $\pm$5\% \\
 $W\gamma$ & $\pm$14\% \\
\end{tabular}
\end{center}
\label{tab:systematics_common}
\end{table}



\section{Signal Region}
\label{sec:ss_signalRegion}

Same-sign lepton pair invariant mass in the signal region is shown in~\FigureRef{fig:signal_mass}.
The last bin in the histogram is an overflow bin, which include number of pairs with invariant mass higher than 600 GeV.
Observed number of pairs is compatible with predicted background.
As one can see, the dominant background arises from charge flip component.
Predicted contribution from each background processes with different invariant mass cuts are shown in \TableRef{tab:2iso_ee_SS}.
In \FigureRef{fig:signal_kinematics} kinematics of the leading lepton is shown. 
In \FigureRef{fig:delta_phi} angle between same-sign electrons in the pair is shown as well.
Background prediction reasonably describes observed numbers in all of these distributions within uncertainties bands.

\begin{figure}[h]
\begin{center}
\includegraphics[width=0.7\textwidth]{SS/paper_draft/2isoSS_ee_mll.eps}
\caption{\toDo Invariant mass distributions for $ee$ pairs in the same-sign signal region.}
\label{fig:signal_mass}
\end{center}
\end{figure}

\begin{table*}[htbp]
\caption{Expected and observed numbers of pairs of isolated same-sign electrons for various cuts on the dielectron invariant mass, \mee. The uncertainties shown include statistical and systematic contributions.}
\begin{center}
\resizebox{\textwidth}{!}{
\begin{tabular}{l|c|c|c|c|c|c|c}
\hline
Sample & \multicolumn{5}{|c}{Number of electron pairs with  $m(e^{\pm}e^{\pm})$} \\
 & $>15$~GeV & $>100$~GeV & $>200$~GeV & $>300$~GeV & $>400$~GeV & $>500$~GeV & $>600$~GeV \\
\hline \hline
Non-prompt	& $ 518.57 \pm 120.17 $	& $ 247.49 \pm 49.5 $	& $ 71.67 \pm 13.15 $	& $ 22.66 \pm 4.8 $	& $ 8.13 \pm 2.42 $	& $ 3.12 \pm 1.49 $	& $ 0.78 \pm 1.01 $	\\[+0.05in]
\hline\hline
$W\gamma$	& $ 175.25 \pm 36.28 $	& $ 74.89 \pm 15.62 $	& $ 22.42 \pm 5.15 $	& $ 8.04 \pm 2.26 $	& $ 3.84 \pm 1.31 $	& $ 2.69 \pm 1.05 $	& $ 1.02 \pm 0.57 $	\\[+0.05in]
\hline\hline
Drell-Yan	& $ 968.61 \pm 145.63 $	& $ 513.53 \pm 77.7 $	& $ 130.91 \pm 26.99 $	& $ 36.1 \pm 12.17 $	& $ 12.8 \pm 7.89 $	& $ 4.79 \pm 4.86 $	& $ 4.79 \pm 4.86 $	\\[+0.05in]
$t\bar{t}$	& $ 36.92 \pm 6.01 $	& $ 30.1 \pm 4.99 $	& $ 14.55 \pm 2.8 $	& $ 5.05 \pm 1.32 $	& $ 2.15 \pm 0.78 $	& $ 1.05 \pm 0.58 $	& $ 1.18 \pm 0.56 $	\\[+0.05in]
$WW$	& $ 13.01 \pm 2.34 $	& $ 10.74 \pm 1.96 $	& $ 4.85 \pm 0.97 $	& $ 1.86 \pm 0.45 $	& $ 0.68 \pm 0.22 $	& $ 0.43 \pm 0.16 $	& $ 0.28 \pm 0.13 $	\\[+0.05in]
\hline
Charge Flip total	& $ 1018.54 \pm 145.78 $	& $ 554.37 \pm 77.89 $	& $ 150.31 \pm 27.16 $	& $ 43.01 \pm 12.25 $	& $ 15.62 \pm 7.93 $	& $ 6.27 \pm 4.89 $	& $ 6.25 \pm 4.89 $	\\[+0.05in]
\hline\hline
$ZZ$	& $ 86.05 \pm 7.21 $	& $ 22.42 \pm 2.11 $	& $ 6.75 \pm 0.84 $	& $ 1.78 \pm 0.37 $	& $ 0.61 \pm 0.2 $	& $ 0.34 \pm 0.16 $	& $ 0.21 \pm 0.12 $	\\[+0.05in]
$WZ$	& $ 234.36 \pm 22.24 $	& $ 132.79 \pm 12.76 $	& $ 37.12 \pm 3.9 $	& $ 10.95 \pm 1.43 $	& $ 3.23 \pm 0.61 $	& $ 1.5 \pm 0.4 $	& $ 0.5 \pm 0.22 $	\\[+0.05in]
$t\bar{t}W$	& $ 5.33 \pm 1.23 $	& $ 3.83 \pm 0.89 $	& $ 1.32 \pm 0.32 $	& $ 0.44 \pm 0.11 $	& $ 0.14 \pm 0.04 $	& $ 0.08 \pm 0.03 $	& $ 0.03 \pm 0.01 $	\\[+0.05in]
$t\bar{t}Z$	& $ 1.73 \pm 0.41 $	& $ 1.2 \pm 0.29 $	& $ 0.4 \pm 0.1 $	& $ 0.11 \pm 0.04 $	& $ 0.03 \pm 0.01 $	& $ 0.02 \pm 0.01 $	& $ 0.01 \pm 0.01 $	\\[+0.05in]
$WWjj$	& $ 14.99 \pm 7.59 $	& $ 12.1 \pm 6.14 $	& $ 5.55 \pm 2.84 $	& $ 2.35 \pm 1.22 $	& $ 1.22 \pm 0.66 $	& $ 0.4 \pm 0.24 $	& $ 0.16 \pm 0.11 $	\\[+0.05in]
MPI	& $ 4.04 \pm 4.06 $	& $ 1.6 \pm 1.61 $	& $ 0.38 \pm 0.39 $	& $ 0.06 \pm 0.07 $	& $ 0.02 \pm 0.02 $	& $ 0 \pm 0 $	& $ 0 \pm 0 $	\\[+0.05in]
\hline
Prompt total	& $ 346.51 \pm 24.95 $	& $ 173.94 \pm 14.44 $	& $ 51.52 \pm 4.93 $	& $ 15.7 \pm 1.92 $	& $ 5.25 \pm 0.92 $	& $ 2.34 \pm 0.49 $	& $ 0.91 \pm 0.28 $	\\[+0.05in]
\hline\hline
Total Background	& $ 2058.86 \pm 193.92 $	& $ 1050.69 \pm 94.67 $	& $ 295.92 \pm 30.99 $	& $ 89.41 \pm 13.49 $	& $ 32.83 \pm 8.44 $	& $ 14.41 \pm 5.25 $	& $ 8.96 \pm 5.04 $	\\[+0.05in]
\hline\hline
Data	& $ 1976 $	& $ 987 $	& $ 265 $	& $ 83 $	& $ 30 $	& $ 13 $	& $ 7 $	\\[+0.05in]

\hline
\end{tabular}
}
\end{center}
\label{tab:2iso_ee_SS}
\end{table*}


\begin{figure}
\begin{subfigure}{.5\textwidth}
  \centering
  \includegraphics[width=\textwidth]{SS/paper_draft/2isoSS_ee_pt1.eps}
\end{subfigure}%
\begin{subfigure}{.5\textwidth}
  \centering
  \includegraphics[width=\textwidth]{SS/paper_draft/2isoSS_ee_eta1.eps}
\end{subfigure}
\caption{\toDo $p_T$ and $\eta$ distributions of leading electron.}
  \label{fig:signal_kinematics}
\end{figure}

\begin{figure}[h]
\begin{center}
\includegraphics[width=0.7\textwidth]{SS/support_note/dielectrons/crs/signal_deltaPhi_v2.eps}
\caption{\toDo $\Delta\phi(ee)$ in the $ee$ signal region.}
\label{fig:delta_phi}
\end{center}
\end{figure}



\subsection{Limit setting}

Background prediction describes data very well and there is no visible significant deviations which can indicate a potential presence of the BSM signal.
The idea of this analysis is to be as general as possible and to perform a search for the a new physics without favoring any BSM model.
This is why next step is to calculate an exclusion limits on the new physics with same-sign lepton pairs.

Due to a limited acceptance of the detector signal region was design to use most efficient phase space possible.
This is why cross section of the process. $\sigma$, relates with measured cross section in the signal region, the so-called fiducial cross section, $\sigma^{fid}$, as:
\begin{equation}
 \sigma = \dfrac{\sigma^{fid}}{<N_{pair}>A}
 \label{eq:cross_section}
\end{equation}
where $<N_{pair}>$ is average number of same-sign pairs produced per event and $A$ is the fiducial acceptance (or volume).
Definition of fiducial volume is discussed in \SectionRef{subsec:fid_volume_eff}.

The cross section limits are derived with using a $CLs$~\cite{CLs_tecnique,CLs_2} prescription with help of RooStat~\cite{RooStat_project} framework 
provided by the ATLAS Statistics Committee. $CLs$ method state that the signal hypothesis is excluded at the confidence level $CL$ when
\begin{equation}
 1 - CL_s \leq CL
\end{equation}
where $CL_s$ is defined as
\begin{equation}
 CL_s \equiv \dfrac{CL_{s+b}}{CL_b}
\end{equation}
where $CL_b$ is a confidence level observed for the background only hypothesis and $CL_{s+b}$ - for background plus signal hypothesis.
Practically $CL_b$ ($CL_{s+b}$) is probability for finding the observed data given an expected background (background plus signal).
This probability is Poisson distributed and is calculated basing on the number of observed and expected same-sign electron pairs in the signal region.
A used test-statistics for the limits is a log likelihood ratio.
\toAsk[is it correct to say like this?] Systematic uncertainties and correlations between them are taken into account during the limit settings.
For example, the charge-flip scale factor uncertainty is correlated among Drell-Yan, $t\bar{t}$, $WW$ and $W\gamma$ background samples.
Experimental errors such as electron reconstruction, identification, energy scale and trigger 
are treated as one uncertainty and correlated among all expected backgrounds and signal.
MC cross section errors are independent except for dibosons ($ZZ$, $WZ$ and $WW$).
The luminosity is common to all background samples, so it is naturally correlated.
Statistical errors are independent.

Following this prescription and using number of expected and observed same-sign lepton pairs one can compute 
an upper limits at the certain confidence level (typically in 95$\%$ level) on the number of same-sign lepton pairs ($N_{95}$)
contributed by the new physics beyond the SM. Limits can be set for different invariant mass thresholds, because it is main observable for the analysis.
Limits on the number of pairs can be translated to the upper limits on fiducial cross section as:
\begin{equation}
 \sigma_{95}^{fid} = \dfrac{N_{95}}{\epsilon_{fid} \times \int \mathscr{L} dt}
 \label{eq:fid_cross_section}
\end{equation}
where $\int \mathscr{L} dt$ is an integrated luminosity of the data and $\epsilon_{fid}$ is a fiducial efficiency for finding a same-sign electron pair from
a possible signal from new physics in the fiducial volume, which is described in \SectionRef{subsec:fid_volume_eff}.

% TODO explain fiducial efficiency

\subsection{Fiducial volume and fiducial efficiency}
\label{subsec:fid_volume_eff}

As can be seen from \EquationRef{eq:cross_section} and \EquationRef{eq:fid_cross_section} in order to translate number of measured and expected lepton pairs 
to the cross section limit of the potential signal from new physics one need to know the fiducial volume and efficiency.
The reason is that detector does not 100$\%$ efficiently reconstruct leptons as well as it does not cover the whole phase space around the interaction point.
Fiducial volume represent the phase space region that is truncated in a manner that mimics detector acceptance.
It is defined by set of cuts on truth (generator) level. 
Kinematic cuts on the electrons are identical to the one used in the signal region definition on the reconstruction level:
\begin{itemize}
 \item Leading electron $p_T > 25$ GeV
 \item Subleading electron $p_T > 20$ GeV
 \item $|\eta|<1.37$ or $1.52<|\eta|<2.47$
\end{itemize}
Requirements on the electron pair are the same as well:
\begin{itemize}
 \item Same-sign pair with $m_{ee} > 15$ GeV
 \item Veto pairs with $70 < m_{ee} < 110$ GeV
 \item No opposite-sign same-flavour pairs with $|m_{ee} - m_{Z}| < 10$ GeV
\end{itemize}
Since electrons are required to be isolated on teh reconstruction level, isolation also has to be applied on the truth level as well.
Track-based isolation are identical as on reconstruction level:
all charged particles within the cone $\Delta R < 0.3$ around the electron with $p_T > 0.4$ GeV are considered and
sum of their $p_T$ has to be not larger than 10$\%$ of the electron $p_T$.
Calorimeter-based isolation was not applied on the truth level due to significantly different behavior of the cut on reconstruction and truth levels.

As one can see cuts on truth level are identical to one used in reconstruction level.
In case of the ideal detector fiducial volume would completely correspond to the geometrical detector acceptance.
But because the real detector does not provide 100$\%$ reconstruction efficiency in order to relate fiducial volume with real geometrical detector acceptance
a fiducial efficiency is used, which is defined as:
\begin{equation}
 \epsilon_{fid} = \dfrac{N_r}{N_f}
\end{equation}
where $N_f$ is number of electron pairs which pass fiducial volume cuts on the truth level and $N_r$ - which pass fiducial volume cuts on the truth level 
as well as all signal selection cuts on the reconstruction level.

In order to perform a search for new physics in the model independent way one want fiducial efficiency to be constant and does not depend on the type of the BSM model.
However, different models provides different $p_T$ and $\eta$ spectra and electron reconstruction efficiency depends both from $p_T$ and $\eta$.
As reported in~\cite{electron_tight} efficiency can vary up to 15$\%$ for ``tight'' identification criteria with respect to electron $p_T$.
Also presence and number of the jets in the final state, which is model dependent, affects electron isolation efficiency, which will have an effect on 
fiducial efficiency as well.

In order to estimate value of the fiducial efficiency, efficiencies for four different BSM models were calculated:
\begin{itemize}
 \item Double charged Higgs. This model assume production of two doubly charged scalar bosons which, decaying leptonically, will provide two pairs of same-sign leptons.
 No jets in the final state. A range of 100-1000 GeV of double charged Higgs were considered.
 \item Colored Zee-Babu model. It assumes production of the diquark, which decays to two leptoquarks with the same charge, 
 which in their turn decay to lepton and quark. Final state consist from one same-sign lepton pair plus two jets.
 Masses considered in the model were 2.5-3.5 TeV for diquark and 1-1.4 TeV for leptoquarks.
 \item Production of heavy right-handed $W_R$ boson and heavy Majorana neutrino. $W_R$ decays to lepton and Majorana neutrino, 
 which decays to $W$ boson and another lepton. Final state consist from one same-sign pair and products of the W boson decay.
 Mass of $W_R$ was between 1 TeV and 2 TeV, while mass of Majorana neutrino was in range 0.25-1.5 TeV
 \item Pair production of forth generation down-type quark. Both quarks decay semi-leptonically to $t$ quark and then to $b$ quark.
 Final state consist from two jets plus four W bosons. At least two same-sign bosons have to decay leptonically in order to provide a same-sign lepton pair.
 This state is characterized by large hadron activity due to high jet multiplicity.
 Mass of forth generation quark was varied from 400 GeV to 1 TeV.
\end{itemize}
Fiducial efficiencies for these model were calculated with different dilepton mass thresholds.
Obtained efficiencies are in range 48-74$\%$. The lowest efficiency were given by forth generation down-type quark, while the highest one 
for heavy right-handed $W_R$ boson and heavy Majorana neutrino. Latter model had larger efficiency with respect to doubly charged Higgs model
because the final state of doubly charged Higgs model contains two same-sign pairs against one pair in Majorana model.
Efficiencies were measured separately for positive and negative same-sign pairs separately but no significant difference were observed.

To provide conservative cross section limit setting for new physics beyond the SM the lowest obtained efficiency, which was 48.3$\%$, have been decided to be used.

\toAsk[why do we need calorimeter-based isolation? Because it helps to reduce pile-up dependence?]

\subsection{Fiducial cross section limits}

Computed upper limits at the 95$\%$ confidence level on the fiducial cross section ($\sigma_{95}^{fid}$)
of the new physics beyond the SM for different invariant mass thresholds based on the values from \TableRef{tab:2iso_ee_SS}
are shown in \FigureRef{fig:inclusive_fid_limit}. Limits separately for positive and negative same-sign pairs are shown in \FigureRef{fig:signal_kinematics}.
Expected limits are shown with 2$\sigma$ bands. Limits are summarized in \TableRef{tab:limits}.

\begin{figure}[h]
\begin{center}
\includegraphics[width=0.7\textwidth]{SS/paper/limit_ee_all.eps}
\caption{\toDo 95\% C.L. upper limits on the cross section for new physics contributing to the fiducial region for $ee$ events.}
\label{fig:inclusive_fid_limit}
\end{center}
\end{figure}


\begin{figure}
\begin{subfigure}{.5\textwidth}
  \centering
  \includegraphics[width=\textwidth]{SS/support_note/limits/limit_ee_neg.eps}
\end{subfigure}%
\begin{subfigure}{.5\textwidth}
  \centering
  \includegraphics[width=\textwidth]{SS/support_note/limits/limit_ee_pos.eps}
\end{subfigure}
\caption{\toDo 95\% C.L. upper limits on the cross section for new physics contributing to the fiducial region 
for $e^{+}e^{+}$ (left) and $e^{+}e^{+}$ (right) events.}
  \label{fig:signal_kinematics}
\end{figure}


\begin{table*}[!ht]
\begin{center}
\begin{tabular}{c||c|c||c|c||c|c}

 & \multicolumn{6}{c}{95\%  CL upper limit [fb]} \\
 & \multicolumn{2}{c||}{$e^{\pm}e^{\pm}$} & \multicolumn{2}{c||}{$e^{+}e^{+}$} & \multicolumn{2}{c}{$e^{-}e^{-}$} \\
Mass range & Expected & Observed & Expected & Observed & Expected & Observed \\
%[+0.05in]
\hline
\rule{0pt}{3ex}
  $>15$~GeV   &  $39^{+10}_{-13}$        &  32    &    $27^{+11}_{-6}$         &  28    &    $23^{+8}_{-5}$          &  19\\
  $>100$~GeV  &  $19^{+6}_{-6}$          &  14    &    $14.3^{+5.4}_{-2.8}$    &  13.5  &    $10.8^{+4.4}_{-2.4}$    &  9.0\\
  $>200$~GeV  &  $6.8^{+2.6}_{-1.7}$     &  5.3   &    $5.4^{+2.0}_{-1.4}$     &  4.6   &    $3.9^{+1.4}_{-1.2}$     &  3.5\\
  $>300$~GeV  &  $3.3^{+1.3}_{-0.4}$     &  3.3   &    $2.5^{+0.9}_{-0.6}$     &  2.0   &    $2.1^{+0.7}_{-0.5}$     &  2.6\\
  $>400$~GeV  &  $2.02^{+0.74}_{-0.21}$  &  2.03  &    $1.59^{+0.47}_{-0.34}$  &  1.64  &    $1.56^{+0.41}_{-0.31}$  &  1.35\\
  $>500$~GeV  &  $1.25^{+0.36}_{-0.26}$  &  1.10  &    $1.44^{+0.34}_{-0.36}$  &  1.55  &    $0.69^{+0.27}_{-0.17}$  &  0.64\\
  $>600$~GeV  &  $0.99^{+0.34}_{-0.20}$  &  1.02  &    $1.27^{+0.37}_{-0.26}$  &  1.10  &    $0.58^{+0.21}_{-0.08}$  &  0.61\\

\end{tabular}
\end{center}
 \caption{Upper limit at 95\% CL on the fiducial cross-section for $\ell^{\pm} \ell^{\pm}$ pairs from non-SM signals. The expected limits and their $1 \sigma$ uncertainties are given together with the observed limits derived from the data. Limits are given inclusively and separated by charge.}
\label{tab:limits}
\end{table*}


\section{Mass limits on doubly charged Higgs}
As an example of the BSM model which produce same-sign lepton pairs in the final state, a pair production of doubly charged Higgs bosons were studied.
The search strategy are the same as described above.
Since the final state of this model has two same-sign pairs of leptons no jet activity or presence of missing transverse energy, 
there is no need to optimize signal selection.
Doubly charged Higgs decay is visible as a sharp peak in the dilepton invariant mass.
This is why using fiducial efficiency calculated in bins of 100 GeV (as it was done for the fiducial limit calculations) is not optimal from point of view signal sensitivity.
Thus, search is performed in mass bins with mass dependent width.

\subsection{MC simulation}
A few signal samples were generated with Pythia8~\cite{pythia8} with different masses of the left-handed and right-handed doubly charged Higgs bosons.
Simulated masses were in range 50-600 GeV in steps of 50 GeV and one sample with 1 TeV mass.
Kinematics of the left-handed and right-handed Higgs bosons are identical, but production rate is different due to different coupling constant to Z boson mediator~\cite{dch_note}.
Cross sections were calculated with NLO precision. \toDo[link]

\subsection{Model acceptance and efficiency}
The width of doubly charged Higgs decay is mainly cause by the detector momentum resolution of the electrons.
Because decay width becomes larger with the invariant mass, search was performed in the mass bins with variable width.
The idea behind mass bin widths optimization lays in two facts. On the one hand, a mass bin have to cover as much signal as possible
but on the other hand, background contribution in the mass bin is desired to be as small as possible.
To satisfy both conditions signal significance, $S$, was chosen as optimization criteria:
\begin{equation}
 S = \sqrt{ 2((s+B)ln(1+s/B)-s } 
\end{equation}
where $s$ is the expected signal and $B=b+\delta b^2$ is the predicted background plus background systematic uncertainty squared.
A bin width is parameterized as a second degree polynomial of the Higgs mass.
Coefficients of the polynomial were parameters to optimize.
During the optimization procedure it become clear that there is a third factor which have to be taken into account.
Due to the limited statistics of the predicted background one cannot have too many mass bins otherwise cross section
exclusion limit will fluctuate very much from bin to bin. 
The optimal bin width parameterization was found to be $\pm(0.04 \times m(\dch) + 0.2 \cdot 10^{-4} \times m(\dch)^{2})$,
where $m(\dch)$ is the Higgs mass.

Next step is to define how many lepton pairs produced from Higgs boson decays are reconstructed, selected and fall into appropriate mass bin.
Number of generated Higgs bosons is known, this is why one need only to count number of reconstructed same-sign pairs which pass signal selection and fall into appropriate mass bins.
The ratio of reconstructed to the total number of generated pairs correspond to the total efficiency, which includes acceptance and efficiency of the signal selection.
Efficiencies for each mass points were calculated but in order to interpolate between the simulated mass points, total efficiency, $\varepsilon_{tot}$, is fitted by
an empirical piecewise function:
\begin{equation}
\varepsilon_{tot}(m) = \begin{cases} p_{0} (1-e^{-(m-p_{1})/p_{2}}), & \mbox{if } m < 450\mbox{ GeV} \\ 
p_{3} + p_{4} m, & \mbox{if } m \geq 450\mbox{ GeV} \end{cases}
\label{eq:ee_eff}
\end{equation}
where $m$ is Higgs mass and $p_{0}, p_{1}, p_{2}, p_{3}, p_{4}$ are fit parameters, which are shown in \TableRef{tab:ee_eff_params}.
\begin{table}[htbp]
    \caption{Fitted parameter values for Equation~\ref{eq:ee_eff}, which gives $\varepsilon_{tot}(m)$.}
    \begin{center}
    \begin{tabular}{ l | l }
        \hline
        Parameter & Value \\
        \hline
        $p_{0}$    & $4.76 \times 10^{-1}$ \\[+0.05in]
        $p_{1}$    & $2.94 \times 10^{+1}$ \\[+0.05in]
        $p_{2}$    & $1.05 \times 10^{+2}$ \\[+0.05in]
        $p_{3}$    & $4.51 \times 10^{-1}$ \\[+0.05in]
        $p_{4}$    & set by requiring continuity \\[+0.05in]
        \hline
    \end{tabular}
    \end{center}
    \label{tab:ee_eff_params}
\end{table}

Computed efficiencies for available mass points and their fit are shown in \FigureRef{fig:signal_efficiency}.

\begin{figure}[h]
\begin{center}
\includegraphics[width=0.6\textwidth]{SS/support_note/DCHLimits/effic_DCH_ee_pairSF.eps}
\caption{\toDo Total acceptance times efficiency ($\varepsilon_{tot}$ in the text) as a function of simulated $H^{\pm\pm}$ mass for the $ee$ channels, 
fitted with piecewise empirical functions described in the text.}
\label{fig:signal_efficiency}
\end{center}
\end{figure}

\toDo[describe systematics for the DCH]

\toDo[CLs = CLs+b/CLb. What is the signal in case of limit on new physics?]

\subsection{Cross section and mass limits}

Invariant mass distribution for same-sign electron pairs together with the signal from left-handed Higgs with masses 300 GeV and 500 GeV superimposed are shown in \FigureRef{fig:signal_mass}.
Branching ratio of $H^{\pm\pm} \to e^{\pm}e^{\pm}$ decay is assumed to be 100$\%$.

\begin{figure}[h]
\begin{center}
\includegraphics[width=0.7\textwidth]{SS/support_note/DCHPlots/ee/2isoSS_ee_mll_dch.eps}
\caption{\toDo Invariant mass distributions for $\ee$ pairs passing the full event selection. 
The data are shown as closed circles. The stacked histograms represent 
the backgrounds composed of pairs of prompt leptons from SM processes, 
pairs with at least one non-prompt lepton and backgrounds arising from charge misidentification and conversions. 
The open histograms show the expected signal from simulated $H^{\pm\pm}$ samples,
assuming a 100\% branching ratio to the decay channel considered. The last bin is an overflow bin.}
\label{fig:signal_mass}
\end{center}
\end{figure}

Upper cross section limit on pair production of doubly charged Higgs are set in the same way as limits on the new physics described earlier.
A cross section is determined as:
\begin{equation}
 \sigma_{HH}\times BR =\frac{N_{H}^{rec}}{2\times A\times \epsilon \times \int \mathscr{L} dt}
\end{equation}
where BR is branching ratio of $H^{\pm\pm} \to e^{\pm}e^{\pm}$ decay, $N_{H}^{rec}$ is number of reconstructed $H^{\pm\pm}$, $A\times \epsilon$ is total efficiency described earlier and
$\int \mathscr{L} dt$ is integrated luminosity.
Factor 2 in the denominator is needed to take into account presence of two same-sign pairs from $H^{++}$ and $H^{--}$ in the event.

Upper cross section limit times branching ratio (which is assumed to be 100$\%$) at 95$\%$ CL is shown in \FigureRef{fig:dch_limits_mass}.
The scatter between mass bins are caused by fluctuations of the predicted background due to low statistics in bins.
A good agreement between expected and observed limit lines can be seen and all deviations lay within $2\sigma$.
The theoretical cross section curves as a function of Higgs mass for left- and right-handed doubly charged Higgs bosons 
are shown as well.
Lower mass limits of the model correspond to intersection of the theoretical curve with expected cross section limit obtained
mass limits are summarized in the \TableRef{tab:limits_mass}.

\begin{figure}[h]
\begin{center}
\includegraphics[width=0.6\textwidth]{SS/paper/limitDCH_ee_all.eps}
\caption{\toDo 95\% C.L. upper limits on the cross section for new physics contribution for $ee$ events.  
Pair production cross sections for left and right-handed $H^{\pm\pm}$ are overlaid.}
\label{fig:dch_limits_mass}
\end{center}
\end{figure}

\begin{table}[htbp]
\caption{Upper limit at 95\% C.L. on mass of \dch, assuming 100\% branching fraction.}
\begin{center}
\begin{tabular}{c||c|c}
& \multicolumn{2}{c}{95\%  C.L. upper limit [GeV]}\\
Signal & expected & observed \\
\hline
$H^{\pm\pm}_L$ & $552.6^{+11.1}_{-49.9}$ & $551.2 \pm 3.1$ \\
\hline
$H^{\pm\pm}_R$ & $424.8^{+1.0}_{-59.7}$ & $374.0 \pm 6.2$ \\
\end{tabular}
\end{center}
\label{tab:limits_mass}
\end{table}

Also limits above can be interpreted like mass limits as a function of the branching ratio for $H^{\pm\pm}_L$ and $H^{\pm\pm}_R$ decays.
They are shown in \FigureRef{fig:dch_limits_BR}.

\begin{figure}
\begin{subfigure}{.5\textwidth}
  \centering
  \includegraphics[width=\textwidth]{SS/paper/limitDCH_ee_allvsBR_LH.eps}
\end{subfigure}%
\begin{subfigure}{.5\textwidth}
  \centering
  \includegraphics[width=\textwidth]{SS/paper/limitDCH_ee_allvsBR_RH.eps}
\end{subfigure}
\caption{\toDo 95\% C.L. limits on the right-handed Doubly Charged Higgs mass vs 
branching fraction of $H^{\pm\pm}_L$ (left) and $H^{\pm\pm}_R$ (right) for $ee$ events.}
  \label{fig:dch_limits_BR}
\end{figure}

\section{Outlook}
\label{sec:ssOutlook}

