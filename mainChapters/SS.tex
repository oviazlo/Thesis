\chapter{Searches for beyond Standard Model physics with same-sign dileptons}
\label{chap:SS}
\section{Motivation}

% TODO write something about lepton signature... why it has benefits wrt to jets and photons...
% TODO write about benefits of the general search (or model-independent? or it's dangerous to say like this?)... theorists can use these results for their models?
% TODO motivation flow: why leptons -> why same-sign dilepton -> why model-independent search?

The idea of the current analysis was to make a general search for a new physics in a same-sign dilepton channel. ???
% Why do we use leptons as final objects?
Leptons as a physics signature are very clean and straighforward observalbles. 
ATLAS detector capable to detect leptons with amazing presition... bla-bla-bla... ???
% Why do we use same-sign dilepton channel
Looking for a same-sign dilepton channel is motivated by the fact that cross section for SM processes with such signature are really small, which make 
it very sensitive possible signals of new physics.  ???
% Why model-independent search
There are other searches in ATLAS which taget the same final signature but with requiring additional cuts on number of jets [link???] of missing transverse energy [link].
In such way these analyses target limited number of specific models.
Analyses presented in this chapter aims to be more general and model-independent. ???
[Which model does this analysis cover?]

\section{Background processes}
\label{sec:wprimeBackgrounds}
% TODO describe that there are hree possibilities to reconstruct same-sign lepton pair

There are three different components which contribute to the signal selection.
The first component is so called prompt contribution when same-sign dilepton originates from the SM processes.
Just a few SM processes contribute to the same-sign dilepton final state. In such processes same-sign leptons originates from semi-weakly decay of top quark
and/or leptonic decays of $W, Z$ bosons. Some example of the Feynman diagrams are shown in \FigureRef{fig:Diboson_feynman}.
Overall SM processes with same-sign dilepton final state have relatively small cross section.

The second component comes from the wrong reconstruction of the lepton itself, when pion or jet is reconstructed as a lepton or from leptons which were born 
not in the pp collision but from decays of secondary particles, e.g. kaons. This component is called non-prompt or fake component.

The third component comes from misidentification of the electrical charge of the lepton, which make processes, where opposing lepton pair is produced, 
contribute to the signal selection as well. The charge misidentification effect become visible for high-momentum leptons, when curvature of the track 
is hard to reconstruct.

% TODO describe connection of written above with table with all used MC samples...
% TODO normalization is just checked for it is not applied as a scale factor. Cross check with Monika!!!
Prompt contribution is described by the MC simulation. 
% with applying the overall normalization of the MC samples to the data in the control region.
Non-prompt contribution is derived with the data-driven method, so called fake factor method.
Charge flip rate is estimated from Z boson lepton decays and is applied to the opposite-sign MC samples.

List of all used MC samples are shown in \TableRef{tab:MC_cross}. All other processes which are not listed in the table don't contribute
significantly to the same-sign selection and are considered negligable.

\begin{table}[ht]
  \begin{center}
    \begin{tabular}{l|c|c|c}

      \hline
Process &  Generator&  PDF set & Normalisation \\
&  + fragmentation/ &  & based on \\
&  hadronisation & &\\
\hline\hline
\multirow{2}{*}{$WZ$ } &  \multirow{2}{*}{{\scshape sherpa-1.4.1} \cite{Sherpa}} &   \multirow{2}{*}{CT10 \cite{ct10}} & NLO QCD \\
 & & &  with {\scshape mcfm-6.2}\cite{mcfm} \\
\hline
\multirow{2}{*}{$ZZ$}  &  \multirow{2}{*}{{\scshape sherpa-1.4.1}} & \multirow{2}{*}{CT10} & NLO QCD  \\
& & &  with {\scshape mcfm}-6.2 \\
\hline
\multirow{2}{*}{\Wpm\Wpm}  & M{\scshape ad}G{\scshape raph}-5.1.4.8 \cite{madgraph4}  &   \multirow{2}{*}{CTEQ6L1 \cite{cteq}}  &  \multirow{2}{*}{LO QCD} \\
&  {\scshape pythia-8.165} \cite{pythia8}& &\\
\hline
\ttbar $V$, & M{\scshape ad}G{\scshape raph}-5.1.4.8  & \multirow{2}{*}{CTEQ6L1} & \multirow{2}{*}{NLO QCD \cite{top9,ttbarW}} \\
$V=W,Z$ &  + {\scshape pythia-6.426} & & \\
\hline
 MPI $VV$ &  \multirow{2}{*}{{\scshape pythia-8.165}\cite{pythia8}}  &  \multirow{2}{*}{CTEQ6L1} &  \multirow{2}{*}{LO QCD} \\
 $V=W,Z$ &  & & \\[+0.025in]
\hline
\hline
\multirow{2}{*}{$Z/\gamma^* +$ jets} & {\scshape alpgen-2.14} \cite{Alpgen}&\multirow{2}{*}{CTEQ6L1}& {\scshape dynnlo-1.1} \cite{dynnlo} with \\
 & + {\scshape herwig-6.520} \cite{Herwig1, Herwig2}& & MSTW2008 NNLO \cite{mstw} \\
\hline
\multirow{2}{*}{\ttbar} & {\scshape mc@nlo}-4.06 \cite{Mcnlo, Mcnlo2} & \multirow{2}{*}{CT10}&{NNLO+NNLL } \\
& + {\scshape herwig-6.520} & & QCD \cite{top1,top2,top3,top4,top5,top6} \\
\hline
\multirow{2}{*}{$Wt$} & {\scshape mc@nlo-4.06}  & \multirow{2}{*}{CT10}& {NNLO+NNLL } \\
   & + {\scshape herwig-6.520} & & QCD \cite{top7,top8}\\
\hline
\multirow{2}{*}{$W^{\pm}W^{\mp}$} & \multirow{2}{*}{{\scshape sherpa-1.4.1}} & \multirow{2}{*}{CT10}& NLO QCD \\
& &  & with {\scshape mcfm-6.2}\\
\hline
\multirow{2}{*}{$W\gamma$} & \multirow{2}{*}{{\scshape sherpa}-1.4.1} & \multirow{2}{*}{CT10}& NLO QCD\\
& &  & with {\scshape mcfm-6.3}\\
\hline
\end{tabular}
\end{center}
  \caption{Generated samples used for background estimates. The generator, PDF set and order of cross-section calculations used for the normalisation
  are shown for each sample.
  The upper part of the table shows the MC samples used for the SM background coming from leptons with the same charge (MPI stands for multiple parton interactions), the lower part gives the background sources arising in the \ee\ or \emu\ channel due to electron charge misidentification.}
\label{tab:MC_cross}
\end{table}

\section{Event selection}
% TODO mention here about size of the dataset and that it was 8 TeV 50 ns data.
% TODO description of the used triggers here!

% TODO ToTHINK MC cross section and sample sizes???


Events of interest are events which have two high-$p_T$ isolated leptons ???

20.3 fb$^{-1}$ of data collected at 8 TeV center of mass energy pp collisions, with mean number of interactions per crossing equal to 21, 
with 50 ns bunch distance were used for the analysis.
Firstly events need to have a reconstructud primary vertex with at least three tracks matched to it. If there are few vertices, one with highest
$\sum p^2_T$, where $p_T$ is transverse momenta of the tracks, is chosen.
Events which pass ??? trigger, which correspond to presense of the two isolated leptons with $p_{T}>25$??? and $p_{T}>20$???, were selected.
% TODO Cleaning and all other event-related cuts to describe here...

\subsection{Electron selection}``s''
% TODO description all cut on electron level
% TODO just tell about $p_T$>20 GeV cut!

\subsection{Definition of same-sign electron pair}
% TODO inv. mass cut; treatment of event with three leptons;
All electrons which passed selection are considered to form a lepton pair.
All combinations are checked and it is allowed to have more than one pair selected in one event.
Electrons in the pair are chatechorized by the $p_T$: electror with higher $p_T$ is called leading lepton, while another one with lower $p_T$ subleading lepton.
Leading lepton have to pass more strict cut $p_T>25$ GeV cut while subleading lepton have to satisfy $p_T>20$ GeV as described above.

If event contains an opposite-sign same flavour lepton pair which satisfies condition $( |m_{\Plepton\Plepton} - m_{Z}| < 10$ GeV$)$,
where $m_{\Plepton\Plepton}$ is lepton pair invariant mass and $m_{Z}$ is mass of Z boson, event is discarded.
This significantly suppress prompt background.

To avoid low-mass hadronic resonances like $J/\psi$ or $\varUpsilon$ to show up in the mass spectrum only same-sign pair with $m_{\Plepton\Plepton}>15$ GeV are selected.
Aditionaly, if same-sign lepton pair satisfies $70 < m_{\Plepton\Plepton}< 110$ GeV, which corresponds to the Z peak region, event is removed.
This region is used for electron charge misidentification background.

\section{Estimation of the non-prompt and charge flip backgrounds}
\subsection{Non-prompt background}
% TODO short description of the fake factor method. Put details on which regions were used.

\subsection{Prompt opposite-sign dilepton with charge flip}
% TODO description how charge flip rate was obtained, and plots if CF rate vs. pT and eta.

\section{Control regions}
\label{sec:wprimeMultijetBackground}
\subsection{Prompt opposite-sign dilepton}
\subsection{Prompt same-sign enhanced control region}
% TODO table with final numbers to show how good MC normalization is wrt to data.
\subsection{Electron charge flip enhanced region}
\subsection{Fake-enriched region}

\section{Signal Region}
\label{sec:wprimeSignalRegion}

\section{Systematic Uncertainties}
\label{sec:wprimeSystematics}
\subsection{Electron ID and momentum scale}
\subsection{Event-level and modelling systematics}
\subsection{Background estimate uncertainty}
\subsection{Systematic uncertainty on the signal}

\section{Cross section and mass limits}
\subsection{Statistical analysis for discovery and exclusion}
\subsection{Fiducial cross section limits on new physics}
\subsection{Mass limits on double charged Higgs}

\section{Outlook}
\label{sec:wprimeConclusion}

