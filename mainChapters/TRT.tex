\chapter{Simulation and performance studies of the Transition Radiation Tracker with Argon-based gas mixture}
\label{chap:TRT}

% TODO write
% 1) explain my personal contribution to the study.

This chapter describes the performance of the ATLAS Transition Radiation Tracker (TRT) detector with an Argon-based gas mixture in some sectors of the detector.
The TRT was designed for a Xenon-based mixture with carbon dioxide and oxygen additions (in proportions 70/27/3).
However, leaks in some drift tubes led to the decision to switch to a cheaper Argon-based mixture for the leaking modules.
This chapter covers a description of the implementation of the Argon mixture in the TRT digitization package.
The chapter consists of three sections. In \SectionRef{sec:TRT_intro} an overview of the detector running conditions is described. In \SectionRef{sec:trt_straw_hw}, the general TRT description is presented with an emphasis on a description of the drift tubes, the gas used and the front-end electronics.
\SectionRef{sec:digi_argon} gives an overview of the digitization package as well as a detailed description of the Argon implementation in the code.
\SectionRef{sec:trt_tracking_perf} describes a performance study of the TRT regarding a hit and track parameters with a focus on an active gas mixture, and particularly the comparison of 
Xenon- and Argon-based mixtures.
In \SectionRef{sec:trt_summary} conclusion of the Argon studies is made.

Modelling of the new Argon gas mixture and the new detector configuration (which allows to simulate the detector with different gas mixtures in different parts), 
were integrated into the official TRT digitization package and are currently used for the Monte Carlo simulation of the TRT detector. 
The previous implementation of the Xenon-filled drift tubes was used as a basis for the modelling of Argon-filled drift tubes.
All Argon gas mixture properties needed for the implementation in the digitization code were measured in the laboratory with a drift tube prototype 
or modelled by the Garfield package~\cite{garfield_program}.
A tracking performance study was done to understand the behavior of the Argon mixture in the TRT compared to the Xenon one.
% Particle identification capabilities (by adsorption of transition radiation photons) of the Argon gas mixture are not discussed in this thesis.

%#######################################################################################################################
% SECTION 0 ############################################################################################################
%#######################################################################################################################
\section{Motivation}
\label{sec:TRT_intro}

During LHC Run-1 the TRT detector was successfully operating and was providing an essential part the of tracking information 
as well as particle identification information by detecting transition radiation photons.
At the end of Run-1 
a few leaks in the drift tubes were observed, which triggered a large investigation in order to understand and fix the problem. 
After investigation of the problem during the technical LHC shutdown some of the problematic tubes were fixed, however, a large part of the leaking tubes were impossible to fix due to the lack of access to them.
Therefore some changes in the gas system were done in order to minimize the leaks, 
and a study of the possibility of using Argon-based gas mixtures (which are significantly cheaper) started. 
With time, the amount of leaking gas increased, and it became inevitable to switch to the Argon mixture 
for the most leaking TRT modules. In \FigureRef{fig:mixed_condition_2015_2016} the gas configurations of the TRT used in years 2015 and 2016 are shown, demonstrating that the problem was gradually increasing with time.

\begin{figure}
\centering
\begin{subfigure}{.5\textwidth}
  \centering
  \includegraphics[width=\linewidth]{TRT/Xe_Ar_map_2015.png}
  \label{fig:sub1}
\end{subfigure}%
\begin{subfigure}{.5\textwidth}
  \centering
  \includegraphics[width=\linewidth]{TRT/Xe_Ar_map_2016.png}
  \label{fig:sub2}
\end{subfigure}
\caption{TRT gas configurations used in 2015 (left) and 2016 (right) during physics data taking. Taken from~\cite{trt_public_plots_2016}}
\label{fig:mixed_condition_2015_2016}
\end{figure}

%#######################################################################################################################
% SECTION 1 ############################################################################################################
%#######################################################################################################################
\section{Detector design}
\label{sec:trt_straw_hw}

The general description of the TRT detector have been described previously in \SectionRef{sec:ID}.
In this chapter the main focus will be on straw tube performance, readout electronics as well as a short overview of the track reconstruction with the TRT.

% \subsection{Detector design}
% \label{subsec:trt_description}
% TODO 
% 1) insert fancy picture of the TRT
% 2) some information about tracking here? change name of the subseciton then?
% 3) some information about transition radiation?
% 4) 
% The ATLAS Transition Radiation Tracker is the outermost of the three tracking subsystems of the ATLAS Inner Detector. 
% % ATLAS is one of two general-purpose detectors built for the Large Hadron Collider at CERN.
% The TRT consists of two barrel parts (Barrel A and Barrel C) and two end-cap parts (End-cap A and End-cap C) 
% which are placed symmetrically with respect to the interaction point.
% The detector contains $\sim$300000 thin-walled proportional-mode drift tubes providing on average 30 two-dimensional 
% space points with $\sim$130 $\mu m$ resolution for charged particle tracks with |$\eta$| < 2 and $p_T$ > 0.5 GeV~\cite{Abat:2008zza,Abat:2008zzb,Abat:2008zz}.
% Along with continuous tracking, the TRT provides electron identification capability through the detection of transition radiation X-ray photons.

% TODO [Oxana] picture of the TRT
% TODO [Oxana] a few words about radiators...

% Implementation of Argon simulation
% Till now TRT simulation code supported only Xenon based gas-mixtures as active gas in the tubes. 
% Current study is focused on implementation of simulation of Argon based gas-mixture, 
% which was used for a few runs of data taking in 2013 in a few TRT modules.

% Performance study
% Performance study of the TRT using Argon and Xenon based gas-mixtures will be presented.
% Hit and track parameters, such as hit reconstruction efficiency, residuals, track momentum resolution and extension fraction
% will be shown for both gas-mixtures and compared to Monte Carlo simulation.

% Current note are dedicated to the performance study of Transition Radiation Tracker (TRT) of ATLAS detector on hit and track parameter with focus on active gas mixture.
% It consist from technical description of implementation of Argon-based active gas mixture to the TRT Digigtization package of ATHENA framework and performance study itself. 
% Aim of current study was to investigate scenario when instead of standart Xenon-based mixture Argon-based mixture will be used in some part of detector. Main parameters of interest
% were hit residuals, track extension fraction ???
% Motivation for current study was leakages in tubes ???

\subsection{The straw tube and operating gas mixture}
% - main idea of signal detection 
% - material of the straw
% - gas mixture and all properties

The main elements of the TRT are proportional straw-shaped drift tubes, hereafter called straw tubes or simply straws.
The tubes need to have good electrical and mechanical properties (the cathode resistance has to be as low as possible). It is important to minimize possibility of gas leaks as well as to avoid bending of the tubes, which will lead to changes in the alignment.
The geometrical factor has to be taken into account as well: large straws provide a better hit efficiency while small straws provide small drift time (the time needed to collect electrons in the straw tube), which is crucial in order to cope with a 25 ns distance between bunch crossings.
Another crucial requirement is a limit on the thickness of the tubes. 
In order to let low-energy transition radiation photons (created mainly by electrons passing through radiators) go through and be adsorbed in the gas, straw walls need to be as thin as possible.
One also want to keep the material budget as low as possible in the ID to reduce multiple scattering,
which is the main source of uncertainty of the track $p_T$ measurement.

All these requirements were considered during the design phase, in order to get an optimal choice.
The straw size is 4 mm in diameter. Walls are made from two layers of multilayer film. A section of the straw wall is shown in \FigureRef{fig:straw_wall_section}.
The multilayer film consists of a 6 $\mu$m thick carbon-polyimide layer, which protects a 0.2 $\mu$m thick Al layer, which is coated by a 25 $\mu$m Kapton film.
These two films are placed back to back and are glued by a 5 $\mu$m thick polyurethane layer.
Mechanically, each straw is supported by 4 carbon fiber bundles.
The anode is a 30 $\mu$m diameter tungsten wire coated with a 0.6-0.7 $\mu$m layer of gold.

\begin{figure}
\centering
\includegraphics[width=.4\textwidth]{TRT/straw_cross_section.png}
\caption{ 
Section of the TRT straw tube wall.
% TODO insert reference
}
\label{fig:straw_wall_section}
\end{figure}

The active detector volume of a straw is a gas mixture which is flushed through the tubes with the help of the gas system.
This mixture was carefully chosen in order to be safe to use (to prevent sparks that can perforate the walls), be compatible with the detector materials, and to not produce dissociation products, after the electron avalanche, with aggressive properties that could chemically damage the straw.
A high electron drift velocity is preferred, as well as a wide operating plateau (the high voltage range where straws work in the proportional regime). 
The latter is important in order to have a
safety margin in case the high voltage has to be changed to correct for temperature variations, to enhance the signal to noise ratio 
% (during operation, due to the radiation damage, the electronics becomes more noise) 
or to adjust for 
heavy-ionizing particle effects~\cite{Abat:2008zza}.
Several noble gases were considered as a main component of the gas mixture. Because they are inert they are safe to use in the detector for a long time, and they have 
excellent photo-adsorption cross sections.
The target energy of transition radiation photons for electron identification is in the range of 1-20 keV.
In \FigureRef{fig:absorption_lenght} the absorption lengths are shown for the four most heavy noble gases.
As one can see, the best choice would be Radon, but due to its radioactivity it can not be used.
The next candidate was Xenon, and it was chosen to be the main gas component.
Because of the high cost of Xenon it was decided to use Argon gas for a large number of tests and during the commissioning phase before pp collisions. This is the reason why the read-out electronics was
designed with a possibility to work with both Argon- and Xenon-based gas mixtures, as described in \SectionRef{subsubsec:front_end_electronics}.

\begin{figure}
\centering
\includegraphics[width=.7\textwidth]{TRT/absorption_lenght_noble_gases.png}
\caption{ 
  Absorption length as a function of photon energy for the heaviest noble gases. The plot is based on data reported in~\cite{Hubbell:353989}.
}
\label{fig:absorption_lenght}
\end{figure}

In order to have a well controlled electron avalanches in the straw, one has to add another gas component, a quencher. The main purpose of the quencher is to absorb UV photons created in the avalanche process and to prevent secondary avalanches, which can lead to an early breakdown.
After a dedicated study, reported in~\cite{Abat:2008zza}, it was decided to use carbon dioxide as a quencher due to its suitability and well known properties.
In order to increase the width of the operating plateau even more, a third gas component, oxygen, was added.
By itself $O_2$ does not interact with UV photons, but ozone ($O_3$), which is created in the electron avalanches, does.
However, $O_2$ is a strongly electronegative gas, which can negatively affect straw performance by capturing drifting primary electrons, as described below,
and therefore only a few percent of it can be used in the mixture.
 
The final choice of the gas mixture to be used in the TRT was decided to be 70$\%$ of Xenon, 27 $\%$ of carbon dioxide and 3$\%$ of oxygen.
When some straws started to leak, it was decided to switch the main gas component from Xenon to Argon for these straws, due to economic reasons.
Argon has significantly higher absorption length with respect to Xenon, as can be seen in \FigureRef{fig:absorption_lenght} (approximately an order of magnitude difference), 
therefore straws with Argon almost completely lose the ability to detect transition radiation, 
which means losing the particle identification capability.

The operating gas gain for the straws was chosen to be 2.5 $\cdot$ 10$^4$~\cite{ID_TDR_vol1} and it had to be the same for all straws. It corresponds to 1530~V for the Xenon mixture, as shown in \FigureRef{fig:gas_gain}.
The gas gain curve for the Argon mixture is significantly different, and to reach the same gain only a voltage of 1420~V had to be applied.

\begin{figure}
\centering
\includegraphics[width=.5\textwidth]{TRT/gas_gain_Xe_Ar.png}
\caption{ 
Straw gas gain as a function of high voltage for different gas mixtures~\cite{Abat:2008zza}.
}
\label{fig:gas_gain}
\end{figure}


% TODO toWrite descfribe gas gain in general. Say about Argon mixture. Explain plot \FigureRef{fig:gas_gain} in the text.

\subsection{Front-end electronics and signal processing}
\label{subsubsec:front_end_electronics}

The main purpose of the TRT is to provide hits for track reconstruction and information for electron identification by detecting transition radiation photons.
In order to get the best possible position resolution of hits one needs not only to have a possibility to identify which straws particle traversed,
but also to measure the closest approach of the charged particle to the straw wire, the so-called track to wire distance. 
For this purpose, the drift time of electron clusters has to be measured.
The TRT front-end electronics was designed to perform these precise time measurements.
One has to measure the time of signal arrival (drift time of the electron clusters to the anode wire) and the duration of the signal, the so-called Time over Threshold (ToT).
Time over Threshold is needed to distinguish signals from particles originating from different bunch crossings 
and can be used for particle identification because it characterizes the amount of the ionization in the straw gas.
Typical time needed for electrons to drift from the straw wall to the anode wire in the Xenon-based gas mixture corresponds to 40-45 ns, which dictates the size of the read-out window to be 
3 BCID or 75 ns. As shown in \FigureRef{fig:pulseDigitization} the signal is sampled with 24 bits (3.125 ns each). 
A value of 1 is assigned to a bit if the signal amplitude is higher than the predefined threshold, 
otherwise a value of 0 is assigned. The first transition in the bit pattern from 0 to 1 corresponds to the so-called leading edge of the signal and the transition from 1 to 0 
corresponds to the trailing edge of the signal. Their difference is equal to the Time over Threshold of the signal. 

\begin{figure}
\begin{center}
 \includegraphics[width=0.8\columnwidth]{TRT/fromPoster/TRTdigi4.pdf}
\caption{Illustration of the digitization of the straw signal.}
\label{fig:pulseDigitization}
\end{center}
\end{figure}

% TODO cross check with Anders, Kostja and Alex if this is true!!!
% Measurements in the laboratory showed that signal from the straw consists from two components, as shown in \FigureRef{fig:ion_tail}: 
% a sharp peak, which correspond to the primary ionization electrons, created by passing particle, reaching the anode wire and the ion tail, which is caused by ion drift towards the cathode.
% When ions hit the wall of the straw there is a probability that it will free a few electrons from the wall which will start to drift towards the anode.
% This effect is partly compensated by the oxygen gas component, which provide recapturing of the electron by oxygen moleulus, as described in the following \SectionRef{subsec:recapture}.
% But some electrons will reach anode and will cause slowly falling component of the signal. To cancel out 

A typical signal from the straw consists of two components, as shown in \FigureRef{fig:ion_tail}: 
a sharp peak, which corresponds to the fast electron drift, and a slowly falling ion tail, which corresponds to the gas ions created in the avalanche process.
Only the electron peak contains information of interest (the drift time of the primary electron clusters), thus the ion tail is canceled out by adding a mirror image pulse to the signal.

\begin{figure}
\centering
\includegraphics[width=.6\textwidth]{TRT/ion_tail.png}
\caption{ 
 Typical signal from the TRT straw~\cite{ID_TDR_vol2}. The sharp peak corresponds to the fast electron drift component.
 The following tail is caused by slow drift of ions in the gas.
}
\label{fig:ion_tail}
\end{figure}

The ion tail cancellation and signal digitization is done by a custom-made ASDBLR (Amplification, Shaping, Discrimination and Base-Line Restoration) chip. 
This chip also performs restoration of the signal baseline and discrimination. 
Electronics make discrimination with two different thresholds. One threshold, the so-called Low Threshold (LT), 
is used to trigger on cases when a high-energetic particle passes through the straw and ionizes the gas along its path, and it is shown in \FigureRef{fig:pulseDigitization}.
The second threshold, the so-called High Threshold (HT), is used for the case of transition radiation photon detection.
Contrary to the high-energetic particles, transition radiation photons deposit all their energy essentially in one point when interacting with gas.
This corresponds to a large number of primary electrons, which leads to a large read-out signal.
Threshold values used for the Xenon-based gas mixture correspond to 300 eV and 6 keV for LT and HT respectively.
Discrimination with HT is done per BCID. 

In total, the ASDBLR chip provides 24+3 bits, which corresponds to 24 LT bits and 3 HT bits over the 75 ns read-out window.
A schematic illustration of the ASDBLR chip operation principle is shown in \FigureRef{fig:nice_asdblr_schematics}.
ASDBLRs are placed close to the straws in order to reduce noise in the signal. One chip reads eight straws.
The next component in the read-out chain is a complementary read-out chip DTMROC (Drift Time Measurements Read-Out Chip), which accepts 
information from two ASDBLRs and holds data in the pipeline until the first level trigger arrives. 
% If the event is accepted, data is sent further to read-out electronics.
The overall TRT read-out chain is shown in \FigureRef{fig:electronics}. The detailed description of the TRT electronics is reported in~\cite{TRT_electronics}.

\begin{figure}
\centering
\includegraphics[width=.8\textwidth]{TRT/nice_ASDBLR_schematics.png}
\caption{ 
 Schematic illustration of the straw signal digitization by the ASDBLR chip~\cite{Aad:2008zzm}.
}
\label{fig:nice_asdblr_schematics}
\end{figure}


\begin{figure}
\centering
\includegraphics[width=.7\textwidth]{TRT/electronics.png}
\caption{ 
 Schematic representation of the TRT read-out electronic~\cite{TRT_electronics}.
}
\label{fig:electronics}
\end{figure}



\subsection{Tracking with the TRT}
\label{subsec:tracking_with_trt}

The track reconstruction algorithm require a set of spatial hits and the corresponding error matrices in order to build a track. The TRT electronics provide drift time measurements which can be transformed to the
drift radius (drift distance of the electron cluster which reached the anode wire first), which can be used in the track reconstruction.

The first task is to properly measure the drift time of the electrons to the anode wire. 
The measured leading edge time has two components: the actual drift time $t$ and the calibration constant $T_{0}$, as illustrated in \FigureRef{fig:pulseDigitization}.
The calibration constant $T_{0}$ has three parts:
\begin{itemize}
 \item The collision time with respect to the ATLAS clock, since there could be some uncompensated delays due to propagation in the wires and electronics.
 \item The time of flight of the particle to the straw. Straws are placed at a distance varying from 0.5 to 2.6 meters from the interaction point, 
 which corresponds to up to 9 ns for a relativistic particle to reach the furthest straws.
 \item Time of the signal propagation in the wire to the front-end electronics, typically a few nanoseconds.
\end{itemize}
The calibration constant $T_{0}$ is obtained for each DTMROC from the calibration procedure which is done after each run.
The next task is to transform the drift time to a drift radius. It is done with the help of the so-called $r-t$ relation which gives the drift radius as a function of the drift time.
This relation is determined from data and is fitted by a third degree polynomial, which is used for calculations of the drift radius itself.
The last missing parameter, the hit measurement error, is also calculated in the special calibration procedure. A detailed description of the calibration procedure
can be found in ~\cite{alonso_thesis}.

The main tracking algorithms used in ATLAS are the global $\chi^2$ fit, when tracks are fitted by minimization of the $\chi^2$ value, and Kalman filters which makes pattern recognition
layer by layer. When a track is fitted, all the used hits in the fit are removed from the list of available hits, and the tracking algorithm is executed iteratively until all tracks are reconstructed.
In \FigureRef{fig:drift_radius} a schematic representation of a track passing two straws is shown. The drift radius obtained from the drift time with the $r-t$ relation is
shown with a dashed line around the wire. The track to wire distance is the distance of the closest approach of the track to the anode wire ($r_{track}$). The difference
between the drift radius and the track to wire distance is called the position residual. By calculating position residuals for each hit and for all
tracks in the barrel or end-cap parts of the detector and by measuring the width of the residual distribution one can estimate the resolution of the detector.
In the same manner, the time residuals can be measured, which are equal to the difference between the drift time and the track time, which are obtained by using $r_{track}$ and the inverse of the
$r-t$ relation.


\begin{figure}
\centering
\includegraphics[width=.7\textwidth]{TRT/drift_radius.png}
\caption{ 
 Illustration of the main hit parameters such as drift radius, track to wire distance ($r_{track}$) and position residual.
}
\label{fig:drift_radius}
\end{figure}

% TODO explain $\frac{d}{\sqrt{12}}$ error, as described in Alex thesis???
There are two types of TRT hits: precision hits and tube hits. Precision hits are the hits described above. Tube hits are hits where the drift time 
information is not used, only the fact that there is a hit in the straw. For such hits, the drift radius is assigned to be 0 mm, and the hit error is $\frac{d}{\sqrt{12}}$, where
$d$ is a diameter of the straw (4 mm). Tubes hits are hits which have a difference between track and drift radius larger than
2.5 times the uncertainty of the drift radius. Since the track fitting is done iteratively, some hits can go back and forth from being tube or precision hits.
% Tube hits decrease quality of the track, which correlates with detector resolution.

The main TRT detector contribution to the tracking is an improvement of the momentum resolution of tracks measured with the Inner Detector. The large number of hits in the TRT provide
a solid handle to measure the curvature of the track which translates into a measurement of the track momentum. Momentum resolution can be expressed by the following formula~\cite{mom_res_book,Gluckstern:1963ng}:
\begin{equation}
%  \dfrac{\delta p}{p^2} = \dfrac{\sigma}{0.3BL^2}\sqrt{4C_N}
\dfrac{\delta p_T}{p_T^2} = \dfrac{\sqrt{([c^2])}}{0.3B}
 \label{eq:momentum_resolution}
\end{equation}
where $p_T$ is the transverse momentum of the track and $\delta p_T$ is its error, $[c^2]$ is the variance of the track curvature, which can be expressed as~\cite{mom_res_book}:

% $\sigma$ is spatial resolution of the hit, $B$ is magnetic field, $L$ is the length of the measured track
% and $C_N$ is factor which depends on the number of hits, N:
\begin{equation}
%  C_N = \dfrac{180N^3}{(N-1)(N+1)(N+2)(N+3)}
[c^2] = \dfrac{\sigma^2}{L^4}\dfrac{720 N^3}{(N-1)(N+1)(N+2)(N+3)}
\end{equation}
where $\sigma$ is the spatial resolution of the hit, $B$ is the magnetic field and $L$ is the length of the measured track.
As can be seen from formula above, the momentum resolution is proportional to the spatial resolution of the hits and is inversely proportional to the square root of 
the number of hits. Therefore spatial resolution and the number of hits are parameters of interest during the tracking performance studies.

The importance of the TRT for the momentum resolution can be seen in \FigureRef{fig:improvement_of_momentum_resolution}.
This plot shows the momentum resolution of cosmic ray data collected in 2008, for Inner Detector tracks reconstructed with and without the TRT.
A clear and significant improvement can be seen, especially in the high momentum region.


\begin{figure}
\centering
\includegraphics[width=.6\textwidth]{TRT/TRT_improvement_momentum_resolution.png}
\caption{ 
Momentum resolution determined from cosmic ray data, taken in 2008, as a function of transverse momentum~\cite{Aad:2010bx}.
The resolution are shown for simulated full ID tracks (stars), full ID tracks (solid triangles) and silicon-only tracks (open triangles).
}
\label{fig:improvement_of_momentum_resolution}
\end{figure}

% TODO
% describe about the momentum resolution and it's dependence from spatial resolution and number of tracks.


%#######################################################################################################################
% SECTION 2 ############################################################################################################
%#######################################################################################################################
\section{Modelling of the new Argon-based gas mixture in the TRT}
\label{sec:digi_argon}

In this section, the description of the digitization code will be presented, with a detailed explanation of the implementation of the Argon-based gas mixture.
The main task of the digitization code is to simulate the number and properties of the primary electron clusters as well as their drift in the gas to the anode wire which corresponds to the one observed in data. 
The response of the front-end electronics, which provides final digitized timing information, is simulated as well. 

\subsection{Monte Carlo simulation of the ATLAS detector}

Before going to the detailed description of the TRT detector modelling, it's important to summarize the
general approach used in ATLAS for detector modelling.

To have a possibility to reconstruct any physics process which happens in high energy particle collisions, or to measure a corresponding physics parameter of a process,
one has to have a clear picture of the response from each detector sub-system. Besides the understanding of the real detector performance, it is equally important to have 
a simulation of the detector which one can trust. Simulation of the detector response allows to perform many sophisticated studies, such as a study of the detector effects and systematic uncertainties,
distinguishing background processes from specific processes of interest or even a prediction of a possibility to ``see'' a signal from a hypothetical model with a given amount of data.
A lot of effort was spent to develop and validate the detector simulation software. In ATLAS, simulation of the detector response is done in a few well-defined steps, which are:
\begin{itemize}
 \item Event generation. In this step, the physics process of interest is modelled. The outcome of this step is a list of final state particles characterized by 4-momenta. 
This is done by specialized Monte Carlo generators. 
This step involves many calculations and modelling, like calculation of the matrix element of the hard process, modelling of the initial- and final-state radiation of incoming/outcoming partons, a creation of the colorless hadrons from the quarks and non-stable particle decays, etc. 
Typical MC generators used in ATLAS
%  TODO references!!!
 are PYTHIA, HERWIG, SHERPA and others.
 \item Simulation. Particles created in collisions traverse the detector and interact with the detector material. These interactions are simulated by software, which defines trajectories of particles and energy depositions in all sub-detectors. 
 %  TODO references!!!
 It is done using the GEANT4~\cite{Agostinelli:2002hh} package, which returns a list of hits, the so-called simulation hits, or simply ``simhits'', which contain position and timing information of 
 the interaction together with the amount of energy lost by the particle due to the interaction (deposited in the detector material).
 \item Digitization. Energy depositions in the detector creates analog signals which are read and processed by the read-out electronics. During the digitization step, the behavior of the front-end electronics is simulated together with the signal processing, in order to simulate the same outputs (measured quantities) as the real electronics provide (e.g. hit in the pixel or strip,
 digitized current from the PMT or calorimeter cell, etc.). Due to the different read-out electronics designs for different sub-detectors, the digitization software is different for each sub-detector.
 Such software is written and maintained by each sub-detector community separately.
 \item Reconstruction. The final step is to collect all information from sub-detectors together, apply all the needed calibration and alignment corrections and reconstruct physics objects. This step is identical to the one applied to the real data, only that the mentioned above corrections can be different for simulated and real 
 data, e.g. some timing shifts in the electronics can be not simulated because they do not play any role, but it means that the timing corrections of the calibrated constants will be different for simulation and real data.
\end{itemize}

More detailed description of the ATLAS detector simulation can be found in ref.~\cite{Aad:2010ah}.

\subsection{Simulation and digitization of the Argon gas mixture in the TRT}
\label{subsec:TRT:argonImpl}
% TODO should I write here information about energy depositoin calculated not with GEANT4 but with PAI modeil in the digitization package???

% - Describe all steps of MC production, as described in the Esben thesis.
% - for simulation part make reference to the Esben thesis, say that it is described there.
% - 

The geometry of the TRT is described in the ATLAS simulation package and during the simulation step it is used to simulate particle interactions with the detector material.
However, because the GEANT4 package does not provide precise enough calculations of energy deposition in thin layers of gas in tubes 
(which is described in the following \SectionRef{subsec:pai_model}), it was decided to take care of this simulation during the digitization step in the digitization package.
Thus from the simulation step, only the list of hits in the TRT is used when calculating energy depositions in the gas inside the tubes during the digitization step.

The digitization package also covers simulation of the following properties:
\begin{itemize}
 \item Number of the primary ionization clusters along the particle path and number of electrons in these clusters
 \item Drift time of the electrons to the anode wire, taking into account magnetic flux density and drift diffusion
 \item Probability of electron recapture by the oxygen molecules
 \item Signal shaping and discrimination by the front-end electronics
 \item Detector white noise
\end{itemize}

% TODO explain how Argon implementation was done.
All these parameters depend on the type and proportion of the components used in the gas mixture. 
To implement the new Argon gas mixture into the digitization package, all these properties were considered in detail 
so proper implementation of all of them was done.

% \subsection{Digitization of the mixed gas configuration}
% 
% % TODO revisit this piece of text
% 
% TRT detector was originally designed as an homogenous detector and it was planned that only one gas mixture will be used in the all straws. 
% Digitization code, as all physics codes, was written with an idea to be as simple as possible, that is why it was designed with assumption that all straws will be simulated with only one type of gas mixture.
% But after leakeges appears at the end of the Run 1, it become higly probable, that TRT may be run in mixed condition, when some part of straws will be run with one gas and other part - with another. That's why 
% possibility to digitize mixed condition become essential feature of the code. 
% 
% To make it work new (to the digitization code) Argon gas mixture was introduced and following simulated detector/electronics parameters as low/high treshold and shaping functions were duplicated.
% Digitization of the TRT hits are done straw by straw and in the beginning of the straw loop flag \mbox{ ``isArgonStraw''} is read from the COOL database. This flag represent is current straw contain Argon or Xenon based 
% gas mixture. The straw map lays under \mbox{/TRT/Cond/StatusHT} COOL folder, and one need to specify dedicated tag to make mixed Ar/Xe digitization.
% According to this flag relevant thresholds and shaping function are picked up and used during digitization following straw. 
% 
% 
% \begin{figure}
% 
% \begin{subfigure}{.5\textwidth}
%   \centering
%   \includegraphics[width=\textwidth]{TRT/mapXY_barA.pdf}
% \end{subfigure}%
% \begin{subfigure}{.5\textwidth}
%   \centering
%   \includegraphics[width=\textwidth]{TRT/mapXY_endA.pdf}
% \end{subfigure}
% 
% \caption{$T_{0}$ calibration constants obtained for MC simulation for Barrel A (left) and End-cap A (right) detectors for the mixed Xenon-Argon condition. 
% 	  Modules with Argon gas have different $T_{0}$ calibration constants.}
%   \label{fig:t0_mixed_condition}
% \end{figure}
% 
% 
% 
% Also two flags which provide simple possibility to change active gas mixture from default one (Xenon) to the optional (Argon) for all straws were implemented. These are:
% \begin{itemize}
%  \item UseArgonStraws
%  \item UseConditionsHTStatus
% \end{itemize}
% 
% First flag says that we want to use optional gas in some part or in full detector. If it is false - default Xenon gas will be used despite of the second flag. 
% Second flag says do we want to read straw map from COOL database or not.
% If it is false full Argon geometry condition will be run. Example of usage these flags can be found at [???].
% 
% Points related to the actual gas mixture implementation are described at the next subsections.

\subsection{Implementation of the TRT mixed gas configuration}
Before the description of the details of the Argon mixture simulation and digitization, one need to briefly explain how digitization code works and how Argon option was embedded in the code itself.
% TODO [Oxana] add some concrete description of the digitization package, as which language it is written, block diagram of the code-flow and so on.
All tubes in the code inherit geometry of the ``ideal'' tube, defined during the designed phase. However, each tube has some unique characteristics. It has a particular position, 
which linked with a value of a magnetic field in the straw. It can be dead, which means that straw will be skipped in the digitization.
When GEANT4 propagate particles, originated in the collision, it simulates the interaction of the particle with a matter in order to obtain an array of simulation hits, where any energy deposition to the 
detector material took place. Digitization code takes list of simulated hits and loop over all of them for each track. If track hit a TRT straw, processing of the straw begins.
The first step is to check if the straw is dead or not by accessing information from the straw database.
If it is not dead, a drift time simulation is done.
Argon implementation was done by assigning each straw additional characteristic, type of used gas mixture. This information is also stored in the straw status database.
After cross check, if the straw is dead or not, the type of gas mixture is defined by taking information from the database.
Then simulation of any parameters, described above, are done accordingly to the type of the gas mixture.
It is worth to mention that such approach allows to simulate detector mixed gas configurations when part of the detector is operated with Xenon and another one with Argon mixtures.

\subsection{Simulation of the energy deposited in the gas and initial number of electrons in a cluster}
\label{subsec:pai_model}
During the development phase of the TRT simulation and digitization software, it was found that the GEANT4 simulation does not provide accurate enough description of the physics
of a charged particle passing through the very thin gas layers (straws). 
% Kittelman thesis, p.92, Figure 9.2.
In ~\cite{kittelmann_thesis} the author made a comparison of the photo-absorption cross section parameterization used in GEANT4 with the more detailed one provided by I. Gavrilenko.
Even though the difference between these two parameterizations was small, using them in the code led to a 7$\%$ difference in the mean free path length, which is significant for the TRT straw simulation.
That is why a dedicated model, the so-called photo-absorption ionization (PAI) model~\cite{pai_model_paper}, is now used to simulate deposited energy in the straw.
This model derives ionization cross section for a charged particle with a given gamma factor numerically from the tabulated values of photoabsorption cross section for the gas.
The missing piece for the Argon-based gas mixture simulation was the lack of the photoabsorption cross section for Argon itself in the code.
Cross section as a function of photon energy reported in~\cite{argon_cross_section} was therefore taken and implemented in the code.

As described previously, propagation of a charged particle through the ATLAS detector is done by GEANT4, and in practice, it corresponds to the list of simulated hits in the detector.
These hits are used as a reference, while the real ionization clusters are calculated by the PAI tool along the ``simhits'' path.
The PAI tool calculates the mean free path, which is the distance between two neighbor electron clusters in the gas, and energy deposited in the cluster. 
The mean free paths for Xenon- and Argon-based gas mixtures, calculated by the PAI tool, are shown in \FigureRef{fig:meanFreePath} as a function of the scaled kinetic energy 
(the kinetic energy of a particle scaled by the factor $\frac{m_{proton}}{m_{particle}}$). As one can see the mean free path for the Argon mixture is approximately 1.5 times longer than the one in the Xenon mixture.
The mean free path for a highly energetic pion is 150-160 $\mu$m in the Xenon mixture. If a pion penetrates a straw centrally close to the anode (will travel 4 mm in the gas), then 25-27 primary clusters will appear.
However, if it will travel 2 mm in the gas, the number of clusters will be only 12-14, and a part of them will not reach anode due to recapture in the oxygen. 
That is why measurement of the time of the first cluster arrived to the anode will not directly correspond to the measurement of the distance of the closest approach between the wire and the track, 
which can be seen in \FigureRef{fig:clusterDriftInTube}. This effect is directly linked to the spatial resolution of the hit.
For the Argon mixture, the number of clusters will be even smaller (8-9 clusters created by a particle traveling 2 mm in the mixture), which leads to even worse spatial hit resolution in the Argon mixture comparing 
to the Xenon one.

\begin{figure}
\centering
 \includegraphics[width=0.7\columnwidth]{TRT/meanFreePath.eps}
\caption{Mean free path of a relativistic muon as a function of scaled kinetic energy ($E_{kin}\frac{m_{proton}}{m_{particle}}$) for Xenon- and Argon-based gas mixtures.}
\label{fig:meanFreePath}
\end{figure}

When a charged particle traverses active gas volume of the straw, it interacts with gas molecules.
It can either ionize them by kicking out an electron or excite them. The energy needed to kick out a least bound shell electron for TRT gasses lays
in the range of 10-20 eV. However, a part of the energy goes to the excitation of gas molecules, and this is why the average energy needed to create an ion pair (the so-called W-value) is higher than the energy required to kick out an electron from a shell. The list of these values for all gasses used in TRT is shown in \TableRef{tab:ionization_energy}.
Using the values from the table and knowing the gas mixture proportions (70/27/3 $\%$) one can estimate the average energy for the Xenon-based mixture, which is equal to 25.3~eV.

The W-value for Argon-based mixture was calculated in the same manner and is equal to 28.3~eV, which is implemented in the digitization code.
The number of primary electrons, $N_{prim.electrons}$, is calculated by the formula:
% TODO probably it's worth to explain why we need +1
\begin{equation}
 N_{prim.electrons} = floor \left[\dfrac{E_{deposited}}{W} + 1\right]
\end{equation}
where $E_{deposited}$ is the energy deposited in the gas, W is the average ionization energy described above, and $floor$ function is a function which returns
The largest integer which does not exceed the argument of the function.
From this formula, one can conclude that the number of primary electrons in clusters in the Argon-based mixture is smaller than in a Xenon-based one, 
assuming the same amount of deposited energy.
% TODO gases have different density, should we expect same energy deposits for both gases? --> TO CHECK!!!

\begin{table}[p]
  \begin{tabular}{c|c}
    Gas & W [eV / ion pair]\\
    \hline
    Xe & 22.1 $\pm$ 0.1 \\
    Ar & 26.4 $\pm$ 0.5 \\
    CO$_2$ & 33.0 $\pm$ 0.7 \\
%     CF$_4$ & 29.2 $\pm$ 1.0 \\
    O$_2$ & 30.8 $\pm$ 0.4 \\
  \end{tabular}
  \caption{The average W-value for electrons and photons in different gases considered to be used in the TRT as reported in~\cite{cwetanski_thesis}.}
  \label{tab:ionization_energy}
\end{table}

\subsection{Electron drift velocity and $r-t$ relation}
% TODO here...

A very crucial part of the digitization is to model the drift time of the primary electrons which appeared after the interaction of the particle with the gas, to the anode wire, because the drift time is directly related to the drift radius, defined as the position of the hit with respect to the straw wire.
During the drift of the electron in the tube, it interacts with gas molecules but continues its path to the wire due to the presence of the electric field.
In \FigureRef{fig:clusterDriftInTube} (left) an example of drift trajectories of electrons in a straw without magnetic field is shown.
Simulation of the drift of all electrons created in the gas in all tubes of the detector is computationally very heavy,
moreover, besides the drift in the gas one has to simulate the creation of the avalanches with a proper simulation of all the secondary effects,
such as the creation of the UV photons and others. However, if one assume that all tubes are kept in the same condition (constant gas gain) during the detector operation, it would
be possible to use tabulated average values of the drift time calculated by the Garfield package (and cross checked in the laboratory with the straw) to simulate it.
This is the approach used by the TRT community. 

\begin{figure}
\begin{center}
 \includegraphics[width=0.85\columnwidth]{TRT/electron_drift.png}
\caption {Drift of primary electron clusters in a straw tube without (left) and with (right) magnetic field~\cite{cwetanski_thesis}. 
}
\label{fig:clusterDriftInTube}
\end{center}
\end{figure}

In \FigureRef{fig:rt_comp} (left) the drift distance as a function of the drift radius obtained with the Garfield package for a case without magnetic field is shown.
It was obtained by making a scan of drift distances (in the range from 0 to 2 mm) and simulating the time it takes for the first electron to reach the anode wire.
The relation shown is called the $r-t$ relation and is used in the code to translate the drift distance to the drift time.
In \FigureRef{fig:clusterDriftInTube} (right) drift trajectories of a cluster in the presence of a magnetic field are shown. One can see that the magnetic field
significantly affects electron drift trajectories and makes the drift time longer. The $r-t$ relation for the case of a 2T magnetic field is shown
in \FigureRef{fig:rt_comp} (right) and it is clearly seen that the drift time is longer in this case. However, the magnetic field is not homogeneous within the detector.
The inhomogeneity effect was studied before (for example in~\cite{esben_thesis}) and it was found that the drift time dependence on the magnetic field can be described by a second-degree polynomial. 
To get the $r-t$ relation for a specific value of the magnetic field, $B_{eff}$, the following interpolation formula is used:
% TODO cross check this formula in the code!
\begin{displaymath}
    RT(B_{eff}) = (RT_{MAX} - RT_{WO}) \cdot \dfrac{B_{eff}^2}{B_{max}^2} + RT_{WO}
\end{displaymath}
where $RT_{MAX}$ corresponds to the $r-t$ relation for the magnetic field $B_{max}$ (which is equal to 2T - the maximum magnetic field in ATLAS) and $RT_{WO}$ is 
$r-t$ relation for the case without magnetic field.

\begin{figure}
\centering
 \includegraphics[width=0.99\columnwidth]{TRT/rt_comp.png}
\caption{r-t relation for the Xenon- (red) and Argon-based (green) gas mixtures. The curve for the Argon-based mixture was obtained from the Garfield simulation. 
Scaled (by ratio of the average drift velocities in the Argon and Xenon mixtures) 
r-t relation of the Xenon-based mixture is shown in gray and was used in the initial studies of the Argon mixture.
}
\label{fig:rt_comp}
\end{figure}

For the Argon gas mixture, two such $r-t$ relation tables were added to the TRT digitization package.
In the early development stage a Xenon gas $r-t$ relations, scaled by a ratio of the average drift velocities in the Argon and Xenon, were used in the code 
(gray points in the \FigureRef{fig:rt_comp}). 
Comparison of the scaled Xenon and Argon curves demonstrates the difference of the shape of distributions for Argon and Xenon gasses (see \FigureRef{fig:rt_comp}).

% Drift diffusion in the code:
% https://svnweb.cern.ch/trac/atlasoff/browser/InnerDetector/InDetSimUtils/TRT_DriftTimeSimUtils/tags/TRT_DriftTimeSimUtils-00-00-27/src/TRT_BarrelDriftTimeData.cxx?order=date&desc=1

Movement of the electron in the gas contains a stochastic component due to collisions with gas atoms. 
That means that the drift time can differ for clusters which drift the same distance.
The spread of the drift time as a function of the drift distance is shown in \FigureRef{fig:diffusion}.
Black points correspond to the drift time spread obtained with the Garfield simulation for the Argon gas mixture.
In the digitization code, the spread is parameterized by a 4-th order polynomial, which was used for the Argon modelling as well,
in order to preserve consistency (the spread of the drift time is shown with red and green points in \FigureRef{fig:diffusion} for Xenon and Argon accordingly).

\begin{figure}
\begin{center}
\includegraphics[width=0.69\columnwidth]{TRT/diffusion.eps}
\caption{Electron drift-diffusion in Xenon- (red) and Argon-based (green) gas mixtures as modelled in the digitization package (with a 4-th degree polynomial). 
Black points correspond to the Garfield simulation for the Argon-based gas mixture.}
\label{fig:diffusion}
\end{center}
\end{figure}


% \subsection{Electron recapture}
\subsection{Electron attachment processes in Oxygen}
\label{subsec:recapture}
When electrons drift towards the anode, there is a probability that they will be captured by the Oxygen molecules in reactions:
$O_2 + e^- \to O_2^-$ or $O_2 + e^- \to O + O^-$. The presence of the magnetic field enhances a cross section of these reactions.
To take into account this effect, a special study was done, reported in~\cite{esben_thesis}.
It was found that the electron capture probability strongly depends on the distance from the primary electron to the anode.
Garfield simulation shows that this dependence can be parameterized with a 4-th order polynomial (see \FigureRef{fig:electron_recapture}), 
which was implemented in the TRT digitization package. 
Because the Argon-based mixture contains the same amount of the Oxygen, the same survival probability
curve was used for it as well. However, a separate implementation for the Argon was done in the code to allow future tuning of this effect separately for Xenon and Argon gas mixtures
in a case when other concentrations of Oxygen will be used.
% TODO ask Kostja: 1) did he look on it for Argon? 2) is it correct that Argon-mixture will have the same suvivl probability as in Xenon?

As it was described previously, Oxygen was added to the mixture to increase operating plateau of the straw, but there is another effect which it takes care of.
Drifting ions might free electrons when they reach the cathode, which can in their turn produce electron avalanches, which is not a desirable effect.
As can be seen from \FigureRef{fig:electron_recapture}, Oxygen addition to the gas mixture provides approximately 65$\%$ probability that these electrons will be recaptured 
and will never reach the anode. 

\begin{figure}
\centering
\includegraphics[width=.7\textwidth]{TRT/electron_recapture.png}
\caption{ 
 Survival probability of the primary electron to reach the anode~\cite{esben_thesis}.
}
\label{fig:electron_recapture}
\end{figure}

\subsection{Signal shaping and discrimination}

% TODO ask Kostja why Argon shaping function has slower tail than Xenon one?
% TODO ask the same Andrew
% TODO what is about HT function?

As it was described previously, 
TRT front-end electronics shape the signal before discriminating by making a convolution with of the signal with a special shaping function. Function for Argon- and Xenon- based functions are different. Thus, one has to simulate them as well in order to simulate the proper response from the electronics. 
This is another missing piece which was implemented for the Argon mixture. 
Drift velocity in the Argon mixture is significantly larger than for Xenon. Thus the straw signal will differ significantly as well.
Argon gas was used for tests and commissioning phases, and this is why compatibility with the Argon mixture was foreseen for the ASDBLR chip. It contains a set of four shaping functions (which can be switched
by a pin on the board itself): two for Xenon and two for Argon mixtures. Two sets of functions are needed for each gas because different shaping functions are used for discrimination on LT and HT. Shaping function for Argon, measured in the lab using a prototype of the ASDBLR chip, is shown in \FigureRef{fig:shaping} together with Xenon functions. Shown Argon functions were implemented in the digitization package for the Argon mixture simulation. In the code, shaping function is convoluted with energy deposits, which are stored in a vector of time bins.

\begin{figure}
\begin{center}
 \includegraphics[width=0.69\columnwidth]{TRT/shaping.eps}
\caption{Argon low threshold shaping function compared to the Xenon low and high threshold shaping functions.}
\label{fig:shaping}
\end{center}
\end{figure}

% TODO here...
% TODO describe here about difference of the LT. Explain why we need different LT (hit efficiency?).
% TODO Why so big difference for Xenon and Argon?

After a convolution with a shaping function, the signal is discriminated against LT and HT, and a 24+3 bit pattern is formed.
Obviously, straws with Xenon and Argon mixture should have different threshold settings.
As was described previously, the Argon-based mixture has a significantly smaller signal, due to a lower number of primary electron clusters and a lower number of electrons in these clusters.
When some parts of the detector work with Argon, and the others with Xenon, their performance must be kept on the same level for
proper operation. In practice, this means that the hit efficiency (probability that there will be a hit if a particle traverse a straw) have to be close enough. 
To get a similar hit efficiency, the low threshold for Argon has to be lower.
Moreover, the threshold can have different values for the barrel and end-cap parts. This is because the end-cap straws are positioned perpendicular to the barrel straws (see \FigureRef{pic???}), % picture of the positioning of end-cap and barrel straws.
which changes a behavior of the drifting electrons in the tubes due to the different orientation of the magnetic field.
Possibility to set separate thresholds for Argon tubes was implemented in the code.
After preliminary studies in the laboratory, it was found that the low threshold for Argon has to be around 100~eV, 
while for Xenon straws this value is 285~eV for Barrel and 300~eV for end-cap detectors. The 100~eV value was used in the code.
Low threshold values are the main parameter of interest for the Monte Carlo simulation tuning.

\subsection{White noise modelling}

During the test beam studies in 2004~\cite{trt_test_beam} many sources of noise in the TRT were observed.
It was found that most noise sources do not affect the operation of the detector significantly, and there is no need to model them in the simulation software.
An example of such sources is a track-induced source, when e.g. a transition radiation photon, created by an electron, can be detected by the neighbor straw which is close 
but doesn't belong to the track, or appearance of a spontaneous electron cluster in the gas due to radioactive decay of unstable nuclei in the gas.
However, it was decided to model the so-called white noise, which consists of high-frequency uncorrelated small fluctuations of the Gaussian nature which can pile up high enough to reach a threshold. This kind of source can appear for many reasons, for example from the thermal noise of capacitors in the front-end electronics
or noise from the anode wire. A noise model was developed by the TRT community to reproduce noise levels in the detector.
The model does not explain the reasons of the noise, but numerically describes noise levels based on few simple assumptions.
% TODO what about rad damage?? will change during ops.
% 	as I understand LTs are changed in order to keep noise level constant --> no need to change modelling of the noise, because it is the same, even if detector become more damaged by radiation with time...
Some of these assumptions are:
\begin{itemize}
 \item Noise amplitude depends on the length of the wire.
 \item Every straw in the simulation has an ``ideal'' design shape, though in the real detector every straw is slightly different due to bending of the wire, 
 fluctuation of the gas density and temperature, the anode being connected to different power supplies, etc. The model does not simulate conditions of each wire, 
 which is not practical. 
 Instead, for every straw, its own value of the low threshold is assigned, which can fluctuate.
 \item The model aims to reproduce noise conditions of the detector, which means that the conditions have to be known and be used as input for the model.
 \item The model is universal, and it can work for any multi-channel detector with threshold discrimination.
\end{itemize}
The full description of the model is given in~\cite{kittelmann_thesis}. 
Below, only information related to the noise modelling in the straws with Argon gas will be discussed.

% - spontaneous cluster in the gas - not implemented because test beam data showed that it's not big effect as well
% - white noise - originating mainly from front-end electronics. Doesn't depend on the type of the gas mixture. But depends on LT.
% - Model don't try to provide detasailed understanding of noise source it only aim to reproduce total effect in the detector.
% - in the simulation all straws are identically ``ideal''. In reality it's not the case. Model doesn't try to describe separately condition for the each straw. 
% Instead it mimic spread of LTs which are originated by imperection of the straws in real life.
% - The main constrain for the model is average noise level. The idea of Argon implementation is to provide the same average noise level for Argon straws as for Xenon one.
% This assumption is taken from private communication with Anatoli Romaniuk.

% TODO describe formula whith explanations in general first and then discuss what was done for the Argon implementation!!!

A noise amplitude, according to the model, is equal to:
\begin{equation}
 \mathcal A_{i} = r_{i} \cdot \dfrac{\langle LT \rangle}{\langle f_i \cdot r_i \rangle}
 \label{eq:ampl_noise}
\end{equation}
where $r_{i}$ is a parameter which depends on the length of the straw and is modelled with a Gaussian with mean $\mu_i = c_1 l + c_2$, 
where $l$ is the length of the wire in the straw and $c_1$ and $c_2$ are empirical parameters.
% TODO add reference to sentence above, about the empirical parameters
The Gaussian width is equal to $\sigma_i = \omega \mu_i$, where $\omega$ is a free parameter which characterizes
relative channel-to-channel spread in noise amplitudes.
$\langle LT \rangle$ is the average of the low thresholds for all tubes. In practice, this value is equal to low thresholds of ``ideal'' straws 
implemented in the code.
$f_{i}$ is a noise level of the $i$'th straw. This value is generated based on the distribution of straw noise level measured in test beam studies in 2004 
(the distribution is reported in~\cite{kittelmann_thesis}).
% TODO what about end-cap??? need to check.
Test beam studies were done with a Xenon-based gas mixture. 

Since the low threshold is significantly different for Xenon and Argon straws and because the model is universal (can be applied to any number of straws), 
simulation of noise in Argon and Xenon straws is done separately. This means that values $\langle LT \rangle$ in \EquationRef{eq:ampl_noise} are 
different and equal to the low thresholds of Argon and Xenon straws implemented in the code. All other parameters are identical for Argon and Xenon straws.
It is important to mention that the noise distribution, which is used as input for the model and was measured in the test beam with Xenon straws,
is utilized for the Argon straw noise modelling as well in the initial code implementation~\cite{anatoli_private_communication}. 
The noise distribution has to be properly measured and tuned for the Argon straws, which
was not a part of the current studies and is a topic for further Argon mixture fine tuning.

%#######################################################################################################################
% SECTION 3 ############################################################################################################
%#######################################################################################################################
\section{TRT tracking performance with focus on active gas mixture}
\label{sec:trt_tracking_perf}

At the end of the Run-1, a part of the TRT, namely,
four of 32 radial sectors in the inner layer of the barrel and one of the 14 wheels in the end-cap A, were filled with Argon mixture, while other sectors were operating 
with the usual Xenon mixture, as shown in \FigureRef{fig:argonModulesIn2013}.
It was an excellent opportunity to test implementation of the Argon mixture in the digitization package.
This section contains an overview of the track and hit performance study of the TRT in this gas configuration.

\begin{figure}
\begin{center}
 \includegraphics[width=0.59\columnwidth]{TRT/Xe_Ar_map_2013_cutted.png}
\caption{Detector gas configuration with few Argon modules used during part of 2013.}
\label{fig:argonModulesIn2013}
\end{center}
\end{figure}

\subsection{TRT hit and track properties}

Key characteristics of a tracking detector are the spatial resolution of hits and their number. 
The better the resolution is, the more precisely a track and its momentum can be measured.
However, there are few other characteristics which are sensitive to the gas mixture used in the detector, which will be discussed below.
% TODO should I include some other parameters?
Distributions considered in this study are:
\begin{itemize}
 \item Reconstruction hit efficiency. This parameter characterizes the probability that a particle which traversing a straw will lead to a hit.
 It heavily depends on the distance of the closest approach of a particle to the anode wire, because longer the path of the particle is, more gas atoms it will ionize and stronger the signal will be. This is why typically it is plotted as a function of the track to wire distance. It is defined as:
 \begin{equation}
  \varepsilon = \dfrac{N^{TRT}_{hits}}{N^{TRT}_{hits} + N^{TRT}_{holes}}
 \label{eq:hit_eff}
 \end{equation}
 where $N^{TRT}_{hits}$ is the number of TRT hits in the track and $N^{TRT}_{holes}$ is the number of holes in the track. A hole corresponds to the case when there is no hit in the straw
 however, the reconstructed track crosses the straw.
 \item Position residuals. As described previously (see \SectionRef{subsec:tracking_with_trt}), the width of this distribution characterizes detector resolution.
 \item Number of precision and tube hits per track. These parameters directly correlate with the track momentum resolution.
 \item Track extension fraction, designed as a probability to find an extension of the track reconstructed in the silicon detector in the TRT.
 \item $r-t$ relation. It relates drift time to the drift radius and is a direct comparison of drift velocities in Xenon and Argon mixtures.
\end{itemize}


\subsection{Tracking performance}
\label{subsec:TRT:trackPerf}
% TODO add:
% 1) RT relation for Argon and Xenon with data!!! (I should have them!).
% 2) time residuals...

To make a performance comparison of Xenon and Argon straws taking into account used gas configuration shown in \FigureRef{fig:argonModulesIn2013},
only straws in the first barrel layer were considered because only the first layer contained some straws with Argon and some - with Xenon 
(while second and third were operating fully with Xenon, as shown in \FigureRef{fig:argonModulesIn2013}).
It is worth to mention that first layer of the TRT consists of two types of straws: regular and short, while second and third layer - only from regular straws.
Short straws were introduced to the first layer during the design phase in order for TRT to
better cope with high-$\mu$ running condition and to keep occupancy on a lower level.
Each layer of the TRT barrel part is split into 32 modules, which are placed radially. 
Modules 1-28 contained Xenon mixture and modules 29-32 contained Argon mixture.
For the end-cap detector part, only one wheel in one side was operating with Argon. Therefore for comparison, the same wheel number from the other side was used in order
to have the identical geometry for both Xenon and Argon cases.

One of the main differences between the Xenon and Argon gas mixtures is the electron drift velocity. Electrons drift faster in Argon. 
This can be seen from the plot in \FigureRef{fig:RT_xenon_argon}, 
%which is constructed and used for the calibration procedure.
which shows $r-t$ relation obtained after the iterative calibration procedure on data.
Argon $r-t$ distribution is shorter by a few nanoseconds than the Xenon one, which means that the drift time for a given drift radius is shorter.

\begin{figure}
\begin{subfigure}{.5\textwidth}
  \centering
  \includegraphics[width=\textwidth]{TRT/fromPoster/rt_FinalXenon_v2.eps}
\end{subfigure}%
\begin{subfigure}{.5\textwidth}
  \centering
  \includegraphics[width=\textwidth]{TRT/fromPoster/rt_FinalArgon_v3.eps}
\end{subfigure}

\caption{Track to wire distance in Xenon (left) and Argon (right) straws in the end-cap detectors.}
  \label{fig:RT_xenon_argon}
\end{figure}

The faster drift velocity of electrons in the Argon leads to a shorter drift time. Since time is digitized in bins of 3.125~ns, shorter drift time will have larger digitization uncertainty. Thus Argon gas will cause to a worse position resolution, which can be observed in \FigureRef{fig:resFit}. Worse position residual means that the momentum resolution will also become worse, according to the
\EquationRef{eq:momentum_resolution}. Nevertheless, the difference between Argon and Xenon is small.

It is worth mentioning that these results were obtained before the fine tuning of Argon simulation parameters. Thus a slightly bigger difference between data and MC distributions can be observed for Argon straws. A dedicated study was performed later by the TRT team to tune Argon MC parameters.

\begin{figure}

\begin{subfigure}{.5\textwidth}
  \centering
  \includegraphics[width=\textwidth]{TRT/fromPoster/resFitXenon_v2.eps}
  \includegraphics[width=\textwidth]{TRT/resFitXenonEndcap.pdf}
\end{subfigure}%
\begin{subfigure}{.5\textwidth}
  \centering
  \includegraphics[width=\textwidth]{TRT/fromPoster/resFitArgon_v2.eps}
  \includegraphics[width=\textwidth]{TRT/resFitArgon_Endcap.pdf}
\end{subfigure}

% \includegraphics[width=0.75\textheight]{monoW/kinematics_simplifiedSChannel_EFT_Wprime.png}
\caption{Position residuals for Xenon (left) and Argon (right) straws in barrel (top) and end-cap (bottom) detectors. Data is shown with black triangles, 
simulation fit - with the solid black line.}
  \label{fig:resFit}
\end{figure}

% \begin{figure}
% \begin{center}
%  \includegraphics[width=0.79\columnwidth]{TRT/grBMPos_mod.pdf}
% \caption{ Position residual width as a function of the TRT module number in the Barrel detector. Red points correspond to simulation of TRT detector
% using Argon-based mixture in all straws, black points correspond to Xenon-based mixture while green point correspond to the mixed detector condition
% which was used during ??? data taking period. Modules with Argon mixture in mixed condition can be clearly seen.}
% \label{fig:meanFreePath}
% \end{center}
% \end{figure}

The second parameter of interest is the reconstructed hit efficiency $\varepsilon$. It is calculated using \EquationRef{eq:hit_eff}.
Distribution of $\varepsilon$ for Argon and Xenon, both for barrel and end-cap parts, are shown in \FigureRef{fig:hit_eff_rtrack_bar}.
Argon straws have higher hit efficiency. To understand this effect, one has to recall that the low threshold settings
for Xenon and Argon straws are quite different (300~eV versus 100~eV). The Smaller low threshold for Argon was chosen in order to compensate for the fact that the same particle will produce a smaller number of primary electron clusters in Argon, 
and a number of electrons in the cluster will be smaller. It means that the signal from Argon tubes is weaker, which means
that the threshold has to be lower as well. Still, we observe higher efficiency for Argon than for Xenon. After some investigation, it was found that quite often for the signal to reach the low threshold, a two or more primary electron clusters have to reach the anode wire in
Xenon straws, while in Argon in most cases only one first cluster is enough to reach a low threshold. This is caused due to a few times lower LT in Argon with respect to the one in Xenon.
To equalize hit efficiency between Argon and Xenon, low threshold for Xenon tubes can be slightly increased.
In \FigureRef{fig:hit_eff_rtrack_bar} hit efficiency is shown as a function of track to wire distance. 
It is stable throughout the straw radius except for the values close to 2 mm, which corresponds
to the case when particle fly just a short path in the gas close to the straw wall. 
The number of primary ionization clusters is thus very small, hence hit efficiency drops significantly.
% TODO explain hit eff distribution vs. track to wire distance. Why do we have some deep close to r=0 ??? Ask Alex!

\begin{figure}

\begin{subfigure}{.5\textwidth}
  \centering
  \includegraphics[width=\textwidth]{TRT/fromPoster/hit_eff_rtrack_bar_xenon_region_v2.eps}
  \includegraphics[width=\textwidth]{TRT/fromPoster/hit_eff_rtrack_ec_xenon_region.eps}
\end{subfigure}%
\begin{subfigure}{.5\textwidth}
  \centering
  \includegraphics[width=\textwidth]{TRT/fromPoster/hit_eff_rtrack_bar_argon_region_v2.eps}
  \includegraphics[width=\textwidth]{TRT/fromPoster/hit_eff_rtrack_ec_argon_region.eps}
\end{subfigure}

% \includegraphics[width=0.75\textheight]{monoW/kinematics_simplifiedSChannel_EFT_Wprime.png}
\caption{Hit reconstruction straw efficiency as a function of track to wire distance 
for Xenon (left) and Argon (right) straws in Barrel (top) and End-cap (bottom) detectors.}
  \label{fig:hit_eff_rtrack_bar}
\end{figure}

% TODO do I want to show T0 caontants for Argon and Xenon? If yes - I have to understand first why they are different...
% As was mentioned above electrons drift faster in the Argon-based gas mixture than Xenon one, which can be observed in the track to wire distance
% distributions shown in \FigureRef{fig:RT_xenon_argon}. 
% This results that timing calibration constants have to be significantly different 
% for Argon and Xenon straws which can be seen in \FigureRef{fig:t0_mixed_condition}. 

Due to a higher hit reconstruction efficiency in Argon straws, the observed number of hits per track is also larger, as can be seen in \FigureRef{fig:nHitsPerTrack}. 
As shown in \EquationRef{eq:momentum_resolution}, larger number of hits leads to better momentum resolution. However, as it was shown before, position residuals are
worse for Argon which negatively affects momentum resolution. Still both factors are quite small, therefore momentum resolution is expected to be similar for
both Argon and Xenon mixtures.

\begin{figure}

\begin{subfigure}{.5\textwidth}
  \centering
  \includegraphics[width=\textwidth]{TRT/fromPoster/xenon_nHitsPerTrack_vs_phi.eps}
\end{subfigure}%
\begin{subfigure}{.5\textwidth}
  \centering
  \includegraphics[width=\textwidth]{TRT/fromPoster/argon_nHitsPerTrack_vs_phi.eps}
\end{subfigure}

\caption{Number of hits per track in configuration with all straws filled with Xenon (left) and Argon (right) gas mixtures in Barrel detector. MC simulation.}
  \label{fig:nHitsPerTrack}
\end{figure}

% TODO ask Alex why fraction of precision hits are better for Argon???
In \FigureRef{fig:precHitFracPerTrack} fractions of precision hits to total number of hits are shown. A hit is considered to be a precision
hit if the difference between track and drift radius is smaller than 2.5 times hit uncertainty. Argon has a higher fraction because hit uncertainty
for Argon hits is larger due to the faster drift time of the electron clusters and digitization binning. 

\begin{figure}

\begin{subfigure}{.5\textwidth}
  \centering
  \includegraphics[width=\textwidth]{TRT/fromPoster/xenon_precHitFracPerTrack_vs_phi.eps}
\end{subfigure}%
\begin{subfigure}{.5\textwidth}
  \centering
  \includegraphics[width=\textwidth]{TRT/fromPoster/argon_precHitFracPerTrack_vs_phi.eps}
\end{subfigure}

\caption{Precision hit fraction in configuration with all straws filled with Xenon (left) and Argon (right) gas mixtures in the barrel detector. MC simulation.}
  \label{fig:precHitFracPerTrack}
\end{figure}

The next set of plots is related directly to the whole tracks. To make a clear comparison of these variables, the whole detector has to operate either with Xenon or with Argon mixtures.
% TODO Alex: I think with your data was possible. You only had to look for extension in eta < 0.8 and then plot the extension fraction vs phi.
Unfortunately, there were no such data available. Thus the comparison is made only with the Monte Carlo simulation for the two detector configurations when all tubes contain either Argon mixture or Xenon mixture. In \FigureRef{fig:track_ext_fraction}, track extension fraction is shown. This parameter characterizes
probability to find the continuation of the track reconstructed in the silicon part of the Inner Detector in the TRT.
In the barrel region, the fraction is flat, except for the area around $\eta = 0$. This can be explained by the inefficiency of the first few layers of straws in the barrel region.
First layers of the barrel region contain shorter straws compared to other layers. 
As it was mentioned before short straws, help to deal with the occupancy at high $\mu$.
These short straws are split into three parts by two glass joints. Moreover, only two outer straw part are read out by electronics, why middle section (which is placed around $\eta=0$) is not read. Thus, it is an inefficient region of the detector. 
This is why a drop in the extension fraction is observed.
Behavior in the end-cap region can be explained by the geometrical factor, because in the intermediate region between the barrel and end-caps particles
traverse the smaller number of straws and with increasing the $\eta$ number of straws is increasing.
One can observe that the fraction is slightly higher for the Argon straws. It can be explained by the higher reconstruction hit efficiency and the larger number of hits per
track for Argon tubes.

\begin{figure}
\begin{center}
 \includegraphics[width=0.9\columnwidth]{TRT/fromPoster/track_ext_frac_eta.eps}
\caption{Track extension fraction as a function of $\eta$ for Xenon and Argon gas mixture. MC simulation.}
\label{fig:track_ext_fraction}
\end{center}
\end{figure}

\section{Summary}
\label{sec:trt_summary}

Implementation of the new Argon gas mixture to the TRT digitization code was presented. As shown above, simulation of Argon gas mixture works very well. though 
fine tuning of the simulation is needed.
Tracking performance of the TRT was compared for the case of Argon and Xenon gas mixtures, and it was found that tracking performance is not significantly affected
by the gas mixture used. That conclude that operating of the TRT with some part with Argon mixture will not affect tracking performance.



