
% TODO define
% - BCID

\chapter{The LHC and The \ATLAS experiment}
\label{chap:MoreStuff}

The Large Hadron Collider (LHC) is a circular accelerator at CERN with a designed center-of-mass (CM) energy of proton-proton collisions of 14~TeV and up to 2.3~TeV per nucleon for heavy ion collisions.
Collisions are registered by the four large experiments: ATLAS, CMS, ALICE, and LHCb. This chapter contains a short overview of the LHC performance as well as a description of the ATLAS experiment.

\section{The LHC performance and beam structure}

In what follows, only proton-proton collisions are considered.
The proton bunches are delivered to the LHC
by an injector chain that consists of Linac2, the Proton Synchrotron Booster (PSB), the Proton Synchrotron (PS), the Super Proton Synchrotron (SPS)~\cite{lhc_tdr_vol3}.
The schematic picture of the full acceleration chain is shown in \FigureRef{fig:accelerator_complex}.

\begin{figure}[]
  \centering
\includegraphics[width=0.99\textwidth]{intro/Cern-Accelerator-Complex_v2.jpg}
\caption{Schematic picture of the LHC accelerator complex at CERN.}
\label{fig:accelerator_complex}
\end{figure}

The maximum number of bunches possible in the LHC is 2808~\cite{lhc_tdr_vol1}.
Bunches are grouped in so-called bunch trains.
The design distance between the bunches in the train is 25 ns.
The longest possible train in the LHC during normal operation consists of 72 bunches with gaps between the trains of at least 12 empty bunches~\cite{lhc_tdr_vol3} 
The energy of the protons in the injected bunches equals 450 GeV and the bunches are injected from the SPS one after another. 
When the injection is done, the fill is complete and an additional acceleration in the LHC starts.
When the bunches have been accelerated to the collision energy (up to 7 TeV) and when all experiments are ready to record data, the so-called physics run starts. During physics runs all the LHC experiments collect physics data for further offline analysis. 

The number of protons in the bunches is decreasing with time due to the proton-proton (pp) collisions.
To achieve maximum delivered luminosity to the experiments 
it is therefore optimal to dump the beams when their intensity becomes too low and to inject new beams into the ring.
After the beams have been dumped, the magnetic field (which is equivalent to the current in the magnets) has to be slowly decreased to make the machine ready for new 450 GeV bunches and a new acceleration cycle.
The period between fills, where there are no bunches in the LHC, is called the interfill period. In the interfill period, the ATLAS sub-detectors perform calibrations to prepare for the next run. 
The luminosity distribution during the physics runs shown in \FigureRef{fig:interfill} demonstrates
the decrease of the luminosity over the run as well as the interfill period between the runs. 

\begin{figure}[]
  \centering
\includegraphics[width=0.6\textwidth]{intro/lhc_record_lumi_v2.eps}
\caption{Example of the luminosity during two runs as a function of time. The period with zero luminosity between the runs corresponds to the interfill period used by the detector groups to make calibrations.}
\label{fig:interfill}
\end{figure}


It is worth mentioning that there are other types of runs, which are used for LHC beam performance studies or special detector tests.
One example of such a special run is the so-called van der Meer (vdM) run which is used to calibrate  the luminometers in the experiments as will be described in \SectionRef{subsec:alfa_run}.

% LHC program description
The LHC running schedule is split into different periods which are called Runs (not to be confused with physics runs mentioned above). Run-1 started at the LHC startup and was using a 50 ns bunch spacing in the trains and had a center-of-mass energy of 7 and 8 TeV. 
Run-2 started in 2014 and is still ongoing at the time of writing. The center-of-mass energy of the collisions is 13 TeV and the bunch spacing is 25 ns (which corresponds to the designed bunch spacing).
During Run-3 and Run-4 LHC will operate with 14 TeV CM energy and with a significantly increased luminosity. The total planned luminosity to be delivered by LHC to the experiments is 3000 fb$^{-1}$.
Information about all running periods is shown in \FigureRef{fig:lhc_program}.

\begin{figure}[]
  \centering
\includegraphics[width=0.99\textwidth]{intro/lhc_program.png}
\caption{An approximate timeline of the scheduled LHC upgrades with planned integrated luminosity to be delivered to the experiments~\cite{Kawamoto:1552862}.}
\label{fig:lhc_program}
\end{figure}


\section{The ATLAS experiment}
% a few words about ATLAS

% ATLAS is a multipurpose detector at LHC
% 
% Reconstruction of the track of the charged particles and primary and secondary vertices is done in the Inner Detector, which consist from high-precision silicon Pixel detector, silicon strip detector (SCT) and transition radiation tracker (TRT).

The ATLAS (A Toroidal LHC ApparatuS) experiment is a multi-purpose detector at the LHC. It has a forward-backward symmetric design with respect to the Interaction Point (IP).
It consists of the Inner Detector (ID), the Electromagnetic (EM) and hadronic calorimeters and the Muon Spectrometer (MS). The magnetic field is provided by the magnet system. A detailed description of each subsystem is presented below.

\subsection{The ATLAS coordinate system}

An illustration of the coordinate system and the track parameters are shown in \FigureRef{fig:coordinate_system}.
The base vectors $e_x$, $e_y$ and $e_z$ shown in the figure represent the x-, y- and z-axes respectively. The vector $e_x$ is pointing to the center of the LHC ring;
vector $e_y$ is pointing vertically upward; vector $e_z$ is pointing along the beam axis. 
Tracks of charged particles in ATLAS are parameterized with these five parameters~\cite{track_parameterization}:
\begin{itemize}
 \item The transverse impact parameter $d_0$, which is the distance to the beam axis in the $x-y$ plane. 
 \item The longitudinal impact parameter $z_0$, which is the distance to the coordinate system origin in z-direction.
 \item The azimuthal angle $\phi$, measured in the $x-y$ plane starting from vector $e_x$. 
 \item The polar angle $\theta$, measured from vector $e_z$.
 \item The charge of the particle divided by its momentum, $q/p$, which characterizes the track curvature.
\end{itemize}

\begin{figure}[]
  \centering
\includegraphics[width=0.6\textwidth]{intro/coordinate_system.png}
\caption{Track parameterization in the ATLAS coordinate system~\cite{Cornelissen:2007aca}.}
\label{fig:coordinate_system}
\end{figure}

\subsection{Magnet System}
The ATLAS Magnet System~\cite{tdr_magnet} provides a magnetic field to bend tracks of charged particles. This is used to measure the momentum of the particles as well as their charge by the curvature of the track.
The magnet system consists of three superconducting magnets:
\begin{itemize}
 \item The Central Solenoid. It is placed around the Inner Detector and provides a 2~T magnetic field directed along the beam axis inside the ID. 
 Since it is placed before the EM calorimeter, the material budget has to be low 
 in order to not distort measurement in the calorimeter.
 \item The Barrel Toroid. Provides $\sim$0.5~T inside the barrel muon spectrometer.
 \item The End-Cap Toroids. Provides $\sim$1.0~T inside the toroid muon spectrometer.
\end{itemize}
The magnet configuration is shown in \FigureRef{fig:atlas_magnets}.

\begin{figure}[]
  \centering
\includegraphics[width=0.8\textwidth]{intro/magnets.eps}
\caption{Illustration of the ATLAS magnet system.}
\label{fig:atlas_magnets}
\end{figure}

\subsection{The inner detector}
\label{sec:ID}
% additional feature - electron identifiation with TRT

The innermost detector of ATLAS and the closest one to the interaction point (IP) is the inner detector (ID).
The main purpose of the ID is to reconstruct tracks of all charged particles which pass through the detector and to measure their momenta.
Tracking detectors also have to provide information on the sign of the electrical charge of the particles, 
which is why a strong magnetic field is applied within the ID, 
which makes tracks of particles with different charges bend into different directions.

Vertices are formed from reconstructed tracks. The primary vertex corresponds to the vertex where the $pp$ collision took place, 
while secondary vertices correspond to the decay of particles.

The ID is done with different layers of detectors. When a particle interacts with one of them it deposits a part of its energy to the sensor,
and this energy is being converted by the sensor readout electronics. 
If the signal is larger than a predefined threshold a hit is recorded.
One wants to have a large number of hits in order to precisely measure the particles track.  
However, too much of detector material can lead to multiple scattering and conversion of photons.
The material  budget of the ID is shown in \FigureRef{fig:material_budget}.

The particle density is falling with the distance from the IP as $1/R^2$, 
thus the layers close to the IP need to have a high granularity in order to be able to distinguish between
hundreds of particles from one \pp collision, while the outermost layers can have a lower granularity.
The ID consists of three subdetectors, listed from innermost to outermost: the high-granular silicon pixel detector, the silicon strip (SCT) detector and the transition radiation tracker (TRT).


\begin{figure}
\centering
\includegraphics[width=0.7\textwidth]{intro/material_budget.eps}
\caption{ 
The material budget of the ATLAS Inner Detector as a function of absolute pseudorapidity in units of radiation length $X_0$.
}
\label{fig:material_budget}
\end{figure}




All subdetectors consist of two barrel parts (Barrel A and Barrel C) and two end-cap parts (End-cap A and End-cap C) 
which are placed symmetrically with respect to the interaction point.
The ID geometry and acceptance is shown in \FigureRef{fig:ID_eta}.

\begin{figure}
\centering
\includegraphics[width=0.99\textwidth]{TRT/TRTeta.png}
\caption{ 
A quarter of the Inner Detector showing the detector acceptance and the geometrical sizes of the layers.
}
\label{fig:ID_eta}
\end{figure}


% Pixel and SCT
The Pixel detector is a semiconductor detector which consists of silicon pixel sensors~\cite{Wermes:381263}.
The detector resolution in the barrel (endcap) region is 10 $\mu m$ in $R-\phi$ and 115 $\mu m$ in $R$ ($z$).

The SCT detector is a semiconductor microstrip detector. Each layer consists of two layers of strips rotated in 40 mrad with respect to each other.
The detector resolution in the barrel (endcap) region is 17 $\mu m$ in $R-\phi$ and 580 $\mu m$ in $z$ ($R$).

% TRT
% electron identifications
The TRT contains $\sim$300000 thin-walled proportional-mode drift tubes providing on average 30 two-dimensional 
space points with $\sim$130 $\mu m$ resolution for charged particle tracks with |$\eta$| < 2 and $p_T$ > 0.5 GeV~\cite{Abat:2008zza,Abat:2008zzb,Abat:2008zz}.
In addition to continuous tracking, the TRT provides electron identification through the detection of transition radiation X-ray photons, which are created by the charged particles passing through layers of radiator material between the tubes.
A detailed description of the detector can be found in \ChapterRef{chap:TRT}.

\subsection{The calorimetry system}

The ATLAS calorimeters are designed to trigger on and to measure accurately the energy and position of photons, electrons and hadrons, as well as to ensure a good missing energy measurement, which is crucial for new physics searches. The calorimeter system is divided into two main parts: an inner electromagnetic (EM) calorimeter aimed at detecting electrons and photons, and an outer hadronic calorimeter designed to detect mesons and baryons which penetrate the EM calorimeter.

The EM calorimeter covers the rapidity region $|\eta| < 3.2$.
The hadronic calorimeter is divided up into a barrel hadronic calorimeter covering $|\eta| < 1.7$, an hadronic end-cap calorimeter covering $1.5 < |\eta| < 3.2$ and a forward calorimeter covering $3.1 < |\eta| < 4.9$. A global view of the ATLAS calorimeter system is illustrated in \FigureRef{fig:Calo}.

\begin{figure}[h!]
\centering
 \includegraphics[width=0.99\textwidth]{intro/0803015_01-A4-at-144-dpi.jpg}
 \caption{Three-dimensional view of the ATLAS calorimetry.}
\label{fig:Calo}
\end{figure}

\subsubsection{EM calorimeter}
The ATLAS EM calorimeter is a sampling calorimeter with accordion-shaped lead absorbers and kapton electrodes.
\FigureRef{fig:EMgran} shows the accordion shaped geometry of the ATLAS EM calorimeter. The accordion geometry provides a fast signal readout and an azimuthal symmetry without cracks. Liquid Argon (LAr) is used as an active material. The EM calorimeter is divided into two barrel parts ($|\eta|<1.475$) and end-caps ($1.375<|\eta|<3.2$).
The end-cap calorimeter on each side consists of two wheels: the Inner Wheel ($1.375<|\eta|<2.5$) is the closest part to the beampipe, while the
Outer Wheel ($2.5<|\eta|<3.2$) is the part furthest from the beampipe.
The amount of material in terms of number of electromagnetic radiation lengths ($X_0$) is shown in \FigureRef{fig:material_budget_calo}.
The thickness of the EM calorimeter is above $24 X_0$ in the barrel and above $26 X_0$ in the end-cap regions.

Both the barrel and the end-cap calorimeters are segmented into three longitudinal layers. The first layer has a thickness of about $6 X_0$ and plays a role as a preshower. It has the finest granularity in $\eta$ with a cell width of about 4~mm. The second layer has a thickness of about $18 X_0$ and is designed to contain almost a full EM shower. It has the finest cell granularity in $\phi$ and provides the azimuth coordinate of the electromagnetic shower direction. The third layer has a two times coarser granularity and the thickness varying between $2 X_0$ and $12 X_0$. The read-out granularity of the LAr system and the accordion shape of the EM calorimeter are schematically illustrated in \FigureRef{fig:EMgran}.

\begin{figure}[h]
\centering
 \includegraphics[width=0.6\textwidth]{intro/LARG3-TDR-barrelM_samplings_presamp_new.png}
 \includegraphics[width=0.39\textwidth]{intro/F3-eps-converted-to.pdf}
 \caption{Read-out granularity and accordion shape of the barrel EM calorimeter.}
\label{fig:EMgran}
\end{figure}

\begin{figure}[h]
\centering
 \includegraphics[width=0.5\textwidth]{intro/caloDepth_eta.png}
 \caption{
 The amount of material traversed by a particle before and in the EM calorimeter, in units of radiation lengths $X_0$ , as a function of $|\eta|$. Different colors represent three different longitudinal layers of the EM calorimeter.}
\label{fig:material_budget_calo}
\end{figure}

The main goal of the lead absorbers in the sampling EM calorimeter is to develop an electromagnetic shower, with a part of the EM shower detected in the LAr sensitive material. The energy deposited in the absorber material is accounted for by the known sampling fraction of the calorimeter. In order to achieve a good performance of an EM calorimeter, an important aspect is the material in front of the calorimeter as it degrades the energy resolution of the calorimeter~\cite{electron_tight}.
A presampler is placed before the ATLAS EM, calorimeter covering the pseudorapidity range of $|\eta|<1.8$. It is needed in order to recover the energy lost in the material before the calorimeter (inner detector, cryostat, etc).
The relative energy resolution for EM objects is parameterized as follows:
\begin{equation}
\frac{\sigma(E)}{E}=\frac{a}{\sqrt{E[GeV]}}\otimes\frac{b}{E[GeV]}\otimes c
\end{equation}
where $a$ is the sampling term which describes the statistical fluctuations of the EM shower, $b$ is the noise term due to electronics and pile-up and $c$ is the constant term which accounts for non-uniformity of the calorimeter response. The sampling term mostly contributes at low energies, whereas at high energies the energy resolution goes asymptotically towards the constant term, which is designed to be 0.7\%.
The transition region between the barrel and the end-cap, $1.37<|\eta|<1.52$, has a significant amount of material in front of the calorimeter (about $ 10 X_0$), making the energy resolution there to be poor and thus it is usually excluded in physics analyses.

The drift of ionisation electrons in the LAr gap is ensured by a high voltage system which generates an electric field of about 1~kV/mm. The induced current on the electrodes is then used to reconstruct the deposited energy in an EM calorimeter cell.

The reconstruction of the electrons and photons starts by reconstructing clusters, i.e., a group of calorimeter cells containing almost the full EM shower. Clusters matched to a well-reconstructed track in the ID and originating from the interaction point (IP) are then classified as electrons. Clusters without corresponding track matching are considered as unconverted photons. If there are two tracks corresponding to the reconstructed cluster, and if a conversion vertex can be reconstructed, the cluster is classified as a converted photon.

\subsubsection{Hadronic calorimeter}

The hadronic calorimeter surrounds the EM calorimeter and is designed to detect the hadrons penetrating the EM calorimeter.
It consists of a Tile calorimeter in the range of $|\eta|<1.7$, constructed with an iron-scintillating-tiles technique, and an hadronic end-cap LAr calorimeter spanning $1.5<|\eta|<3.2$. The acceptance of the hadronic calorimeter is extended by the LAr Forward calorimeter up to $|\eta|<4.9$ (see \FigureRef{fig:Calo}). The LAr technology for large $|\eta|$ is chosen because of its intrinsic radiation hardness.
The signal in the Tile calorimeter is provided by scintillating tiles as an active material, while the absorbers are made of iron. The calorimeter is divided into a barrel and two extended barrels with the inner radius of 2.28~m and the outer radius of 4.23~m. Similarly to the EM calorimeter, the Tile calorimeter is longitudinally segmented into three layers, which are needed for triggering and reconstruction of jets. The readout of the tiles is performed by optical fibers. The tiles are grouped into readout cells, which are designed to be projective with respect to the interaction point.

The end-cap hadronic calorimeters are constructed with copper as an absorber and LAr as an active material. The absorber plates are orthogonal to the beam axis and consist of two consecutive wheels
with a thickness of 25 and 50 mm. The forward calorimeter is placed at a distance of about 5 meters from the interaction point. It consists of three longitudinal sections: the first is made of copper absorbers, while the next two are made of tungsten absorbers. The forward calorimeter also provides an electron reconstruction capability.

% TODO Few sentenceas how jets are reconstructed?

\subsection{Muon Spectrometer}

The Muon Spectrometer (MS) is the outermost part of the ATLAS detector and is designed to trigger on and detect muons, the only charged particles that penetrate the calorimeter system, and it covers the pseudorapidity range of $|\eta|<2.7$. It is a tracking detector which measures the muon tracks deflected in the strong magnetic field. The MS consists of one barrel ($|\eta|<1.05$) and two end-cap sections. The system of large superconducting air-core toroid magnets provides the magnetic field of 0.5~T and 1~T in the barrel and end-cap, respectively, resulting in a bending power between 2.0 and 7.5~Tm~\cite{MCPrun1}. It is equipped with three cylindrical layers of Monitored Drift Tube Chambers (MDT) and Cathode Strip Chambers (CSC) providing track measurement; three doublet layers for $|\eta|<1.05$ of Resistive Plate Chambers (RPC) and three triplet and doublet layers for $1.0 < |\eta| < 2.4$ of Thin Gap Chambers (TGC) providing triggering and ($\eta, \phi$) measurements of the muon track momentum.
The CSC is used in the forward region instead of MDT due to the high background conditions.
A combination of four complementary technologies is needed to provide a precise muon measurement over a large $\eta-$range. The layout of the muon spectrometer is shown in \FigureRef{fig:MS}.

\begin{figure}[h!]
\centering
 \includegraphics[width=0.64\textwidth]{intro/MS1.pdf}
 \caption{Cut-view of one quadrant of the Muon Spectrometer.}
\label{fig:MS}
\end{figure}

The MS allows for a precise muon momentum measurement in the pseudorapidity region up to 
$|\eta|<2.7$ and provides a relative resolution better than 3\% over a wide range of $p_T$. It deteriorates to 10\% at $p_T\sim 1$~TeV.
The MDT and the CSC provide a single hit resolution in the bending plane of about 80~$\mu m$ and 60~$\mu m$, respectively.

The muon track in the MS is reconstructed in two steps. In the first step the muons are triggered in the RPC/TGC and local track segments are defined in each layer of chambers. In the next step, the local track segments from different layers are combined through a $\chi^2-$fit forming a full MS track. To reduce the probability of background tracks penetrating the calorimeter, the fitted tracks of the muon candidates are required to point towards the interaction point.

\subsection{Trigger system}
% TRIGGER
% Why do we need triggers
The maximal theoretically possible collision rate at the LHC is 40MHz.
Since the size of one \pp collision event recorded by the ATLAS detector is 1-2 Mbytes, there is no physical possibility to record such a huge stream of data and store it permanently. However, more than 99$\%$ of the events happening in \pp collisions have no interest for physicists because they have been well studied previously due to their high cross section. Thus one wants to ``hunt'' only for events which occur very rarely. This is why non-interesting events have to be rejected by the trigger system.
% Three-level trigger system with description of the L1, ROI, L2 and EF
The ATLAS experiment has three levels of triggers: a hardware-based level 1 (L1), a software-based level 2 (L2) and the Event Filter (EF)~\cite{tdr_tdaq}.
The reduction of the rate of the accepted events for the three trigger levels is shown in \FigureRef{fig:trigger_rate}.
The L1 trigger uses trigger chambers in the MS and the coarse-granulated calorimeter information to find high-$p_T$ charged leptons, photons or large missing transverse energy. The so-called regions of interest (ROI) are then formed corresponding to the ($\eta$,$\phi$) phase space of the detector where large signals have been observed. These regions are used by the L2, which uses the full granularity information within a ROI. The EF takes events which passed L2 and apply the same algorithms as in the offline analysis to make a final decision to store or discard the event.

\begin{figure}[h!]
\centering
 \includegraphics[width=0.7\textwidth]{intro/trigger_rate_anthony.png}
 \caption{The acceptance rate as a function of time needed to make a decision about the event by the trigger. A comparison with the rate of the SM processes occurring in the \pp collision are shown as well~\cite{anthony_thesis}.}
\label{fig:trigger_rate}
\end{figure}

In this manner, one can significantly reduce the rate and accept only events of interest. A comparison of the  trigger acceptance rate with the rate of some SM processes is shown in \FigureRef{fig:trigger_rate}.

% TODO DAQ

% TODO DCS (from paper)
