\chapter{LUCID}
\label{chap:LUCID}

% ToWRITE:
% \begin{itemize}
%  \item what this chapter is about
%  \item List of publications related to this topic
%  \item My personal contribution related to the described topic
% \end{itemize}

This chapter describe the LUCID detector which was built specially for the Run-2 phase of the LHC program.
It covers aspects such as the design of the detector and its key components,the assembly and testing of the new detector as well as
operation and performance of the detector during the 2015-2016 data taking.
Special attention is given to the calibration system of the detector and the development of the calibration procedure.

During the design phase a lot of tests were done in order to find an optimal design and the parameters for various detector components.
A lot of tests were done in order to understand the behavior and performance of the detector components.
During the assembly and installation phase a number of tests were done to make sure that all components performed they should.
An overall testing of the system was done to make sure that no damage had been done during the installation of the detector.
During the operation phase, which is still ongoing, a lot of studies have been made to understand the performance of the calibration system and the detector.

I contributed to all the steps mentioned above. I took part in the development of the LUCID design and in particularly the design of the calibration system. 
I made a series of tests to find the optimal design parameters of the LED and Laser diffusers used to evenly distribute LED and laser signals and deliver it to all detector 
channels. 
I spent a lot of time on understanding the behavior of the LED system as well as the PMT and PIN-diode signal behavior.
Tests with Bi-207 radioactive sources which are used as one of the way to monitor photomultiplier (PMT) gain were done as well.
I participated in the detector assembly in the clean room and did testing of the detector during this process.
Testing of LED and laser diffusers were done in order to cross check the integrity of fibers and the homogeneity of signals between all PMTs.
Also temperature stress-test was performed in order to understand what maximum temperature could be allowed without destroing the detector during the 
beam pipe bake-out procedure.
In the operational phase the main focus was on understanding the aging of PMTs and the possibility to improve the calibration system.

The LUCID group published a paper with a description of the choice and the characterization of photomultipliers for the new LUCID detector for Run-2~\cite{Alberghi:2016tad}.
My contribution was in understanding and developing the monitoring using a Bi-207 source.

During my PhD studies I was the ATLAS Forward detectors Run Coordinator for a period of 5 months.


% \section{Plan of the LUCID chapter}
% 
% \begin{itemize}
%  \item Lucid operation (should I write about it? if yes - what can I write?)
%  \begin{itemize}
%   \item performance of calibration system.
%   \item HV changes in 2015/2016. As an interesting fact how PMTs are aging.
%   \item interesting observations from LUCID operatins.
%   \item performance of detector - luminosity in 2015 (as a conlusion of LUCID chapter).
%  \end{itemize}
%  
%  \item PMT temperature test (should I describe experimental setup and procedure of development a circuit?);
%  \begin{itemize}
%   \item have a look on temperature change during the nights...
%  \end{itemize}
% 
%  \item Fiber boiling test (see sent letter by october 15, 2014):
%  \begin{itemize}
%   \item picture of bake-out phase
%  \end{itemize}
%  
%  \item Testing of LUCID after installation:
%  \begin{itemize}
%   \item testing of LED diffuser and integrity of fibers
%  \end{itemize}
%  
%  \item Detector design tests:
%  \begin{itemize}
%   \item Understanding LED and PIN-diode:
%   \begin{itemize}
%    \item tests charge or PMT/PIN vs. LED DAC (no conclution reached?)
%    \item angular distribution (lucid$\_$july22.pdf)
%    \item distance measurements 
%    \begin{itemize}
%     \item was done to proof that PMT is linear with light intensity
%     \item this because we saw that PMT charge is not linear vs. LED DAC.
%     \item conclusion: LED is not linear vs. DAC
%    \end{itemize}
%    
%    \item LED frequency test
%   \end{itemize}
%   
%   \item LED difuser geometry (no presentation?).
%   \item Laser diffuser distance.
%   \item Filter choice for LED diffuser (lucid$\_$september10.pdf; also ask Carla about 100 mV restriction).
% %   \item 
% %   \item 
% %   \item 
%  \end{itemize}
%  
%  \item Time tests. Long LED runs. (Though Anders also did it and Vincent is using his results to show, but I also was working for quite some time on it, 
% so I think I should write about it as well):
%  \begin{itemize}
%   \item find presentation of Anders. Check what he did.
%   \item from my presentation (lucid15$\_$apr15.pdf) it looks like we have strong effect in high rate and no effect in low rate.
%  \end{itemize}
% \end{itemize}
% 
% \newpage

\section{The new LUCID-2 detector}
\label{sec:LUCID}

[TODO: rewrite thoroughly this subsection!].

LUCID (LUminosity Cherenkov Integrating Detector) is a luminosity monitor with two detectors placed around the beam-pipe on both forward ends of the ATLAS detector. 
Each detector consists of 16 photo multipliers and 4 quartz fiber bundles. The Photo Multiplier Tubes (PMTs) detect charged 
particles that traverse their quartz windows, where Cherenkov light is produced. Cherenkov light is produced in 
the fiber bundles as well and carried to PMTs that are protected by shielding some 2 meters away 
(see \FigureRef{fig:LucidDrawing}). To increase the detector lifetime, only a subset of the PMTs is used at a 
given time, the others being available as spares. In addition, 4 PMTs have a reduced window opening to decrease 
their acceptance and thus avoid saturation of some luminosity algorithms.

\begin{figure}
\centering
\begin{subfigure}{.5\textwidth}
  \centering
  \includegraphics[width=\linewidth]{LUCID/LUCIDdesign.png}
\end{subfigure}%
\begin{subfigure}{.5\textwidth}
  \centering
  \includegraphics[width=\linewidth]{LUCID/FourPMTs_zoomed.png}
\end{subfigure}
\caption{(left) Schematic drawing of one of the two detectors, showing the position of the photomultiplier tubes 
and quartz fibers with respect to the LHC beampipe; (right) A quarter of one of the detectors. All tubes are 
placed inside mu-metal shielding to protect the PMTs from a stray magnetic field. Cooling pipes carrying water were installed in order 
to protect the PMTs from overheating during the beampipe bake-out procedure. Three of four tubes have fiber connectors, which
transfer LED and laser pulses for calibration. The fourth tube is equipped with a Bi-207 source and is completely 
sealed.}
\label{fig:LucidDrawing}
\end{figure}


\subsection{Motivation for the new LUCID-2 detector}
\label{subsec:motivationNewLucid}

[TODO: rewrite thoroughly this subsection!].

With respect to the detector used in Run I~\cite{Aad:2013ucp}, the new LUCID has a reduced material budget, 
an increased dynamic 
range and will measure luminosity with additional algorithms based on PMT charge integration. It also has a 
completely new calibration system.

\subsection{Choice of photomultipliers}
\label{subsec:PMTChoice}

[TODO: rewrite thoroughly this subsection!].

The new LUCID uses R760 Hamamatsu PMTs, a smaller version of the previously used R762 model. These PMTs have a 
10 mm quartz window diameter, while the old ones had a 14 mm diameter. A smaller PMT model has been chosen to reduce acceptance 
which will help to cope with the increased occupancy and to avoid saturation of the luminosity algorithms.
In addition, 4 PMTs per side have a specially reduced sensitive window with a 7 mm diameter which roughly 
corresponds to a factor 2 decrease in rate (see \FigureRef{fig:modPMT}). They provide luminosity algorithms that will not saturate at 
increased luminosity. Detailed description of choice and characterization of PMTs used in detector can be found in~\cite{Alberghi:2016tad}.

[ToASK: Should I add additional information how choise of PMTs was done (as described in the paper~\cite{Alberghi:2016tad})?].
% TODO also mention that below in the text there will be some description of tests done to understand behaviour of PMts?

\begin{figure}
\centering
\includegraphics[width=.6\textwidth]{LUCID/mod_PMT.jpg}
\caption{R760 Hamamatsu PMT with specially reduced sensitive window size used in the LUCID detector.}
\label{fig:modPMT}
\end{figure}


\subsection{Read-out electronics}
\label{subsec:LUCIDElectronics}

[TODO: rethink this subsection!].

New readout electronics have been built that consist of VME boards that digitize the PMT signals with FADCs. 
The electronics record hits if the pulseheight is above a threshold and integrate the pulses in each 25 ns 
interval that correspond to a LHC bunch crossing. \FigureRef{fig:pulseShape} shows a typical PMT signal shape in 
a physics run. The duration of the pulses is less than 25 ns.

\begin{figure}
\centering
\includegraphics[width=.6\textwidth]{LUCID/pulseShape_run_267367_preliminary.eps}
\caption{Digitized pulse shape of a signal from one of the PMTs of the LUCID detector during a run recorded on 
the 10th of June 2015 at $\sqrt{s}$ = 13 TeV. The polarity of the pulse is inverted. The FADCs measure the 
pulse amplitude in time bins that are 3.125 ns long.}
\label{fig:pulseShape}
\end{figure}

The LUCID read-out consists of four (two per side) custom made so-called LUCROD boards of VME type which sit close to the detector in the ATLAS experimental hall.
The decision to place electronics close to the detector in the experimental hall was motivated by preventing signals to develop long tails in the cables.
Signal from PMTs are transferred with thick cables which prevent distortion along their path.
Every LUCROD board has 16 input channels and every channel consists of a low noise amplifier, a filter and a Flash ADC.
A block diagram of LUCROD is shown in \FigureRef{fig:LUCROD_schematics}.
Channels are grouped in pairs and for each pair there is a dedicated channel FPGA.
All information from all channel FPGAs are collected and processed by main FPGA.
After that information are sent to the ATLAS as well as to so-called LUMAT boards which are placed in the counting room of the experiment.

\begin{figure}
\centering
\includegraphics[width=.95\textwidth]{LUCID/LUCROD_schematics.jpg}
\caption{Block diagram of the LUCROD board. Every board host two input channels. Every channel consist from low noise amplifier, flash ADC and FPGA. 
There are 16 channels (8 units) per LUCROD. All units are controlled by main FPGA, which collect information from all channels and make needed calculations.}
\label{fig:LUCROD_schematics}
\end{figure}

There are two LUCROD board per side and it was decided to couple different sets of sensors to the different boards as shown in 
\FigureRef{fig:Eletronics_schematics}.
BI, VDM and SPARE PMTs were connected to one board while MOD and FIB PMTs were connected to the another board.
The same connection scheme was used on the other side as well.

% TODO change this picture with one which is in lumi note.
\begin{figure}
\centering
\includegraphics[width=.95\textwidth]{LUCID/Eletronics_schematics.eps}
\caption{Block diagram of the LUCID electronics. Signals from all photomultiplier tubes are collected by 4 \mbox{LUCROD} cards 
(two per side) that digitize the signals with FADCs. Some of the luminosity algorithms are implemented in the LUCRODs. 
The number of events that fullfil different luminosity algorithms are counted and a copy of all digitized PMT 
signals are sent to the LUMAT cards, which perform calculations with algorithms that combine data from 
both detectors and publish the results to the Information Server (IS) database.}
\label{fig:Eletronics_schematics}
\end{figure}

Every LUCROD boards receive information from PMTs from only one of the two detectors.
In order to implement the possibility of requiring signals in both detectors, two additional boards which are called LUMAT boards are used
as shown in \FigureRef{fig:Eletronics_schematics}.
Digital signals from PMTs of the same family from LUCRODs from both sides are sent to LUMAT boards, 
which then perform logical operation with signals from both sides.

Information from the LUCROD and LUMAT boards are then published to the Information Server (IS) which is a database.
From these the data is accessed by programs that calculate the luminosity online.

\section{Design of the PMT gain monitoring system}

% TODO add reference to some book with desciption of PMT aging 

[TODO: describe why do we need calibration system. What is the purpose of it. Stress that it's important for correct measurements of luminosity.]

The PMT gain is monitored in 3 independent ways (see \FigureRef{fig:calibrationSystem}):
\begin{itemize}
 \item by LED signals carried by optical fibers;
 \item by laser signals transferred from the calibration system of ATLAS Tile Calorimeter;
 \item by radioactive sources (Bi-207).
\end{itemize}

\begin{figure}
\centering
\includegraphics[width=.7\textwidth]{LUCID/calibrationSystem.png}
\caption{The LUCID PMT gain monitoring system. 16 PMTs per side receives light from LEDs and the Tile laser calibration 
system. 
For redundancy, two fibers come from two different LED diffusers (with three LEDs each, monitored by 
PIN-diodes), and two fibers come from one laser diffuser. The four remaining PMTs in each detector are calibrated 
with Bi-207 sources.}
\label{fig:calibrationSystem}
\end{figure}

The availability of three independent calibration methods increase the robustness of the calibration system 
and provide a possibility to cross-check calibration results between the methods.

LED signals provide peaks in the amplitude and charge distributions that are recorded by LUCID in data acquisition runs between LHC fills.
The stability of the PMT gain is controlled by measuring the mean value of these distributions 
and then changing the high voltage to the photomultiplier in order to keep their mean values constant. 
The stability of the LEDs themselves is controlled by PIN-diodes and an alternative way of calibration is 
to use the ratio of the mean charge measured by PMTs with that of the PIN-diode. 
This charge is proportional to the LED intensity and by using this charge ratio allows to rule out any 
dependence of the calibration results on LED intensity fluctuations.
In order to provide the same amount of light simultaneously to all PMTs a special LED diffuser was designed and manufactured.
This is discussed in details at \SectionRef{subsec:LEDDiffuser}.

The Tile calorimeter laser system provides an alternative source of stable light and is treated in the same way
as the LED signals in the calibration procedure. 
The stability of the laser signals is monitored by the Tile calibration system \cite{atlasGeneral}.
The laser light has to be distributed between the PMTs in the same way as the LED light.
Laser light is provided by the Tile calibration system via optical fibers which means that another type of diffuser
has to be used in order distribute the light to the PMTs as described in \SectionRef{subsec:laserDiffuser}.

Bi-207 radioactive sources provide monoenergetic electrons from an internal conversion process with energies 
above the Cherenkov threshold in quartz. These electrons
have enough kinetic energy to penetrate the quartz window of the PMT and produce signals similar to the signals 
from high energetic particles in physics runs. The truncated mean of the charge and amplitude 
distributions from the Bi-207 sources are used in the same way as for the two methods described above. 
This method does not suffer from any instability issues~\cite{Alberghi:2016tad}.
It was decided to use a liquid Bi-207 source as described in \SectionRef{subsec:bi207Calibration}.

In \SectionRef{subsec:calibPerformance} the calibration strategy during the  2015 and 2016 data-taking periods are discussed.

\subsection{The LED diffuser}
\label{subsec:LEDDiffuser}

% TODO structure of the section:
% - to describe why we have 2 LED diffusers: why we use filters on one and don't use on another on
% - geometry studies
% - show LED stability plot over time (PIN-diode measurements)
% - filter studies - explain why we have to be below 100mV for PMTs (discuss this with Carla)
% - 
% - 


% light has to be evenly distributed among PMTs
% LED has to be monitored by PIN-diode in order to verify its stability
% geometrical constrain
%   taking into account size of PIN-diode
%   size of support structure for fibers
%   bending of fibers - whole LED diffuser has to be hosted in limited space and fiber cannot be bend to much.
% constrain during production of the diffuser:
%   complexity of drilling holes for fiber for some angles
%   gluing fibers to the diffuser


Despite the simple purpose of the LED diffuser to evenly distribute light among the PMTs 
there was a number of constrains that made it necessary necessity to make dedicated studies 
to define the optimal parameters of the diffuser. The main points which were considered during the design phase were:
\begin{itemize}
 \item light had to be evenly distributed among PMTs;
 \item the LED had to be monitored by PIN-diode in order to verify its stability;
 \item geometrical constrains such as the size of the PIN-diode, 
       limited space for the whole diffuser and limits on the bending of fibers had to be taken into account;
 \item manufacturing constrains such as the complexity of drilling small holes and gluing fiber at specific angles had to be kept in mind.
\end{itemize}

To meet all these requirements a radial design was proposed with a PIN-diode facing three LEDs and located aligned with LEDs axis. Fibers surround the PIN-diode 
evenly with a certain angle to the LED axis. A schematic sketch is shown in \FigureRef{fig:AngularMeasurementSetup} and in \FigureRef{fig:LEDDiffuser}.
% TODO insert sketch used for testing.
Such a geometry assume that the distance between the LEDs and the fibers would be the same for all fibers in order for 
the fibers to pick up the same amount of light.
  
\begin{figure}
\centering
\includegraphics[width=.7\textwidth]{LUCID/LED_diffuser_schematic.pdf}
\caption{Schematic of the experimental setup which was used in the LED diffuser design phase. LED and PIN-diode are aligned and face each other. 
	 Angles represent possible positions of the fibers around the PIN-diode. Fibers are not shown in the sketch.}
\label{fig:AngularMeasurementSetup}
\end{figure}

\begin{figure}
\begin{subfigure}{.48\textwidth}
  \centering
  \includegraphics[width=\textwidth]{LUCID/LEDdiffuser_rightPart.png}
\end{subfigure}
\begin{subfigure}{.48\textwidth}
  \centering
  \includegraphics[width=.8\textwidth]{LUCID/photo_LED_diffuser_v2.jpg}
\end{subfigure}

\caption{The LED diffuser. Schematic drawing (left), photo (right).}
\label{fig:LEDDiffuser}
\end{figure}
  
% As described above main purpose of LED diffuser is to evenly distribute light between PMT-channels. 

% consists from LED itself, set of fibers which deliver light to the PMTs and PIN-diode which is used to monitor
% stability of LED during long time. LED diffuser has to be design in such way, that all PMT channels have to receive the same amount of light.

% TODO define what is LED axis. Try to explain it better.
The manufacturer provided information about LED light intensity as a function of the angle between an observer and the LED axis 
(\FigureRef{fig:AngularDistributionOfLED} (right)).
% But in order to make sure that real LEDs correspond to specified characteristics a test in the lab was done.
A set of measurements was also done.
A sketch of the experimental setup is shown in \FigureRef{fig:AngularMeasurementSetup}.
Measurements of the PMT anode current as a function of fiber angle with respect to LED axis for different distances between 
the fiber surface (PIN-diode) and the LEDs were done and the results are shown in \FigureRef{fig:AngularDistributionOfLED} (left).
With angles below to 30\degree the intensity is relatively homogeneous and above 30\degree the light intensity startsg to drop off.

\begin{figure}
\begin{subfigure}{.46\textwidth}
  \centering
  \includegraphics[width=\textwidth]{LUCID/current_vs_angle.pdf}
\end{subfigure}
\begin{subfigure}{.51\textwidth}
  \centering
  \includegraphics[width=\textwidth]{LUCID/LED_radial_intensity_chart.pdf}
\end{subfigure}

\caption{Angular homogeneity of LED. Measurements of PMT anode current as a function of angle between fiber and LED axis for different distances between fiber and LED are shown in the left.
	 Stated by manufacturer angular homogeneity is shown on the right.}
\label{fig:AngularDistributionOfLED}
\end{figure}

The sensitivity of the PIN-diodes is significantly smaller than that of the PMTs.
That's why, in order to have enough light intensity to see a clear signal with PIN-diodes, one needs as small as possible
distance between the LEDs and the PIN-diode. However, since the fibers sit around the PIN-diode case and due to geometrical constrains 
(the diffuser has certain space requirements in order to fit in the limited space inside the shielding) 
the angle between the fibers and the LEDs - PIN-diode axis cannot be very large because then
the fibers have to be bent too much which can damage them during the detector installation.

The final design had a 6 mm distance between the PIN-diode and the LEDs and a 30\degree fiber angle.

The different sensitivity of the PMT and the PIN-diode led to another limitation caused by the dynamic range of the LUCROD board.
% TODO I am not sure if this sentence is correct... Probably Vincent had something else in mind.
The dynamic range is the maximum possible input voltage (after amplification) the read-out card has an output is linear and thus not saturating.
The dynamic range of the LUCROD card is 1.5 V. 
The input signal are amplified with a low noise amplifier with an amplification factor of 14, as described in \SectionRef{subsec:LUCIDElectronics} 

The maximum possible amplitude of the PMT signals which can be handled by the electronics without any saturation is therefore slightly above 100 mV.
This introduce limitation for the diffuser because the intensity of the LED cannot be too high so that it produces larger than 100 mV PMT signals.
However, for this LED intensity the signal from the PIN-diode will be very small and barely measurable.
It was therefore needed to suppress the amount of light which goes to the fibers while keeping high intensity for the PIN-diode.
The solution was to place a ring of optical filters which covers the fibers but not the PIN-diode.

Two sets of filters with the optical densities 0.15 and 0.6 were used. 
% TODO ``lab'' is slang. is laboratory OK instead???
Measurements with many possible combinations of filters were done in the laboratory.
The setup was the same as showed in \FigureRef{fig:AngularMeasurementSetup}. The distance between the LED and the PIN-diode 
and the fiber angle were set to the values 
decided to be used in the diffuser. 
Filter were inserted between the LED and the fiber to reduce the amount of light picked up by the fiber.
Measurements were repeated three times for each filter configuration. The results of the measurements are showed in \TableRef{tab:FilterChoice}.
It was decided to choose a combination with one filter with 0.15 optical density and one with 0.6 which gave a PMT signal amplitude of 85.33 $\pm$ 0.29 mV.
This amplitude is slightly smaller than the threshold value of 100 mV in order to have a safety margin in case fibers 
are slightly off from the nominal position in the diffuser.


\begin{table}[bp]
  \begin{tabular}{l|c}
    Filter configuration & Signal amplitude [mV]\\
    \hline
    2x0.15       	&	147.90	$\pm$	1.05	\\
    3x0.15       	&	129.67	$\pm$	0.72	\\
    4x0.15       	&	106.47	$\pm$	1.77	\\
    1x0.6          	&	109.93	$\pm$	0.12	\\
    1x0.6 + 1x0.15 	&	85.33	$\pm$	0.29	\\
    1x0.6 + 2x0.15 	&	61.67	$\pm$	0.62	\\
    1x0.6 + 3x0.15 	&	41.83	$\pm$	0.35	\\
    2x0.6	        &	30.13	$\pm$	0.41	\\
  \end{tabular}
  \caption{}
  \label{tab:FilterChoice}
\end{table}

The LED diffuser had to be placed on top of the so-called shielding monoblock, where radiation levels are expected to be low compared to the ares 
close to the beampipe where LUCID sits.
However, significant amount of radiation will be present also in the location of the LED diffusers during operation.
No estimation of the radiation hardness of the filters were done which is why it was decided to make one diffuser with filters and another one without.

% \begin{figure}
% \centering
% \includegraphics[width=.7\textwidth]{LUCID/LED_diffuser_schematic.pdf}
% \caption{Schematic drawing of LED diffuser.}
% \label{fig:LEDDiffuser}
% \end{figure}

\subsection{The laser diffuser}
\label{subsec:laserDiffuser}

% TODO Read here: https://www.rp-photonics.com/numerical_aperture.html

The stability of the light source used for PMT gain monitoring is a crucial factor in the calibration procedure.
Instead of relying on only one light source from LEDs it was decided to also use laser light
provided and monitored by the Tile calibration system. 
To distribute the light between the PMTs one cannot use the same diffuser as for LED light, 
since laser light is very well collimated which is not the case for the LEDs.
In order to handle laser light a new diffuser was made which is shown in \FigureRef{fig:laserDiffuserSchematics}.
% TODO personaly I don't like this phrasing... because it sound that diffuser is just air gap... but in my opinion diffuser is air gap + ferryl connector 
% with 48 fibers
The diffuser connects a fiber bundle of 48 quartz fibers encased in ferrule connector of 2 mm diameter with a signle fiber with 0.6 mm diameter
since the laser light is delivered from the Tile calibration system by single fiber.



\begin{figure}
\centering
\includegraphics[width=.6\textwidth]{LUCID/laserDiffuserAirGap.pdf}
% TODO need to move top text from picture to the main text
\caption{Schematic drawing of a laser diffuser that couples a fiber bundle to a single quartz fiber which delivers laser light from the Tile calibration system.}
\label{fig:laserDiffuserSchematics}
\end{figure}

\begin{figure}
\centering
\begin{subfigure}{.45\textwidth}
  \centering
  \includegraphics[width=0.9\linewidth]{LUCID/mapping_bundle1_color.png}
\end{subfigure}%
\begin{subfigure}{.45\textwidth}
  \centering
  \includegraphics[width=0.9\linewidth]{LUCID/mapping_bundle2_color.png}
\end{subfigure}
\caption{(left) Picture of a laser diffuser with the fiber buncdle which delivers laser signals to PMTs on the side A detector. 
Numbers correspond to PMT numbers which the fiber is connected to. 
Two fibers from a bundle are connected to each PMT. 
Fiber pairs are divided into three categories based on the distance from the center of the connector to the closest fiber 
in a pair: central (orange color), intermediate (yellow) and peripheral fiber pair (green).
(right) Laser diffuser which correspond to the side C detector.}
\label{fig:laserDiffuserMapping}
\end{figure}

Light in the fiber undergos multiple total internal reflections in the interface between the fiber core and the cladding. 
Due to this, the light from the fiber will come out within a certain cone. 
The size of the cone is characterized by the numerical aperture $NA$ of the fiber which is given by

\begin{equation}
\label{eq:numericalApperture}
 NA = n \sin{\theta_{max}} = \sqrt{n_{core}^2 - n_{cladding}^2}
\end{equation}

where $n$ is the refractive index of the medium (air in our case), $n_{core}$ is the refractive index of the fiber core, $n_{cladding}$ is the refractive index 
of the cladding and $\theta_{max}$ is half-angle of light cone. 
% TODO Vincent suggest to remove this sentence... need to rethink is it worth to let it be.
% That's why light spot produced by light from fiber will become bigger with distance from fiber edge. 
To distribute the light from the single fiber to the fiber bundle one need to introduce some air gap between them that depends on $NA$ 
as shown at \FigureRef{fig:laserDiffuserSchematics}.
The fiber used to deliver the laser light has a numerical aperture of 0.22, which corresponds to a 3.1 mm air gap distance 
in order to cover the 2 mm surface of the diffuser by a light spot (according to \EquationRef{eq:numericalApperture}).
The light received by each PMT have to be similar for all photomultipliers.
However, the intensity of light is not constant within the light spot and dedicated measurements were needed to determine the optimal air gap distance.
Two conditions had to be met:
\begin{itemize}
 \item The light has to be evenly distributed between PMTs,
%  TODO rephrase ``dimmest channel'', see comments bny Vincent
 \item A preference is given for a configuration in which the dimmest channel receives a maximum amount of light.
\end{itemize}
% TODO is this sentence correct???
Two fibers in each bundle goes to each PMT. 
For small distances between the fiber with the laser signal and the fiber bundle one expected the biggest intensity for teh central 
region of the diffuser, while for the peripheral region one expects the lowest intensity.
The diffuser is divided into three regions: central, intermediate and peripheral. 
If one of the channel fibers from pair is in the central or intermediate region, another fiber from the pair was placed in the most peripheral ring of the diffuser.
If both fibers in a pair are in the peripheral region they were placed as close as possible to the center of the diffuser.
Pairs which include one fiber in the central region are called central fibers and marked in orange color in \FigureRef{fig:laserDiffuserMapping}.
Pairs with one fiber in the intermediate region are marked with yellow and pairs marked with green color have both fibers in the peripheral region.



Measurement were done with different air gap distances for fiber pairs from each category. 
Due to the identical structure of the fiber bundle on side A and side C (or bundle 1 and 2), detailed measurements
were done only for one bundle and only a few points were measured with the second bundle to verify the results of the first one. 
The amplitude of the PMT signals were measured by the oscilloscope in the ATLAS experimental hall
and the results are shown in \FigureRef{fig:laserDiffuserDistanceTest}. 
PMT 1 represent PMTs with a central fiber pair, PMT 7 an intermediate pair and PMT 16 and 19 peripheral fibers for bundle 1 and 2. 
With increasing distance the signal amplitude is decreasing for central fibers as expected. 
% TODO Vincent suggest to remove this sentence... but probably he said to not later - have to check comments in my notebook...
Intermediate and peripheral fibers has maximums. Maximum for peripheral fibers correspond to the distance of 4 cm.
Homogeneity within all categories is acceptable at 4 cm, so this distance was chosen for the final version of the diffuser.
% TODO conclusions above are very short... Make it longer!!!

% TODO describe some parameters of Tile system with reference to the atlasGeneral


\begin{figure}
\centering
\includegraphics[width=.6\textwidth]{LUCID/laserDiffuserDistanceTest.pdf}
\caption{Measurement of the PMT signal amplitude as a function of air gap distance for different categories of fibers. Fiber pairs connected to PMTs 1 represent 
central fiber pairs, PMT 7 intermediate pair, PMTs 16 and 19 to peripheral pairs.}
\label{fig:laserDiffuserDistanceTest}
\end{figure}


\subsection{PMTs with Bi-207 source}
\label{subsec:bi207Calibration}

To use LED or laser light sources as a reference for the PMT gain monitoring system one has to be sure that intensity of the light delivered to the PMT 
will stay constant over long period of time. In other words one has to be sure that the light source itself is stable and condition of 
the optical fibers which deliver light to the PMTs stays constant.
Alternative option which is robust againt mentioned effects above is to use a radioactive source.

It was decided to use Bi-207 source because it provides monoenergetic electrons from an internal conversion process 
with energies above the Cherenkov threshold in quartz.
In order to use Bi-207 for gain monitoring one needs to put the radioactive source close to the PMT.
A set of measurements with Bi-207 radioactive sources were done in a test set-up and it was found that peak from the electron source 
(shown in \FigureRef{fig:pulseheight_Bi207}) corresponds to approximately 30 photo electrons~\cite{Alberghi:2016tad}.

\begin{figure}
\centering
\includegraphics[width=.6\textwidth]{LUCID/Bi_amplitude_distribtion.png}
\caption{Pulseheight distribution of Bi-207 radioactive source signals measured in a test with a Hamamatsu R760 PMT~\cite{Alberghi:2016tad}.}
\label{fig:pulseheight_Bi207}
\end{figure}

% TODO new text: cross check grammar
A radioactive source was enclosed in a circular case with 25 mm diameter 
(the source itself was enclosed in the disc with a diameter of 5 mm) as shown in \FigureRef{fig:Bi207_case}.
During the measurements the source was put in contact with the PMT quartz window.
Size of the source is bigger than the size of the PMT quartz window (25 mm and 10 mm respectively) 
which led to an observed dependence of the shape of the charge and pulseheight of distributions (such as in \FigureRef{fig:pulseheight_Bi207}) 
depending on the relative position of the source with respect to the center of PMT quartz window.
It leads to possible difficulties to position sources in the same way for all sets of PMTs main problem arise from the 
Despite this issue the main problem with using Bi-207 source was 2.5 times bigger size of the source itself than the size of quartz windoe.
It was decided to use liquid Bi-207 source and put it on the whole surface of the quartz window which will eliminate geometrical factor as well.
In order to prevent leaking of the source a special cup was glued on top of the PMT.
 
\begin{figure}
\centering
\includegraphics[width=.6\textwidth]{LUCID/Bi207_CERN_upperPart.png}
\caption{Schematic picture the of case of a Bi-207 radioactive source used in the measurements during the characterization of the PMTs~\cite{Alberghi:2016tad}.}
\label{fig:Bi207_case}
\end{figure}



\subsection{Calibration strategy during 2015-2016}
\label{subsec:calibPerformance}

LED and laser calibrations suffered from the two following effects:
the stability of the light source; the wavelength of both LED and laser were different from the wavelength of Cherenkov radiation. 
The quantum efficiency of the cathode depends on the wavelength and this efficiency can potentially change with the PMT aging
differently for a different wavelenght.
% TODO not clear word --> check with Vincent
Bi-207 calibrations doesn't suffer from these two effects and it was the best in 2015.

The purpose of the calibration is to keep the gain of the PMT constant with time.
% TODO rewrite (see comments from Vincent, bottom of page 16)
Due to the fact that luminosity during 2015-2016 are higher compared to 2011-2012
aging effect become visible already after a few hours of the beam collisions.
% end rewriting here
The strategy was to make calibration runs after each physics run.
After the calibration run the truncated mean pf the charge distribution from the Bi-207 sources were measured for each PMT.
These values directly correlate with the gain of the PMTs. 
In order to keep the gain of the PMT constant it is enough to keep the truncated mean of the charge distribution constant.
After each calibration run the mean for each tube is compared to the reference values and if there is a significant difference a program 
adjusts the high voltage in order to correct the PMT gain.

This procedure was used for 2015 and 2016 and was shown to be highly effective and robust.

The LED calibrations were not as succesful in keeping the gain constant.

% TODO write this explanation
[ToASK: should I try to explain the reason why LED calibration doesn't work?].

The laser calibrations were not studied much since the laser source is controlled by the Tile subsystem and require 
coordination with the Tile sub-detector group to share this resource between the two detectors.

[ToDO: add trending plot of the mean truncated charge over the time for Bi-207 tubes].

% TODO: explain why Bi-207 calibration is preferred calibration method and only more or less understood.

% TODO 
% describe how we did Bi-207 calibration during 2015-2016 years
% attempt to do LED calibration
% situation with Laser calibration


% \section{Detector performance}
% \label{sec:DetPerf}
% 
% *** some text here ***
% 
% \subsection{Luminosity measurements}
% \label{subsec:lumMeas}


\section{Temperature dependence of the PMT gain and the temperature tolerance of calibration fibers}
\label{sec:tempMeas}

During the first long LHC shutdown (LS1) between 2013 and 2015 the beampipe at the interaction point of ATLAS
had to be removed and to be replaced with a new one made of aluminum instead of stainless steel.
The new material was chosen to minimize the induced radioactivity of the beampipe (see \FigureRef{fig:metalSupportAndTempProbes} (left)).
The new beampipe also had a reduced aperture in the inner detector region to allow for a new pixel inner barrel layer (IBL) of the inner tracker.

\begin{figure}
\centering
\begin{subfigure}{.6\textwidth}
  \centering
  \includegraphics[width=0.95\textwidth]{LUCID/PMT_metal_support.jpg}
\end{subfigure}%
\begin{subfigure}{.4\textwidth}
  \centering
  \includegraphics[width=0.95\linewidth]{LUCID/temperatureProbes_1_v2.jpg}
\end{subfigure}
\caption{Metal support for the PMT tubes (left). Positions of the temperature probes close to the beam pipe flange (right).}
\label{fig:metalSupportAndTempProbes}
\end{figure}


In order to remove residual gas molecules from the inner walls of the new beampipe (which otherwise will lead to significant beam-gas interactions during operation)
the beampipe had to undergo a so called ``bake-out'' process, which consists of heating up the walls of the beam pipe from the outside.

Since LUCID sits close to the beampipe there was a need to understand the temperature tolerance of the detector components.
During the design phase a special cooling of the detector was made in order to protect PMTs, cables and calibration fibers against potential overheating.
The PMTs were placed on a special metal support which were cooled down by water pipes.
Fibers and cables goes along the beampipe for a few meters and were more exposed to heat than PMTs.
% TODO redo a little bit following sentence
They are also protected with cooling pipes but it was practically impossible to provide cooling along the whole path of the fibers 
and so there are some areas were fibers are not cooled down by the water pipes.

Special studies was done in order to understand the temperature tolerance of the quartz fibers used in the detector and ths is described in the current section.

% TODO Describe temperate probes... probably somewhere before!!!

Another temperature test described in this section was focused on understanding the temperature dependence of the PMTs gain during their operation.



% In current section measurements of temperature dependence of PMT gain and behaviour of quartz fiber at high temperatures will be discussed.

% TODO ask Davide how freuqent we archive temperature from sensors.

% Another test described in this section was focused on understanding how big 
% temperature can quartz fiber, used to provide LED and laser calibration signals, 
% tolerate during the bake-out procedure.

% TODO insert references to the sub-sections here?


% [TODO: describe scheme of temperature sensonrs installed in LUCID].

\subsection{Temperature controller}
\label{subsec:tempController}

In order to perform any temperature tests one need to have a reliable method to measure and control temperature in the testing area.
Measurements with PMTs are typically done in a light tight black box to make sure that no external light will accidentally get on the PMT photo-cathode.
The black box is well sealed which prevent air from the box to circulate outside which make it easy to control temperature inside the box with good 
precision.
A dedicated temperature controller, based on the Arduino Mega 2560 microcontroller~\cite{arduino} (programmed by Arduino software) was built.
To measure temperature the LM35CAH sensor was used which is a precision integrated-circuit device with an output voltage linearly-proportional to the 
temperature C\degree.
In order to interface the sensor with the readout input channel from the Arduino board an electrical circuit 
(see \FigureRef{fig:tempReadOutCircuit}) has been made, 
in order to match the output voltage from the sensor to the readout of the Arduino board.
As a heater element a simple chain of resistors were used which were dissipating heat produced by the current flowing thorough them. 
The voltage on the resistors was controlled by a controller with help of a MOSFET transistor. 
In order to verify that the temperature will be homogeneous within the black box,
a pc fan was used. A heating profile was programmed in the Arduino microcontroller and 
the measured temperature values were sent directly to a personal computer by a serial port or were stored on an external microSD card 
that was connected to the Arduino board.
 
 \begin{figure}
\centering
\includegraphics[width=.95\textwidth]{LUCID/LM35_TempSensor_cutted.png}
\caption{Electrical circuit of connection temperature sensor LM35CAH to the Arduino read-out.}
\label{fig:tempReadOutCircuit}
\end{figure}

 
\subsection{The PMT gain dependence}
\label{subsec:pmtGainTempDep}

It is known that a PMT is more sensitive to ambient temperature than ordinary 
electronic components (such as capacitors and resistors)~\cite{hamamatsu}.
It is caused mainly by two factors: the cathode quantum efficiency is sensitive to 
temperature variations and gain of the dynode chain depends on the temperature as well.

To estimate the temperature effect on the PMT gain a dedicated measurement was done. 
A R760 Hamamatsu PMT was placed in the black box together with a Bi-207 radioactive source
to provide stable input signals over time. The temperature in the box was controlled by the temperature controller described in 
\SectionRef{subsec:tempController}.

The interior of a PMT is at vacuum and heat conducts through it very slowly temperature gradient has to be very small in order to make sure 
that the temperature of the tube reaches the same level as the ambient (measured) temperature.
In order to satisfy this condition the temperature gradient was chosen to be 0.2\degree~C per hour.

% TODO link. Also put some information about v1720 (probably from Vincents note).
% TODO explain what is the difference between this one and the one which is used in final verison of LUCRODs.
% TODO probably make it not here.
The PMT signal was digitized by a 12 bit VME Flash ADC (FADC) which was operated by the ATLAS central Trigger and Data Acquisition (TDAQ) framework.
Ir order to record only signals from the Bi-207 source, a triggering was done that require a signal above a certain threshold.
The digitized pulse height of every triggered event was stored in ROOT files.
The shape of the digitized pulses from a Bi-207 source is shown at \FigureRef{fig:bi207DigitizedPulse}. 

\begin{figure}
\centering
\includegraphics[width=.7\textwidth]{LUCID/rndmPulse_1_mod.pdf}
\caption{Shape of a digitized pulse from a Bi-207 source. For pulseheight measurement the baseline has to be measured and subtracted in a region without signals.}
\label{fig:bi207DigitizedPulse}
\end{figure}

In order to calculate the charge and the amplitude of the Bi-207 signal, the baseline had to be subtracted. 
To estimate the baseline value the last 30 (out of 80) FADC samples were used as shown in \FigureRef{fig:bi207DigitizedPulse}.

The recorded rate of the Bi-207 signals was around 150-200 Hz. In order to collect enough statistics to measure the mean of the charge and 
the amplitude distributions every measurement was done for 5 minutes. 
After one measurements was finished another one was immediately started.
Due to the very small temperature gradient (0.2\degree~C per hour) the temperature within one measurement was considered to be stable.

The mean of the charge distribution and the temperature as a function of time is shown in \FigureRef{fig:PMTChargeTempDep}.
Black points represent the measurements of the ambient temperature. The temperature was slowly increasing and the total change was 6\degree~C over 30 hours.
Red points shows the mean of the charge distribution in 5 minute intervals.
A clear decreasing trend is observed which represent the temperature dependence of the PMT gain.
The measured temperature dependence was 0.25 $\%$ of gain per 1\degree~C. 




\begin{figure}
\centering
\includegraphics[width=.7\textwidth]{LUCID/goodSlowTemp_Feb20_Feb22_charge.eps}
\caption{The temperature dependence of the PMT gain represented by measurements of the mean of the charge distribution the Bi-207 source signals 
for different temperature values.
Black markers correspond to the temperature measurement; red markers correspond to mean of charge distribution of Bi-207 signals collected for 5 minutes.}
\label{fig:PMTChargeTempDep}
\end{figure}


\subsection{Bake-out tests of the calibration fibers}

% From https://indico.cern.ch/event/315665/contributions/1690160/attachments/605816/833747/LS2_LHC_paper_rev_final-1.pdf
% 
% In 2015 year during first long shutdown (LS1) a central ATLAS beampipe made from Ferum (???) 
% was changed with a new one made from Beryllium and Aluminium to better deal with material activation and background ~\ref{Baglin:1967027}.
% New beampipe has to be purified in order to prevent dust particles appear inside the pipe.
% Beampipe is heated up to 350 \degree~C for a few hours. This procedure is called beampipe bakeout.
% 
% The LUCID detector sits on supporting structure (gray structure in \FigureRef{???})
% 
% dedicated cooling has to be installed to protect PMTs and calibration fibers.
% To understand how efficient cooling is needed 
% one need to know temperature robustness of detector components.
% % TODO what about temperature robustness of PMTs? do we know level at which they start to destroy? why we only did this test for fibers?
% % Becuase used PMTs in detector are quite small it makes them quite easy to cool down. 
% Calibration fibers goes from PMTs close to beampipe all along to LED and laser diffusers which are placed on top on monoblock structure 
% %  TODO some other picture???
% (see \FigureRef{fig:LucidDrawing} (left)).
% That's why fibers can be potentially overheated at some parts. 
% In order to understand which temperaure fibers can handle dedicated measurement was done. 


The goal of this test was to estimate the temperature threshold at which fibers started to loose their optical properties
due to heat damage of the cladding and/or the fiber core.

The experimental setup consisted of a PMT placed in the black box, a LED source and a fiber bundle.
Light was transmitted from the LED to the PMT by the fiber.
The fiber bundle was placed in the thermoinsulated box, while the LED and the PMT were outside and not heated.

A high ambient temperature was obtained inside the box with the help of a thermogun, which was blowing hot air into the box constantly.
The temperature was measured with the Arduino-based temperature controller described in \SectionRef{subsec:tempController}, with the temperature probe inside the box.
% TODO do we really use Arduino-based temperature controller in this test?

The first set of measurements were done at room temperature inside the box in order to use this value as a reference (first phase of the experiment).
In the second phase the thermogun was switched on and the temperature raised to around 95\degree~C.
This temperature was kept constant for slightly more than one hour to make sure that the fiber was exposed to a high temperature for
a long enough time period to make any changes observable. 
In the third phase the power of the thermogun was increased further and the temperature of the air was raised to 110\degree~C and kept at this level for one hour.
In the last phase the thermogun was switched off and the measurements were done for another half an hour.
% TODO find out which rate of LED was used in the test
The LED was pulsed by pulse generator with a rate of ??? kHz and the FADC was triggered 
with the same signal in order to make sure that only signals originating from the LED source were stored.
Every measurement was done for a 5 minute interval and the next one was started as soon as the last one had finished.
The mean of the charge distribution was calculated for each measurement and is shown as function of time with red points in \FigureRef{fig:fiberBakeOut}.
Black points correspond to the temperature measurements done during the same time.
% TODO see orange question sign on page 22 on first Vincent comments
The charge measurements during the second phase (when the temperature in the box was 95\degree~C) are compatible with measurements done during the first phase 
(at room temperature) and they are stable during all of phase two (which was one hour long) which shows that fibers can operate normally at this temperature.
During the third phase (110\degree~C) a significant decrease of the measured charge is observed, which demonstrates a change in the
optical properties of the fiber. In the last phase the temperature went back to room temperature and the charge increased but didn't reach 
nominal values which indicate a unrecoverable damage of the fiber at the 110\degree~C temperature.

\begin{figure}
\centering
\includegraphics[width=.95\textwidth]{LUCID/fiberBakeOut.png}
\caption{Measurements of the mean of the charge distribution of LED signals which goes through a heated fiber bundle.
Black points correspond to the temperature of the fiber bundle; red points correspond to the mean of the charge collected in 5 minute intervals.}
\label{fig:fiberBakeOut}
\end{figure}

% TODO rephrase a little bit here...
This test demonstrated that the calibration fibers could be damaged if they will be exposed 
to temperatures above 95\degree~C for long period of time.

% TODO probably it's better to remove this...
% But due to fact that bake-out procedure was planned to be done with around 220\degree~C and early
% simulations showed that expected temperature on LUCID area can be around 140\degree~C 
% % TODO find proof of this!!!
% efficient cooling for fibers became mandatory requirement for the system.
% 
% Decision was to use water cooling. 
% To make cooling of PMTs efficient dedicated metal support was designed to host PMTs shown in \FigureRef{fig:PMTMetalSupport}.
% Water pipes were positioned under this support.
% To protect calibration fibers from overheating they were bundled and glued around cooling pipe which was placed along 
% the beampipe (see \FigureRef{fig:temperatureProbes_VJCone}).
% 
% Also to provide circulation of air in the detector region additional air cooling was installed. 
% [according to Giulio air circulation was essential in order to make sure that some gas will not get into PMTs???]

 
\subsection{Temperature conditions during the beampipe bake-out and detector operational period}

The LUCID detector was equipped with 18 temperature sensors per side which were installed (\FigureRef{fig:metalSupportAndTempProbes} right)
to monitor the temperature of the detector components during the bake-out procedure of the new ATLAS beampipe and during the detector operation.
Temperature from all sensors were recorded and archived every 10 seconds.

The temperature from the sensor placed in the cable tray and at the beam pipe flange during the beampipe bake-out is shown in \FigureRef{fig:NEG_ctflangeA}.
The temperature were well within the safety margin and didn't exceed 40\degree~C. The temperature from other sensors didn't exceed 40\degree~C as well.
It demonstrated the high efficiency of the beampipe insulation and the LUCID cooling system and no damage to the LUCID detector components were done.

\begin{figure}
\centering
\includegraphics[width=.9\textwidth]{LUCID/NEG_ctflangeA.png}
\caption{The temperature reading from the temperature probes during the bake-out procedure.}
\label{fig:NEG_ctflangeA}
\end{figure}
 

% Temperature of the sensors during data-taling:
After the bake-out procedure had finished the temperature was constantly monitored to make sure that there was no large fluctuations in the cavern.
A few sensors were installed close to the PMTs and their measurements can be treated as an approximate temperature of the PMT tubes.
% TODO insert picture mentioned in the sentence below!
The temperature during the data-taking period in 2015 and 2016 years are shown in \FigureRef{fig:???}. 
There are no significant fluctuation over the entire period and they are all within ?\degree~C.
According to the measurements described in \SectionRef{subsec:pmtGainTempDep} the PMT gain dependence in temperature correspond
to 0.25 $\%$ per 1\degree~C, which is negligible for the observed temperature variations during the data-taking period.



% TODO describe:
% - water cooling - see in the figure
% - additional air cooling by blowing air with fans to make circulation of air around lucid detecor
% - temperature probes installed - add picture with probes
% - temperature plots during bakeout procedure

\begin{figure}
\centering
\includegraphics[width=.9\textwidth]{LUCID/temperatureProbes_2.jpg}
\caption{Positions of temperature probes along carbon support cone.}
\label{fig:temperatureProbes_VJCone}
\end{figure}



% TODO insert picture which show that LUCID sits on VJ cone around beampipe


 

\section{The LUCID luminosity measurements}

[WARNING: this section is work in the progress].

The luminosity is measured by LUCID from a measurement of the number of PMT-hits, the number of bunch crossings 
with at least one PMT-hit and the integrated pulseheight (charge). These measurements are done over a time 
period called a luminosity block, which are typically 1 minute long and they are done for each of the 
individual bunch crossings in the LHC. 
The new electronics provides luminosity measurements using 124 different algorithms which take as input 
different combination of hits or charge from different tubes. Algorithms, which are based on PMT-hits from only 
one of the detectors (either A or C), are calculated by the LUCROD VME custom made boards, while algorithms which depend on 
a combination of hits from both detectors are calculated by the LUMAT boards (see 
\FigureRef{fig:Eletronics_schematics}).
The luminosity is proportional to the measured charge and to the logarithm of the measured 
% TODO *** add log formula for lumi ***
number of PMT-hits *** add log formula for lumi ***. The two types of measurements therefore have different limitations. 
The main issue with the 
charge measurement is PMT gain stability while the hit measurements can suffer from pile-up of several signals 
below threshold giving a signal above threshold.
% Figure~\ref{fig:hitCount} shows the number of PMT-hits from different LHC bunches. The large peaks correspond 
% to six trains of 
% six colliding bunches each, plus two isolated colliding bunches. Two smaller peaks that correspond to bunches 
% with only one beam are also seen. The baseline background level is due to the Bi-207 source used for monitoring 
% of the photomultiplier gain. 


\subsection{Luminosity algorithms}



\subsection{The first 13 TeV collisions at the LHC}
\label{sec:physics}

% TODO Vincent comment. See orange question mark in comments. Probalby one has to put reference for plots?
The LHC reported the first 13 TeV pp collision in May of 2015~\cite{publicPlots} and these were recorded by ATLAS and other LHC 
experiments. 
Starting from that time the new LUCID was succesfully operating and provided information about the 
luminosity delivered to ATLAS. 
% More than a femtobarn of luminosity is already recorded by ATLAS, which provides a
% sufficient amount of data about the performace of the LUCID detector with a high number of pp-interaction per bunch 
% crossing.

The PMT pulseheight distribution in a physics run is shown in the left plot of \FigureRef{fig:Pulseheight} (blue) 
together with the same distribution 
during a Bi-207 calibration run (red). In both distributions a peak due to Cherenkov photons is visible. The 
calibration distribution is cut due to the threshold in the electronics that define a PMT-hit.

\begin{figure}
\centering
\begin{subfigure}{.5\textwidth}
  \centering
  \includegraphics[width=\linewidth]{LUCID/LowMuWithBi_ampl_preliminary_v2.pdf}
  \label{fig:sub1}
\end{subfigure}%
\begin{subfigure}{.5\textwidth}
  \centering
  \includegraphics[width=\linewidth]{LUCID/ComparisonPhysRuns_ampl_preliminary_v2.pdf}
  \label{fig:sub2}
\end{subfigure}
\caption{Comparison of pulseheight distribution in a physics run with low-$\mu$ with the same distributions 
during a Bi-207 calibration run (left) and with distributions during a high-$\mu$ physics run.}
\label{fig:Pulseheight}
\end{figure}

In the right plot of \FigureRef{fig:Pulseheight} a comparison of the pulseheight distributions in a physics run 
with
low-$\mu$ (red) and high-$\mu$ (blue) are shown. The pulse height is shifted towards higher values when at high 
luminosity several particles traverse the photomultiplier window in the same bunch crossing.
A second peak correspond to events with two particles going through the PMT window is clearly seen.

LUCID can measure luminosity in many ways and \FigureRef{fig:InternalConsistency} shows a comparison of the 
luminosity measured by the A and C detector in different ATLAS data taking runs. The two measurements agree to better than 0.5$\%$.

\begin{figure}
\centering
\includegraphics[width=.7\textwidth]{LUCID/InternalConsistency_preliminary.pdf}
\caption{Fractional difference in measured luminosity between the forward (A) and backward (C) arms of the LUCID 
detector. The agreement between the two LUCID arms is better than 1$\%$.}
\label{fig:InternalConsistency}
\end{figure}

The left plot of \FigureRef{fig:LumiVsTime} shows a measurement of the average number of inelastic pp collisions 
using different ATLAS 
luminometers and the right plot shows the ratio of this measurement with respect to a LUCID measurement. One of the 
detectors shows a deviation of up to 2$\%$ during this LHC fill but the other measurements are all in agreement 
with LUCID to better than 0.5$\%$.
The first month of data taking with the new detector therefore showed that LUCID could measure the relative 
luminosity with a precision of about 0.5$\%$.

\begin{figure}
\centering
\begin{subfigure}{.5\textwidth}
  \centering
  \includegraphics[width=\linewidth]{LUCID/LumiVsTime_preliminary_v2.pdf}
  \label{fig:sub3}
\end{subfigure}%
\begin{subfigure}{.5\textwidth}
  \centering
  \includegraphics[width=\linewidth]{LUCID/DeviationSubsystems_preliminary_v2.pdf}
  \label{fig:sub4}
\end{subfigure}
\caption{(left) Average number of inelastic proton-proton collisions per bunch crossing during a 13 TeV fill; 
(right) Comparison of the measured luminosity by different luminometers in ATLAS with respect to LUCID.}
\label{fig:LumiVsTime}
\end{figure}

\subsection{Luminosity measurements during 2015-2016}

