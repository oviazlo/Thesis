\chapter{Wprime}
\label{chap:Wprime}

% ***Explanation of my personal contribution to the analysis.***
% TODO this is not a real text to be used - just a plan of what I want to write,,,
% \begin{itemize}
%  \item MC simultaion
%  \item second analysis of muon channel
%  \begin{itemize}
%   \item comparison of results with Magnar
%  \end{itemize}
%  \item mono-W investigation
% \end{itemize}


% PLAN
% <What this chapter about - short intro to Wprime analysis>
% <What was my personal contribution to the analysis> 


This chapter describe search for new charged boson (namely $\PWprime$) in final state with one lepton and missing transverse energy ($E_T^{miss}$).
Search was done with first $\sqrt{s}$ = 13 TeV data collected in 2015 by ATLAS with corresponded luminosity of 3.2 fb$^{-1}$.
% TODO describe more about the analysis.
% Analysis focuses on high-p$_T$ 

This analysis have been made public as a conference note~\cite{ATLAS-CONF-2015-063} with early reasults and later as a paper~\cite{Aaboud:2016zkn} where final results were presented.

My personal contribution to the analysis can be concluded in following three parts.
I performed complete analysis in muon channel. Signal selection, control and validation plots, estimation of systematics and other.
This part was done in parrallel with other collaborator to make sure that results are robust and are the same from two independent analyses and there are no mistaken done.

% TODO discuss with Monika, should I include it or not. If yes - how should I argument about why it was not use and how to explain differnce comparing to inclusive sample?
Secondly I was investigating ways to decrease systematics caused by limited statistic of inclusive diboson and ttbar backgound samples at high-$m_{T}$ region.
One option was to generate lepton-neutrino mass-binned samples. After investigation it was found out that these samples doesn't populate high-$m_{T}$ region
good enough, so it was decided to develop unique (first ever?) samples in bins of lepton-$E_T^{miss} $ $m_T$.

Third part was related to investigation of use another models which can be potentially sensitive in signal selection.
Main focus of this investigation was mono-W Dark Matter (DM) models, where pair of DM particles candidates are produced in final state in association with SM W boson.
Two sets of model were considered: simplified and Effective Field Theory models [links???]. 

% TODO discuss with Monika - how should I position analysis - as a dedicated search of Wprime or as more model-independent search?

\section{Introduction}
\label{sec:wprimeIntro}

\section{Wprime signal}
\label{sec:wprimeSignal}

\section{Background processes}
\label{sec:wprimeBackgrounds}

\section{Event Selection}
\label{sec:wprimeEventSelection}

\section{Data driven multijet background estimate}
\label{sec:wprimeMultijetBackground}

\section{Signal Region}
\label{sec:wprimeSignalRegion}

\section{Systematic Uncertainties}
\label{sec:wprimeSystematics}

\section{Conclusion}
\label{sec:wprimeConclusion}
