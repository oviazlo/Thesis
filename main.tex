%% For normal draft builds (figs undisplayed hence fast compile)
%\documentclass[hyperpdf,nobind,draft,oneside]{hepthesis}
%\documentclass[hyperpdf,nobind,draft,twoside]{hepthesis}

%% For short draft builds (breaks citations by necessity)
%\documentclass[hyperpdf,nobind,draft,hidefrontback]{hepthesis}

%%For Cambridge soft-bound version
\documentclass[hyperpdf,bindnopdf]{hepthesis}
% \documentclass[hyperpdf,nobind,draft,twoside]{hepthesis}
% \documentclass[hyperpdf,nobind,draft,oneside]{hepthesis}
%% For Cambridge hard-bound version (must be one-sided)
%\documentclass[hyperpdf,oneside]{hepthesis}

%% Load special FONT packages here if you wish
%\usepackage{lmodern}
%\usepackage{euler}

%% Put package includes etc. into preamble.tex for convenience
\usepackage{xspace}
\usepackage{tikz}
\usepackage{morefloats,afterpage}
\usepackage{mathrsfs} % script font
\usepackage{verbatim}
% \usepackage{gensymb}
\usepackage{mathpazo}
\usepackage{amsmath}
\usepackage{multirow} % for tables
% \usepackage{atlasphysics}
\usepackage{ssdefs} % defitions for inclusive Same Sign dilepton search note

%% Using Babel allows other languages to be used and mixed-in easily
%\usepackage[ngerman,english]{babel}
\usepackage[english]{babel}
\selectlanguage{english}

%% Citation system tweaks
\usepackage{cite}
% \let\@OldCite\cite
% \renewcommand{\cite}[1]{\mbox{\!\!\!\@OldCite{#1}}}

%% Maths
% TODO: rework or eliminate maybemath
\usepackage{abmath}
\DeclareRobustCommand{\mymath}[1]{\ensuremath{\maybebmsf{#1}}}
% \DeclareRobustCommand{\parenths}[1]{\mymath{\left({#1}\right)}\xspace}
% \DeclareRobustCommand{\braces}[1]{\mymath{\left\{{#1}\right\}}\xspace}
% \DeclareRobustCommand{\angles}[1]{\mymath{\left\langle{#1}\right\rangle}\xspace}
% \DeclareRobustCommand{\sqbracs}[1]{\mymath{\left[{#1}\right]}\xspace}
% \DeclareRobustCommand{\mods}[1]{\mymath{\left\lvert{#1}\right\rvert}\xspace}
% \DeclareRobustCommand{\modsq}[1]{\mymath{\mods{#1}^2}\xspace}
% \DeclareRobustCommand{\dblmods}[1]{\mymath{\left\lVert{#1}\right\rVert}\xspace}
% \DeclareRobustCommand{\expOf}[1]{\mymath{\exp{\!\parenths{#1}}}\xspace}
% \DeclareRobustCommand{\eexp}[1]{\mymath{e^{#1}}\xspace}
% \DeclareRobustCommand{\plusquad}{\mymath{\oplus}\xspace}
% \DeclareRobustCommand{\logOf}[1]{\mymath{\log\!\parenths{#1}}\xspace}
% \DeclareRobustCommand{\lnOf}[1]{\mymath{\ln\!\parenths{#1}}\xspace}
% \DeclareRobustCommand{\ofOrder}[1]{\mymath{\mathcal{O}\parenths{#1}}\xspace}
% \DeclareRobustCommand{\SOgroup}[1]{\mymath{\mathup{SO}\parenths{#1}}\xspace}
% \DeclareRobustCommand{\SUgroup}[1]{\mymath{\mathup{SU}\parenths{#1}}\xspace}
% \DeclareRobustCommand{\Ugroup}[1]{\mymath{\mathup{U}\parenths{#1}}\xspace}
% \DeclareRobustCommand{\I}[1]{\mymath{\mathrm{i}}\xspace}
% \DeclareRobustCommand{\colvector}[1]{\mymath{\begin{pmatrix}#1\end{pmatrix}}\xspace}
\DeclareRobustCommand{\Rate}{\mymath{\Gamma}\xspace}
\DeclareRobustCommand{\RateOf}[1]{\mymath{\Gamma}\parenths{#1}\xspace}

%% High-energy physics stuff
\usepackage{abhep}
\usepackage{hepnames}
\usepackage{hepunits}
\DeclareRobustCommand{\arXivCode}[1]{arXiv:#1}
\DeclareRobustCommand{\CP}{\ensuremath{\mathcal{CP}}\xspace}
\DeclareRobustCommand{\CPviolation}{\CP-violation\xspace}
\DeclareRobustCommand{\CPv}{\CPviolation}
\DeclareRobustCommand{\LHCb}{LHCb\xspace}
\DeclareRobustCommand{\ATLAS}{ATLAS\xspace}
\DeclareRobustCommand{\LHC}{LHC\xspace}
\DeclareRobustCommand{\LEP}{LEP\xspace}
\DeclareRobustCommand{\CERN}{CERN\xspace}
\DeclareRobustCommand{\bphysics}{\Pbottom-physics\xspace}
\DeclareRobustCommand{\bhadron}{\Pbottom-hadron\xspace}
\DeclareRobustCommand{\Bmeson}{\PB-meson\xspace}
\DeclareRobustCommand{\bbaryon}{\Pbottom-baryon\xspace}
\DeclareRobustCommand{\Bdecay}{\PB-decay\xspace}
\DeclareRobustCommand{\bdecay}{\Pbottom-decay\xspace}
\DeclareRobustCommand{\BToKPi}{\HepProcess{ \PB \to \PK \Ppi }\xspace}
\DeclareRobustCommand{\BToPiPi}{\HepProcess{ \PB \to \Ppi \Ppi }\xspace}
\DeclareRobustCommand{\BToKK}{\HepProcess{ \PB \to \PK \PK }\xspace}
\DeclareRobustCommand{\BToRhoPi}{\HepProcess{ \PB \to \Prho \Ppi }\xspace}
\DeclareRobustCommand{\BToRhoRho}{\HepProcess{ \PB \to \Prho \Prho }\xspace}
\DeclareRobustCommand{\X}{\thesismath{X}\xspace}
\DeclareRobustCommand{\Xbar}{\thesismath{\overline{X}}\xspace}
\DeclareRobustCommand{\Xzero}{\HepGenParticle{X}{}{0}\xspace}
\DeclareRobustCommand{\Xzerobar}{\HepGenAntiParticle{X}{}{0}\xspace}
\DeclareRobustCommand{\epluseminus}{\Ppositron\!\Pelectron\xspace}
\DeclareRobustCommand{\protonproton}{\Pproton\APantiproton\xspace}

\DeclareOldFontCommand{\sc}{\normalfont\scshape}{\@nomath\sc}

% Some custom declarations needed for pieces of text taken from internal notes...
\def\Wpm{\ensuremath{W^{\pm}}}%
\def\antibar#1{\ensuremath{#1\bar{#1}}}
% \def\tbar{\ensuremath{\bar{t}}}
\def\ttbar{\antibar{t}}

%% Fix for \bf problem in bibliography
\DeclareOldFontCommand{\bf}{\normalfont\bfseries}{\mathbf}
%% Other possible declarations to use
% \DeclareOldFontCommand{\rm}{\normalfont\rmfamily}{\mathrm}
% \DeclareOldFontCommand{\sf}{\normalfont\sffamily}{\mathsf}
% \DeclareOldFontCommand{\tt}{\normalfont\ttfamily}{\mathtt}
% \DeclareOldFontCommand{\it}{\normalfont\itshape}{\mathit}
% \DeclareOldFontCommand{\sl}{\normalfont\slshape}{\@nomath\sl}
% \DeclareOldFontCommand{\sc}{\normalfont\scshape}{\@nomath\sc}
% \DeclareRobustCommand*\cal{\@fontswitch\relax\mathcal}
% \DeclareRobustCommand*\mit{\@fontswitch\relax\mathnormal}

\def\dch{\ensuremath{H^{\pm\pm}}}
\def\dchl{\ensuremath{H^{\pm\pm}_{L}}}
\def\dchr{\ensuremath{H^{\pm\pm}_{R}}}

\def\dynnlo{{\sc dynnlo}}
\def\phozpr{{\sc phozpr}}
\def\horace{{\sc Horace}}
\def\photos{{\sc Photos}}
\def\powheg{{\sc Powheg}}
\def\powhegbox{{\sc Powheg-Box}}
\def\sanc{{\sc Sanc}}
\def\mcsanc{{\sc Mcsanc}}


\def\vrap{{\sc VRAP}}

\usepackage{caption}
\usepackage{subcaption}

\graphicspath{ {/home/oviazlo/PhD_study/THESIS/pictures/} }
\DeclareGraphicsExtensions{.eps, .pdf, .png}

%% You can set the line spacing this way
%\setallspacing{double}
%% or a section at a time like this
%\setfrontmatterspacing{double}


%% Define the thesis title and author
\title{Same-sign, $\PWprime$, LUCID and TRT}
\author{Oleksandr Viazlo}

%% Doc-specific PDF metadata
\makeatletter
\@ifpackageloaded{hyperref}{%
\hypersetup{%
  pdftitle = {Same-signa and $\PWprime$, LUCID and TRT},
  pdfsubject = {Oleksandr Viazlo's PhD thesis},
  pdfkeywords = {ATLAS, BSM, physics, LHC, LUCID, TRT},
  pdfauthor = {\textcopyright\ Oleksandr Viazlo}
}}{}
\makeatother

%% User-defined commands
\newcommand{\toAsk}[1][?]{\colorbox{red!40}{[#1]}}
\newcommand{\toFix}[1][!]{\colorbox{blue!40}{[#1]}}
\newcommand{\toDo}[1][]{\colorbox{green!40}{[TODO: #1]}}

% \linenumbers

%% Start the document
\begin{document}

% TODO this line is to remove page numbers from lines in the table of contents
% \let\Contentsline\contentsline
% \renewcommand\contentsline[3]{\Contentsline{#1}{#2}{}}


%% Define the un-numbered front matter (cover pages, rubrik and table of contents)
% \begin{frontmatter}
%   %% Title
\titlepage[of Lund Universoty]{%
  A dissertation submitted to the Lund University\\ for the degree of Doctor of Philosophy}

%% Abstract
\begin{abstract}%[\smaller \thetitle\\ \vspace*{1cm} \smaller {\theauthor}]
  %\thispagestyle{empty}
  Abstract here
\end{abstract}


%% Acknowledgements
\begin{acknowledgements}
  Of the many people who deserve thanks.
\end{acknowledgements}


%% Preface
\begin{preface}
  This thesis describes my research on various aspects of the \ATLAS
  particle physics program, centred around the \ATLAS detector and \LHC
  accelerator at \CERN in Geneva.

  \noindent
  For this example, I'll just mention \ChapterRef{chap:SomeStuff}
  and \ChapterRef{chap:MoreStuff}.
\end{preface}

%% ToC
\tableofcontents


%% Strictly optional!
\frontquote{%
  Writing in English is the most ingenious torture\\
  ever devised for sins committed in previous lives.}%
  {James Joyce}
%% I don't want a page number on the following blank page either.
\thispagestyle{empty}

% \end{frontmatter}

% TODO there is tableofcontents in frontmatter. Just temprorary
\tableofcontents


%% Start the content body of the thesis
\begin{mainmatter}
  %% Actually, more semantic chapter filenames are better, like "chap-bgtheory.tex"
  \chapter{Physics of the Standard Model and Beyond}
\label{chap:Theory}

%% Restart the numbering to make sure that this is definitely page #1!
\pagenumbering{arabic}

%% Note that the citations in this chapter use the journal and
%% arXiv keys: I used the SLAC-SPIRES online BibTeX retriever
%% to build my bibliography. There are also quite a few non-standard
%% macros, which come from my personal collection. You can have them
%% if you want, or I might get round to properly releasing them at
%% some point myself.

\section{Introduction}
% TODO add one or two sentences
Particle physics studies the elementary constituents of the universe and how they interact with each other. One of many questions which it attempts to address is the origin of our universe and what it consists of. 
Like any field of science, particle physics is based on the two pillars: experiment and theory.
By making experimental observations, one obtains certain information, and by systematizing this information and making relations between the pieces, one aims to develop a theory or a model which would explain everything.
The more different observations the model can explain, the more confidence one can have that the model works and that it can be used to predict as-yet unmeasured effects. 

This was the case with the so-called Standard Model (SM) which became extremely popular in the last century because it was able to describe hundreds of newly observed particles in collider experiments.
The SM describes with remarkable precision three types of particle interactions, 
with only the gravitational interaction left unincluded.
Thus a significant effort is ongoing to try to incorporate gravity into the SM to obtain a complete model.

However, despite the fact that the SM agrees amazingly well with experimental measurements, and even if gravity is not taken into account, 
there are many reasons to think that the SM is not complete.
There is a set of unsolved questions which the SM cannot address (e.g. neutrino masses, dark matter, matter-antimatter asymmetry).
Because all known elementary particles fit well in the SM, these problems make us believe that there is
potentially new physics (and correspondingly new particles) in TeV or above-TeV regimes which can solve these problems.
% TODO Will comment: I think either "provides a possibility" or "provides us with a possibility" are correct, but "provides us a possibility" seems not quite right to me 
The Large Hadron Collider provides us with a possibility to explore a TeV energy frontier.
This is why many analyses are focused on an investigation of new energy regime and search for possible deviations from the SM which can be hints of new physics.
Since the SM was introduced more than 30 years ago, many theorists spent a tremendous amount of time and effort to create
a plethora of models which extend the SM and address multiple unsolved problems. These models can predict signatures and criteria which are most sensitive to the possible new physics and motivate strategies of the searches.

This chapter contains a brief description of the SM, its problems and 
some Beyond Standard Model (BSM) models which have been developed to address several of these problems.

\section{The Standard Model}

\subsection{Elementary particles}

% *****************
% INTRO
% *****************

% TODO Will: You don't define "e" when you use it to talk about quark and lepton charges. Maybe talk about lepton charge first as this ~ defines it 

The SM is a very successful model which describes all the known particles in existence to a remarkable degree of accuracy.
All of the SM particles are fundamental particles (they have no internal structure) that make up the matter and forces in the universe.
The elementary particles can be classified into two groups: 
% TODO Will comment:
% - the spin-1/2 fermions which & the bosons which > "the spin-1/2 fermions, which" and "the bosons, which" - if you miss the comma, it suggests there are other fermions which do not obey Fermi-Dirac statistics 
\begin{itemize}
 \item the spin-1/2 fermions, which obey Fermi-Dirac statistics,
 \item the bosons, which obey Bose-Einstein statistics and have integer spin values.
\end{itemize}

% *****************
% FERMIONS
% *****************
The matter in the universe is made of fermions, which are classified into three generations.
Each generation consists of two electromagnetically charged quarks and one charged lepton as well as an associated neutral neutrino.
Fermions from different generations have the same charges but differ by mass.
One quark from a generation has charge +2/3$e$, while the other has -1/3$e$. Charged leptons carry an integer charge -1$e$.
% TODO Will: Technically Z measurements support the hypothesis that there are three *light* generations 
Experimental evidence supports the hypothesis that the number of fermion generations has to be three~\cite{three_lepton_generations}.
% TODO Will comment: - "Each higher generation" - you haven't mentioned any hierarchy of generations thus far 
Each higher generation consists of more heavy particles which decay to lighter particles from lower generations, and that can be
interpreted as an explanation of 
% why matter is made from the first generation particles.
why the observed stable matter in the universe is made exclusively from particles of the first generation.
% TODO Will: I would also say it's a bit more than "allows" - it's what holds hadrons together and stops electromagnetism blasting them apart 
In addition to electric charge, quarks also carry colour charge, which allows two otherwise identical quarks to jointly occupy an energy and spin state inside a hadron.

% TODO Will: - different “colours” within the SM -> I think I would say "colour charges" without quotation marks, but put quotation marks the first time you mention colour charge i.e. " "colour" charge" 
There are three different ``colours'' within the SM, namely: red, green and blue.
The number of colours has been experimentally confirmed by measurements of the 
ratio of hadronic to $\mu^+\mu^-$ production cross sections
in the electron-positron annihilation~\cite{pdg_2014}. 
Each fermion has an associated antiparticle, which is the particle with opposite charges but with identical mass and spin.
A list of fermions and their properties is presented in \TableRef{tab:fermions}.

% TODO Will Table 1.1: "nu_e" isn't really a flavour. Maybe "flavour" -> "particle"? 
\begin{table*}[h]
\begin{center}
\begin{tabular}{c||c|c|c||c|c|c}
\multirow{2}{*}{Generation} & \multicolumn{3}{c||}{Leptons} & \multicolumn{3}{c}{Quarks} \\
\cline{2-7}
 & Flavour & Mass [GeV] & Charge [$e$] &  Flavour & Mass [GeV] & Charge [$e$] \\
\hline
\multirow{2}{*}{1st} & $\nu_e$ & $<2\times10^{-9}$ & 0 & up & $2.2^{+0.6}_{-0.4}\times10^{-3}$ & 2/3 \\
 & $e$ & 0.000511 & -1 & down & $4.7^{+0.5}_{-0.4}\times10^{-3}$ & -1/3 \\
\cline{1-7} 
\multirow{2}{*}{2nd} & $\nu_{\mu}$ & $<0.00019$ & 0 & charm & $1.27\pm0.03$ & 2/3 \\
 & $\mu$ & 0.106 & -1 & strange & $96^{+8}_{-4}\times10^{-3}$ & -1/3 \\
\cline{1-7}
\multirow{2}{*}{3rd} & $\nu_{\tau}$ & $<0.0182$ & 0 & top & $174.2\pm1.4$ & 2/3 \\
 & $\tau$ & 1.777 & -1 & bottom & $4.18^{+0.04}_{-0.04}$ & -1/3 \\
\end{tabular}
\end{center}
 \caption{The fermion particle generations with their electrical charges and masses.}
\label{tab:fermions}
\end{table*}

% *****************
% BOSONS
% *****************
The remaining particles described by the SM are bosons. These particles have integer values of spin.
A list of all known elementary bosons is shown in \TableRef{tab:bosons}.
Many experimental measurements obtained with the help of colliders prove the existence of the bosons.
The first evidence of the Z and W bosons was obtained by the UA1 and UA2 collaborations~\cite{ARNISON1983103, BAGNAIA1983130}
which made the first measurements of their masses.
The gluon has been experimentally confirmed 
in electron-positron annihilation at the PETRA storage ring by observation
of the three-jet topology~\cite{three_jet_event}.
The Higgs boson was discovered by the ATLAS and CMS collaborations in 2012~\cite{Aad2012tfa, Chatrchyan2012ufa}.

\begin{table}[]
\begin{center}
{
\begin{tabular}{|l|c|c|c|c|c|c|}\cline{1-7}
 Boson & Mass &  Charge [$e$] & Spin  & Interaction & Range & Interact with \\ \hline
    photon  &  0  &  0 & 1 &  Electromagnetism & $ \infty $  &  charge \\ \hline
  8 gluons  &  0  &  0 & 1 &  Strong & $10^{-15}$~m &  colour \\ \hline 
%     $W^{\pm}$   &  80.4~GeV  &  $\pm$1 & 1 &  \multirow{2}{*}{Weak}  & \multirow{2}{*}{$10^{-18}$~m} & \multirow{2}{*}{isospin \toAsk[+hypercharge?] }  \\
    $W^{\pm}$   &  80.4~GeV  &  $\pm$1 & 1 &  \multirow{2}{*}{Weak}  & \multirow{2}{*}{$10^{-18}$~m} & weak isospin  \\
    $Z$   &  91.2~GeV  &  0 & 1 &  &  & + hypercharge \\ \hline
    Higgs   &  125~GeV  &  0 & 0 &  &  &   \\ 
\hline
\end{tabular}
}
\caption{\label{tab:bosons}The Standard Model bosons with their masses and charges, and corresponding interaction types. }
\end{center}
\end{table}




\subsection{Types of interactions and fields}
% quanta of interaction fields

The SM is a Quantum Field Theory that describes interactions between particles.
There are three fundamental forces which are incorporated into the SM framework: the electromagnetic, the weak and the strong forces. These forces are mediated between matter particles (elementary or composite) by carrier particles with spin 1, which are called gauge bosons.
Each fundamental force has its own gauge boson(s): the electromagnetic force is mediated via an exchange of massless photons, the strong force is mediated by massless gluons, while the weak force is transmitted by massive W and Z bosons.
% TODO Snizhko suggest just to not use this sentence...
% TODO comment it after Will will read it.
In quantum field theories the forces are given by the dynamics of the quantised, relativistic and locally interacting fields.
% \toAsk[is ``interpreted as dynamics'' correct?]
The electromagnetic, weak and strong forces have different strengths and act over different ranges (see \TableRef{tab:bosons}). 
Both types of fundamental matter particles, leptons and quarks, interact through the electromagnetic and weak forces (except neutrinos, which interact only weakly), whereas only quarks interact strongly.

The SM combines two main theories built to describe all three fundamental interactions, the force-carrier bosons and the matter particles. The first one is the Glashow-Salam-Weinberg (GSW) theory which unifies the electromagnetic and weak interactions. The second theory, quantum chromodynamics (QCD), describes the strong interactions of quarks and gluons. Both GSW and QCD theories are constrained by principles of local gauge invariance of the fields based on the $U(1)_Y\times SU(2)_{L}$ weak isospin and hypercharge, and $SU(3)_C$ color symmetry groups, respectively. Thus, the SM is a Quantum Field Theory based on the $U(1)_Y\times SU(2)_{L}\times SU(3)_C$ symmetry group with $1+3+8=12$ generators that correspond to 12 massless gauge bosons (see \TableRef{tab:bosons}), if the gauge symmetry is unbroken. 
% TODO Oxana was not sure about word ``determines''... consult with Oleg!
Both GSW and QCD theories, unlike the quantum electrodynamics (QED), are non-abelian, which determines the property of self-interactions between the corresponding gauge boson fields. In particular, self-interaction of the gluon field leads to essential characteristics of the QCD, such as asymptotic freedom and confinement.

% TODO Will: "indicating that the electro-weak symmetry of the SM is spontaneously broken" - I'm not sure it directly indicates that, more that SSB is a way to give masses to the EW bosons while maintaining the gauge invariance of the EW lagrangian. But I'm not certain about that distinction. 
The weak W and Z bosons have been experimentally proven to be massive, indicating that the electro-weak symmetry of the SM is spontaneously broken. 
In the SM the spontaneous symmetry breaking is implemented by the Higgs mechanism. 
% TODO Will: "doublet of complex fields is introduced which interacts with all the SM particles" > not neutrinos unless there are RH ones, which are not included in the SM. And certainly not photons or gluons. Ah you get to this in the next paragraph. But still I wouldn't say "all". 
In this approach, a doublet of complex fields is introduced which interacts with the SM particles and has a potential with an infinite number of degenerate vacuum states. 
The choice of one particular ground state with nonzero vacuum expectation value, $v$, breaks the $SU(2)$ symmetry and results in one scalar neutral particle (Higgs boson) that must be present in the SM. As was mentioned above, the existence of the Higgs boson has been confirmed recently.

% TODO cross-check validity of this statement
The Higgs mechanism gives masses not only to $Z$ and $W$ gauge bosons but also to fermions through the corresponding Yukawa coupling terms. 
% TODO Will: "flips the left-handed fermions into right-handed fermions" - this is a bit of a vague statement. I would say "it couples left and right handed fermions" 
An important property of the Higgs field is that it flips the left-handed fermions into right-handed fermions and vice versa. 
As right-handed neutrinos have not been observed and, thus, are not included in the SM, the left-handed
neutrinos can not interact with Higgs boson and remain massless within the SM. 
However, as will be discussed further, there is experimental evidence which proves that neutrinos are massive.

% TODO check is paragraph below is well connected to the text above
The SM has been tested in thousands of measurements by many different experiments.
All measurements are in remarkably good agreement with theoretical predictions.
For example, comparison of the latest measurements of different SM process cross sections by the ATLAS collaboration with theoretical predictions is shown in
\FigureRef{fig:SM_theory_vs_data}.

\begin{figure}[]
  \centering
  \includegraphics[width=0.99\textwidth]{intro/ATLAS_a_SMSummary_TotalXsect.eps}
  \caption{
  Summary of the Standard Model total production cross section measurements made by the ATLAS experiment, compared to the corresponding theoretical expectations~\cite{sm_atlas_public_plots_2016}. 
  Theoretical predictions and their errors are shown with gray bars, while experimental
  measurements are shown with hollow markers with colored bars, which represent measurement errors.
  All theoretical expectations were calculated at next-to-leading order or higher in perturbative QCD.
  }
  \label{fig:SM_theory_vs_data}
\end{figure}


% \begin{figure}[]
%   \centering
%   \includegraphics[width=0.4\textwidth]{intro/ew_fit.pdf}
%   \caption{
%   ??? Taken from~\cite{ew_fit_sm_success}.
%   }
%   \label{fig:ew_fit}
% \end{figure}




% \subsection{Symmetries}
% symmetries --> their combination --> SM (which combine three type of interactions)

% \subsection{Experimental confirmation of SM}
% great agreement between SM and experimental results

\section{Physics Beyond the Standard Model}

\subsection{Problems of the Standard Model}

The SM is currently the best description of the micro-world. It predicted many particles before their discovery. However, despite its great phenomenological success, the SM does not describe the full picture and is believed to be incomplete. The following $big$ $questions$~\cite{Gershtein:2013iqa} remain unsolved within the SM:
\begin{itemize}
\item \textit{The particle content of the dark matter.} An existence of the dark matter is confirmed in the astrophysics and cosmology. The most compelling hypothesis is that the dark matter is made out of massive neutral particles weakly interacting with the matter (WIMPs). They are expected to have a mass of order less than a TeV. In this scenario, the WIMPs can be directly produced at high energy colliders.
\item \textit{Origin of the mass.} The Higgs Mechanism is introduced in the SM ad hoc. The Higgs boson is the first scalar fundamental particle observed in nature. It gives masses to the fermions, $W$, and $Z$ bosons. However, the SM does not tell us why this happens. It is still not clear whether this particle is fundamental or composite, or if there are other Higgs bosons.
% TODO - harmonize it with Little Higgs model description... 
% TODO - I remember that I didn't agree with Oleg about description of this problem!!!
\item \textit{Hierarchy problem.} The mass of the Higgs boson contains large contributions from radiative loop corrections. In particular, it is sensitive to heavy particles of the SM (as well as to hypothetical particles of new physics that might lie at the TeV scale). 
% TODO I don't like the sentence below...
% TODO only some BSM models take care of this problem, by introdusing superpartners
% which give the same radioative correction and they cancel out with each other...
% but other models are not designed to deal with this model, so they will not change anything!!!
In the SM these corrections are quadratically divergent which leads to an unnaturally high mass of the Higgs boson. It is possible to restore the Higgs mass to a proper value through $fine$-$tuning$, but this is considered to be unnatural. Some models which predict new physics and new particles at TeV or above scale allow avoiding this problem
by compensating problematic terms in the Higgs mass formula (e.g. little Higgs model~\cite{Brak}).
\item \textit{Origin of matter-antimatter asymmetry.} Amounts of matter and antimatter created after the Big Bang are expected to be equal. However, the universe visible from Earth is made almost entirely of matter.
A possible explanation could come from new physics predicting baryon number violation, CP-violation and new scalar particles at the TeV scale.
\item \textit{Origin of mass hierarchy of fermion masses.} The mass of the top quark is almost $10^6$ times larger than masses of up and down quarks, which calls for an explanation. Discovery of new particles would provide additional clues to this puzzle.
\item \textit{Neutrino mass.} The experimental results on neutrino oscillations~\cite{Fukuda:1998fd} confirm that neutrinos are massive particles. However, in the SM no mass term for the neutrino particles is incorporated. An extension of the SM with a model containing a massive right-handed, sterile neutrino can solve this problem. In such a model, the SM neutrinos acquire mass and the so-called seesaw mechanism explains the smallness of their masses~\cite{Mohapatra:1979ia}.
% TODO Oxana commented that graviton is not something specific to a String Theory, but just
% an analog of the force carrier as gauge bosons of the SM.
\item \textit{Gravity is not part of the SM.} The Quantum Field Theory, which provides a theoretical framework for the SM, is used to describe the micro-world, while the General Relativity, used to describe gravity, works in the macro-world and cannot be easily quantized. It is not an easy task to fit them into a single framework of the SM comfortably. However, a unification is possible in the context of the String Theory.
\end{itemize}

These questions unsolved by the SM motivate us to continue searches for new physics beyond the SM at the TeV scale and higher.

% \subsection{Strategies for finding new physics}
% Oleg: (look in paper with prospects of BSM searches)
% say about my final state (with leptons)

% \toDo[This subsection is in progress. Preliminary content of the subsection:]
% \begin{itemize}
%  \item search of new physics is done mainly with a high energetic object or rare final states.
%  \item Electrons and muons are most well-measured objects at colliders.
%  \item Jets are more complicated objects to use in the analysis than $e$ and $\mu$.
%  \item Transverse missing momentum - powerful object for the search of new physics, the only way to measure weak particles.
% \end{itemize}


% TODO write something like this:
% General-purpose particle detectors measure the kinematics of all particles with strong or electromagnetic couplings,
% as well as the ‘missing’ momentum due to weakly interacting particles. This gives them the ability to discover
% almost any possible decay of new particles, as well as new stable particles.
% TODO also describe my final states...
% TODO general approach to look for new physics - deviation from the SM...
% TODO direct and undirect searches...

\subsection{Models beyond the Standard Model}
\label{subsec:bsm_models}
Despite the described problems above, the incredible accuracy of the SM, thus far measured up to TeV scale, leads to the understanding that SM 
is simply incomplete rather than incorrect. 
This is why a first step to build a new model which could address some of the problems is first to verify that it agrees with the SM predictions. 
This is why many new models aim to expand the SM rather than to provide an entirely new approach. Such models are typically called Beyond Standard Models (BSM).
There is a plethora of models which address SM problems in many different ways. 

This thesis covers two searches. One of them is focused on the search for new physics with same-sign dilepton signature. Such final state can be produced in many BSM models, e.g. models which predict double charged Higgs or Majorana neutrino. Another one is focused on the search for new heavy spin-1 gauge boson, namely $\PWprime$, with lepton plus missing transverse momentum signature. BSM models which potentially can be confirmed by these searches are discussed below.

% GUT based models, particularly LRSM
In general, there are many possibilities and ways to go beyond the SM. 
Some famous examples can be found in ref.~\cite{Ellis:2011jb}.
Many models are based on the idea of Grand Unified Theory~\cite{GUT_bigPaper},
which seeks to find a simple symmetry group which are based on the SM symmetry group
and contains all known interactions~\cite{Langacker:1984dc,Cvetic:1995zs}.
One such model is the model based on the $SO(10)$ group, which leads to intermediate symmetry:
\begin{equation}
SO(10) \to SU(3)_C \times SU(2)_L \times SU(2)_R \times U(1)_{B-L}
\end{equation}
It yields the so-called Left-Right Symmetric Model (LRSM), which adds
right-handed weak interaction to the SM and accordingly new right-handed $W_R$ and $Z_R$ gauge bosons.
The breaking of $SU(2)_R \times U(1)_{B-L} \to U(1)_Y$ occurs due to a triplet of complex Higgs fields, consisting of $\Delta^0_R, \Delta^+_R$ and $\Delta^{++}_R$, at a high energy scale~\cite{Azuelos:2004mwa}.
Doubly-charged Higgs $\Delta^{++}_R$ can decay to two same-sign lepton pair, which makes it a signal candidate for the same-sign dilepton analysis described in \ChapterRef{chap:SS}. Spontaneous breaking of this symmetry provides mass to the right-handed $W_R$.
Since $SU(2)_R$ symmetry is broken at higher energy scale than $SU(2)_L$, it makes
$W_R$ a signal candidate for the analysis described in \ChapterRef{chap:Wprime}.
This model addresses a few SM problems. Firstly, it assumes Majorana nature of the neutrinos and assigns mass to them by a seesaw mechanism and allows scenarios in which the neutrino masses are naturally light~\cite{Mohapatra:1979ia}. 
Secondly, LRSM provides spontaneous parity breaking~\cite{Grimus:1993fx}, while in the SM parity is broken explicitly.

% Little Higgs model
Another set of models is little Higgs models.
These models address hierarchy and fine-tuning problems described above.
The idea of the models is that the Higgs boson becomes a pseudo-Goldstone boson due to some global symmetry breaking at a TeV energy scale.
The quadratic divergence corrections present in the SM Higgs scalar mass calculation from one-loop contributions from the SM top quark and gauge bosons will be canceled 
by the identical contributions from new heavy gauge bosons and new heavy fermionic states (to oppose the top contribution) introduced by the model~\cite{Han:2003wu,Brak}.
These models assume production of doubly charged Higgs and new heavy charged gauge bosons, which are particles of interest for the searches presented in this thesis.

% Mention other models
In addition to the models described above there are many other theories and models, such as Kaluza-Klein, Zee-Babu, SUSY and others which either assume new charged spin-1 boson or same-sign dilepton final state, which proves the high potential of the BSM searches presented in this thesis.
%   \chapter{The LHC and The \ATLAS experiment}
\label{chap:MoreStuff}

% TODO describe
% 1) Run-1 and Run-2 definitions
% 2) BCID, bunch spasing, bunch length
% 3) define physics run, interfill period
% 4) ATLAS coordination system

% coordinate system

\section{The \LHC}
\subsection{The LHC performance and beam structure}




\section{Magnet System}
\section{The inner detector}
\label{sec:ID}
% additional feature - electron identifiation with TRT

The innermost detector of ATLAS and the closest one to the interaction point (IP) is the inner detector (ID).
The main purpose of the ID is to reconstruct tracks of all charged particles which pass through the detector.
Also tracking detector have to provide information on the sign of the electrical charge of the particles, 
this is why a strong magnetic field is maintained within ID, which 
which makes tracks of particle with different charges be bent into different directions.

From reconstructed tracks vertices are formed. The primary vertex correspond to the vertex where $pp$ collision took place, 
while secondary vertices correspond to the decay of the particles.

ID is done with layers of the sensitive detectors. When particle interact with one of them it deposits the part of its energy to the sensor,
and this energy afterwards is being read and multiplied by the sensor readout electronics. The collected signal is trigerred against the predefined threshold and if 
it signal is larger a hit have been recorded.
One want to have the large number of hits in order to precisely measure the particles track, however, more interaction particles experienced with the detector the more
energy it left, the more distorted track of the particle will be. This is why one prefer as less amount of the material in the ID as possible.
Material  budget of the ID is shown in \FigureRef{fig:material_budget}.

\begin{figure}
\centering
\includegraphics[width=0.7\textwidth]{intro/material_budget.eps}
\caption{ 
The material budget of the ATLAS Inner Detector as a function of absolute pseudorapidity in units of radiation length $X_0$.
}
\label{fig:material_budget}
\end{figure}


Particle density is falling with the distance from the IP as a $1/R^2$, this is why layers close to the IP need to have a large granularity in order to be able to distinguish 
hundreds of particle from one pp collisions, while outermost layers can have lesser granularity to provide the same occupancy as innermost layers.
Thus ID consists from three subdetectors, listed from innermost to outermost one: high-granular silicon pixel detector, silicon strip (SCT) detector and transition radiation tracker (TRT).

All subdetectors consist of two barrel parts (Barrel A and Barrel C) and two end-cap parts (End-cap A and End-cap C) 
which are placed symmetrically with respect to the interaction point.
Detector geometry and acceptance is shown in \FigureRef{fig:ID_eta}.

\begin{figure}
\centering
\includegraphics[width=0.99\textwidth]{TRT/TRTeta.png}
\caption{ 
The Inner Detector quarter-section showing detector acceptance and geometrical sizes of the layers.
}
\label{fig:ID_eta}
\end{figure}


% Pixel and SCT
The Pixel detector is a semiconductor detector which consists of pixels~\cite{Wermes:381263}.
Detector resolution in the barrel region is 10 $\mu m$ in $R-\phi$ and 115 in $z$,
while in the endcap region is 10 $\mu m$ in $R-\phi$ and 115 in $R$.

The SCT detector is a semiconductor microstrip detector. Each layer consists from
two layers of strips rotated in 40 mrad with respect to each other.
Detector resolution in the barrel region is 17 $\mu m$ in $R-\phi$ and 580 in $z$,
while in the endcap region is 17 $\mu m$ in $R-\phi$ and 580 in $R$.

% TRT
% electron identifications
The TRT contains $\sim$300000 thin-walled proportional-mode drift tubes providing on average 30 two-dimensional 
space points with $\sim$130 $\mu m$ resolution for charged particle tracks with |$\eta$| < 2 and $p_T$ > 0.5 GeV~\cite{Abat:2008zza,Abat:2008zzb,Abat:2008zz}.
Along with continuous tracking, the TRT provides electron identification capability through the detection of transition radiation X-ray photons, which is created by the charged particles passing through layers of the radiator material between the tubes.
Detailed description of the detector can be found in \ChapterRef{chap:TRT}.




\subsection{The calorimetry system}

The ATLAS calorimeter is designed to trigger and to measure accurately the energy and position of photons, electrons and hadrons, as well as to ensure a good missing energy measurement, which is crucial for new physics searches. The calorimeter system is divided into two different parts: an inner electromagnetic (EM) calorimeter is aimed to detect electrons and photons, an outer hadronic calorimeter is designed to detect mesons and baryons which escape the EM calorimeter.

The EM calorimeter covers the rapidity region $|\eta| < 3.2$.
In order to meet the physics requirements and to operate properly at high radiation environment, the hadronic calorimeter is further divided into barrel hadronic calorimeter covering $|\eta| < 1.7$, hadronic end-cap calorimeter covering $1.5 < |\eta| < 3.2$, and forward calorimeter covering $3.1 < |\eta| < 4.9$. A global view of the ATLAS calorimeter is illustrated in \FigureRef{fig:Calo}.

\begin{figure}[h!]
\centering
 \includegraphics[width=0.99\textwidth]{intro/0803015_01-A4-at-144-dpi.jpg}
 \caption{ Three-dimensional view of the ATLAS calorimetry.}
\label{fig:Calo}
\end{figure}

\subsubsection{EM calorimeter}
The ATLAS EM calorimeter is a sampling calorimeter with accordion-shaped lead absorbers and Kapton electrodes.
\FigureRef{fig:EMgran} shows illustrates the accordion shape geometry of the ATLAS EM calorimeter. The accordion geometry benefits from fast signal readout and the azimuthal symmetry without cracks. The liquid Argon is used as an active material. The EM calorimeter is divided into two barrel parts ($|\eta|<1.475$) and end-caps ($1.375<|\eta|<3.2$).
The end-cap calorimeter on each side is built of two wheels: Inner Wheel ($1.375<|\eta|<2.5$) is the closest part to the beam pipe,
Outer Wheel ($2.5<|\eta|<3.2$) is the part further from the beam pipe.
An amount of the material in terms of number of electromagnetic radiation lengths ($X_0$) is shown in Figure 2.
The thickness of the EM calorimeter is above $24 X_0$ in the barrel and above $26 X_0$ in the end-cap regions.

Both, barrel and end-cap calorimeters, are segmented into three longitudinal layers. The first layer has about $6 X_0$ thickness with upstream material and plays a role as preshower compartment. It has the finest granularity in $\eta$ with cell width of about 4~mm. The second layer has the thickness of about $18 X_0$ and is designed to contain almost full EM shower. It has the finest cell granularity in $\phi$ allowing to provide the azimuth coordinate of the electromagnetic shower direction. The third layer has two times coarser granularity and the thickness varying between $2 X_0$ and $12 X_0$. The read-out granularity of the LAr system and the accordion shape of the EM calorimeter are schematically illustrated in \FigureRef{fig:EMgran}.

\begin{figure}[h!]
\centering
 \includegraphics[width=0.5\textwidth]{intro/caloDepth_eta.png}
 \caption{Amount of material traversed by a particle before and in EM calorimeter, in units of radiation length $X_0$ , as a function of $|\eta|$.}
\label{fig:Calo}
\end{figure}

\begin{figure}[h!]
\centering
 \includegraphics[width=0.6\textwidth]{intro/LARG3-TDR-barrelM_samplings_presamp_new.png}
 \includegraphics[width=0.39\textwidth]{intro/F3-eps-converted-to.pdf}
 \caption{Read-out granularity and accordion shape of the barrel EM calorimeter.}
\label{fig:EMgran}
\end{figure}

The main goal of the lead absorbers in the sampling EM calorimeter is to develop an electromagnetic shower, while a part of EM shower is collected in the lAr sensitive material. The energy deposited in the absorber material is accounted through the known sampling fraction of the calorimeter. In order to achieve a good performance of the EM calorimeter, an important aspect is the material budget in front of the calorimeter as it degrades the energy resolution of the calorimeter~\cite{electron_tight}.
Before the EM calorimeter the presampler is placed covering the pseudorapidity range of $|\eta|<1.8$. It is needed to recover the energy lost in the material before the calorimeter (inner detector, cryostat, etc).
The relative energy resolution for EM objects is parameterized as follows:
\begin{equation}
\frac{\sigma(E)}{E}=\frac{a}{\sqrt{E[GeV]}}\otimes\frac{b}{E[GeV]}\otimes c
\end{equation}
where $a$ is the sampling term which describes the statistical fluctuations of the EM shower, $b$ is the noise term due to electronics and pile-up, and $c$ is the constant term which accounts for non-uniformity of the calorimeter response. The sampling term mostly contributes at low energies, whereas at high energies theenergy resolution tends asymptotically to the constant term, which is designed  to be of 0.7\%.
The transition region between the barrel and the end-cap, $1.37<|\eta|<1.52$, has significant amount of material in front of the calorimeter (about $ 10 X_0$), making the energy resolution to be poor and thus is usually excluded in physics analyses.

The drift of ionisation electrons in the lAr gap is ensured by high voltage system which generates an electric field of about 1~kV/mm. The induced current on the electrodes is then reconstructed into the deposited energy in an EM calorimeter cell.

The reconstruction of the electrons and photons starts from reconstructing of clusters, a group of the calorimeter cells ($3\times 5$ cells of middle sampling) containing almost full EM shower. Clusters matched to a well-reconstructed track in the ID and originating from the interaction point (IP) are classified as electrons. Clusters without corresponding track matching are considered as unconverted photons. If there are two tracks corresponding to the reconstructed cluster, and if in addition the conversion vertex can be reconstructed, the considered candidate is reconstructed as converted photon.

\subsubsection{Hadronic calorimeter}

The hadronic calorimeter surrounding the EM calorimeter, is designed to measure the hadrons penetrating the EM calorimeter.
It consists of Tile calorimeter in the range of $|\eta|<1.7$ constructed with iron-scintillating-tiles technique, and hadronic end-cap lAr calorimeter spanning $1.5<|\eta|<3.2$. The acceptance of the hadronic calorimeter is extended by lAr Forward calorimeter up to $|\eta|<4.9$ (see \FigureRef{fig:Calo}). The lAr technology for large $|\eta|$ is chosen because of the
intrinsic radiation hardness.
In the Tile calorimeter the signal is provided by scintillating tiles as an active material, while the absorbers are made out of iron. It is divided into of barrel and two extended barrels with inner radius of 2.28~m and outer radius of 4.23~m. Similarly to EM calorimeter, the Tile calorimeter is longitudinally segmented into three layers, which are needed for triggering and reconstruction of jets. The readout of the tiles is performed using optical fibers. The tiles are grouped into the readout cells, which are designed to be projective with respect to the interaction point.

The Hadronic end-cap calorimeters are constructed using coppers as an absorber material and lAr as an active material. The absorber plates are orthogonal to the beam axis and consists of two consecutive wheels
with absorber thickness of 25 and 50 mm; respectively. The forward calorimeter is placed at a distance of about 5 meters from the interaction point. It consists of three longitudinal sections: the first is made of copper absorbers, while the next two are made of tungsten absorbers. The forward calorimeter also provides the electron reconstruction capability.

% Few sentenceas how jets are reconstructed?





% 
% \section{Calorimeters}
% 
% The ATLAS calorimeter is designed to trigger and to measure accurately the energy and position of photons, electrons and hadrons, as well as to ensure a good missing energy measurement, which is crucial for new physics searches. The calorimeter system is divided into two different parts: an inner electromagnetic (EM) calorimeter is aimed to detect electrons and photons, an outer hadronic calorimeter is designed to detect mesons and baryons which escape the EM calorimeter.
% 
% The EM calorimeter covers the rapidity region $|\eta| < 3.2$.
% In order to meet the physics requirements and to operate properly at high radiation environment, the hadronic calorimeter is further divided into barrel hadronic calorimeter covering $|\eta| < 1.7$, hadronic end-cap calorimeter covering $1.5 < |\eta| < 3.2$, and forward calorimeter covering $3.1 < |\eta| < 4.9$. A global view of the ATLAS calorimeter is illustrated in Figure \ref{fig:Calo}.
% 
% \begin{figure}[h!]
% \centering
% \includegraphics[width=0.84\textwidth]{intro/0803015_01-A4-at-144-dpi.jpg}
% \caption{ Three-dimensional view of the ATLAS calorimetry.}
% \label{fig:Calo}
% \end{figure}
% 
% \subsubsection{EM calorimeter}
% The ATLAS EM calorimeter is a sampling calorimeter with accordion-shaped lead absorbers and Kapton electrodes.
% Figure \ref{fig:EMgran} shows illustrates the accordion shape geometry of the ATLAS EM calorimeter. The accordion geometry benefits from fast signal readout and the azimuthal symmetry without cracks. The liquid Argon is used as an active material. The EM calorimeter is divided into two barrel parts ($|\eta|<1.475$) and end-caps ($1.375<|\eta|<3.2$).
% Before the EM calorimeter the presampler is placed covering the pseudorapidity range of $|\eta|<1.8$. It is needed to correct the energy lost in the material before the calorimeter (inner detector, cryostat). An amount of the material in terms of number of electromagnetic radiation lengths ($X_0$) is shown in Figure 2.
% The thickness of the EM calorimeter is above $24 X_0$ in the barrel and above $26 X_0$ in the end-cap regions.
% 
% Both, barrel and end-cap calorimeters, are segmented into three longitudinal layers. The first layer has about $6 X_0$ thickness with upstream material and plays a role as preshower compartment. It has the finest granularity in $\eta$ with cell width of about 4~mm. The second layer has the thickness of about $18 X_0$ and is designed to contain almost full EM shower. The third layer has two times coarser granularity and the thickness varying between $2 X_0$ and $12 X_0$.
% 
% \begin{figure}[h!]
% \centering
% \includegraphics[width=0.5\textwidth]{intro/caloDepth_eta.png}
% \caption{Amount of material in the EM calorimeter (with upstream material), in units of radiation length $X_0$, as a function of $|\eta|$.}
% \label{fig:Calo}
% \end{figure}
% 
% \begin{figure}[h!]
% \centering
% \includegraphics[width=0.44\textwidth]{intro/LARG3-TDR-barrelM_samplings_presamp_new.png}
% \includegraphics[width=0.44\textwidth]{intro/F3-eps-converted-to.pdf}
% \caption{Read-out granularity and accordion shape of the EM calorimeter.}
% \label{fig:EMgran}
% \end{figure}

% TODO to use this paragraph
% The particles entering the EM calorimeter develop EM showers through their interactions with absorbers.
% The ionization electrons drift to the electrode under electric field generated
% by the high voltage of 2000 V. The size of the drift gap on each side of the electrode is 2.1 mm.
% The induced current on the electrode has triangular shape and is initially proportional
% to the deposited energy in the cell. The time of charge collection has order of 400 ns. The physical triangular
% signal is then amplified, shaped by bipolar shaper and digitized every 25 nanoseconds. If signal is accepted
% by trigger, the signal amplitude is determined  from signal samples and transformed to the cell energy.
% The energy response of the calorimeter needs to be calibrated in advance. The energy deposited in the absorber
% can be taken into account by knowing the sample fraction of the calorimeter. 

% TODO formula of energy resolution...
% The energy resolution in a calorimeter is parametrized as the following: 

\section{Muon Spectrometer}
\section{Luminometers and beam monitors}
\section{Trigger, Data Acquisition and Detector Control Systems}


%   \chapter{TRT}
\label{chap:TRT}

\section{Introduction}
\label{sec:TRT:introduction}




The ATLAS Transition Radiation Tracker (TRT) is the outermost of the three tracking subsystems of the ATLAS Inner Detector. 
ATLAS is one of two general-purpose detectors built for the Large Hadron Collider at CERN.

% About TRT
The TRT contains $\sim$300000 thin-walled proportional-mode drift tubes providing on average 30 two-dimensional 
space points with $\sim$130 $\mu m$ resolution for charged particle tracks with |$\eta$| < 2 and $p_T$ > 0.5 GeV~\cite{Abat:2008zza,Abat:2008zzb,Abat:2008zz}.
Along with continuous tracking, the TRT provides electron identification capability through the detection of transition radiation X-ray photons.

% Implementation of Argon simulation
Till now TRT simulation code supported only Xenon based gas-mixtures as active gas in the tubes. 
Current study is focused on implementation of simulation of Argon based gas-mixture, 
which was used for a few runs of data taking in 2013 in a few TRT modules.

% Performance study
% Performance study of the TRT using Argon and Xenon based gas-mixtures will be presented.
% Hit and track parameters, such as hit reconstruction efficiency, residuals, track momentum resolution and extension fraction
% will be shown for both gas-mixtures and compared to Monte Carlo simulation.

Current note are dedicated to the performance study of Transition Radiation Tracker (TRT) of ATLAS detector on hit and track parameter with focus on active gas mixture.
It consist from technical description of implementation of Argon-based active gas mixture to the TRT Digigtization package of ATHENA framework and performance study itself. 
Aim of current study was to investigate scenario when instead of standart Xenon-based mixture Argon-based mixture will be used in some part of detector. Main parameters of interest
were hit residuals, track extension fraction ???
Motivation for current study was leakages in tubes ???





\begin{figure}
\begin{center}
 \includegraphics[width=0.35\columnwidth]{TRT/fromPoster/driftTube.pdf}
\caption{\label{fig:clusterDriftInTube} Example of ionisation and electron drift in a straw tube. Taken from ~\cite{Cwetanski:962570}. 
	  [Peter Cwetanski thesis ???]}
\end{center}
\end{figure}

\begin{figure}
\begin{center}
 \includegraphics[width=0.6\columnwidth]{TRT/fromPoster/TRTdigi4.pdf}
\caption{\label{fig:pulseDigitization} Illustration of the digitization of a TRT low threshold signal.}
\end{center}
\end{figure}



\section{Argon implementation into Digitization package}
\label{sec:TRT:argonImpl}


\subsection{Simulation and Digitization of TRT}

To simulate propagation of charged partiles trough TRT two steps are using: Simulation and Digitization. 
Roughly speaking on simulation step .... and digitization step should take care of processing all physical energy contributions which were left by those particles.
But due to the fact that standart-configured GEANT4 package make poor job on calculating energy deposition in thin layer of gas in tubes it was decided to take care 
of this in Digitization step. So from simulation step only list of hits in TRT are used when energy deposition to the gas inside the tubes are calculated in Digitization step.


\begin{figure}
\begin{center}
 \includegraphics[width=0.59\columnwidth]{TRT/fromPoster/ArgonModules2013.png}
\caption{\label{fig:argonModulesIn2013} Detector modules which were operated with Argon gas mixture during 2013 runs}
\end{center}
\end{figure}





\subsection{Implementation of mixed condition}
TRT detector was originally designed as an homogenous detector and it was planned that only one gas mixture will be used in the all straws. 
Digitization code, as all physics codes, was written with an idea to be as simple as possible, that's why it was designed with assumption that all straws will be simulated with only one type of gas mixture.
But after leakeges appears at the end of the Run 1, it become higly probable, that TRT may be run in mixed condition, when some part of straws will be run with one gas and other part - with another. That's why 
possibility to digitize mixed condition become essential feature of the code. 

To make it work new (to the digitization code) Argon gas mixture was introduced and following simulated detector/electronics parameters as low/high treshold and shaping functions were duplicated.
Digitization of the TRT hits are done straw by straw and in the beginning of the straw loop flag \mbox{ ``isArgonStraw''} is read from the COOL database. This flag represent is current straw contain Argon or Xenon based 
gas mixture. The straw map lays under \mbox{/TRT/Cond/StatusHT} COOL folder, and one need to specify dedicated tag to make mixed Ar/Xe digitization.
According to this flag relevant thresholds and shaping function are picked up and used during digitization following straw. 


\begin{figure}

\begin{subfigure}{.5\textwidth}
  \centering
  \includegraphics[width=\textwidth]{TRT/mapXY_barA.pdf}
\end{subfigure}%
\begin{subfigure}{.5\textwidth}
  \centering
  \includegraphics[width=\textwidth]{TRT/mapXY_endA.pdf}
\end{subfigure}

\caption{$T_{0}$ calibration constants obtained for MC simulation for Barrel A (left) and End-cap A (right) detectors for the mixed Xenon-Argon condition. 
	  Modules with Argon gas have different $T_{0}$ calibration constants.}
  \label{fig:t0_mixed_condition}
\end{figure}



Also two flags which provide simple possibility to change active gas mixture from default one (Xenon) to the optional (Argon) for all straws were implemented. These are:
\begin{itemize}
 \item UseArgonStraws
 \item UseConditionsHTStatus
\end{itemize}

First flag says that we want to use optional gas in some part or in full detector. If it is false - default Xenon gas will be used despite of the second flag. 
Second flag says do we want to read straw map from COOL database or not.
If it is false full Argon geometry condition will be run. Example of usage these flags can be found at [???].

Points related to the actual gas mixture implementation are described at the next subsections.

\subsection{Simulation of energy deposition to the gas in straw: PAI tool}


Average ionization energy. Argon gas was studied before   [Explain how I count it from table from Perer Cwetanski thesis, see presentation by Andrew]


%%% Average ionization energy. %%%
%%% explanation how does average inonization energy effect simulation and in which steps (Andrew presentation).
%%% 


\begin{itemize}
 \item Average ionization energy
 \item Primary electron survival probability (???)
 \item ???
\end{itemize}



\begin{figure}
\begin{center}
 \includegraphics[width=0.59\columnwidth]{TRT/meanFreePath.eps}
\caption{\label{fig:meanFreePath} ???}
\end{center}
\end{figure}


\subsection{Simulation of timing variables}

To simulate drift time of the primary cluster of electrons from place of origin to the wire Athena package 
use empirically measured (TODO not from Garfield?) tables of dependence of drift distance versus drift time (so called RT tables). 
Due to the facts that drift velocity depends from magnetic field in the tube and field itself is not homogenous within detector 
there are two sets of tables: one which corespond to the maximum magnetic field (2T for the ATLAS experiment) and another - without magnetic field. To get RT table
for some specific value of magnetic field following interpolation formula are used:

\begin{displaymath}
    RT(B_{eff}) = (RT_{MAX} - RT_{WO}) \cdot \dfrac{B_{eff}^2}{B_{max}^2} + RT_{WO}
\end{displaymath}

[TODO] explain variables in formula above.

That's why to implement Argon gas mixture two such tables were added to the TRT digitization package.
% \\ \mbox{InDetSimUtils/TRT$\_$DriftTimeSimUtils} package (\mbox{TRT$\_$BarrelDriftTimeData.cxx} file). 
These tables were obtained from Garfield simulation [reference ???].
% by Konstantin Vorobyev 
The comparison of Argon and Xenon RT tables are shown in figure \ref{fig:rt_comp}. 
Also scaled Xenon RT distribution are shown to highlight different shape of distributions for Argon and Xenon gases.


\begin{figure}
\begin{center}
 \includegraphics[width=0.99\columnwidth]{TRT/rt_comp.png}
\caption{\label{fig:rt_comp} }
\end{center}
\end{figure}

Another variable which was implemented for Argon mixture is signal shaping function. 

Different active gases give diferent signals. These signals differs in leading front, amplitude and shape and size of tail. But TRT electronics expect to recieve signal with specific configuration.
That's why ASDBLR chip~\cite{TRT_electronics} convulve signal from straw with dedicated shaping function to produce final signal which will be acceptable by further chain of electronics. This leads to the fact that we need
to have different shaping functions for Argon and Xenon based mixtures.
% TODO do I want to include this cite???
% ~\cite{Vorobev_private}. 
Due to the fact, that datataking ASDBLR chip was used different shaping function for Xenon and Argon 
digitization code should also provide such kind of functionality. 

\begin{figure}

\begin{subfigure}{.5\textwidth}
  \centering
  \includegraphics[width=\textwidth]{TRT/fromPoster/rt_FinalXenon.eps}
\end{subfigure}%
\begin{subfigure}{.5\textwidth}
  \centering
  \includegraphics[width=\textwidth]{TRT/fromPoster/rt_FinalArgon.eps}
\end{subfigure}

\caption{Track to wire distance in Xenon (left) and Argon (right) straws in Endcap detectors.}
  \label{fig:RT_xenon_argon}
\end{figure}

Shaping functions are used in the simulation of electronics. It effect simulation of discriminator responce and accordingly TRT bit pattern.
Shaping function are stored at \mbox{TRTDigSettings.amplitudes} file in the \mbox{TRT$\_$Digitization} package. During digitization according to ``isArgonStraw'' flag relevant shaping function is used. 
All available shaping functions are shown in figure \ref{fig:shaping}.

\begin{figure}
\begin{center}
 \includegraphics[width=0.69\columnwidth]{TRT/shaping.eps}
\caption{\label{fig:shaping} Argon low threshold shaping function in comparison with Xenon low and high treshold shaping functions}
\end{center}
\end{figure}

% TODO new text starts from here [August 2016]:




\subsection{Noise simulation}

% TODO I am not sure if it is this note....
Noise note:~\cite{Kittelmann:987854}

[TODO] see comments in TRTDigCondFakeMap::setStrawStateInfo() function.


\begin{equation}
 LT_i = A_i \cdot (\alpha + \beta \cdot ErfcInv(\gamma \cdot f_i)) %TODO make EffcInv as in Noise note (eq. 2)
\end{equation}

%%% OR 
%%% 
%%% just say that according to the Noise revision note LT_i is proportional to the noise signal amplitude
%%%

% LTi = Ai · ( +  · ErfcInv(  · fi)) ,

%%% formula from ``Revision of Noise and Threshold Description in MC Simulation'':
%%% --> averagenoiseampforstrawlength = ( ( (3000.-1350.)/(70*CLHEP::cm) ) * strawlength + 1350.0 ) / 3000.0;
%%% according to Anatoli, we need to scale noise amplitude for Argon straws by factor LT_argon/LT_xenon
%%% --> scaleAmplitudeFactor = m_settings->lowThreshold(true)/m_settings->lowThreshold(false);
%%% --> averagenoiseampforstrawlength *= scaleAmplitudeFactor; 
%%% Found in file: ./InDetDigitization/TRT_Digitization/src/TRTDigCondFakeMap.cxx *

%%% WARNING Sasha: this function used hardcoded noise signal shape in TRTSignalShape.cxx line 58.
%%% WARNING Sasha: should noise be different for Argon? should I change something here?  
%%% WARNING Sasha: for now I will assume that it is okay for Argon
%%% Found in file: ./InDetDigitization/TRT_Digitization/src/TRTNoise.cxx *

\begin{figure}
\begin{center}
 \includegraphics[width=0.79\columnwidth]{TRT/grBMPos_mod.pdf}
\caption{ Position residual width as a function of the TRT module number in the Barrel detector. Red points correspond to simulation of TRT detector
using Argon-based mixture in all straws, black points correspond to Xenon-based mixture while green point correspond to the mixed detector condition
which was used during ??? data taking period. Modules with Argon mixture in mixed condition can be clearly seen.}
\label{fig:meanFreePath}
\end{center}
\end{figure}

\section{Tracking performance study}
\label{sec:TRT:trackPerf}

During 2013 year a few runs with Argon gas mixture in the detetector was done in order to investigate performace of the detector with Argon as active gav.
Four of 32 phi sectors in the inner layer for the barrels and one of 14 wheel in the endcap A were filled with Argon mixture while other sectors were operating 
with usual Xenon mixture (as it shown in \FigureRef{fig:argonModulesIn2013}). Used Argon mixture had following cofiguration: Ar/CO$_{2}$/O$_{2}$ (70$\%$/27$\%$/3$\%$). In order to operate it high voltage had
to be higher [??? explain why HV has to be different] and it was 1470 V.




\begin{figure}

\begin{subfigure}{.5\textwidth}
  \centering
  \includegraphics[width=\textwidth]{TRT/fromPoster/resFitXenon.eps}
\end{subfigure}%
\begin{subfigure}{.5\textwidth}
  \centering
  \includegraphics[width=\textwidth]{TRT/fromPoster/resFitArgon.eps}
\end{subfigure}

% \includegraphics[width=0.75\textheight]{monoW/kinematics_simplifiedSChannel_EFT_Wprime.png}
\caption{Position residuals. Barrel A.}
  \label{fig:resFit}
\end{figure}





\begin{figure}

\begin{subfigure}{.5\textwidth}
  \centering
  \includegraphics[width=\textwidth]{TRT/fromPoster/hit_eff_rtrack_bar_xenon_region.eps}
\end{subfigure}%
\begin{subfigure}{.5\textwidth}
  \centering
  \includegraphics[width=\textwidth]{TRT/fromPoster/hit_eff_rtrack_bar_argon_region.eps}
\end{subfigure}

% \includegraphics[width=0.75\textheight]{monoW/kinematics_simplifiedSChannel_EFT_Wprime.png}
\caption{Hit reconstruction straw efficiency as a function of track to wire distance.}
  \label{fig:hit_eff_rtrack_bar}
\end{figure}

As was mentioned above electrons drift faster in the Argon-based gas mixture than Xenon one, which can be observed in the track to wire distance
distributions shown in \FigureRef{fig:RT_xenon_argon}. 
This results that timing calibration constants have to be significantly different 
for Argon and Xenon straws which can be seen in \FigureRef{fig:t0_mixed_condition}. 


\begin{figure}
\begin{center}
 \includegraphics[width=0.9\columnwidth]{TRT/fromPoster/track_ext_frac_eta.eps}
\caption{Track extension fraction as a function of $\eta$ for Xenon and Argon active gas mixture obtained with MC simulation.}
\label{fig:meanFreePath}
\end{center}
\end{figure}



Due to a lower threshold a larger hit reconstruction efficiency is seen for Argon straws. This leads to a
larger number of hits per track and larger fraction of precision hits. Also position residuals are seen to be 
slightly larger. All these observations indicate that additional threshold tuning is required.

[TODO] think which plots can be added to this section.

\begin{figure}

\begin{subfigure}{.5\textwidth}
  \centering
  \includegraphics[width=\textwidth]{TRT/fromPoster/xenon_nHitsPerTrack_vs_phi.eps}
\end{subfigure}%
\begin{subfigure}{.5\textwidth}
  \centering
  \includegraphics[width=\textwidth]{TRT/fromPoster/argon_nHitsPerTrack_vs_phi.eps}
\end{subfigure}

\caption{Number of hits per track. MC simulation.}
  \label{fig:nPrecHitsPerTrack}
\end{figure}


\begin{figure}

\begin{subfigure}{.5\textwidth}
  \centering
  \includegraphics[width=\textwidth]{TRT/fromPoster/xenon_precHitFracPerTrack_vs_phi.eps}
\end{subfigure}%
\begin{subfigure}{.5\textwidth}
  \centering
  \includegraphics[width=\textwidth]{TRT/fromPoster/argon_precHitFracPerTrack_vs_phi.eps}
\end{subfigure}

\caption{Precision hit fraction. MC simulation.}
  \label{fig:precHitFracPerTrack}
\end{figure}


%   \chapter{LUCID - The ATLAS Luminosity Monitor}
\label{chap:LUCID}

% ToWRITE:
% \begin{itemize}
%  \item what this chapter is about
%  \item List of publications related to this topic
%  \item My personal contribution related to the described topic
% \end{itemize}

This chapter describes the LUCID-2 detector which was built specially for the Run-2 phase of the LHC program.
It covers aspects such as the design of the detector and its key components,the assembly and testing of the new detector as well as
operation and performance of the detector during the 2015-2016 data taking periods.
Special attention is given to the calibration system of the detector and the development of the calibration procedure.

During the design phase many tests were done in order to find an optimal design and the optimal parameters for various detector components.
During the assembly and installation phase a number of tests were done to make sure that all components performed as they should.
An overall test of the system was done to make sure that no damage had been done during the installation of the detector.
During the operation phase, which is still ongoing, a lot of studies have been made to understand the performance of the calibration system and the detector.

I contributed to all the steps mentioned above. I took part in the development of the LUCID design and in particularly the design of the calibration system. 
I made a series of tests to find the optimal design parameters of the LED and Laser diffusers used to evenly distribute LED and laser signals and deliver it to all detector 
channels. 
I spent a lot of time on understanding the behavior of the LED system as well as the PMT and PIN-diode signal behavior.
Tests with Bi-207 radioactive sources which are used as one of the way to monitor photomultiplier (PMT) gain were done as well.
I participated in the detector assembly in the clean room and did testing of the detector during this process.
Testing of LED and laser diffusers were done in order to cross check the integrity of fibers and the homogeneity of signals between all PMTs.
Also temperature stress-test was performed in order to understand what maximum temperature could be allowed without destroying the detector during the 
beam pipe bake-out procedure.
In the operational phase the main focus was on understanding the aging of PMTs and the possibility to improve the calibration system.

The LUCID group published a paper with a description of the choice and the characterization of photomultipliers for the new LUCID detector for Run-2~\cite{Alberghi:2016tad}.
My contribution was in understanding and developing the monitoring using a Bi-207 source.

During my PhD studies I was the ATLAS Forward detectors Run Coordinator for a period of 5 months.


% \section{Plan of the LUCID chapter}
% 
% \begin{itemize}
%  \item Lucid operation (should I write about it? if yes - what can I write?)
%  \begin{itemize}
%   \item performance of calibration system.
%   \item HV changes in 2015/2016. As an interesting fact how PMTs are aging.
%   \item interesting observations from LUCID operatins.
%   \item performance of detector - luminosity in 2015 (as a conlusion of LUCID chapter).
%  \end{itemize}
%  
%  \item PMT temperature test (should I describe experimental setup and procedure of development a circuit?);
%  \begin{itemize}
%   \item have a look on temperature change during the nights...
%  \end{itemize}
% 
%  \item Fiber boiling test (see sent letter by october 15, 2014):
%  \begin{itemize}
%   \item picture of bake-out phase
%  \end{itemize}
%  
%  \item Testing of LUCID after installation:
%  \begin{itemize}
%   \item testing of LED diffuser and integrity of fibers
%  \end{itemize}
%  
%  \item Detector design tests:
%  \begin{itemize}
%   \item Understanding LED and PIN-diode:
%   \begin{itemize}
%    \item tests charge or PMT/PIN vs. LED DAC (no conclution reached?)
%    \item angular distribution (lucid$\_$july22.pdf)
%    \item distance measurements 
%    \begin{itemize}
%     \item was done to proof that PMT is linear with light intensity
%     \item this because we saw that PMT charge is not linear vs. LED DAC.
%     \item conclusion: LED is not linear vs. DAC
%    \end{itemize}
%    
%    \item LED frequency test
%   \end{itemize}
%   
%   \item LED difuser geometry (no presentation?).
%   \item Laser diffuser distance.
%   \item Filter choice for LED diffuser (lucid$\_$september10.pdf; also ask Carla about 100 mV restriction).
% %   \item 
% %   \item 
% %   \item 
%  \end{itemize}
%  
%  \item Time tests. Long LED runs. (Though Anders also did it and Vincent is using his results to show, but I also was working for quite some time on it, 
% so I think I should write about it as well):
%  \begin{itemize}
%   \item find presentation of Anders. Check what he did.
%   \item from my presentation (lucid15$\_$apr15.pdf) it looks like we have strong effect in high rate and no effect in low rate.
%  \end{itemize}
% \end{itemize}
% 
% \newpage

\section{The new LUCID-2 detector}
\label{sec:LUCID}

% TODO: rewrite thoroughly this subsection!.
% Is this relevant comment?

LUCID (LUminosity Cherenkov Integrating Detector) is a luminosity monitor with two detectors placed around the beam-pipe on both forward ends of the ATLAS detector.
It is a relative luminosity detector which detects inelastic $pp$ scattering events in order to measure luminosity and provide online monitoring of the instantaneous luminosity.
The first version of the detector was installed in 2008 and it is described in ~\cite{Aad:2008zzm}.
It was used as the main luminosity detector for Run-1 of the LHC program in 2009-2010 ~\cite{Aad:2013ucp} and in combination with other luminosity detectors in 2011-2013.
The second version of the detector and read-out electronics was designed to cope with 
the increased luminosity and the decreased bunch spacing (from 50 ns to 25 ns) for Run-2 of the LHC program.
In this section the design of the new detector and its electronics is presented. 

\begin{figure}
\centering
\begin{subfigure}{.5\textwidth}
  \centering
  \includegraphics[width=\linewidth]{LUCID/LUCIDdesign.png}
\end{subfigure}%
\begin{subfigure}{.5\textwidth}
  \centering
  \includegraphics[width=\linewidth]{LUCID/FourPMTs_zoomed.png}
\end{subfigure}
\caption{(left) Schematic drawing of one of the two detectors, showing the position of the photomultiplier tubes 
and quartz fibers with respect to the LHC beampipe; (right) A quarter of one of the detectors. All tubes are 
placed inside mu-metal shielding to protect the PMTs from a stray magnetic field. Cooling pipes carrying water were installed in order 
to protect the PMTs from overheating during the beampipe bake-out procedure. Three of four tubes have fiber connectors, which
transfer LED and laser pulses for calibration. The fourth tube is equipped with a Bi-207 source and is completely 
sealed.}
\label{fig:LucidDrawing}
\end{figure}

\subsection{The detector design}
\label{subsec:newLucidDesign}

LUCID consists of two identical parts which are placed on both sides of the ATLAS intersection point at a distance of 17 meters.
Each detector consists of 20 photomultipliers (PMTs).
16 of them, grouped four by four, are placed close to the beampipe. The center of these photomultipliers is at a distance of 125 mm from the beamline.
These PMTs detect charged particles that traverse their quartz windows, where Cherenkov light is produced.
Four other PMTs are placed 1.2 m away from the beampipe and are protected by the massive muon shielding; 
Cherenkov light is produced in quartz fiber bundles that runs parallel to the beamline and that are coupled to the PMTs (see \FigureRef{fig:LucidDrawing} left).

The 20 PMTs are grouped in 5 different families:
\begin{itemize}
 \item FIB PMTs are the one which are protected by the shielding. Cherenkov light are produced and delivered by quartz fibers bundles 
 (which consists of 37 optical fibers each with a 0.8 mm quartz core).
 \item BI PMTs are equipped with a Bi-207 radioactive source which is used for PMT gain monitoring. 
 \item VDM PMTs were meant to be used during so-called van der Meer (vdM) scan ~\cite{vanderMeer:296752,Rubbia:1025746} taken to determine the absolute luminosity.
 \item SPARE PMTs are spare PMTs and can be turned on if the VDM PMTs have aged too much.
 \item MOD PMTs have a thin ring shaped layer of aluminium deposited between the quartz window and the photocathode. 
 The center hole of this ring has a diameter of 7 mm which can be compared to the 10 mm diameter of the photocathode 
 and the layer thus reduces the acceptance of these photomultipliers with a factor of 2 (in comparison with other used PMTs) 
 which will help to avoid saturation of some luminosity algorithms.
\end{itemize}

BI, VDM, SPARE and MOD PMTs are grouped together and 16 PMTs form four such groups which are placed equidistantly from each other around the beampipe (see \FigureRef{fig:LucidDrawing} left).

The gain of the PMTs is monitored by the dedicated gain monitoring system which is described in \SectionRef{sec:pmtGainMonitoringSystem}.

With respect to the detector used in Run-1~\cite{Aad:2013ucp}, the new LUCID has a reduced material budget, 
an increased dynamic range and can measure luminosity with additional algorithms based on so-called PMT charge integration in which the pulses are integrated with 
flash analog-to-digital converter (FADC).

\subsection{Choice of photomultipliers}
\label{subsec:PMTChoice}

% Is comment below relevant???
% [TODO: rewrite thoroughly this subsection!].

The new LUCID uses R760 Hamamatsu PMTs, a smaller version of the previously used R762 model. These PMTs have a 
10 mm quartz window diameter, while the old ones had a 14 mm diameter. A smaller PMT model has been chosen to reduce acceptance 
which will help to cope with the increased occupancy and to avoid saturation of the luminosity algorithms.
In addition, 4 PMTs per side have a specially reduced sensitive window with a 7 mm diameter which roughly 
corresponds to a factor 2 decrease in acceptance (see \FigureRef{fig:modPMT}). They provide luminosity algorithms that will saturate at 
higher luminosity than the standard photomultipliers. A detailed description of choice and characterization of the PMTs used in the detector can be found in~\cite{Alberghi:2016tad}.

% [ToASK: Should I add additional information how choise of PMTs was done (as described in the paper~\cite{Alberghi:2016tad})?].
% TODO also mention that below in the text there will be some description of tests done to understand behaviour of PMts?

\begin{figure}
\centering
\includegraphics[width=.6\textwidth]{LUCID/mod_PMT.jpg}
\caption{R760 Hamamatsu PMT with specially reduced sensitive window size used in the LUCID detector.}
\label{fig:modPMT}
\end{figure}

\subsection{Read-out electronics}
\label{subsec:LUCIDElectronics}
% TODO is this comment relevant?
% [TODO: rethink this subsection!].

New readout electronics have been built that consist of VME boards that digitize the PMT signals with FADCs. 
The electronics record hits if the pulseheight is above a threshold and integrate the pulses in each 25 ns 
interval that correspond to a LHC bunch crossing. \FigureRef{fig:pulseShape} shows a typical PMT signal shape in 
a physics run. The duration of the pulses is less than 25 ns.

\begin{figure}
\centering
\includegraphics[width=.6\textwidth]{LUCID/pulseShape_run_267367_preliminary.eps}
\caption{Digitized pulse shape of a signal from one of the PMTs of the LUCID detector during a run recorded on 
the 10th of June 2015 at $\sqrt{s}$ = 13 TeV. The polarity of the pulse is inverted. The FADCs measure the 
pulse amplitude in time bins that are 3.125 ns long.}
\label{fig:pulseShape}
\end{figure}

The LUCID read-out consists of four (two per side) custom made so-called LUCROD (LUCid ReadOut Driver) boards of VME type which sit close to the detector in the ATLAS experimental hall.
The decision to place electronics close to the detector in the experimental hall was motivated by preventing signals to develop long tails in the cables.
Signal from PMTs are transferred with thick cables which prevent distortion along their path.
Every LUCROD board has 16 input channels and every channel consists of a low noise amplifier, a filter and a FADC.
A block diagram of a LUCROD module is shown in \FigureRef{fig:LUCROD_schematics}.
Channels are grouped in pairs and for each pair there is a dedicated channel FPGA.
All information from all channel FPGAs are collected and processed by the main FPGA.
After that the information is sent to ATLAS as well as to the so-called LUMAT (LUminosity Monitor And Trigger) boards 
which are placed in the counting room of the experiment.

\begin{figure}
\centering
\includegraphics[width=.95\textwidth]{LUCID/LUCROD_schematics.jpg}
\caption{Block diagram of the LUCROD board. Every board host two input channels. Every channel consist of a low noise amplifier, a flash ADC and a shared FPGA. 
There are 16 channels (8 units) per LUCROD. All units are controlled by the main FPGA, which collect information from all channels and make needed calculations.}
\label{fig:LUCROD_schematics}
\end{figure}

There are two LUCROD board per side and it was decided to couple different sets of sensors to the different boards as shown in 
\FigureRef{fig:Eletronics_schematics}.
BI, VDM and SPARE PMTs were connected to one board while MOD and FIB PMTs were connected to the another board.
The same connection scheme was used on the other side as well.

% TODO change this picture with one which is in lumi note.
\begin{figure}
\centering
\includegraphics[width=.95\textwidth]{LUCID/Eletronics_schematics.eps}
\caption{Block diagram of the LUCID electronics. Signals from all photomultiplier tubes are collected by 4 \mbox{LUCROD} cards 
(two per side) that digitize the signals with FADCs. Some of the luminosity algorithms are implemented in the LUCRODs. 
The number of events that fullfil different luminosity algorithms are counted and a copy of all digitized PMT 
signals are sent to the LUMAT cards, which perform calculations with algorithms that combine data from 
both detectors and publish the results to the Information Server (IS) database.}
\label{fig:Eletronics_schematics}
\end{figure}

Every LUCROD boards receive information from PMTs situated in one of the two detectors.
In order to implement the possibility of requiring signals in both detectors, two additional boards which are called LUMAT boards are used
as shown in \FigureRef{fig:Eletronics_schematics}.
Digital signals from PMTs of the same family from LUCRODs on both sides are sent to LUMAT boards, 
which then perform logical operations with signals from both sides.

Information from the LUCROD and LUMAT boards are then published to the Information Server (IS) which is a database.
From here the data is accessed by programs that calculate the luminosity online.

\section{Design of the PMT gain monitoring system}
\label{sec:pmtGainMonitoringSystem}
% TODO add reference to some book with desciption of PMT aging 
% Are these comments relevant?
% [TODO: describe why do we need calibration system. What is the purpose of it. Stress that it's important for correct measurements of luminosity.]

The PMT gain is monitored in 3 independent ways (see \FigureRef{fig:calibrationSystem}):
\begin{itemize}
 \item by LED signals carried by optical fibers;
 \item by laser signals transferred from the calibration system of the ATLAS Tile Calorimeter;
 \item by radioactive sources (Bi-207).
\end{itemize}

\begin{figure}
\centering
\includegraphics[width=.7\textwidth]{LUCID/calibrationSystem.png}
\caption{The LUCID PMT gain monitoring system. 16 PMTs per side receives light from LEDs and the Tile laser calibration 
system. 
For redundancy, two fibers come from two different LED diffusers (with three LEDs each, monitored by 
PIN-diodes), and two fibers come from one laser diffuser. The four remaining PMTs in each detector are calibrated 
with Bi-207 sources.}
\label{fig:calibrationSystem}
\end{figure}

% The availability of three independent calibration methods increase the robustness of the calibration system 
% and provide a possibility to cross-check calibration results between the methods.
\toAsk[the paragraph below is new]

PMTs which are coupled with Bi-207 radioactive sources constantly see signals of radioactive decays which is treated as a background
during luminosity measurements. This is why activity of the radioactive sources have to be as low as possible on the one hand, 
but on the other hand it has to be large enough to collect needed statistics for the calibration during 20-30 minutes available between the LHC interfills.
Source activity is completely negligible during physics runs with many $pp$ interactions per bunch crossing. However, sources activity could become 
a problem for special beam-separation runs (so-called van der Meer runs), used for the absolute luminosity calibration, due to theirs tiny luminosity.
For these runs VDM PMTs were expected to be used. However, analysis of the first set of van der Meer runs, described in \SectionRef{subsec:alfa_run}, 
demonstrated that activity of the Bi-207 sources is small exactly enough to allow use BI PMTs for absolute luminosity calibration as well. 
The gain of the BI PMTs is monitored by the radioactive sources, while the gain for all other PMTs by LED and laser light.

LED signals provide peaks in the amplitude and charge distributions that are recorded by LUCID in data acquisition runs between LHC fills.
The stability of the PMT gain is controlled by measuring the mean value of these distributions 
and then changing the high voltage to the photomultiplier in order to keep their mean values constant. 
The stability of the LEDs themselves is controlled by PIN-diodes and an alternative way of calibration is 
to use the ratio of the mean charge measured by PMTs with that of the PIN-diode. 
This charge is proportional to the LED intensity and by using this charge ratio allows to rule out any 
dependence of the calibration results on LED intensity fluctuations.
In order to provide the same amount of light simultaneously to all PMTs a special LED diffuser was designed and manufactured.
This is discussed in details in \SectionRef{subsec:LEDDiffuser}.

The Tile calorimeter laser system provides an alternative source of stable light and is treated in the same way
as the LED signals in the calibration procedure. 
The stability of the laser signals is monitored by the Tile calibration system \cite{Aad:2008zzm}.
The laser light has to be distributed between the PMTs in the same way as the LED light.
Laser light is provided by the Tile calibration system via optical fibers which means that another type of diffuser
has to be used in order distribute the light to the PMTs as described in \SectionRef{subsec:laserDiffuser}.

Bi-207 radioactive sources provide monoenergetic electrons from an internal conversion process with energies 
above the Cherenkov threshold in quartz. These electrons
have enough kinetic energy to penetrate the quartz window of the PMT and produce signals similar to the signals 
from high energetic particles in physics runs. The truncated mean of the charge and amplitude 
distributions from the Bi-207 sources are used in the same way as for the two methods described above. 
This method does not suffer from any instability issues~\cite{Alberghi:2016tad}.
It was decided to use a liquid Bi-207 source as described in \SectionRef{subsec:bi207Calibration}.

In \SectionRef{subsec:calibPerformance} the calibration strategy during the  2015 and 2016 data-taking periods are discussed.

\subsection{The LED diffuser}
\label{subsec:LEDDiffuser}
% TODO structure of the section:
% - to describe why we have 2 LED diffusers: why we use filters on one and don't use on another on
% - geometry studies
% - show LED stability plot over time (PIN-diode measurements)
% - filter studies - explain why we have to be below 100mV for PMTs (discuss this with Carla)
% - 
% - 


% light has to be evenly distributed among PMTs
% LED has to be monitored by PIN-diode in order to verify its stability
% geometrical constrain
%   taking into account size of PIN-diode
%   size of support structure for fibers
%   bending of fibers - whole LED diffuser has to be hosted in limited space and fiber cannot be bend to much.
% constrain during production of the diffuser:
%   complexity of drilling holes for fiber for some angles
%   gluing fibers to the diffuser

Despite the simple purpose of the LED diffuser to evenly distribute light among the PMTs 
there was a number of constrains that made it necessary to make dedicated studies 
to define the optimal parameters of the diffuser. The main points which were considered during the design phase were:
\begin{itemize}
 \item light had to be evenly distributed among PMTs;
 \item the LED had to be monitored by PIN-diode in order to verify its stability;
 \item the sensitivity of PMTs and PIN-diodes to light is very different;
 \item geometrical constrains such as the size of the PIN-diode, 
       limited space for the whole diffuser and limits on the bending of fibers had to be taken into account;
 \item manufacturing constrains such as the complexity of drilling small holes and gluing fiber at specific angles had to be kept in mind.
\end{itemize}

To meet all these requirements a radial design was proposed with a PIN-diode facing three LEDs and located aligned with LEDs axis. Fibers surround the PIN-diode 
evenly with a certain angle to the LED axis. A schematic sketch is shown in \FigureRef{fig:AngularMeasurementSetup} and in \FigureRef{fig:LEDDiffuser}.
% TODO insert sketch used for testing.
Such a geometry assume that the distance between the LEDs and the fibers would be the same for all fibers in order for 
the fibers to pick up the same amount of light.
  
\begin{figure}
\centering
\includegraphics[width=.7\textwidth]{LUCID/LED_diffuser_schematic.pdf}
\caption{Schematic of the experimental setup which was used in the LED diffuser design phase. LED and PIN-diode are aligned and face each other. 
	 Angles represent possible positions of the fibers around the PIN-diode. Fibers are not shown in the sketch.}
\label{fig:AngularMeasurementSetup}
\end{figure}

\begin{figure}
\begin{subfigure}{.48\textwidth}
  \centering
  \includegraphics[width=\textwidth]{LUCID/LEDdiffuser_rightPart.png}
\end{subfigure}
\begin{subfigure}{.48\textwidth}
  \centering
  \includegraphics[width=.8\textwidth]{LUCID/photo_LED_diffuser_v2.jpg}
\end{subfigure}

\caption{The LED diffuser. Schematic drawing (left), photo (right).}
\label{fig:LEDDiffuser}
\end{figure}
  
% As described above main purpose of LED diffuser is to evenly distribute light between PMT-channels. 

% consists from LED itself, set of fibers which deliver light to the PMTs and PIN-diode which is used to monitor
% stability of LED during long time. LED diffuser has to be design in such way, that all PMT channels have to receive the same amount of light.

% TODO define what is LED axis. Try to explain it better.
The manufacturer provided information about LED light intensity as a function of the angle between an observer and the LED axis 
(\FigureRef{fig:AngularDistributionOfLED} (right)).
% But in order to make sure that real LEDs correspond to specified characteristics a test in the lab was done.
A set of measurements was also done.
A sketch of the experimental setup is shown in \FigureRef{fig:AngularMeasurementSetup}.
Measurements of the PMT anode current as a function of fiber angle with respect to the LED axis for different distances between 
the fiber surface (PIN-diode) and the LEDs were done and the results are shown in \FigureRef{fig:AngularDistributionOfLED} (left).
With angles below 30\degree the intensity is relatively homogeneous and above 30\degree the light intensity starts to drop off.

\begin{figure}
\begin{subfigure}{.46\textwidth}
  \centering
  \includegraphics[width=\textwidth]{LUCID/current_vs_angle.pdf}
\end{subfigure}
\begin{subfigure}{.51\textwidth}
  \centering
  \includegraphics[width=\textwidth]{LUCID/LED_radial_intensity_chart.pdf}
\end{subfigure}

\caption{Angular homogeneity of LED. 
Measurements of PMT anode current as a function of angle between fiber and LED axis for different distances between fiber and LED is shown in the left.
A measurements by the manufacturer of the angular homogeneity is shown on the right.}
\label{fig:AngularDistributionOfLED}
\end{figure}

The sensitivity of the PIN-diodes is significantly smaller than that of the PMTs.
That's why, in order to have enough light intensity to see a clear signal with PIN-diodes, one needs as small as possible
distance between the LEDs and the PIN-diode. However, since the fibers sit around the PIN-diode case and due to geometrical constrains 
(the diffuser has certain space requirements in order to fit in the limited space inside the ATLAS shielding) 
the angle between the fibers and the LEDs - PIN-diode axis cannot be very large because then
the fibers have to be bent too much which can damage them during the detector installation.

The final design had a 6 mm distance between the PIN-diode and the LEDs and a 30\degree fiber angle.

The different sensitivity of the PMT and the PIN-diode led to another limitation caused by the dynamic range of the LUCROD board.
% TODO I am not sure if this sentence is correct... Probably Vincent had something else in mind.
The dynamic range is the maximum possible input voltage (after amplification) at which the read-out card has an output that is linear and thus not saturating.
The dynamic range of the LUCROD card is 1.5 V. 
The input signal are amplified with a low noise amplifier with an amplification factor of 14, as discussed previously. 

The maximum possible amplitude of the PMT signals which can be handled by the electronics without any saturation is therefore slightly above 100 mV.
This introduce limitation for the diffuser because the intensity of the LED cannot be too high so that it produces larger than 100 mV PMT signals.
However, for this LED intensity the signal from the PIN-diode will be very small and barely measurable.
It was therefore needed to suppress the amount of light which goes to the fibers while keeping high intensity for the PIN-diode.
The solution was to place a ring of optical filters which covers the fibers but not the PIN-diode.

Two sets of filters with the optical densities 0.15 and 0.6 were used. 
Measurements with many possible combinations of filters were done in the laboratory.
The setup was the same as showed in \FigureRef{fig:AngularMeasurementSetup}. The distance between the LED and the PIN-diode 
and the fiber angle were set to the values 
decided to be used in the diffuser. 
Filter were inserted between the LED and the fiber to reduce the amount of light picked up by the fiber.
Measurements were repeated three times for each filter configuration. The results of the measurements are showed in \TableRef{tab:FilterChoice}.
It was decided to choose a combination with one filter with 0.15 optical density and one with 0.6 which gave a PMT signal amplitude of 85.33 $\pm$ 0.29 mV.
This amplitude is slightly smaller than the threshold value of 100 mV in order to have a safety margin in case fibers 
are slightly off from the nominal position in the diffuser.

\begin{table}[bp]
  \begin{tabular}{l|c}
    Filter configuration & Signal amplitude [mV]\\
    \hline
    2x0.15       	&	147.90	$\pm$	1.05	\\
    3x0.15       	&	129.67	$\pm$	0.72	\\
    4x0.15       	&	106.47	$\pm$	1.77	\\
    1x0.6          	&	109.93	$\pm$	0.12	\\
    1x0.6 + 1x0.15 	&	85.33	$\pm$	0.29	\\
    1x0.6 + 2x0.15 	&	61.67	$\pm$	0.62	\\
    1x0.6 + 3x0.15 	&	41.83	$\pm$	0.35	\\
    2x0.6	        &	30.13	$\pm$	0.41	\\
  \end{tabular}
  \caption{Results of the measurements in the laboratory with a LED diffuser prototype to choose proper combination of optical filters to be used.}
  \label{tab:FilterChoice}
\end{table}

The LED diffuser had to be placed on top of the so-called shielding monoblock, where radiation levels are expected to be low compared to the ares 
close to the beampipe where LUCID sits.
However, some amount of radiation will be present also in the location of the LED diffusers during operation.
No estimation of the radiation hardness of the filters were done which is why it was decided to make one diffuser with filters and another one without.

% \begin{figure}
% \centering
% \includegraphics[width=.7\textwidth]{LUCID/LED_diffuser_schematic.pdf}
% \caption{Schematic drawing of LED diffuser.}
% \label{fig:LEDDiffuser}
% \end{figure}

\subsection{The laser diffuser}
\label{subsec:laserDiffuser}
% TODO Read here: https://www.rp-photonics.com/numerical_aperture.html

The stability of the light source used for PMT gain monitoring is a crucial factor in the calibration procedure.
Instead of relying on only one light source from LEDs it was decided to also use laser light
provided and monitored by the Tile calibration system. 
To distribute the light between the PMTs one cannot use the same diffuser as for LED light, 
since laser light is very well collimated which is not the case for the LEDs.
In order to handle laser light a new diffuser was made which is shown in \FigureRef{fig:laserDiffuserSchematics}.
% TODO personaly I don't like this phrasing... because it sound that diffuser is just air gap... but in my opinion diffuser is air gap + ferryl connector 
% with 48 fibers
The diffuser connects a fiber bundle of 48 quartz fibers encased in a ferrule connector of 2 mm diameter with a single fiber with 0.6 mm diameter
since the laser light is delivered from the Tile calibration system by a single fiber.

\begin{figure}
\centering
\includegraphics[width=.6\textwidth]{LUCID/laserDiffuserAirGap_v2.pdf}
% TODO need to move top text from picture to the main text
\caption{Schematic drawing of a laser diffuser that couples a fiber bundle to the single quartz fiber which delivers laser light from the Tile calibration system.}
\label{fig:laserDiffuserSchematics}
\end{figure}

\begin{figure}
\centering
\begin{subfigure}{.45\textwidth}
  \centering
  \includegraphics[width=0.9\linewidth]{LUCID/mapping_bundle1_color.png}
\end{subfigure}%
\begin{subfigure}{.45\textwidth}
  \centering
  \includegraphics[width=0.9\linewidth]{LUCID/mapping_bundle2_color.png}
\end{subfigure}
\caption{(left) Picture of a laser diffuser with the fiber bundle which delivers laser signals to PMTs on the side A detector. 
Numbers correspond to PMT numbers which the fiber is connected to. 
Two fibers from a bundle are connected to each PMT. 
Fiber pairs are divided into three categories based on the distance from the center of the connector to the closest fiber 
in a pair: central (orange color), intermediate (yellow) and peripheral fiber pair (green).
(right) Laser diffuser for the side C detector.}
\label{fig:laserDiffuserMapping}
\end{figure}

Light in the fiber undergos multiple total internal reflections in the interface between the fiber core and the cladding. 
Due to this, the light from the fiber will come out within a certain cone. 
The size of the cone is characterized by the numerical aperture $NA$ of the fiber which is given by

\begin{equation}
\label{eq:numericalApperture}
 NA = n \sin{\theta_{max}} = \sqrt{n_{core}^2 - n_{cladding}^2}
\end{equation}

where $n$ is the refractive index of the medium (air in our case), $n_{core}$ is the refractive index of the fiber core, $n_{cladding}$ is the refractive index 
of the cladding and $\theta_{max}$ is half-angle of light cone. 
% TODO Vincent suggest to remove this sentence... need to rethink is it worth to let it be.
% That's why light spot produced by light from fiber will become larger with distance from fiber edge. 
To distribute the light from the single fiber to the fiber bundle one need to introduce some air gap between them that depends on $NA$ 
as shown in \FigureRef{fig:laserDiffuserSchematics}.
The fiber used to deliver the laser light has a numerical aperture of 0.22, which corresponds to a 3.1 mm air gap distance 
in order to cover the 2 mm surface of the diffuser by a light spot (according to \EquationRef{eq:numericalApperture}).
The light received by each PMT have to be similar for all photomultipliers.
However, the intensity of light is not constant within the light spot and dedicated measurements were needed to determine the optimal air gap distance.
Two conditions had to be met:
\begin{itemize}
 \item The light has to be evenly distributed between PMTs,
%  TODO rephrase ``dimmest channel'', see comments bny Vincent
 \item A preference is given for a configuration in which the dimmest channel is given as much light as possible.
\end{itemize}
% TODO is this sentence correct???
Two fibers in each bundle goes to each PMT. 
For small distances between the fiber with the laser signal and the fiber bundle one expected the largest intensity for the central 
region of the diffuser, while for the peripheral region one expects the lowest intensity.
The diffuser is divided into three regions: central, intermediate and peripheral. 
If one of the fibers from a pair is in the central or intermediate region, another fiber from the pair was placed in the most peripheral ring of the diffuser.
If both fibers in a pair are in the peripheral region they were placed as close as possible to the center of the diffuser.
Pairs which include one fiber in the central region are called central fibers and marked in orange color in \FigureRef{fig:laserDiffuserMapping}.
Pairs with one fiber in the intermediate region are marked with yellow and pairs marked with green color have both fibers in the peripheral region.

Measurement were done with different air gap distances for fiber pairs from each category. 
Due to the identical structure of the fiber bundle on side A and side C (or bundle 1 and 2), detailed measurements
were done only for one bundle and only a few points were measured with the second bundle to verify the results of the first one. 
The amplitude of the PMT signals were measured by the oscilloscope in the ATLAS experimental hall
and the results are shown in \FigureRef{fig:laserDiffuserDistanceTest}. 
PMT 1 represent PMTs with a central fiber pair, PMT 7 an intermediate pair and PMT 16 and 19 peripheral fibers for bundle 1 and 2. 
With increasing distance the signal amplitude is decreasing for central fibers as expected. 
% TODO Vincent suggest to remove this sentence... but probably he said to not later - have to check comments in my notebook...
Intermediate and peripheral fibers has maximums. Maximum for peripheral fibers correspond to the distance of 4 cm.
Homogeneity within all categories is acceptable at 4 cm, so this distance was chosen for the final version of the diffuser.
% TODO conclusions above are very short... Make it longer!!!
% TODO describe some parameters of Tile system with reference to the atlasGeneral

\begin{figure}
\centering
\includegraphics[width=.6\textwidth]{LUCID/laserDiffuserDistanceTest.pdf}
\caption{Measurement of the PMT signal amplitude as a function of air gap distance for different categories of fibers. Fiber pairs connected to PMTs 1 represent 
central fiber pairs, PMT 7 intermediate pairs, PMTs 16 and 19 to peripheral pairs.}
\label{fig:laserDiffuserDistanceTest}
\end{figure}

\subsection{PMTs with Bi-207 source}
\label{subsec:bi207Calibration}

To use LED or laser light sources as a reference for the PMT gain monitoring system one has to be sure that intensity of the light delivered to the PMT 
will stay constant over a long period of time. In other words one has to be sure that the light source itself is stable and the condition of 
the optical fibers which deliver light to the PMTs stays the same.
An alternative option which is robust against the effects above is to use a radioactive source.

It was decided to use a Bi-207 source because it provides monoenergetic electrons from an internal conversion process 
with energies above the Cherenkov threshold in quartz.
In order to use Bi-207 for gain monitoring one needs to put the radioactive source close to the PMT.
A set of measurements with Bi-207 radioactive sources were done in a test set-up and it was found that the peak from the electron source 
(shown in \FigureRef{fig:pulseheight_Bi207}) corresponds to approximately 30 photo electrons~\cite{Alberghi:2016tad}.

\begin{figure}
\centering
\includegraphics[width=.6\textwidth]{LUCID/Bi_amplitude_distribtion.png}
\caption{Pulseheight distribution of Bi-207 radioactive source signals measured in a test with a Hamamatsu R760 PMT.}
\label{fig:pulseheight_Bi207}
\end{figure}

% TODO new text: cross check grammar
A radioactive source was enclosed in a circular case with 25 mm diameter 
(the source itself is a disc with a diameter of 5 mm enclosed in a plastic film) as shown in \FigureRef{fig:Bi207_case}.
During measurements the source was put in contact with the PMT quartz window.
The size of the source is larger than the size of the PMT quartz window (25 mm compared to 10 mm respectively) which led to an observed dependence of the shape of the charge and pulseheight distributions (such as in \FigureRef{fig:pulseheight_Bi207}) with respect to the relative position of the source 
with respect to the center of the PMT quartz window.
It led to difficulties in positioning sources in the same way for all sets of PMTs.
Despite this issue, the main problem with using Bi-207 source was the fact that the source has a 2.5 times larger size than the quartz window
and it could therefore not be used in the final detector.
It was decided to use liquid Bi-207 source and put one drop on the surface of the quartz window which eliminated geometrical issue as well.
In order to prevent contamination by the source a special cap was glued on top of the PMT.
 
\begin{figure}
\centering
\includegraphics[width=.6\textwidth]{LUCID/Bi207_CERN_upperPart_v2.png}
\caption{Schematic picture the of case of a Bi-207 radioactive source used in the measurements during the characterization of the PMTs~\cite{Alberghi:2016tad}.}
\label{fig:Bi207_case}
\end{figure}

\subsection{Calibration strategy during 2015-2016}
\label{subsec:calibPerformance}

LED and laser calibrations could be influenced by the two following effects:
the stability of the light source and the wavelength of both LED and laser which were different from the wavelength of Cherenkov radiation. 
The quantum efficiency of the cathode depends on the wavelength and this efficiency can potentially change during PMT aging
differently for different wavelengths.

The purpose of the calibration is to keep the gain of the PMT constant with time.
Due to the fact that luminosity during 2015-2016 are higher compared to 2011-2012
aging effect become visible already after a few hours of the beam collisions.
The strategy was to make calibration runs after each physics run.
After the calibration run the truncated mean of the charge distribution from the Bi-207 sources were measured for each PMT.
These values directly correlate with the gain of the PMTs. 
In order to keep the gain of the PMT constant it is enough to keep the truncated mean of the charge distribution constant.
After each calibration run the mean charge for each tube is compared to a reference value and if there is a significant difference a program 
automatically adjusts the high voltage in order to correct the PMT gain.

This procedure was used for 2015 and 2016 and was shown to be highly effective and robust.
In \FigureRef{fig:hv_trending_plot_2016} the high voltage change for Bismuth PMTs during 2016 are shown. 
Voltage was increased up to 170 V for some tubes.

[ToASK: ask Vincent for table of HV (or time) vs. integrated luminosity].

The LED calibrations were not as succesful in keeping the gain constant, results were overestimated all the time.
A few possible effects can help to explain it. First of all, the wavelenght of the used LED light are diferent from the Cherenkov radiation 
(which is prodused both in physics runs and from Bismuth source).
Quantum efficiency of the photocathode depends from the wavelenght of the incoming light and it probably can depends from PMT aging.
One possibility to proove this explanation can be to make a dedicated test in the laboratory to study behaviour of the PMT with different sources of light, which have different wavelenght.
Another possible effect can be radiation damage of the optical fibers which deliver LED light or activation of the material of the ferrule connector, which mounts the fiber to PMT 
(as shown in \FigureRef{fig:LucidDrawing} (right)). LED calibration for the FIB detectos behave in the same way, which make assumption about radiation damage of the fibers more stronger.
The reason of the effect is still not clear and studies are ongoing.

The laser calibrations were not studied much since the laser source is controlled by the Tile subsystem and require 
coordination with the Tile sub-detector group to share this resource between the two detectors.

\begin{figure}
\centering
\begin{subfigure}{.5\textwidth}
  \centering
  \includegraphics[width=\linewidth]{LUCID/hv_2016_sideA_trendingPlot.png}
  \label{fig:sub3}
\end{subfigure}%
\begin{subfigure}{.5\textwidth}
  \centering
  \includegraphics[width=\linewidth]{LUCID/hv_2016_sideC_trendingPlot.png}
  \label{fig:sub4}
\end{subfigure}
\caption{Trending plot of high voltage change for Bismuth PMTs on side A (left) and side C (right) during 2016.}
\label{fig:hv_trending_plot_2016}
\end{figure}

% TODO: explain why Bi-207 calibration is preferred calibration method and only more or less understood.

% TODO 
% describe how we did Bi-207 calibration during 2015-2016 years
% attempt to do LED calibration
% situation with Laser calibration


% \section{Detector performance}
% \label{sec:DetPerf}
% 
% *** some text here ***
% 
% \subsection{Luminosity measurements}
% \label{subsec:lumMeas}

\section{Temperature dependence of the PMT gain and the temperature tolerance of calibration fibers}
\label{sec:tempMeas}

During the first long LHC shutdown (LS1) between 2013 and 2015 the beampipe at the interaction point of ATLAS
had to be removed and to be replaced with a new one made of aluminum instead of stainless steel.
The new material was chosen to minimize the induced radioactivity of the beampipe (see \FigureRef{fig:metalSupportAndTempProbes} (left)).
The new beampipe also had a reduced aperture in the inner detector region to allow for a new pixel inner barrel layer (IBL) of the inner tracker.

\begin{figure}
\centering
\begin{subfigure}{.6\textwidth}
  \centering
  \includegraphics[width=0.95\textwidth]{LUCID/PMT_metal_support.jpg}
\end{subfigure}%
\begin{subfigure}{.4\textwidth}
  \centering
  \includegraphics[width=0.95\linewidth]{LUCID/temperatureProbes_1_v2.jpg}
\end{subfigure}
\caption{Metal support for the PMT tubes (left). Positions of the temperature probes close to the beam pipe flange (right).}
\label{fig:metalSupportAndTempProbes}
\end{figure}

In order to remove residual gas molecules from the inner walls of the new beampipe (which otherwise will lead to significant beam-gas interactions during operation)
the beampipe had to undergo a so called ``bake-out'' process, which consists of heating up the walls of the beam pipe from the outside.

Since LUCID sits close to the beampipe there was a need to understand the temperature tolerance of the detector components.
During the design phase a special cooling of the detector was made in order to protect PMTs, cables and calibration fibers against potential overheating.
The PMTs were placed on a special metal support which were cooled down by water pipes.
Fibers and cables goes along the beampipe for a few meters and were more exposed to heat than PMTs.
% TODO redo a little bit following sentence
They are also protected with cooling pipes but it was practically impossible to provide cooling along the whole path of the fibers 
and so there are some areas were fibers are not cooled down by the water pipes.

Special studies was done in order to understand the temperature tolerance of the quartz fibers used in the detector and ths is described in the current section.

% TODO Describe temperate probes... probably somewhere before!!!

Another temperature test described in this section was focused on understanding the temperature dependence of the PMTs gain during their operation.
% In current section measurements of temperature dependence of PMT gain and behaviour of quartz fiber at high temperatures will be discussed.
% TODO ask Davide how freuqent we archive temperature from sensors.
% Another test described in this section was focused on understanding how big 
% temperature can quartz fiber, used to provide LED and laser calibration signals, 
% tolerate during the bake-out procedure.
% TODO insert references to the sub-sections here?
% [TODO: describe scheme of temperature sensonrs installed in LUCID].

\subsection{Temperature controller}
\label{subsec:tempController}

In order to perform any temperature tests one need to have a reliable method to measure and control temperature in the testing area.
Measurements with PMTs are typically done in a light tight black box to make sure that no external light will accidentally get on the PMT photo-cathode.
The black box is well sealed which prevent air from the box to circulate outside which make it easy to control temperature inside the box with good 
precision.
A dedicated temperature controller, based on the Arduino Mega 2560 microcontroller~\cite{arduino} (programmed by Arduino software) was built.
To measure temperature the LM35CAH sensor was used which is a precision integrated-circuit device with an output voltage linearly-proportional to the 
temperature C\degree.
In order to interface the sensor with the readout input channel from the Arduino board an electrical circuit 
(see \FigureRef{fig:tempReadOutCircuit}) has been made, 
in order to match the output voltage from the sensor to the readout of the Arduino board.
As a heater element a simple chain of resistors were used which were dissipating heat produced by the current flowing thorough them. 
The voltage on the resistors was controlled by a controller with help of a MOSFET transistor. 
In order to verify that the temperature will be homogeneous within the black box,
a pc fan was used. A heating profile was programmed in the Arduino microcontroller and 
the measured temperature values were sent directly to a personal computer by a serial port or were stored on an external microSD card 
that was connected to the Arduino board.
 
\begin{figure}
\centering
\includegraphics[width=.95\textwidth]{LUCID/LM35_TempSensor_cutted.png}
\caption{Electrical circuit of connection temperature sensor LM35CAH to the Arduino read-out.}
\label{fig:tempReadOutCircuit}
\end{figure}
 
\subsection{The PMT gain dependence}
\label{subsec:pmtGainTempDep}

It is known that a PMT is more sensitive to ambient temperature than ordinary 
electronic components (such as capacitors and resistors)~\cite{hamamatsu}.
It is caused mainly by two factors: the cathode quantum efficiency is sensitive to 
temperature variations and gain of the dynode chain depends on the temperature as well.

To estimate the temperature effect on the PMT gain a dedicated measurement was done. 
A R760 Hamamatsu PMT was placed in the black box together with a Bi-207 radioactive source
to provide stable input signals over time. The temperature in the box was controlled by the temperature controller described in 
\SectionRef{subsec:tempController}.

The interior of a PMT is at vacuum and heat conducts through it very slowly temperature gradient has to be very small in order to make sure 
that the temperature of the tube reaches the same level as the ambient (measured) temperature.
In order to satisfy this condition the temperature gradient was chosen to be 0.2\degree~C per hour.

% TODO link. Also put some information about v1720 (probably from Vincents note).
% TODO explain what is the difference between this one and the one which is used in final verison of LUCRODs.
% TODO probably make it not here.
The PMT signal was digitized by a 12 bit VME Flash ADC which was operated by the ATLAS central Trigger and Data Acquisition (TDAQ) framework.
Ir order to record only signals from the Bi-207 source, a triggering was done that require a signal above a certain threshold.
The digitized pulse height of every triggered event was stored in ROOT files.
The shape of the digitized pulses from a Bi-207 source is shown at \FigureRef{fig:bi207DigitizedPulse}. 

\begin{figure}
\centering
\includegraphics[width=.7\textwidth]{LUCID/rndmPulse_1_mod.pdf}
\caption{Shape of a digitized pulse from a Bi-207 source. For pulseheight measurement the baseline has to be measured and subtracted in a region without signals.}
\label{fig:bi207DigitizedPulse}
\end{figure}

In order to calculate the charge and the amplitude of the Bi-207 signal, the baseline had to be subtracted. 
To estimate the baseline value the last 30 (out of 80) FADC samples were used as shown in \FigureRef{fig:bi207DigitizedPulse}.

The recorded rate of the Bi-207 signals was around 150-200 Hz. In order to collect enough statistics to measure the mean of the charge and 
the amplitude distributions every measurement was done for 5 minutes. 
After one measurements was finished another one was immediately started.
Due to the very small temperature gradient (0.2\degree~C per hour) the temperature within one measurement was considered to be stable.

The mean of the charge distribution and the temperature as a function of time is shown in \FigureRef{fig:PMTChargeTempDep}.
Black points represent the measurements of the ambient temperature. The temperature was slowly increasing and the total change was 6\degree~C over 30 hours.
Red points shows the mean of the charge distribution in 5 minute intervals.
A clear decreasing trend is observed which represent the temperature dependence of the PMT gain.
The measured temperature dependence was 0.25 $\%$ of gain per 1\degree~C. 

\begin{figure}
\centering
\includegraphics[width=.7\textwidth]{LUCID/goodSlowTemp_Feb20_Feb22_charge.eps}
\caption{The temperature dependence of the PMT gain represented by measurements of the mean of the charge distribution the Bi-207 source signals 
for different temperature values.
Black markers correspond to the temperature measurement; red markers correspond to mean of charge distribution of Bi-207 signals collected for 5 minutes.}
\label{fig:PMTChargeTempDep}
\end{figure}

\subsection{Bake-out tests of the calibration fibers}
% From https://indico.cern.ch/event/315665/contributions/1690160/attachments/605816/833747/LS2_LHC_paper_rev_final-1.pdf
% 
% In 2015 year during first long shutdown (LS1) a central ATLAS beampipe made from Ferum (???) 
% was changed with a new one made from Beryllium and Aluminium to better deal with material activation and background ~\ref{Baglin:1967027}.
% New beampipe has to be purified in order to prevent dust particles appear inside the pipe.
% Beampipe is heated up to 350 \degree~C for a few hours. This procedure is called beampipe bakeout.
% 
% The LUCID detector sits on supporting structure (gray structure in \FigureRef{???})
% 
% dedicated cooling has to be installed to protect PMTs and calibration fibers.
% To understand how efficient cooling is needed 
% one need to know temperature robustness of detector components.
% % TODO what about temperature robustness of PMTs? do we know level at which they start to destroy? why we only did this test for fibers?
% % Becuase used PMTs in detector are quite small it makes them quite easy to cool down. 
% Calibration fibers goes from PMTs close to beampipe all along to LED and laser diffusers which are placed on top on monoblock structure 
% %  TODO some other picture???
% (see \FigureRef{fig:LucidDrawing} (left)).
% That's why fibers can be potentially overheated at some parts. 
% In order to understand which temperaure fibers can handle dedicated measurement was done. 

The goal of this test was to estimate the temperature threshold at which fibers started to loose their optical properties
due to heat damage of the cladding and/or the fiber core.

The experimental setup consisted of a PMT placed in the black box, a LED source and a fiber bundle.
Light was transmitted from the LED to the PMT by the fiber.
The fiber bundle was placed in the thermoinsulated box, while the LED and the PMT were outside and not heated.

A high ambient temperature was obtained inside the box with the help of a thermogun, which was blowing hot air into the box constantly.
The temperature was measured with the Arduino-based temperature controller described in \SectionRef{subsec:tempController}, with the temperature probe inside the box.
% TODO do we really use Arduino-based temperature controller in this test?

The first set of measurements were done at room temperature inside the box in order to use this value as a reference (first phase of the experiment).
In the second phase the thermogun was switched on and the temperature raised to around 95\degree~C.
This temperature was kept constant for slightly more than one hour to make sure that the fiber was exposed to a high temperature for
a long enough time period to make any changes observable. 
In the third phase the power of the thermogun was increased further and the temperature of the air was raised to 110\degree~C and kept at this level for one hour.
In the last phase the thermogun was switched off and the measurements were done for another half an hour.
% TODO find out which rate of LED was used in the test
%      I put 1 kHz, cause it's most likely number,
The LED was pulsed by pulse generator with a rate of 1 kHz and the FADC was triggered 
with the same signal in order to make sure that only signals originating from the LED source were stored.
Every measurement was done for a 5 minute interval and the next one was started as soon as the last one had finished.
The mean of the charge distribution was calculated for each measurement and is shown as function of time with red points in \FigureRef{fig:fiberBakeOut}.
Black points correspond to the temperature measurements done during the same time.
% TODO see orange question sign on page 22 on first Vincent comments
The charge measurements during the second phase (when the temperature in the box was 95\degree~C) are compatible with measurements done during the first phase 
(at room temperature) and they are stable during all of phase two (which was one hour long) which shows that fibers can operate normally at this temperature.
During the third phase (110\degree~C) a significant decrease of the measured charge is observed, which demonstrates a change in the
optical properties of the fiber. In the last phase the temperature went back to room temperature and the charge increased but didn't reach 
nominal values which indicate a unrecoverable damage of the fiber at the 110\degree~C temperature.

\begin{figure}
\centering
\includegraphics[width=.95\textwidth]{LUCID/fiberBakeOut.png}
\caption{Measurements of the mean of the charge distribution of LED signals which goes through a heated fiber bundle.
Black points correspond to the temperature of the fiber bundle; red points correspond to the mean of the charge collected in 5 minute intervals.}
\label{fig:fiberBakeOut}
\end{figure}

% TODO rephrase a little bit here...
This test demonstrated that the calibration fibers could be damaged if they will be exposed 
to temperatures above 95\degree~C for long period of time.
% TODO probably it's better to remove this...
% But due to fact that bake-out procedure was planned to be done with around 220\degree~C and early
% simulations showed that expected temperature on LUCID area can be around 140\degree~C 
% % TODO find proof of this!!!
% efficient cooling for fibers became mandatory requirement for the system.
% 
% Decision was to use water cooling. 
% To make cooling of PMTs efficient dedicated metal support was designed to host PMTs shown in \FigureRef{fig:PMTMetalSupport}.
% Water pipes were positioned under this support.
% To protect calibration fibers from overheating they were bundled and glued around cooling pipe which was placed along 
% the beampipe (see \FigureRef{fig:temperatureProbes_VJCone}).
% 
% Also to provide circulation of air in the detector region additional air cooling was installed. 
% [according to Giulio air circulation was essential in order to make sure that some gas will not get into PMTs???]
 
\subsection{Temperature conditions during the beampipe bake-out and detector operational period}

The LUCID detector was equipped with 18 temperature sensors per side which were installed (\FigureRef{fig:metalSupportAndTempProbes} (right) 
and \FigureRef{fig:temperatureProbes_VJCone})
to monitor the temperature of the detector components during the bake-out procedure of the new ATLAS beampipe and during the detector operation.
Temperature from all sensors were recorded and archived every 10 seconds.

\begin{figure}
\centering
\includegraphics[width=.99\textwidth]{LUCID/temperatureProbes_2.jpg}
\caption{Positions of temperature probes along carbon support cone.}
\label{fig:temperatureProbes_VJCone}
\end{figure}

The temperature from the sensor placed in the cable tray and at the beam pipe flange during the beampipe bake-out is shown in \FigureRef{fig:NEG_ctflangeA}.
The temperature were well within the safety margin and didn't exceed 40\degree~C. The temperature from other sensors didn't exceed 40\degree~C as well.
It demonstrated the high efficiency of the beampipe insulation and the LUCID cooling system and no damage to the LUCID detector components were done.

\begin{figure}
\centering
\includegraphics[width=.8\textwidth]{LUCID/NEG_ctflangeA.png}
\caption{The temperature reading from the temperature probes during the bake-out procedure.}
\label{fig:NEG_ctflangeA}
\end{figure}
 
% Temperature of the sensors during data-taling:
After the bake-out procedure had finished the temperature was constantly monitored to make sure that there was no large fluctuations in the cavern.
A few sensors were installed close to the PMTs and their measurements can be treated as an approximate temperature of the PMT tubes.
% TODO insert picture mentioned in the sentence below!
The temperature during the data-taking period in 2015 and 2016 years are shown in \FigureRef{fig:tempTrendingPlot}. 
There are no significant fluctuation over the entire period and they are all within 1\degree~C.
According to the measurements described in \SectionRef{subsec:pmtGainTempDep} the PMT gain dependence in temperature correspond
to 0.25 $\%$ per 1\degree~C, which is negligible for the observed temperature variations during the data-taking period.

\begin{figure}
\centering
\includegraphics[width=.9\textwidth]{LUCID/pmt1_sideA_temperature_2015_2016.png}
\caption{Temperature trending plot recordered by T01 probe (side A, close to one of the Bismuth PMTs)
for 2015 (purple) and 2016 (green) years. This plot demonstrates that the temperature is stable in 
the experimental cavern during physics runs and fluctuations are smaller than one degree.}
\label{fig:tempTrendingPlot}
\end{figure}

% TODO describe:
% - water cooling - see in the figure
% - additional air cooling by blowing air with fans to make circulation of air around lucid detecor
% - temperature probes installed - add picture with probes
% - temperature plots during bakeout procedure
% TODO insert picture which show that LUCID sits on VJ cone around beampipe

\section{The LUCID performance and luminosity measurements during 2015-2016}
\label{sec:lucid_performance}

The LUCID detector have different sets of sensors which make possible to LUCID provide many independent measurements of the luminosity by many algorithms.
Electronics provides luminosity measurements using 124 different algorithms which are based on combinations of signals from different tubes.
This section covers the luminosity algorithm related topics such as description of the luminosity counting method, absolute calibration of the luminometers, 
results of the luminosity measurement in 2015 and 2016 and overview of the main systematic effects and uncertainties.

% The luminosity is measured by LUCID from a measurement of the number of PMT-hits, the number of bunch crossings 
% with at least one PMT-hit and the integrated pulseheight (charge). These measurements are done over a time 
% period called a luminosity block, which are typically 1 minute long and they are done for each of the 
% individual bunch crossings in the LHC. 
% The new electronics provides luminosity measurements using 124 different algorithms which take as input 
% different combination of hits or charge from different tubes. Algorithms, which are based on PMT-hits from only 
% one of the detectors (either A or C), are calculated by the LUCROD VME custom made boards, while algorithms which depend on 
% a combination of hits from both detectors are calculated by the LUMAT boards (see 
% \FigureRef{fig:Eletronics_schematics}).
% The luminosity is proportional to the measured charge and to the logarithm of the measured 
% % TODO *** add log formula for lumi ***
% number of PMT-hits *** add log formula for lumi ***. The two types of measurements therefore have different limitations. 
% The main issue with the 
% charge measurement is PMT gain stability while the hit measurements can suffer from pile-up of several signals 
% below threshold giving a signal above threshold.

% Figure~\ref{fig:hitCount} shows the number of PMT-hits from different LHC bunches. The large peaks correspond 
% to six trains of 
% six colliding bunches each, plus two isolated colliding bunches. Two smaller peaks that correspond to bunches 
% with only one beam are also seen. The baseline background level is due to the Bi-207 source used for monitoring 
% of the photomultiplier gain. 



\subsection{Luminosity algorithms}
% TODO change ``process of interest'' to ``inelastic collisions''
The rate of the inelastic processes in hard collisions at the collider experiment can be expressed as
\begin{equation}
R_{inel} = \mathscr{L} \sigma_{inel}
\label{eq:simpleLumi}
\end{equation}
where $\sigma_{inel}$ is an inelastic cross section and $\mathscr{L}$ is a luminosity delivered by the collider.
A typical task in particle physics experiments is to measure the cross section of some specific process of interest.
The number of collisions observed by the detector where process of interest took place over some period can be expressed as:
\begin{equation}
N_{process} = \int_{t_{1}}^{t_{2}} \mathscr{L} \sigma_{process} dt = \sigma_{process} \int_{t_{1}}^{t_{2}} \mathscr{L} dt
\label{eq:simpleLumi2}
\end{equation}
where $\mathscr{L}$ is the same luminosity as in \EquationRef{eq:simpleLumi}.
The cross section of the process of interest will look like:
\begin{equation}
\sigma_{process} = \dfrac{N_{process}}{  \int_{t_{1}}^{t_{2}} \mathscr{L} dt}
\label{eq:sigma_proc}
\end{equation}
To measure the cross section of process of interest one need not only count number of events where process took place but also know the luminosity of the used dataset.
This is valid for any type of studied process. 
That is why precision measurements of the luminosity is an important task for the experiment.
At the LHC collider a few pp interactions can occur in one bunch crossing. Number of the interactions can vary from bunch crossing to bunch crossing and by the assumption this number is distributed by the Poisson law with the mean $\mu$ - the average number of interactions per bunch crossing.
Taking into account revolution frequency of the LHC, $f_{LHC}$, \EquationRef{eq:simpleLumi} will transform to
\begin{equation}
\mathscr{L} = \dfrac{f_{LHC} \mu}{\sigma_{inel}}
\label{eq:lumi_average_mu}
\end{equation}
Luminometers, used to measure the number of interactions, have a limited efficiency and acceptance, 
that is why they can provide only visible number of interactions ($\mu^{vis}$) which is the number of interactions 
times detector acceptance and efficiency of the method ($\mu^{vis} = \varepsilon \mu$).
In the same way it is possible to introduce the visible cross section ($\sigma^{vis}$) which is the inelastic 
cross section times acceptance and efficiency ($\sigma^{vis} = \varepsilon \sigma_{inel}$).
Using notions above \EquationRef{eq:lumi_average_mu} will look:
\begin{equation}
\mathscr{L} = \dfrac{f_{LHC}}{\sigma^{vis}} \mu^{vis}
\label{eq:lumi_bunch_sum_visible}
\end{equation}
There are two numbers needed for the luminosity calculation: a visible number of interactions ($\mu^{vis}$) and a visible cross section ($\sigma^{vis}$). 
Former number are measured by LUCID and other luminometers while latter is measured in the special runs which will be described in \SectionRef{subsec:alfa_run}. 
The number of protons in bunches can be different and can vary up to 20$\%$~\cite{Aad:2013ucp}, that is why the number of interactions has to be measured separately for each bunch crossing. Each bunch pair is identified numerically by a Bunch-Crossing Identifier (BCID) which is set for each of 3564 possible 25 ns slots in one full revolution of the LHC fill.
LUCID electronics were designed to perform counting separately for each BCID. 
Taking this into account the luminosity can be expressed as:
\begin{equation}
\mathscr{L} = \dfrac{f_{LHC}}{\sigma^{vis}} \sum_{j=1}^{n_{b}} \mu_{j}^{vis}
\label{eq:lumi_bunch_sum}
\end{equation}
where $n_{b}$ is the number of colliding bunches and $\mu_{j}^{vis}$ is the average number of interactions per BCID.
Due to the injection process not all slots can be occupied by bunches and the maximum possible number of colliding bunches equal to 2808.
The luminosity is measured over a certain period which is called a luminosity block (LB) and typically it is equal for one minute in the ATLAS experiment. 

The luminosity can be measured in three different ways with the LUCID detector: by counting number of events, counting number of hits or measuring total charge collected by the PMTs. When particle, created in the pp inelastic collision, penetrate the quartz window a clear signal from the PMT can be seen and if it is larger than a specific threshold value we call it a hit. If there is a hit in the at least one PMT in the detector, we call it event. LUCID consists from two identical detectors placed from both sides (side A and side C) of the interaction point. 
If there is at least one hit in the side A (side C) detector, ``Event ORA'' (``Event ORC'') take place. ``Event OR'' require at least one hit in either detectors, while ``Event AND'' require at least one hit in both side A and side C detectors simultaneously. These four combinations are used in the same way in a hit counting algorithms where the number of PMT hits are counted instead of the number of events.
LUCID electronics count number of bunch crossings during one LB in which hits or events are present.
Let’s have a look on the ``Event OR'' algorithm. If one assume that number of pp interactions in one bunch crossing follow Poisson distribution, the probability to observe an event which satisfies the ``Event OR'' criteria can be computed as
\begin{equation}
P_{``Event OR''} (\mu_{OR}^{vis}) = N_{OR} / N_{BC} = 1 - e^{-\mu_{OR}^{vis}}
\label{eq:poissonProb}
\end{equation}
where $\mu_{OR}^{vis}$ is the mean of the distribution, $N_{BC}$ is the number of bunch crossings with a certain BCID during one LB and $N_{OR}$ the number of bunch crossings 
(with the same BCID) where ``Event OR'' took place. One can express $\mu_{OR}^{vis}$ from \EquationRef{eq:poissonProb}
\begin{equation}
\mu_{OR}^{vis} = -ln( 1 - \dfrac{N_{OR}}{N_{BC}})
\label{eq:logFormula}
\end{equation}
There is no analytical solution for $\mu$ for ``Event AND'' algorithms, that is why it can only be solved numerically. 

Luminosity measurements can be used to monitor beam conditions as well.
For example, it is useful during so-called emittance scans when LHC make optimizations of beam parameters.
It requires fast luminosity measurements over short period, that is why LUCID provide additional set of luminosity measurements every second, which are called instantaneous luminosity.
Instantaneous luminosity measurements are used for specialized LHC tests when ATLAS detector does not collect data. In order to not occupy resources of the central trigger processor (CTP), which distribute LHC clock to all sub-detectors, instantaneous luminosity measurements were implemented in a way that LUCID can provide them by running in standalone mode.
In this case LHC clock is not used by LUCID electronics and LHC clock emulator is used instead.
It means that LUCID does not know which slots are occupied by bunches and when they arrive.
That is why instantaneous luminosity measurements are called to be BCID blind.

\begin{figure}
\centering
\includegraphics[width=.99\textwidth]{LUCID/lumiAlgs_2015.jpg}
% TODO update this caption and label
\caption{List of luminosity algorithms provided by the LUCID detector in 2015. Only a subset of algirithms are stored in the COOL database.}
\label{fig:InternalConsistency}
\end{figure}

\subsection{The absolute $\sigma^{vis}$ calibration}
\label{subsec:alfa_run}
Let's go back to the luminosity formula. The only unknown component in \EquationRef{eq:lumi_bunch_sum_visible} is the visible inelastic cross section $\sigma^{vis}$.
This cross section is measured in van der Meer runs where the absolute luminosity are measured directly from beam parameters ~\cite{vanderMeer:296752,Rubbia:1025746}.
The absolute luminosity for centered bunches, with respect to each other, can be expressed with transverse proton density functions of two beams ($\rho_{1}(x,y)$ and $\rho_{2}(x,y)$) as:
\begin{equation}
\mathscr{L} = f_{LHC} n_{p1} n_{p2} \int \rho_{1}(x,y) \rho_{2}(x,y) dx dy
\label{eq:lumi_vs_intensity}
\end{equation}
where $n_{p1}$ and $n_{p2}$ are the number of protons in each of two colliding bunches.

The number of protons in bunches (which is equivalent to beam currents) are measured by special direct current (DC) transformers: 
a pair of DC current transformers (DCCT) and a pair of fast beam current transformers (FBCT). 
DCCT is used to measure the total beam current while FBCT is used to measure the relative intensity of separate bunches.

Van der Meer scan is performed by changing the relative position of beams in one direction while keeping them centered in another direction.
By doing such the scan in the both directions and measuring the interaction rate as a function of the beam shift one can obtain two scan curves (with curve widths $\Sigma_{x}$ and $\Sigma_{y}$). One of such curves, normalized to the number of protons in beams, 
is shown in \FigureRef{fig:vdmScanCurve}.
The peak luminosity during van der Meer run can be written as:

\begin{figure}
\centering
\includegraphics[width=.7\textwidth]{LUCID/vdM_scan.pdf}
\caption{Visible interaction rate for the LUCID algorithm (BI$\_$OR$\_$A) in one bunch crossing and per unit bunch population, 
	 versus nominal beam separation during horizontal scan 1 in the August 2015 luminosity-calibration session. }
\label{fig:vdmScanCurve}
\end{figure}
\begin{equation}
\mathscr{L}^{peak} = f_{LHC} n_{p1} n_{p2} \int \rho_{1}(x,y) \rho_{2}(x,y) dx dy =  f_{LHC} n_{p1} n_{p2} \dfrac{1}{2\pi \Sigma_{x} \Sigma_{y}}
\label{eq:lumi_vdm}
\end{equation}
By comparing \EquationRef{eq:lumi_bunch_sum_visible} and \EquationRef{eq:lumi_vdm} one can show that the visible cross section equal to:
\begin{equation}
\sigma^{vis} = \dfrac{ 2 \pi \Sigma_{x} \Sigma_{y} \mu_{max}  }{ n_{p1} n_{p2} }
\label{eq:sigma_vis}
\end{equation}
And, by recalling the definition of the visible cross section, one can express efficiency of the luminosity algorithm as:
\begin{equation}
\varepsilon = \dfrac{\sigma^{vis}}{\sigma^{ineal}}
\label{eq:alg_method_eff}
\end{equation}
In \TableRef{tab:sigma_vis_efficiency} measured visible cross sections and efficiencies for some LUCID luminosity algorithms are shown.
The last column in the table contain the maximum possible number of interactions which algorithms can deal with without saturation.
These numbers were calculated for values of $\mu_{vis} = 10$, which was chosen arbitrary.
All counting algorithms can start to saturate if number of interactions became large enough. As example let's take a look on the ``Event OR'' algorithm.
If in all bunch crossings during LB at least one hit will be detected by LUCID (which correspond to ``Event OR'' definition) 
the ratio $\frac{N_{OR}}{N_{BC}}$ will be equal to 1 and it will be impossible to use \EquationRef{eq:logFormula} to calculate 
the number of interactions $\mu$. In the \FigureRef{fig:peakMuByFill} one can see that the maximum number of interactions per beam crossing 
for some physics runs in 2016 were higher than 40, which demonstrate that BI$\_$OR$\_$A algorithm, which was mainly used in 2015, can potentially saturate in some LB.
% TODO describe how often vdM runs are done and explain that parameters of the beams have to stay more or less constant.

\begin{table}[p]
  \begin{tabular}{|l|c|c|c|}
    Algorithm & $\sigma^{vis}$ (mb) & $\varepsilon$ ($\%$) & $\mu_{max}$ \\
    \hline
    BI$\_$OR & 32.4 & 40.5 & 24 \\
    \hline
    BI$\_$OR$\_$A & 19.3 & 24.2 & 41 \\
    \hline
    BI$\_$AND & 6.38 & 8.0 & 125 \\
    \hline
    BI$\_$OR$\_$C9 & 6.44 & 8.0 & 125 \\
    \hline
    MOD$\_$OR & 21.7 & 27.1 & 36 \\
  \end{tabular}
  \caption{The result of the vdM calibrations for some of the LUCID luminosity algorithms.}
  \label{tab:sigma_vis_efficiency}
\end{table}

\begin{figure}
\centering
\includegraphics[width=.7\textwidth]{LUCID/peakMuByFill.pdf}
\caption{The maximum number of inelastic collisions per beam crossing during 2016 physics runs.}
\label{fig:peakMuByFill}
\end{figure}


\subsection{The first 13 TeV collisions at the LHC}
\label{sec:physics}

The LHC reported the first 13 TeV pp collision in May of 2015 and these were recorded by ATLAS and other LHC experiments. 
Starting from that time the new LUCID was succesfully operating and provided information about the 
luminosity delivered to ATLAS. 
% More than a femtobarn of luminosity is already recorded by ATLAS, which provides a
% sufficient amount of data about the performace of the LUCID detector with a high number of pp-interaction per bunch 
% crossing.

The PMT pulseheight distribution in a physics run is shown in the left plot of \FigureRef{fig:Pulseheight} (blue) 
together with the same distribution 
during a Bi-207 calibration run (red). In both distributions a peak due to Cherenkov photons is visible. The 
calibration distribution is cut due to the threshold in the electronics that define a PMT-hit.

\begin{figure}
\centering
\begin{subfigure}{.5\textwidth}
  \centering
  \includegraphics[width=\linewidth]{LUCID/LowMuWithBi_ampl_preliminary_v2.pdf}
  \label{fig:sub1}
\end{subfigure}%
\begin{subfigure}{.5\textwidth}
  \centering
  \includegraphics[width=\linewidth]{LUCID/ComparisonPhysRuns_ampl_preliminary_v2.pdf}
  \label{fig:sub2}
\end{subfigure}
\caption{Comparison of pulseheight distribution in a physics run with low-$\mu$ with the same distributions 
during a Bi-207 calibration run (left) and with distributions during a high-$\mu$ physics run.}
\label{fig:Pulseheight}
\end{figure}

In the right plot of \FigureRef{fig:Pulseheight} a comparison of the pulseheight distributions in a physics run 
with low-$\mu$ (red) and high-$\mu$ (blue) are shown. The pulse height is shifted towards higher values when at high 
luminosity several particles traverse the photomultiplier window in the same bunch crossing.
A second peak correspond to events with two particles going through the PMT window is clearly seen.

LUCID can measure luminosity in many ways and \FigureRef{fig:InternalConsistency} shows a comparison of the 
luminosity measured by the A and C detector in different ATLAS data taking runs. The two measurements agree to better than 0.5$\%$.

\begin{figure}
\centering
\includegraphics[width=.7\textwidth]{LUCID/InternalConsistency_preliminary.pdf}
\caption{Fractional difference in measured luminosity between the forward (A) and backward (C) arms of the LUCID 
detector. The agreement between the two LUCID arms is better than 1$\%$.}
\label{fig:InternalConsistency}
\end{figure}

The left plot of \FigureRef{fig:LumiVsTime} shows a measurement of the average number of inelastic pp collisions 
using different ATLAS 
luminometers and the right plot shows the ratio of this measurement with respect to a LUCID measurement. One of the 
detectors shows a deviation of up to 2$\%$ during this LHC fill but the other measurements are all in agreement 
with LUCID to better than 0.5$\%$.
The first month of data taking with the new detector therefore showed that LUCID could measure the relative 
luminosity with a precision of about 0.5$\%$.

\begin{figure}
\centering
\begin{subfigure}{.5\textwidth}
  \centering
  \includegraphics[width=\linewidth]{LUCID/LumiVsTime_preliminary_v2.pdf}
  \label{fig:sub3}
\end{subfigure}%
\begin{subfigure}{.5\textwidth}
  \centering
  \includegraphics[width=\linewidth]{LUCID/DeviationSubsystems_preliminary_v2.pdf}
  \label{fig:sub4}
\end{subfigure}
\caption{(left) Average number of inelastic proton-proton collisions per bunch crossing during a 13 TeV fill; 
(right) Comparison of the measured luminosity by different luminometers in ATLAS with respect to LUCID.}
\label{fig:LumiVsTime}
\end{figure}

\subsection{Luminosity measurements in the ATLAS experiment during 2015-2016}
\label{subsec:lumi_2015_2016}
During 2015-2016 the preferred luminosity algorithm was based on LUCID BI photomultiplier, PMTs with the Bismuth radioactive source, which were the most stable PMTs due to the radioactive source used for the PMT gain stability monitoring.
Beside the LUCID detector there are other detectors which can provide luminosity measurements. They are: 

\begin{itemize}
 \item The beam condition monitor (BCM) consists from four small diamond sensors (~1 cm$^2$ in the cross section) which perform hit counting and provide luminosity on a bunch-by-bunch basis same as LUCID. It is placed at $|\eta| = $ 4.2.
 \item The Tile calorimeter consists from iron plates and plastic scintillators. 
 Signal from latter are read by PMTs. Particle flux can be estimated by measuring current drawn by these PMTs. 
 It can not provide the luminosity for each bunch crossing, only the luminosity summed over all colliding bunches, because the current are read out every 10 ms. Calorimeter covers the pseudorapidity range $|\eta| < $ 1.7.
 \item The Electromagnetic end-cap calorimeter (EMEC) and the forward calorimeter (FCAL) cover the pseudorapidity range 1.375 $< |\eta| < $ 3.2 and 3.2 $< |\eta| < $ 4.9 accordingly. 
 They consist from adsorbers with gaps filled with liquid Argon. High voltage is distributed to the gaps by inserted sets of electrodes.
 HV drops induced by particle flux is counterbalanced by a continuous injection of electrical current, which is proportional to the particle flux and thereby provide a 
 relative luminosity measurements~\cite{Aaboud:2016hhf}, which are BCID blind. 
 \item The Inner detector covers $|\eta| < $ 2.5 and performs track and vertex reconstructions. 
 The luminosity is measured by counting number of reconstructed primary vertices.
 \item Minimum Bias Trigger Scintillators (MBTS). Scintillator detector which was designed specially 
 for low luminosity runs (with instantaneous luminosity $\mathscr{L} < 10^{33} cm^{-2}s^{-1}$). 
 Cover rapidity range 2.09 $< |\eta| < $ 3.84. Measured luminosity is BCID blind.
\end{itemize}

The variety of detectors and methods to measure luminosity provide a great handle to cross check measurements and estimate systematic errors as a difference between results from different detectors.

As for any other detector, LUCID measurements are monitored constantly by the ATLAS shift crew and on-call experts. 
Many online checks are implemented to verify that detector and electronics conditions are stable and quality of the collected data are good. 
The detector development is still constantly ongoing along with detector operation because many unexpected new issues can show up. 
One of such cases for LUCID was related to the photomultiplier transit time. 

The central trigger processor makes the overall level 1 accept decision and also provides the central ATLAS clock, distributed via the Timing, Trigger and Control (TTC) system to sub-detectors. 
Time which signal propagate to the different sub-detectors are different. That is why readout electronics have a programmable delays which are used to compensate timing differences. It is the case for LUCID as well. 
This delay can be used to position the signal inside the 25 ns time window as well. 
Once this delay was set it was expected that it will be constant and the signal will not move with respect to the border of the BCID. 
As described in \SectionRef{subsec:calibPerformance} in order to keep the gain of PMTs stable daily calibrations runs are performed and automatic high voltage adjustment takes places. 
For 2015 running period high voltage was increased up to 100V for some PMTs.
Apparently 100 V increasement of high voltage significantly changed the transit time in PMTs (up to 6 ns). 
It led to the effect that part of signals moved outside of the timing window and where not detected, which led to decreasing efficiency of the detector.
The problem was different for different photomultipliers. There was one PMT (BI$\_$C9) [??? it was BI$\_$A9 instead?] which was hardly affected. 
This PMT was used to derive transit-time correction for other PMTs.
Effect of correction can be seen in \FigureRef{fig:transitTime} where fractional difference of the ratio of LUCID measurements to the track counting measurements with and without transit-time corrections are shown.  
The effect started to become visible in the late of October, were 1.5 $\%$ drop in luminosity ratio is observed.
%% TODO to ask Vincent to send one of the same plot but as a function of HV, the one which was used to calculate correction for LUCID luminosity.
The ratio in \FigureRef{fig:transitTime} was plotted as a function of high voltage and was fitted by the linear function to the data to obtain transit-time correction.
This correction significantly improved the consistency of the LUCID data. 
In order to prevent this effect in future a special cross check was implemented for the LUCID data quality monitor.
Program define the position of pulses (part of which are stored in the online histograms) 
and if the peak of the pulse is too close to the BCID border an automatic warning appears and notify on-call expert that delay has to be adjusted.

\begin{figure}
\centering
\includegraphics[width=.7\textwidth]{LUCID/LuminosityRatio_NoTimeCorr_Preliminary.pdf}
\caption{ 
Fractional difference of the measured luminosity per run between LUCID BI$\_$OR$\_$A and track-counting algorithms in 2015.
By the end of 2015 high voltage for some PMTs were changed up to 100 V, which caused to the significant changing of the transit time of electrons in the tubes.
As a result part of signals moved outside the timing windows and in turn decreased efficiency of the detector.
Some PMTs were barely affected and they were used to derive transit-time correction.
The black circles correpond to the LUCID data with these corrections and the red squares without.
}
\label{fig:transitTime}
\end{figure}

\subsection{Backgrounds}
\label{subsec:backgrounds}
In order to make correct measurements one need to be sure that signals, detected by detector, were originated only from proton collisions from current bunch crossing.
Any other signals which originates from other sources (or from pp collisions from previous bunch crossings) are considered as a background.
The first possible background is the signal from the Bismuth source used in PMT gain monitoring system.
During the design phase one of requirements for the source was that its activity has to be small to not significantly affect luminosity measurements.
Effect of the Bismuth source during vdM scans is shown in \FigureRef{fig:vdmScanCurve}. Effect is on permille level comparing with the signal.
The luminosity for typical 25 ns physics run are 20 times higher, which make effect of the Bismuth source completely negligible. 

During Run-1 period it was observed by both BCM and LUCID detectors that there is some activity detected in the BCIDs following immediately after a collision~\cite{Aad:2011dr,Aad:2013ucp}
and it was called an afterglow background. It is most likely caused by photons from nuclear de-excitation, which happens because of hadronic showers from pp interactions, which excite detector material ~\cite{Aaboud:2016hhf}.
In Run-1 50 ns train bunches were used which means that there was one empty bunch slot between filled bunches in the train.
This empty slot was used to estimate the afterglow background.
25 ns trains, used in Run-2, does not have empty bunch slots because all slots are filled with bunches, which make it impossible to use the same approach as for Run-1.
For this purpose so-called template method was invented. The idea is to measure afterglow ``template'' using data from single colliding bunches (as shown in \FigureRef{fig:afterglow}).
By assuming that every colliding bunches in the train produce the same afterglow background one can add up templates for each colliding bunches according to the train structure and estimate the total afterglow background caused be the bunch train. As it can be seen from \FigureRef{fig:afterglow} the maximum rate of afterglow background for BI$\_$OR counting algorithm is at level 
3$\times10^{-5}$. The maximum possible train in the LHC can consist from 144 bunches. After detailed studies it was concluded that total afterglow rate is not larger than $10^{-4}$ and is 
negligible for physics runs.
\begin{figure}
\centering
\includegraphics[width=.7\textwidth]{LUCID/afterglow.png}
\caption{Afterglow background for LUCID BI$\_$OR$\_$A counting algorithm for single colliding bunches. It it used as a ``template'' for the so-called template method, which was design
to address afterglow background estimation for bunch train with 25 ns distance between bunches.}
\label{fig:afterglow}
\end{figure}

\subsection{Overview of systematic effects of the luminosity measurements}

The main approach, used by ATLAS, to investigate systematic effect is to compare measurements of several luminosity detectors which use different algorithms to measure luminosity.
As described in \SectionRef{subsec:lumi_2015_2016}, different detectors have different acceptance coverage, sensitivity and methods to measure luminosity, which means different detector behavior and response to the pile-up and beam-induced backgrounds.

In \FigureRef{fig:mu_dependence} ratio of the average number of pp interactions per bunch crossing measured by Tile Calorimeter to that of BI$\_$OR$\_$A LUCID measurements is shown.
Values have been normalized in a way that ratio has to be equal 1 for the vdM fill.
One can observe a clear $\mu$-dependence in both 50 and 25 ns runs. Effect is equal to 0.1$\%$ of ratio decreasement for one unit of $\mu$ for 50 ns runs and approximately 0.2$\%$
decreasement for 25 ns runs.
% TODO ask Vincent for possible reasons of this (sentence above)?
A similar dependence have been observed with respect to the track counting algorithm as well.
One of the effects which contribute to the $\mu$-dependence is the assumption that the probability of an individual pp interaction to give a hit (or event) does not depend on the number of interactions in the bunch crossing, which was made during the derivation of the \EquationRef{eq:logFormula}.
But in case of many pp interactions signals from separate proton collisions which do not provide large enough signal to reach the threshold value can sum up and give a hit in the end. 
This effect is known as a pile-up (or migration) problem.
It potentially can be present in any counting algorithms because they rely on definition of the hit - a signal larger than a certain threshold.
Algorithms which rely on measurements of values proportional to the $\mu$, such as current (charge) from the PMTs, in case of Tile calorimeter, or track counting does not suffer from this effect.



\begin{figure}
\centering
\includegraphics[width=.7\textwidth]{LUCID/mu_dependence.png}
\caption{The ratio of measured average $\mu$ by the Tile calorimeter to the measurements by LUCID BI$\_$OR$\_$A algorithm. Data is scaled in a way to provide ratio equal to 1 for vdM scan.
This figure demonstrate $\mu$-dependence of the LUCID counting algorithm.}
\label{fig:mu_dependence}
\end{figure}

Another effect which can contribute to the $\mu$-dependence is related with LUCID PMT gain decreasement.
After each physics run high voltage for PMTs are changed in order to compensate for the decreasement of the PMT gain (see \SectionRef{sec:pmtGainMonitoringSystem}).
A test was performed to investigate a profile of the PMT gain decreasement during the runs. Due to fact that there is no possibility to monitor 
PMT gain online during the run the idea was to mimic physics run with help of LED light. A long LED run was done along during which many LED calibrations were done as well.
The mean of the integrated pulse charge distribution as a function of time is shown in \FigureRef{fig:pmt_gain_profile}. Every point in the plot correspond to one LED calibration.
This plot represent PMT gain decreasement during the long physics run. The gain is decreasing very rapidly over short period in the beginning of run and later it is decreasing slowly.
Taking this into account as well as that average $\mu$ is slowly decreasing over the run (due to the fact that the number of protons in the beams is constantly decreasing with time) it will lead to additional contribution to the $\mu$-dependence effect.
% TODO write here explanation why mu-dependence is not corrected...

\begin{figure}
\centering
\includegraphics[width=.6\textwidth]{LUCID/pmt_gain_profile.png}
\caption{Mean of the charge distribution of the pulsing constant LED light as a function of time. The trend represent change of the PMT gain during the test.
LED setup was used in order to mimic normal conditions during physics run.}
\label{fig:pmt_gain_profile}
\end{figure}

By extrapolating curves of 25 and 50 ns runs to low-$\mu$ values one can see that extrapolation from high-$\mu$ region will not match with values obtained in vdM runs.
It means that the calibration constant $\sigma^{vis}$ (measured in vdM runs) is not directly applicable for high-$\mu$ regime, which was the case for 2012 as well~\cite{Aaboud:2016hhf}.
In order to correct for this effect so-called calibration transfer corrections have to be applied for LUCID counting algorithms.
To estimate this correction for LUCID one can use measurements from calorimeter-based luminometers or track counting.
In 2015 Tile calorimeter luminosity was showing inconsistency between its luminosity algorithms that is why track counting was used to derive the corrections.
Runs which had both Tile and track counting luminosity available and which were happened before the vdM scan were used to made comparison of average luminosity for LUCID and track algorithms
and to derive corrections from it. [ToAsk: is statement above correct?] Calibration transfer correction was found to be equal 1.2$\%$ for 50 ns and 2.5$\%$ for 25 ns runs in 2015.

% TODO explain that aging of PMTs become slower with age. Show one of the Vincents plots.

Taking into account all systematic effect of different luminometers one has to choose the preferred algorithm, which is most stable, in order to be used as a reference for high-$\mu$ runs.
For both 2015 and 2016 LUCID Bismuth algorithms were used as references. In \FigureRef{fig:runByRunStability} run-to-run stability of measurements from other detectors with respect to the 
LUCID measurements in 2015 are shown. 
All luminometers were cross-calibrated to a LUCID measurement in order to have the same integrated luminosity in the reference fill, which is shown by the arrow in the figure.
The relative variations in measurements of all ATLAS luminometer has been found to be within $\pm$ 2.5$\%$ level.

\begin{figure}
\centering
\includegraphics[width=.7\textwidth]{LUCID/RunByRunStability.pdf}
\caption{
Fractional difference of the measured luminosity per run between LUCID BI$\_$OR$\_$A algorithm and other luminometer measurements in 2015.
All luminometers are cross-calibrated with respect to the LUCID measurements in a way that luminosity reported by them in a reference run (indicated by arrow) are
equal to one of the LUCID detector.
}
\label{fig:runByRunStability}
\end{figure}

\subsection{Estimation of total systematic errors}

\subsection{Future prospects of the LUCID measurements}

As described above, luminosity counting algorithms suffer from number of effects. Most of these effects were discovered during Run-1 operation.
That is why design of the new LUCID detector for Run-2 was aimed to minimize these effects but they still can be observed with the new detector.
That is why it was decided to implement the possibility to measure the charge collected by PMTs for each bunch crossing.
Charge measurements do not suffer from the migration effect ($\mu$-dependence) and they are promising luminosity algorithms, because
LUCID could make measurements for each BCID, when other ATLAS detectors which exploit charge for luminosity measurements, such as Tile calorimeter, 
can only provide average values over all BCIDs. Unfortunately, charge luminosity algorithms suffer from other effects.
They are far more sensitive to the PMT gain variation than counting algorithms.
If PMT gain decrease for 1$\%$ luminosity measured with ``Event OR C'' algorithm will decrease only for ???$\%$.
For charge algorithm the same drop of the gain will affect luminosity for ???$\%$.
That is why a lot of effort was made to develop the robust monitoring system with a few independent methods as described in \SectionRef{sec:pmtGainMonitoringSystem}.
The PMT gain can decrease up to a few $\%$ during one long physics run (especially for new PMTs), that is why in order to have better precision information about the PMT gain change during the physics run would be useful.
One possible way to obtain this information can be to pulse the LED light to PMTs during so-called ``forbidden gap'' where no bunches are present.
Signals from LED can be used in the same way as in PMT gain monitoring system to estimate the change of the gain.

Another critical effect for the charge method is the afterglow background, which was described for counting algorithms in \SectionRef{subsec:backgrounds}.
For 25 ns physics runs this affect can reach up to 3$\%$, which make it very important to develop precise subtraction procedure of the background.
Possibility to use template method for charge algorithm is currently under study.




%   \chapter{Searches for beyond Standard Model physics with same-sign dileptons}
\label{chap:SS}
\section{Motivation}
% ******************************************************************************
% TODO
% - write something about lepton signature... why it has benefits wrt to jets and photons...
% - write which models we are looking for. how signal looks like with comparison with background
% - write about benefits of the general search (or model-independent? or it's dangerous to say like this?)... theorists can use these results for their models?
% - motivation flow: why leptons -> why same-sign dilepton -> why model-independent search?
% ******************************************************************************

Despite the great success of the Standard Model in high-precision description of interactions of the elementary particles with high precision, there is a plethora of phenomena
which cannot be explained by the SM. Gravity, Dark Energy and Matter, neutrino masses and other observations are not predicted by the
SM, which indicates its incompleteness.
% TODO comment from Monika
This is why many studies at LHC experiments are focused on probing the new energy regime to search for new physics, 
its observation or non-observation, which can prove or reject the validity of different extensions of the SM.

Many BSM models predict same-sign dilepton in the final state.
% TODO maybe ``clear'' not ``clean''?
It is a very clean signature as only few SM processes predict such final state.
This leads to high sensitivity for new physics.
% TODO some more ATLAS 8 TeV papers with same-sign leptons?

% TODO comment from Monika
There is a number of searches in ATLAS which target same-sign dilepton final state
signatures, either inclusive ones or searches with additional requirements such as the number of jets or missing transverse energy~\cite{heavy_majorana_neutrino_paper,floderus_paper,Aad:2014pda}.
% There is number of inclusive searches which target same-sign dilepton final state as well as searches with additional requirements on number of jets or missing transverse energy in ATLAS~\cite{heavy_majorana_neutrino_paper,floderus_paper,Aad:2014pda}.
The analysis presented here aims to be as inclusive as possible and to probe exclusively the same-sign dilepton signature and all the BSM models which predict it.

The signal selection is based on minimal kinematic requirements on leptons.
Electrons are very clean objects which can be precisely reconstructed with high efficiency in ATLAS.
The search observable for the analysis is the same-sign dilepton invariant mass distribution.

If no signal from new physics is found, an upper limit on fiducial cross section for new physics can be set.
As an example of a BSM model with the same-sign dilepton final state, the pair production of doubly charged Higgs bosons is studied in detail within the search presented in this thesis.

The analysis searches in three channels: $e^{\pm}e^{\pm}$, $\mu^{\pm}\mu^{\pm}$ and $e^{\pm}\mu^{\pm}$.
This chapter will describe the search only for the $e^{\pm}e^{\pm}$ channel.

\section{Background processes}
\label{sec:wprimeBackgrounds}
% ******************************************************************************
% TODO
% - describe that there are hree possibilities to reconstruct same-sign lepton pair
% - 
% ******************************************************************************

There are four different sources of background which contribute to the signal region.
The first source is the so-called prompt background which corresponds to prompt same-sign dileptons originating from the SM processes.
Just a few SM processes have same-sign dileptons in the final state. 
For example, they can originate from semileptonic decays of top quarks
and/or leptonic decays of $W$ or $Z$ bosons. The corresponding Feynman diagrams are shown in \FigureRef{fig:prompt_bkg_feynman_diag}.
% TODO is sentence below OK? It was modified after Monika's comment.
In general, SM processes with the same-sign dilepton final state have relatively small cross sections at high dilepton invariant mass region of interest.

The second source comes from wrong identification of the lepton itself, when a pion or a jet is reconstructed as a lepton, or from leptons which were produced 
not in the pp collision but from decays of secondary particles, e.g. kaons. This source is called non-prompt or fake background.

% TODO add trident event Feynman diagram
% TODO bremm - italik
The third source comes from the misidentification of the electric charge of the lepton, which makes processes in which an opposite-sign prompt lepton pair is produced 
contributing to the signal selection as well. The charge misidentification effect becomes significant for high-momentum leptons, when the curvature of the track 
is difficult to reconstruct. 
This source also includes events in which prompt electron emits a photon due to hard bremsstrahlung. Subsequently the photon creates an electron-positron pair in which one of the two leptons receives most of the energy.
One lepton from the pair can form a same-sign electron pair with the original prompt lepton. These two processes are referred to as charge misidentification 
background later in the text.

% TODO rewrite like: 
% from photon conversion which provides a ss pair by a combination of one of the converted lepton with a lepton from $W$ decay.
The last source arises from $W\gamma$ processes, where an electron-positron pair is created from photon conversion, and by combining with an electron from $W$ decay
a same-sign lepton pair can be formed. 
% This source is similar to the charge misidentification background, however, is estimated in a different way, as will be described further in the text.
% It is considered separately from the charge misidentification background because it does not contain prompt opposite-sign electron pairs.

% TODO describe connection of written above with table with all used MC samples...
% TODO normalization is just checked for it is not applied as a scale factor. Cross check with Monika!!!
Prompt background is modelled by the MC simulation. 
% with applying the overall normalization of the MC samples to the data in the control region.
Non-prompt background is derived using a data-driven method, the so-called fake factor method.
Charge misidentification and $W\gamma$ backgrounds are estimated from the MC simulation as well. However, as will be shown in \SectionRef{subsec:CF_definition}, electron charge misidentification probability is not properly described by the MC simulation, this is why a special correction, the so-called charge misidentification scale factor, obtained from the data, is used.

The list of all the MC samples used in the analysis is shown in \TableRef{tab:MC_cross}. 
All listed samples were centrally produced by the ATLAS simulation group.
All other processes which are not listed in the table don't contribute
significantly to the same-sign signal region and are considered negligible.
This table summarizes which MC generators and parton distribution function (PDF) sets were used, and shows an order of cross section calculations.


\begin{table}[ht]
  \begin{center}
    \begin{tabular}{l|c|c|c}

      \hline
Process &  Generator&  PDF set & Normalization \\
&  + fragmentation/ &  & based on \\
&  hadronization & &\\
\hline\hline
\multirow{2}{*}{$WZ$ } &  \multirow{2}{*}{{\scshape sherpa-1.4.1} \cite{Sherpa}} &   \multirow{2}{*}{CT10 \cite{CT10}} & NLO QCD \\
 & & &  with {\scshape mcfm-6.2}\cite{mcfm} \\
\hline
\multirow{2}{*}{$ZZ$}  &  \multirow{2}{*}{{\scshape sherpa-1.4.1}} & \multirow{2}{*}{CT10} & NLO QCD  \\
& & &  with {\scshape mcfm}-6.2 \\
\hline
\multirow{2}{*}{\Wpm\Wpm}  & M{\scshape ad}G{\scshape raph}-5.1.4.8 \cite{madgraph4}  &   \multirow{2}{*}{CTEQ6L1 \cite{cteq}}  &  \multirow{2}{*}{LO QCD} \\
&  {\scshape pythia-8.165} \cite{pythia8}& &\\
\hline
\ttbar $V$, & M{\scshape ad}G{\scshape raph}-5.1.4.8  & \multirow{2}{*}{CTEQ6L1} & \multirow{2}{*}{NLO QCD \cite{top9,ttbarW}} \\
$V=W,Z$ &  + {\scshape pythia-6.426} & & \\
\hline
 MPI $VV$ &  \multirow{2}{*}{{\scshape pythia-8.165}\cite{pythia8}}  &  \multirow{2}{*}{CTEQ6L1} &  \multirow{2}{*}{LO QCD} \\
 $V=W,Z$ &  & & \\[+0.025in]
\hline
\hline
\multirow{2}{*}{$Z/\gamma^* +$ jets} & {\scshape alpgen-2.14} \cite{Alpgen}&\multirow{2}{*}{CTEQ6L1}& {\scshape dynnlo-1.1} \cite{dynnlo} with \\
 & + {\scshape herwig-6.520} \cite{Herwig1, Herwig2}& & MSTW2008 NNLO \cite{mstw} \\
\hline
\multirow{2}{*}{\ttbar} & {\scshape mc@nlo}-4.06 \cite{Mcnlo, Mcnlo2} & \multirow{2}{*}{CT10}&{NNLO+NNLL } \\
& + {\scshape herwig-6.520} & & QCD \cite{top1,top2,top3,top4,top5,top6} \\
\hline
\multirow{2}{*}{$Wt$} & {\scshape mc@nlo-4.06}  & \multirow{2}{*}{CT10}& {NNLO+NNLL } \\
   & + {\scshape herwig-6.520} & & QCD \cite{top7,top8}\\
\hline
\multirow{2}{*}{$W^{\pm}W^{\mp}$} & \multirow{2}{*}{{\scshape sherpa-1.4.1}} & \multirow{2}{*}{CT10}& NLO QCD \\
& &  & with {\scshape mcfm-6.2}\\
\hline
\multirow{2}{*}{$W\gamma$} & \multirow{2}{*}{{\scshape sherpa}-1.4.1} & \multirow{2}{*}{CT10}& NLO QCD\\
& &  & with {\scshape mcfm-6.3}\\
\hline
\end{tabular}
\end{center}
  \caption{List of MC generated samples used for background prediction. 
  The used MC generator, the PDF set and the order of cross section calculations used for the normalization are listed for each sample.
  The upper part of the table contains MC samples which provide same-sign dilepton in the final state
  (MPI stands for multiple parton interactions), 
  while the lower part contains samples which contribute to the signal selection due to electron charge misidentification.}
\label{tab:MC_cross}
\end{table}

\begin{figure}

\begin{subfigure}{.5\textwidth}
  \centering
  \includegraphics[width=\textwidth]{SS/feynman/WZ_electron_v2.eps}
\end{subfigure}%
\begin{subfigure}{.5\textwidth}
  \centering
  \includegraphics[width=\textwidth]{SS/feynman/ttbar_electron_v2.eps}
\end{subfigure}

\caption{Production diagrams for diboson (left) and $\ttbar W$ processes leading to the same-sign dileptons.}
  \label{fig:prompt_bkg_feynman_diag}
\end{figure}


\section{Event selection}
% TODO mention here about size of the dataset and that it was 8 TeV 50 ns data.
% TODO description of the used triggers here!
% TODO [no jet selection is described]

The analysis is based on the pp collision data collected in 2012 by the ATLAS detector with 8 TeV center of mass energy.
The integrated luminosity of the sample corresponds to 20.3 fb$^{-1}$, the mean number of interactions per bunch crossing was 21.

An event selected for the analysis has at least one reconstructed vertex with at least three tracks matched to it. 
If there are several vertices, the one with the highest
$\sum p^2_T$, where $p_T$ are transverse momenta of the matched tracks, is chosen.
Events should have at least two electron candidates and fire the dilepton trigger, that requires the presence of two electrons with $p_T > 12$ GeV.
In order to exclude ambiguities between electron and jet reconstruction, the electrons are required to
be isolated. Isolation is calculated by summing up the particle momenta around the electron candidate within a defined cone size, 
$\Delta R = \sqrt{ (\eta_e-\eta_i)^2 + (\phi_e-\phi_i)^2 }$,
where $\eta_e$ and $\phi_e$ are rapidity and azimuthal angle of the electron candidate while $\eta_i$ and $\phi_i$ are rapidity and azimuthal angle of track or calorimeter cluster, not related to the electron candidate. Detailed explanations of the isolation and other requirements on electron candidates are given later in this thesis.

\subsection{Electron selection}
\label{subsec:electron_selection}
The next step is the selection of isolated high-$p_T$ electrons present in the event.
Electron reconstruction in the ATLAS detector central region ($\eta<2.5$) is done by matching tracks from the inner detector with energy deposits in the EM calorimeter.
The spatial resolution of the electron candidate in ($\eta$,$\phi$) plane is taken from the parameters of the matched track,
while energy is calculated from energy deposits in the EM calorimeter.

% ATLAS use three sets of electron identification criteria, which are ordered by background-rejection power and identification efficiency~\cite{electron_tight}.
% These sets are labelled as loose, medium and tight. Loose set provide has poor background-rejection power while tight - has the best.
% But background-rejection power comes at cost of identification efficiency

% so as to provide increasing
% background-rejection power at some cost to the identi-fication efficiency. 

The electrons of interest are well-reconstructed candidates which satisfy
the following requirements:
\begin{itemize}
 \item $p_T > 20$ GeV: to ensure a high and constant trigger efficiency as a function of $p_T$ and to harmonize $p_T$ requirement between the three analysis channels ($ee$, $\mu\mu$ and $e\mu$) 
 \item $|\eta|<1.37$ or $1.52<|\eta|<2.47$: to be within high-granularity acceptance of the EM calorimeter, but excluding barrel-end-cap transition region.
 \item Electron tracks have to originate from the primary vertex.
 Transverse impact parameter significance which is defined as the ratio of the absolute transverse impact parameter ($d_0$) to its uncertainty parameters ($\sigma(d_0)$) have to be below three. The distance between $z$-coordinate of the primary vertex and $z$-position of the point of closest approach of the electron track in the ID to the beamline to be less than 1 mm.
 These requirements also reject electrons originating from decays of long-lived particles.
 \item Pass ``tight'' electron set of identification criteria defined in~\cite{electron_tight}. \\ 
 This set provides the best background rejection, a factor of two with respect to the medium selection, 
 at the cost of less than 10$\%$ electron efficiency with respect to the medium selection.
 \item No reconstructed jet within $\Delta R < 0.4$: 
 to ensure that electron is not part of a jet.
%  to remove ambiguous classification of the track, 
%  when track is classified as an electron and as a jet simultaneously,
%  or when a jet is too close to the electron, leading to poor reconstruction of the latter. `
 \item Pass isolation requirement: to distinguish prompt electrons from those associated with jet activity.
\end{itemize}

The isolation requirement was chosen in order to reach pile-up independent efficiency of more than 99$\%$ for electrons 
with $p_T >$ 40 GeV. The requirement has two parts.
Firstly, the sum of the transverse energies in the EM and hadronic calorimeters around the electron within 
$\Delta R < 0.2$, with core electron energy subtracted from the sum, has to be less than 
3 GeV $+ (p_T - 20$ GeV$) \times 0.037$
\footnote{
Core electron energy corresponds to the electron energy deposit in the calorimeter in the core cluster, to which electron candidate is assigned. However, not all electron energy is contained in the core cluster, part of it leaks to the neighbor clusters which are used for electron isolation requirement. This effect is taken into account in the isolation requirement formula.}, 
where $p_T$ is electron transverse momentum.
Secondly, the sum of $p_T$ of all tracks with $p_T > 0.4$ GeV within $\Delta R < 0.3$ around the electron track has
to be less than 10$\%$ of electron $p_T$.

Electrons used in the so-called validation regions defined in \SectionRef{sec:bkg_validation}, used for validation of the background estimation, 
are required to fail one or several requirements above but pass looser one, as will be described further in the text.

\subsection{Electron pair selection}
% TODO inv. mass cut; treatment of event with three leptons;
All the electrons which passed the selection are used in the considered lepton pair formation step.
All combinations are considered, and it is allowed to have more than one pair selected in one event.
Electrons in the pair are classified by $p_T$: the electron with a higher $p_T$ is called leading lepton, while another one is the subleading lepton.
The leading lepton has to pass a stricter cut of $p_T>25$ GeV, while the subleading lepton has to satisfy $p_T>20$ GeV as described above.

To avoid low-mass hadronic resonances like $J/\psi$ or $\varUpsilon$ showing up in the invariant mass spectrum, 
only same-sign pairs with $m_{\Plepton\Plepton}>15$ GeV are selected.
Additionally, if same-sign lepton pairs with the invariant mass $70 < m_{\Plepton\Plepton}< 110$ GeV 
which corresponds to the Z peak region, are found in the event, such event is removed.
This region is used for estimation of the electron charge misidentification background as described further.

Since more than one pair is allowed in one event, an additional requirement which significantly suppresses the prompt background contribution is required.
If an event contains an opposite-sign same-flavour lepton pair which satisfies the condition ($|m_{\Plepton\Plepton} - m_{Z}| < 10$ GeV,
where $m_{\Plepton\Plepton}$ is the lepton pair invariant mass and $m_{Z}$ is the mass of Z boson), the event is discarded.

\section{Estimation of the charge misidentification and non-prompt backgrounds}

\subsection{Prompt opposite-sign dilepton with charge misidentification}
\label{subsec:CF_definition}
% TODO
% subsection flow:
% CF sources --> method and used regions --> validation of the method --> deriving scale factors --> appling to specific MC samples

% \toAsk[why we define CF rate in 80..100 GeV region, while we cut 70..110 GeV from signal region?]
% \toAsk[CF at high pT. extrapolation? of one large bin with huge error? look it up!]

Since the charge of one of the reconstructed electrons from the pair can be misidentified, 
processes with opposite-sign dilepton final state can contribute to the signal region.
As the prompt background is relatively small in the signal region, the possibility of a charge misidentification cannot be neglected and has to be precisely estimated. 
Two cases are considered as a charge misidentification background. 
The first one is when the electron charge is truly misidentified due to matching of the wrong track to the EM cluster 
or due to a small curvature of the high-momentum track. 
The second one arises when an electron emits a photon by bremsstrahlung, which in turn decays to an electron-positron pair. 
Instead of the original electron, an electron from the photon decay can be attributed to the pair and can have an opposite charge.

In order to estimate how many events with opposite-sign dileptons contribute to the same-sign dilepton signal region, 
one has to know the electron charge misidentification rate. It is expected that the charge misidentification depends on the electron momentum (due to the track curvature) 
and on $\eta$ (due to the different amount of material in the detector versus $\eta$).

Electrons from Z boson decays provide a huge sample of opposite-sign lepton pairs which can be used to estimate the charge misidentification rate. 
By applying the signal selection criteria
and requiring the invariant mass of the lepton pair to be around the Z mass, $80 < m_{ee} < 100$ GeV, one obtains a pure
sample of electron pairs, where the charge of one electron is misidentified. Knowing the number of opposite-sign pairs which pass the signal selection, 
one can extract the charge misidentification rate.

In \FigureRef{fig:chargeFlip_structure} the charge misidentification rate obtained from MC simulation as a function of $p_T$ is shown.
By using generator level information one can distinguish charge misidentification due to photon conversion from other effects as shown in the figure.
The contribution from photon conversions from bremsstrahlung dominates 
the total charge misidentification rate over the entire $p_T$ range except in the high-$p_T$ region, where misidentification 
of charge due to very small curvature of tracks becomes the dominant effect.

% TODO remove ``ATLAS internal sim'' from picture
\begin{figure}
\begin{center}
 \includegraphics[width=0.7\columnwidth]{SS/support_note/chargeflip/misidratept_ZDY.eps}
\caption{Electron charge misidentification rate, obtained from MC simulation using electrons from Z boson decay. 
Contribution from charge misidentification due to photon conversion is shown separately on the plot and as a fraction of the total rate in the ratio plot below.}
\label{fig:chargeFlip_structure}
\end{center}
\end{figure}

% TODO to describe method??? --> see Monika comment
To verify the MC estimation of the charge misidentification, a special data-driven technique, the so-called likelihood method, 
is used in the same way as in ref.~\cite{same_sign_paper_7tev}.
This method is based on using a maximum likelihood fit to extract the charge misidentification rates in different kinematic regions simultaneously~\cite{cf_top_paper}.
This method is found to be the most precise one due to using most statistics available and providing kinematically unbiased results~\cite{anthony_thesis}
for same-sign dilepton analysis with 7 TeV data.

To cross-check the applicability of the method, it was first used 
in reconstructed MC events, and then the charge misidentification rates were compared with the ones from the generator level (truth) information. 

The comparison demonstrates that the likelihood method provides a very reliable result as shown in
\FigureRef{fig:likelihood_cross_check}.

\begin{figure}
\begin{subfigure}{.5\textwidth}
  \centering
  \includegraphics[width=\textwidth]{SS/support_note/chargeflip/eta_mctruth.eps}
\end{subfigure}%
\begin{subfigure}{.5\textwidth}
  \centering
  \includegraphics[width=\textwidth]{SS/support_note/chargeflip/pt_mctruth.eps}
\end{subfigure}
\caption{Electron charge misidentification rate, obtained from MC simulation using truth (generated) information and the likelihood method. 
Ratio between the rate obtained with the likelihood method to the rate obtained with truth MC information is shown below in the ratio plot. }
\label{fig:likelihood_cross_check}
\end{figure}

The charge misidentification rate as a function of $p_T$ and $\eta$ is shown in \FigureRef{fig:charge_flip_data_vs_mc}
for the MC simulation and the likelihood data-driven predictions.
The dependence of the rate on $p_T$ is well described by MC simulation, while some difference is observed
in high-$\eta$ region. This is why $\eta$-dependent correction factors were calculated as a ratio of the charge misidentification rate extracted from collision date 
to the rate predicted by the MC simulation.

\begin{figure}
\begin{subfigure}{.5\textwidth}
  \centering
  \includegraphics[width=\textwidth]{SS/support_note/chargeflip/misidrate_datamc.eps}
\end{subfigure}%
\begin{subfigure}{.5\textwidth}
  \centering
  \includegraphics[width=\textwidth]{SS/support_note/chargeflip/misidratept_datamc.eps}
\end{subfigure}
\caption{Electron charge misidentification rate, obtained from MC simulation and from real collision data with the likelihood method.
Ratio plot shows charge misidentification scale factor (SF), which correspond to the ratio between rates from data and MC simulation.}
\label{fig:charge_flip_data_vs_mc}
\end{figure}

The charge misidentification background consists of opposite-sign lepton pairs which were reconstructed as same-sign pairs.
The background is estimated from MC simulation (all the considered processes are shown in the lower part of the \TableRef{tab:MC_cross}) 
plus is corrected by a charge misidentification scale factor to properly reproduce the $\eta$-dependence of the charge misidentification rate.

To estimate the systematic error of the derived scale factors, the following uncertainty sources were considered:
\begin{itemize}
 \item The width of the invariant mass window, used to preselect electrons from the $Z$ peak, was varied by 10 GeV.
 The largest difference between the nominal scale factors and scale factor obtained from $Z$ peak window width variation was taken as a systematic error.
 \item The electron isolation requirement was loosened by 4 GeV in both track- and calorimeter-based isolation criteria.
\end{itemize}
Electrons from Z decays have a limited $p_T$-range of up to around 100 GeV. As can be seen in~\FigureRef{fig:charge_flip_data_vs_mc} (right), 
the MC simulation describes the $p_T$ dependence of the charge misidentification rate very well, thus it was decided to rely on MC simulation 
for high-$p_T$ leptons. Additional studies were done to estimate systematic uncertainty in this case.
Special MC samples with varied detector alignment and amount of detector material (with 5-20$\%$ of variations depending on the sub-detector) 
were used in order to estimate the effect on the charge misidentification rate.
% TODO Monika commetns
This resulted in a conservative estimate of a 20$\%$ error, which was assigned to the high-$p_T$ lepton charge misidentification rate.

The total systematic uncertainty of the background prediction is shown in~\TableRef{tab:syst}.


%TODO use one of this: ``scale factor'' or ``scale factorS''
The charge misidentification scale factor is also applied for the MC prediction of the background from $W\gamma$ process
% TODO Monika comment
as the origin of creating a same-sign electron pair is very similar.
In this case the electron is produced in the $W$-decay and the $\gamma$ conversion.
% \toAsk[should I say smth. more about $W\gamma$ process]


\subsection{Non-prompt background}
\label{subsec:fakes_description}

% TODO comment from Monika
Another type of background present after the signal selection is the so-called non-prompt background.
Main sources of this background are jets misidentified as electrons and electrons which do not originate from 
the primary vertex, e.g. electrons from semi-leptonic decays of heavy flavor quarks ($b$, $c$).
To estimate this background a data-driven method, the so-called fake factor method, is used.

The first step of the methods is to define a background region, which does not overlap with the signal region in which the
contribution of non-prompt electrons is dominant, while the contribution from prompt electrons is minimal.
Events in this region must contain exactly one reconstructed electron (probably a jet misreconstructed as electron) with $p_T > 20$ GeV.
To counterbalance the electron, a jet in the opposite azimuthal direction ($\Delta \phi (e,jet) > 2.4$) is required.
By requiring strictly one electron one can make sure that the background region will not overlap with the signal selection and processes 
like Drell-Yan and \ttbar~will be suppressed.
To make sure that the jet and the non-prompt electron are well balanced in terms of energy, the jet is
requested to have $p_T > 30$ GeV.
To suppress contributions from W boson production, a requirement on the transverse invariant mass of 
$m_\mathrm{T}$
\footnote{$m_\mathrm{T} = \sqrt{2 p_\mathrm{T} p_T^{miss} (1-\cos\varphi_{\Plepton\nu})}$, where
$p_\mathrm{T}$ is the transverse momentum of the electron, $p_T^{miss}$ is the transverse missing energy of the event
and $\varphi_{\Plepton\nu}$ is the angle in transverse plane between electron direction and direction of the missing energy.}
$ > 40$ GeV is applied.

In this background region the fake factor $f$ is defined as:
\begin{equation}
f = \frac{N_{\mathrm{P}} - N_{\mathrm{P}}^{\mathrm{prompt}}}{N_{\mathrm{F}}  - N_{\mathrm{F}}^{\mathrm{prompt}}}
\label{eq:fakefactor}
\end{equation}
where $N_{\mathrm{P}}$ is the number of reconstructed electrons in the background region which pass the electron signal selection
described above in \SectionRef{subsec:electron_selection}, and $N_{\mathrm{F}}$ is the number of electrons which do not fulfill the 
signal electron selection requirements but satisfy a looser selection. This looser selection is identical to that for the signal selection, except 
that the electron only has to satisfy ``medium'' set of electron identification criteria~\cite{electron_tight} instead of the ``tight'' one,
and it has to fail the calorimeter- or track-based isolation criterion (or both).
$N_{\mathrm{P}}^{\mathrm{prompt}}$ and $N_{\mathrm{F}}^{\mathrm{prompt}}$ are the numbers (obtained from MC simulation) of real prompt
leptons which pass the signal and looser selections respectively.
The contribution of prompt leptons has to be subtracted in order to make sure that 
the fake factor is evaluated as a ratio of non-prompt electrons that passed the signal selection to
the number of non-prompt electrons that passed the looser selection and there is no contamination from real prompt leptons.
The fake factor measured in this way can then be used to predict the non-prompt background in the signal region.
In order to take into account possible different kinematics of a lepton in the region where the fake factor was derived 
and the region were it will be applied, the fake factor is measured as a function of electrons $p_T$ and $\eta$.
The total number of non-prompt same-sign pairs, $N_{\mathrm{NP}}$, in the signal region can be estimated as:
\begin{equation}
N_{\mathrm{NP}} = \sum_{i}^{\mathrm{N_{P_l F_s}}} f_{\mathrm{s}}(p_{\mathrm{Ti}},|\eta_{i}|) + \sum_{i}^{\mathrm{N_{F_l P_s}}} f_{\mathrm{l}}(p_{\mathrm{T}i},|\eta_{i}|) - \sum_{i}^{\mathrm{N_{F_l F_s}}} f_{\mathrm{l}}(p_{\mathrm{T}i},|\eta_{i}|) \times f_{\mathrm{s}}(p_{\mathrm{T}i},|\eta_{i}|)
\label{eq:fake_pred}.
\end{equation}
The first term corresponds to the number of electron pairs ($N_{P_l F_s}$) 
where the leading electron (denoted by ``l'') passes selection criteria ($P_l$) and the subleading electron (denoted by ``s'')
fails to fulfill it ($F_s$) but passes the looser selection used for the fake factor calculation. 
Contribution of every such pair to the signal region is scaled by the fake factor 
$f_{\mathrm{s}}(p_{\mathrm{Ti}},|\eta_{i}|)$, where $p_\mathrm{Ti}$ and $\eta_{i}$ are transverse momentum and pseudorapidity
of the subleading electron of a pair which fails the signal selection. 
Similarly, the second term represents the number of pairs where
the leading electron fails the signal selection while the subleading one passes it. 
The last term corresponds to the case in which both the leading
and subleading electrons fail the signal selection. This term has to be subtracted to correct for the double
counting of non-prompt electron pairs.

To estimate the systematic errors of the method one can test all the assumptions made in the method and take into account possible differences between 
the region used to derive fake factors and the region in which they are applied. The following sources were considered:
\begin{itemize}
 \item statistical uncertainty of the data sample used to derive fake factors;
 \item prompt MC subtraction, which is done to verify that there is no contamination from prompt leptons when deriving/applying fake factors.
 Due to the luminosity uncertainty of 2.8$\%$ and the uncertainty of MC cross section (7$\%$ on major prompt processes), the prompt MC subtraction
 was varied by 10$\%$;
 \item requirement on the away side jet $p_T$. It was varied up to $>50$ GeV in order to test the dependence of the fake factor on kinematics of the jets faking electrons;
 \item difference in the non-prompt background composition in the region used to derive the fake factors and the region where they were applied.
 Non-prompt background can originate from jets which were created by gluons or light quarks as well as from heavy flavour jets.
 The fake factor depends on the proportion of these two categories of jets in the region. Therefore the fake factors were derived separately
 for heavy and light flavour jets, and the difference between the results was taken as systematic error.
\end{itemize}
A more detailed description of the evaluated systematic uncertainties and the fake factor method can be found in~\cite{anthony_thesis}.

The electron fake factor as a function of electron $p_T$ together with the statistical and systematic errors is
shown in \FigureRef{fig:ff_e_errs}.
\begin{figure}[h]
\begin{center}
%\includegraphics[width=0.48\textwidth]{figures/electrons/electron-syst}
\includegraphics[width=0.65\textwidth]{SS/support_note/electrons/electron_fake_plot.eps}
\caption{Electron fake factor $f$ as a function of electron $p_{T}$. Combined statistical and systematic errors are shown as shaded areas.}
\label{fig:ff_e_errs}
\end{center}
\end{figure} 
% TODO[say that fake factors were also measured not only for signal selection, but also for less strong selections which are used for fake validation regions].

Verification of the fake factor method is done with the help of validation regions which will be described in \SectionRef{subsec:fake_validation}.

% The fake background includes all background processes where at least one of the leptons is fake. The dominant sources are $W$+jets and QCD multijets with smaller contributions arising from $Z$+jets and $t\bar{t}$. To assess this background, a data-driven method, known as the fake factor method, is employed. This method is used in several ATLAS analyses, particularly diboson measurements and searches\footnotemark. The method employed was covered in great depth in the previous 7 TeV incarnation of this analysis.

% \footnotetext{The following analyses, among others, use the fake factor method: SM $WW \ra \ell\nu\ell\nu$ and $WZ \ra \ell\nu\ell\ell$, $H \ra WW \ra \ell\nu\ell\nu$, and exotics $WZ \ra \ell\nu\ell\ell$ search.}

\section{Background validation regions}
\label{sec:bkg_validation}

In order to make sure that all backgrounds are modelled/predicted correctly in the signal region, one can define and use special validation regions.
These regions should be kinematically close to the signal region but should not overlap with it. 
The first validation region tests overall normalization of the background prediction to the data.
Other described regions are designed to test one given background type at a time,
which means that in each of these regions only one background type has to give a dominant contribution in comparison to all other background sources.

\subsection{Prompt opposite-sign dileptons}

The first validation region is defined by using exactly the same event selection as for the signal region,
but requiring the two leptons to have opposite charges. 
This validation region does not test any background specifically.
MC-based background estimation is normalized by the luminosity of the data sample.
This is why overall normalization of the simulation to the data in the opposite-sign region verifies trigger and lepton reconstruction efficiencies.
The correct description of the Z peak shape in data by the MC simulation tests electron scale momentum and resolution.
In \FigureRef{fig:OS_CR} the invariant mass of opposite-sign electron pairs is shown.
\TableRef{tab:dilep_isoOS} gives the observed and the expected number of the electron pairs.
Good agreement between data and simulation is observed.

\begin{table}[htbp]
\begin{center}
\begin{tabular}{l|c}

Process & Number of lepton pairs \\\hline\hline
        Drell-Yan	& $ 4701110 \pm 329108 $	\\[+0.05in]
	$t\bar{t}$	& $ 14580.8 \pm 874.92 $	\\[+0.05in]
	Dibosons	& $ 12210.9 \pm 545.6 $	\\[+0.05in]
	Non-prompt	& $ 8321.28 \pm 244.4 $	\\[+0.05in]
	$W\gamma$	& $ 243.03 \pm 35.2 $	\\[+0.05in]
	MPI	& $ 32.74 \pm 32.74 $	\\[+0.05in]
	\hline
	Total expectation	& $ 4736500 \pm 329109 $	\\[+0.05in]
	\hline
	Observation in data	& $ 4895830 $	\\[+0.05in]
	\hline
	Agreement ($\sigma$) & -0.48 \\[+0.05in]

\hline  
\end{tabular}
\end{center}
\caption{Observed and expected number of lepton pairs for the control region with opposite-sign, isolated leptons.
Agreement between the observed and the expected number of pairs is quoted in the bottom of the table as a fraction of the total uncertainty on the prediction.
} %The significance of the difference between the number of data events observed and that predicted is calculated considering the statistical error on the data and the systematic uncertainty on the prediction.}
\label{tab:dilep_isoOS}
\end{table}

\begin{figure}[h]
\begin{center}
\includegraphics[width=0.65\textwidth]{SS/support_note/dielectrons/crs/OS_mod_v4.pdf}
\caption{Invariant mass of the opposite-sign electron pairs that passed signal selection.
The data is shown as closed circles. The stacked histograms represent the background estimations. 
The last bin is an overflow bin.
}
\label{fig:OS_CR}
\end{center}
\end{figure} 

\subsection{Prompt same-sign dileptons}
% TODO table with final numbers to show how good MC normalization is wrt to data.

The same-sign dilepton prompt background originates dominantly from $WZ$ and $ZZ$ processes, where both Z and W bosons decay leptonically.
In order to check the normalization of these processes, prompt validation region is used.
In a fully reconstructed event, in which one of these processes took place, one can find at least one same-sign and one opposite-sign lepton pair.
In order to enhance the $WZ$ and $ZZ$ contributions in the validation region, at least three leptons are required in the event, where one lepton pair
has to be a same-sign electron pair and the other one - an opposite-sign same-flavour pair (from $Z$ boson decay). 
The invariant mass of the opposite-sign pair has to be close to the $Z$ boson mass $(|m_{\Plepton\Plepton} - m_{Z}| < 10$ GeV$)$.

The expected and the observed numbers of same-sign pairs in this region are listed in \TableRef{tab:promptCR_yields}, 
and the ratios between the observed and the expected number of pairs are summarized in \TableRef{tab:prompt_ratios}. 
The expectations are in good agreement with the observation.

\begin{table*}[htbp]
\begin{center}
\resizebox{\textwidth}{!}{
\begin{tabular}{l|c|c|c|c|c|c|c}
\hline 
Sample & \multicolumn{6}{|c}{Number of electron-electron pairs with  $m(e^{\pm}e^{\pm})$} \\
 & $>15$~GeV & $>100$~GeV & $>200$~GeV & $>300$~GeV & $>400$~GeV & $>500$~GeV & $>600$~GeV \\
\hline \hline
Non-prompt & $49 \pm 14$ & $31.1 \pm 8.1$ & $11.1 \pm 3.0$ & $3.4 \pm 1.3$ & $1.22 \pm 0.72$ & $0.81 \pm 0.63$ & $0.41 \pm 0.44$ \\
\hline
Prompt total & $226 \pm 18$ & $133.8 \pm 9.2$ & $36.7 \pm 3.0$ & $11.6 \pm 1.3$ & $3.44 \pm 0.63$ & $1.15 \pm 0.34$ & $0.38 \pm 0.18$ \\
\hline
$W/ \gamma$ & $0.0 \pm 0.0$ & $0.0 \pm 0.0$ & $0.0 \pm 0.0$ & $0.0 \pm 0.0$ & $0.0 \pm 0.0$ & $0.0 \pm 0.0$ & $0.0 \pm 0.0$ \\
\hline
Charge Flip total & $0.00036 \pm 0.00068$ & $0.0 \pm 0.0$ & $0.0 \pm 0.0$ & $0.0 \pm 0.0$ & $0.0 \pm 0.0$ & $0.0 \pm 0.0$ & $0.0 \pm 0.0$ \\
\hline \hline
Sum of Backgrounds & $275 \pm 23$ & $165 \pm 12$ & $47.9 \pm 4.2$ & $15.0 \pm 1.9$ & $4.65 \pm 0.95$ & $1.96 \pm 0.71$ & $0.78 \pm 0.47$ \\
\hline \hline
Data  & $268 \pm 16$ & $156 \pm 12$ & $46.0 \pm 6.8$ & $14.0 \pm 3.7$ & $6.0 \pm 2.4$ & $3.0 \pm 1.7$ & $1.0 \pm 1.3$ \\
\hline 
\end{tabular}
}
\end{center}
\caption{Expected and observed numbers of pairs for various cuts on the dilepton invariant mass. The uncertainties shown are quadratic sums of the statistical and systematic uncertainties.}
\label{tab:promptCR_yields}
\end{table*}

\begin{table*}[htbp]
\begin{center}
\resizebox{\textwidth}{!}{
\begin{tabular}{c|c|c|c|c|c|c}
\multicolumn{6}{c}{Ratio between observed and expected for $m(e^{\pm}e^{\pm})$} \\
$>15$~GeV & $>100$~GeV & $>200$~GeV & $>300$~GeV & $>400$~GeV & $>500$~GeV & $>600$~GeV \\
\hline \hline
$0.97 \pm 0.09$ & $0.95 \pm 0.10$ & $0.96 \pm 0.17$ & $0.93 \pm 0.27$ & $1.3 \pm 0.6$ & $1.5 \pm 1.0$ & $1.3 \pm 1.9$ \\
\hline \hline
\end{tabular}
}
\end{center}
\caption{Ratio between observed and expected same-sign pairs in the $WZ$ and $ZZ$ control region for various cuts on the dielectron invariant mass. 
The uncertainties account for both statistical and systematic errors.}
\label{tab:prompt_ratios}
\end{table*}

\FigureRef{fig:prompt_CR} shows the invariant mass distribution in the prompt validation region. As was mentioned above, 
the invariant mass $70 < m_{\Plepton\Plepton}< 110$ GeV region has been excluded
to be used for the estimation of the electron charge misidentification background.
The simulation agrees well with data.

\begin{figure}[h]
\begin{center}
%\includegraphics[width=0.48\textwidth]{figures/electrons/electron-syst}
\includegraphics[width=0.65\textwidth]{SS/support_note/PromptCR/ElEl/2isoSS_ee_mll_pr.eps}
\caption{Invariant mass os reconstructed same-sign electron pairs in the prompt background validation region. The last bin is an overflow bin.}
\label{fig:prompt_CR}
\end{center}
\end{figure} 


\subsection{Electron charge misidentification}

As was described above, events with opposite-sign lepton pairs in the final state can be reconstructed as same-sign lepton pairs, 
if charge of one of the leptons was wrongly identified.
Misidentification probability is well modelled as a function of $p_T$ by MC simulation, 
but $\eta$-dependence has to be corrected by scale factors obtained with a data-driven method.
One can make a sanity check, comparing data from $Z$ peak window (same-sign pairs with invariant mass $80 < m_{ee} < 100$~GeV)
with the MC simulation corrected by the charge misidentification scale factors.

The invariant mass of the same-sign pairs within the $Z$ peak window is shown in \FigureRef{fig:charge_flip_CR_inv_mass}. 
The $p_T$ and $\eta$ distributions for the leading electron are shown in \FigureRef{fig:charge_flip_CR_kinematics}.
A good agreement is observed between the data and MC expectations, which demonstrates correctness of the derived charge misidentification scale factor.

The observed and the expected numbers of electron pairs are also shown in \TableRef{tab:ee_isoSS_Z} for all same-sign electron pairs and separately for positively and negatively charged pairs.

\begin{figure}[h]
\begin{center}
%\includegraphics[width=0.48\textwidth]{figures/electrons/electron-syst}
\includegraphics[width=0.65\textwidth]{SS/paper_draft/2isoSS_ee_mll_ssz.eps}
\caption{Invariant mass of reconstructed same-sign electron in the validation region for the charge misidentification background prediction.
Dominant background contribution is arising due to electron charge misidentification.
}
\label{fig:charge_flip_CR_inv_mass}
\end{center}
\end{figure} 

\begin{figure}
\begin{subfigure}{.5\textwidth}
  \centering
  \includegraphics[width=\textwidth]{SS/paper_draft/2isoSS_ee_pt1_ssz.eps}
\end{subfigure}%
\begin{subfigure}{.5\textwidth}
  \centering
  \includegraphics[width=\textwidth]{SS/paper_draft/2isoSS_ee_eta1_ssz.eps}
\end{subfigure}
\caption{Leading electron $p_T$ (left) and $\eta$ (right) distributions in the charge misidentification validation region. The last bin is an overflow bin in the left figure.}
  \label{fig:charge_flip_CR_kinematics}
\end{figure}


\begin{table}[htbp]
\begin{center}
\begin{tabular}{l|c}
\hline
Process & Number of $ee$ pairs \\\hline\hline
%
\multicolumn{2}{c}{\textbf{Same-sign $ee$ $Z$ mass window.}} \\\hline 
        Non-prompt      & $200 \pm 110$ \\[+0.05in]
        Charge Flips & $12400 \pm 1300$ \\[+0.05in]
        Prompt Electrons & $143.4 \pm 8.1$ \\[+0.05in]
        $W\gamma$  & $26.8 \pm 5.6$ \\[+0.05in]
            \hline
        Total Prediction & $12700 \pm 1300$ \\[+0.05in]
            \hline
        Data       &       $11793 \pm 110$ \\[+0.05in]
            \hline
        Agreement ($\sigma$)  &      0.8 \\[+0.05in]
\hline \hline
\multicolumn{2}{c}{\textbf{Same-sign $e^{+}e^{+}$ $Z$ mass window.}} \\\hline 
        Fakes      & $66 \pm 60$ \\[+0.05in]
        Charge Flips & $6380 \pm 670$ \\[+0.05in]
        Prompt Electrons & $82.0 \pm 5.0$ \\[+0.05in]
        $W\gamma$  & $17.5 \pm 4.0$ \\[+0.05in]
            \hline
        Total Prediction & $6540 \pm 680$ \\[+0.05in]
            \hline
        Data       &        $5908 \pm 77$ \\[+0.05in]
            \hline
        Agreement ($\sigma$)  &     1.0 \\[+0.05in]
%
\hline \hline
\multicolumn{2}{c}{\textbf{Same-sign $e^{-}e^{-}$ $Z$ mass window.}} \\\hline 
        Fakes      & $131 \pm 63$ \\[+0.05in]
        Charge Flips & $5990 \pm 630$ \\[+0.05in]
        Prompt Electrons & $61.4 \pm 3.9$ \\[+0.05in]
        $W\gamma$  & $9.4 \pm 2.3$ \\[+0.05in]
            \hline
        Total Prediction & $6190 \pm 630$ \\[+0.05in]
            \hline
        Data       &        $5885 \pm 77$ \\[+0.05in]
            \hline
        Agreement ($\sigma$)  &     0.5 \\[+0.05in]
\hline 
\end{tabular}
\end{center}
\caption{Observed and expected numbers of lepton pairs for the control region with same-sign, isolated electrons falling inside the $Z$ mass window. 
The uncertainties of the predictions are combined statistical and systematic ones.
Agreement between observed and expected number of pairs is quoted as a fraction of the total uncertainty on the prediction.
}
\label{tab:ee_isoSS_Z}
\end{table}


\subsection{Non-prompt background validation region}
\label{subsec:fake_validation}
% TODO [explain here how fake factors for validation regions were calculated]

To verify the fake factor method, a set of validation regions is checked. 
These regions have to be as much as possible kinematically close to
the nominal signal selection as well as to the looser signal selection, 
which is used in non-prompt background estimation with \EquationRef{eq:fake_pred}.

The looser signal selection requires weaker isolation and
electron identification requirement compared to the nominal selection.
Two validation regions were defined, as shown in a schematic representation in \FigureRef{fig:fake_validation_regions}. 
Both regions are identical to the nominal signal selection 
except either the weaker identification requirement (shown as ``VR1''), or 
the weaker isolation cut (``VR2'').
Looser selections for the
validation regions which were used for fake factor calculation are shown
as well. Such a design of the validation region provides similar kinematics to the one in the signal region.

\begin{figure}[h]
\begin{center}
%\includegraphics[width=0.48\textwidth]{figures/electrons/electron-syst}
\includegraphics[width=0.7\textwidth]{SS/fake_regions_black_diagram.pdf}
\caption{Schematic representation of the kinematic phase space of non-prompt validation regions with respect to signal region.}
\label{fig:fake_validation_regions}
\end{center}
\end{figure}

In total, four different validation regions are defined.
One validation region consists of same-sign electron pairs, where both electrons pass the signal region isolation requirement, 
but fail the ``tight'' electron identification requirement while passing ``medium'' one.
The second validation region consists of same-sign electron pairs, where both electrons fail the signal isolation requirement
but pass a looser intermediate isolation cut (loosened by 4 GeV) instead.
The third and fourth validation regions are identical to the second one but require only the leading (third region) or the subleading (fourth region)
electron to fail the signal isolation requirement but pass a looser intermediate isolation cut instead.

\FigureRef{fig:fakeCR_part1} and \FigureRef{fig:fakeCR_part2} show the invariant mass distributions for the validation regions described above. 
The agreement between observation and prediction is generally good.  
\TableRef{tab:ee_fakeCR} shows the expected and the observed numbers of electron pairs. 
The uncertainties quoted are statistical only. 

\begin{table}[htbp]
    \centering
    \resizebox{\textwidth}{!}{
    \begin{tabular}{ l | r r r r r r r }
        \hline
        Validation region 		& Fakes 			& Prompt 	& Charge Flip 			& W$\gamma$ 		& Total Pred 			& Data		&  Agreement($\sigma$) \\
            \hline

        Medium electron identification 	& $ 111.04 \pm 27.4 $	&	$ 2.9 \pm 0.5 $	& $ 72.46 \pm 16.75 $		& $ 8.78 \pm 2.3 $	& $ 195.18 \pm 32.2 $		& $ 217 \pm 15 $&  -0.62\\
        Weak isolation on both electrons 	& $ 252.9 \pm 133.64 $	& $ 1.23 \pm 0.3 $	& $ 29.07 \pm 10.1 $		& $ 0.27 \pm 0.28 $	& $ 283.47 \pm 134.02 $		& $285 \pm 17 $	&  -0.01\\
        Weak isolation on subleading electron 	& $ 519.21 \pm 120.72 $ & $ 32.88 \pm 2.14 $	& $ 52.69 \pm 14.87 $		& $ 17.64 \pm 4.32 $	& $ 622.42 \pm 121.72 $		& $574 \pm 24 $ &  0.39\\
        Weak isolation on leading electron 	& $ 154.97 \pm 58.67 $  & $ 13.28 \pm 1.21 $	& $ 15.96 \pm 7.5 $		& $ 5.12 \pm 1.72 $	& $ 189.33 \pm 59.19 $		& $ 224 \pm 15 $&  -0.57\\
        
        \hline
    \end{tabular}
    }
\caption{Expected and observed numbers of electron pairs for the different same-sign $ee$ fake control regions. 
The uncertainties on the predictions include the statistical and systematic uncertainties (fake factor and charge misidentification 
uncertainties have been included; other systematic uncertainties are negligible in these regions).
Agreement between observed and expected number of pairs is quoted in the last column of the table as a fraction of the total uncertainty on the prediction.
% For the fake predictions, a systematic uncertainty derived for the signal region is assumed.
}
\label{tab:ee_fakeCR}
\end{table}



\begin{figure}
\begin{subfigure}{.5\textwidth}
  \centering
  \includegraphics[width=\textwidth]{SS/support_note/dielectrons/crs/dec13_fake_medium_v2.eps}
\end{subfigure}%
\begin{subfigure}{.5\textwidth}
  \centering
  \includegraphics[width=\textwidth]{SS/support_note/dielectrons/crs/dec13_fake_bothInter_v2.eps}
\end{subfigure}
\caption{Invariant $e^{\pm}e^{\pm}$ mass distributions for non-prompt background prediction with medium electron identification (left) and with weak isolation on both electrons (right).
The hatched areas show the statistical uncertainty of the background prediction.}
  \label{fig:fakeCR_part1}
\end{figure}

\begin{figure}
\begin{subfigure}{.5\textwidth}
  \centering
  \includegraphics[width=\textwidth]{SS/support_note/dielectrons/crs/dec13_fake_leadNom_sublInter_v2.eps}
\end{subfigure}%
\begin{subfigure}{.5\textwidth}
  \centering
  \includegraphics[width=\textwidth]{SS/support_note/dielectrons/crs/dec13_fake_leadInter_sublNom_v2.eps}
\end{subfigure}
\caption{Invariant $e^{\pm}e^{\pm}$ mass distributions for non-prompt background prediction with weak isolation on subleading electron (left) and on leading electron (right).
The hatched areas show the statistical uncertainty of the background prediction.}
  \label{fig:fakeCR_part2}
\end{figure}



\section{Systematic Uncertainties}
\label{sec:ss_Systematics}
% TODO [add Fast/Full sim. systematics]

A set of possible systematic sources which can affect background predictions were studied.
These sources are presented below.
Systematic uncertainties related to the data-driven methods for non-prompt and charge misidentification background estimations
were described already in \SectionRef{subsec:fakes_description} and \SectionRef{subsec:CF_definition}, respectively.

\subsection{Electron reconstruction}
\label{subsec:elec_reco_system}
Several systematic uncertainties are related to the electron reconstruction procedure.
These uncertainties are provided by the ATLAS e/gamma working group which studies electron and photon identification performance of the ATLAS detector.
They provide recommendations and uncertainty estimations for all physics analyses which use electron or photon final states.

Electron reconstruction and ``tight'' identification efficiencies are obtained with so-called tag-and-probe data-driven method.
This method allows to measure from the data the efficiency of a studied electron selection using $Z \to e e$ and $J/\psi \to e e$ resonance decays. 
One electron from a pair is selected by requiring very tight criteria, 
while the second one is required to pass the studied selection.
By counting a number of pairs from the resonance decays (by fitting an invariant mass resonance peak) which were selected or were failed to be selected due to the electron required to pass studied selection, one can extract efficiency of the studied selection. Detailed information on the method can be found, for example, in ref.~\cite{tag-and-probe}.
The reconstruction efficiency uncertainty range is between 1.3-2.4\% depending on $\eta$, while the 
``tight'' identification efficiency uncertainty range is between 2.0-2.8\% depending on both $p_T$ and $\eta$~\cite{electron_reco_id_2011}.

Reconstruction of the electron energy is optimized using multivariate algorithms.
The electron energy scale and energy resolution are obtained using electrons from Z boson decays.
Their uncertainties are provided as a function of $p_T$ and $\eta$~\cite{electron_energy_errors_Run1} 
by the working group as well.

The total effect of these uncertainties on the total background prediction 
is shown under the name ``Electron reconstruction and identification'' in \TableRef{tab:syst}.

\subsection{Trigger and luminosity}
The electron trigger efficiency varies with $p_T$ and $\eta$ and is measured with respect of the offline identification.
This uncertainty is estimated to be at $\sim1\%$ level by the ATLAS trigger groups.
The resulting uncertainty on the yield in the signal region is different, since there are two leptons that can pass the trigger requirement. 

To scale the background prediction obtained with MC simulation to the data 
one has to know the integrated luminosity of the collected data sample. 
Therefore the luminosity uncertainty propagates to all the backgrounds measured using MC simulation.
The integrated luminosity uncertainty in 2012 is equal to 2.8$\%$~\cite{Aad:2013ucp} 
and it was obtained in a similar way to that described in \SectionRef{sec:lucid_performance}.

\subsection{Statistics and theoretical cross section}
% TODO describe how much statistics (wrt data lumi) samples had!
The limited number of simulated events in the Monte Carlo samples leads to additional uncertainty which is listed as ``MC statistics'' in \TableRef{tab:syst}.
This uncertainty also includes the effect of the limited number of events in data sets used in data-driven methods to measure the charge misidentification rate and the fake factor.
The statistical uncertainty is significant in the high-mass region.

% TODO describe who did all this tests and variations! Who it was???
As one can see from \TableRef{tab:MC_cross}, different processes were simulated using different MC generators, PDF sets and level of perturbative higher order calculations.
Additional uncertainty arises from  their choice.
To estimate these uncertainties, different MC generators, parton shower and hadronization models are tested.
Uncertainty on PDF and strong coupling constant $\alpha_{\mathrm{s}}$ are estimated by using different PDF sets following recommendations from~\cite{pdf4lhc}.
Also renormalization and factorization scales are varied by a factor of two to estimate the effect on the cross section.
The summary list of uncertainties used in the analysis is shown in \TableRef{tab:systematics_common}.
Detailed information about cross section calculations and their errors for some processes 
are reported in~\cite{diboson_cross_section,ttW_cross_section,ttV_cross_section}.

\begin{table}[ht]
\begin{center}
\begin{tabular}{l|l|c}
Source & Process & Uncertainty \\
\hline
\multirow{2}{*}{Trigger} & Signal and background & \multirow{2}{*}{2.1-2.6\%}  \\
& from MC simulations &\\
\hline
Electron reconstruction & Signal, prompt &\multirow{2}{*}{1.9--2.7\%}\\
and identification & background&\\\hline
Electron charge  & Opposite-sign& \multirow{2}{*}{9\%} \\
misidentification& backgrounds&\\\hline
Determination of & Non-prompt &\multirow{2}{*}{22\%}\\
fake factor $f$& backgrounds&  \\\hline
\multirow{2}{*}{Luminosity} & Signal and background& \multirow{2}{*}{2.8\%}\\
& from MC simulations&\\\hline
\multirow{2}{*}{MC statistics} & Backgrounds from &  \multirow{2}{*}{5\%}\\
& MC simulations &\\\hline
%Differences between fast & \multirow{2}{*}{Signal}& \multirow{2}{*}{1.8\%}& \multirow{2}{*}{5\%}& \multirow{2}{*}{0.7\%}\\
%and normal simulation &&&&\\\hline
Photon misidentification & \multirow{2}{*}{$W\gamma$} & \multirow{2}{*}{13\%}\\
as electron&&\\\hline
\multirow{2}{*}{MC cross sections} & Prompt, opposite-& \multirow{2}{*}{4\%}\\
& sign backgrounds & \\
\end{tabular}
\end{center}
\caption{Sources of systematic uncertainty (in \%) of the signal yield and the expected background predictions, described in the second column, for the mass range $m_{ee} > 15$ GeV.
}
\label{tab:syst}
\end{table}


\begin{table}[ht]
\begin{center}
\begin{tabular}{l|c}
Processes affected & Uncertainty \\
\hline
 Drell-Yan (Charge flips) & $\pm$7\% \\
 $WZ$ & $\pm$7\% \\
 $ZZ$ & $\pm$5\% \\
 $t\bar{t}W$, $t\bar{t}Z$  & $\pm$22\% \\
 $W^{\pm}W^{\pm}$ & $\pm$50\% \\
 MPI $WW$, $WZ$, $ZZ$ & $\pm$100\% \\
 $t\bar{t}$ & $\pm$5\% \\
 $W\gamma$ & $\pm$14\% \\
\end{tabular}
\end{center}
\caption{Theoretical uncertainties of the production cross section of SM processes modelled by MC.}
\label{tab:systematics_common}
\end{table}



\section{Signal Region}
\label{sec:ss_signalRegion}

The same-sign lepton pair invariant mass in the signal region is shown in~\FigureRef{fig:signal_mass}.
The last bin in the histogram is an overflow bin, which includes pairs with invariant mass higher than 600 GeV.
The observed number of pairs is compatible with the predicted background.
As one can see, the dominant background arises from the charge misidentification component.
The predicted contributions from each background process with different invariant mass cuts are shown in \TableRef{tab:2iso_ee_SS}.
In \FigureRef{fig:signal_kinematics} the kinematics of the leading lepton is shown. 
In \FigureRef{fig:delta_phi} the angle between same-sign electrons in the pair is shown as well.
The background prediction describes the observed numbers of all of these distributions reasonably well within the uncertainties bands.

\begin{figure}[h]
\begin{center}
\includegraphics[width=0.7\textwidth]{SS/paper_draft/2isoSS_ee_mll.eps}
\caption{Invariant mass distribution for $e^{\pm}e^{\pm}$ pairs in the signal region. 
The shaded band in the lower plot corresponds to the combination of the statistical and systematic uncertainties of the background prediction.
The last bin is an overflow bin.}
\label{fig:signal_mass}
\end{center}
\end{figure}

\begin{table*}[htbp]
\begin{center}
\resizebox{\textwidth}{!}{
\begin{tabular}{l|c|c|c|c|c|c|c}
\hline
Sample & \multicolumn{5}{|c}{Number of electron pairs with  $m(e^{\pm}e^{\pm})$} \\
 & $>15$~GeV & $>100$~GeV & $>200$~GeV & $>300$~GeV & $>400$~GeV & $>500$~GeV & $>600$~GeV \\
\hline \hline
Non-prompt	& $ 518.57 \pm 120.17 $	& $ 247.49 \pm 49.5 $	& $ 71.67 \pm 13.15 $	& $ 22.66 \pm 4.8 $	& $ 8.13 \pm 2.42 $	& $ 3.12 \pm 1.49 $	& $ 0.78 \pm 1.01 $	\\[+0.05in]
\hline\hline
$W\gamma$	& $ 175.25 \pm 36.28 $	& $ 74.89 \pm 15.62 $	& $ 22.42 \pm 5.15 $	& $ 8.04 \pm 2.26 $	& $ 3.84 \pm 1.31 $	& $ 2.69 \pm 1.05 $	& $ 1.02 \pm 0.57 $	\\[+0.05in]
\hline\hline
Drell-Yan	& $ 968.61 \pm 145.63 $	& $ 513.53 \pm 77.7 $	& $ 130.91 \pm 26.99 $	& $ 36.1 \pm 12.17 $	& $ 12.8 \pm 7.89 $	& $ 4.79 \pm 4.86 $	& $ 4.79 \pm 4.86 $	\\[+0.05in]
$t\bar{t}$	& $ 36.92 \pm 6.01 $	& $ 30.1 \pm 4.99 $	& $ 14.55 \pm 2.8 $	& $ 5.05 \pm 1.32 $	& $ 2.15 \pm 0.78 $	& $ 1.05 \pm 0.58 $	& $ 1.18 \pm 0.56 $	\\[+0.05in]
$WW$	& $ 13.01 \pm 2.34 $	& $ 10.74 \pm 1.96 $	& $ 4.85 \pm 0.97 $	& $ 1.86 \pm 0.45 $	& $ 0.68 \pm 0.22 $	& $ 0.43 \pm 0.16 $	& $ 0.28 \pm 0.13 $	\\[+0.05in]
\hline
Charge Flip total	& $ 1018.54 \pm 145.78 $	& $ 554.37 \pm 77.89 $	& $ 150.31 \pm 27.16 $	& $ 43.01 \pm 12.25 $	& $ 15.62 \pm 7.93 $	& $ 6.27 \pm 4.89 $	& $ 6.25 \pm 4.89 $	\\[+0.05in]
\hline\hline
$ZZ$	& $ 86.05 \pm 7.21 $	& $ 22.42 \pm 2.11 $	& $ 6.75 \pm 0.84 $	& $ 1.78 \pm 0.37 $	& $ 0.61 \pm 0.2 $	& $ 0.34 \pm 0.16 $	& $ 0.21 \pm 0.12 $	\\[+0.05in]
$WZ$	& $ 234.36 \pm 22.24 $	& $ 132.79 \pm 12.76 $	& $ 37.12 \pm 3.9 $	& $ 10.95 \pm 1.43 $	& $ 3.23 \pm 0.61 $	& $ 1.5 \pm 0.4 $	& $ 0.5 \pm 0.22 $	\\[+0.05in]
$t\bar{t}W$	& $ 5.33 \pm 1.23 $	& $ 3.83 \pm 0.89 $	& $ 1.32 \pm 0.32 $	& $ 0.44 \pm 0.11 $	& $ 0.14 \pm 0.04 $	& $ 0.08 \pm 0.03 $	& $ 0.03 \pm 0.01 $	\\[+0.05in]
$t\bar{t}Z$	& $ 1.73 \pm 0.41 $	& $ 1.2 \pm 0.29 $	& $ 0.4 \pm 0.1 $	& $ 0.11 \pm 0.04 $	& $ 0.03 \pm 0.01 $	& $ 0.02 \pm 0.01 $	& $ 0.01 \pm 0.01 $	\\[+0.05in]
$WWjj$	& $ 14.99 \pm 7.59 $	& $ 12.1 \pm 6.14 $	& $ 5.55 \pm 2.84 $	& $ 2.35 \pm 1.22 $	& $ 1.22 \pm 0.66 $	& $ 0.4 \pm 0.24 $	& $ 0.16 \pm 0.11 $	\\[+0.05in]
MPI	& $ 4.04 \pm 4.06 $	& $ 1.6 \pm 1.61 $	& $ 0.38 \pm 0.39 $	& $ 0.06 \pm 0.07 $	& $ 0.02 \pm 0.02 $	& $ 0 \pm 0 $	& $ 0 \pm 0 $	\\[+0.05in]
\hline
Prompt total	& $ 346.51 \pm 24.95 $	& $ 173.94 \pm 14.44 $	& $ 51.52 \pm 4.93 $	& $ 15.7 \pm 1.92 $	& $ 5.25 \pm 0.92 $	& $ 2.34 \pm 0.49 $	& $ 0.91 \pm 0.28 $	\\[+0.05in]
\hline\hline
Total Background	& $ 2058.86 \pm 193.92 $	& $ 1050.69 \pm 94.67 $	& $ 295.92 \pm 30.99 $	& $ 89.41 \pm 13.49 $	& $ 32.83 \pm 8.44 $	& $ 14.41 \pm 5.25 $	& $ 8.96 \pm 5.04 $	\\[+0.05in]
\hline\hline
Data	& $ 1976 $	& $ 987 $	& $ 265 $	& $ 83 $	& $ 30 $	& $ 13 $	& $ 7 $	\\[+0.05in]

\hline
\end{tabular}
}
\end{center}
\caption{Expected and observed numbers of pairs of isolated same-sign electrons for various cuts on the dielectron invariant mass, \mee. The uncertainties shown include statistical and systematic contributions.}
\label{tab:2iso_ee_SS}
\end{table*}


\begin{figure}
\begin{subfigure}{.5\textwidth}
  \centering
  \includegraphics[width=\textwidth]{SS/paper_draft/2isoSS_ee_pt1.eps}
\end{subfigure}%
\begin{subfigure}{.5\textwidth}
  \centering
  \includegraphics[width=\textwidth]{SS/paper_draft/2isoSS_ee_eta1.eps}
\end{subfigure}
\caption{$p_T$ and $\eta$ distributions of the leading electron in the signal region. The last bin is an overflow bin in the left figure.}
  \label{fig:signal_kinematics}
\end{figure}

\begin{figure}[h]
\begin{center}
\includegraphics[width=0.7\textwidth]{SS/support_note/dielectrons/crs/signal_deltaPhi_v2.eps}
\caption{Azimuthal angle difference between two same-sign electrons from the pair in the signal region. The uncertainties shown include statistical and systematic contributions.}
\label{fig:delta_phi}
\end{center}
\end{figure}



\subsection{Limit setting}

The background prediction describes data very well and there are no significant visible deviations which could indicate a potential presence of the BSM signal.
The idea of this analysis is to be as general as possible and to perform a search for the a new physics without favoring any BSM model.
This is why the next step is to calculate exclusion limits on any type of new physics with prompt same-sign lepton pairs.

Due to the limited acceptance of the detector the signal region was designed to use the phase space which is used in ATLAS for precision measurements.
This is why the cross section, $\sigma$, of the process relates with the measured cross section in the signal region, the so-called fiducial cross section $\sigma^{fid}$, as:
\begin{equation}
 \sigma = \dfrac{\sigma^{fid}}{<N_{pair}>A}
 \label{eq:cross_section}
\end{equation}
where $<N_{pair}>$ is the average number of same-sign pairs produced per event and $A$ is the fiducial acceptance (or volume).
The definition of the fiducial volume is discussed in \SectionRef{subsec:fid_volume_eff}.

The cross section limits are derived using a $CLs$~\cite{CLs_tecnique,CLs_2} prescription with the help of the RooStat~\cite{RooStat_project} framework 
provided by the ATLAS Statistics Committee. $CLs$ method state that the signal hypothesis is excluded at the confidence level $CL$ when
\begin{equation}
 1 - CL_s \leq CL
\end{equation}
where $CL_s$ is defined as
\begin{equation}
 CL_s \equiv \dfrac{CL_{s+b}}{CL_b}
\end{equation}
where $CL_b$ is a confidence level observed for the background-only hypothesis and $CL_{s+b}$ - for the background plus signal hypothesis.
In practise, $CL_b$ ($CL_{s+b}$) is a probability to find the observed data given an expected background (background plus signal).
This probability is Poisson-distributed and is calculated based on the number of observed and expected same-sign electron pairs in the signal region.
The test-statistics used for the limit settings is a log likelihood ratio test.
Systematic uncertainties and their correlations between each other are taken into account with this method.
For example, the charge misidentification scale factor uncertainty is correlated across Drell-Yan, $t\bar{t}$, $WW$ and $W\gamma$ background samples.
Experimental errors such as electron reconstruction, identification, energy scale and trigger 
are treated as one uncertainty and correlated across all the expected backgrounds and signal.
The MC cross section errors are independent except for the diboson ($ZZ$, $WZ$ and $WW$) samples.
The luminosity is common to all the background samples, thus it is correlated.
The statistical errors are independent.

Following this prescription and using the number of expected and observed same-sign lepton pairs one can compute 
upper limits at a given confidence level (typically at 95$\%$ level) on the number of same-sign lepton pairs ($N_{95}$)
arising from the new physics beyond the SM. Limits can be set for different invariant mass thresholds, 
because the dilepton invariant mass is the main observable in the analysis.
Limits on the number of pairs can be translated into upper limits on the fiducial cross section as:
\begin{equation}
 \sigma_{95}^{fid} = \dfrac{N_{95}}{\epsilon_{fid} \times \int \mathscr{L} dt}
 \label{eq:fid_cross_section}
\end{equation}
where $\int \mathscr{L} dt$ is an integrated luminosity of the data and $\epsilon_{fid}$ is a fiducial efficiency for finding a same-sign electron pair from
a possible signal from new physics in the fiducial volume, which is described in \SectionRef{subsec:fid_volume_eff}.

% TODO explain fiducial efficiency

\subsection{Fiducial volume and fiducial efficiency}
\label{subsec:fid_volume_eff}

As can be seen from \EquationRef{eq:cross_section} and \EquationRef{eq:fid_cross_section}, in order to translate the number of measured and expected lepton pairs 
to the cross section limit of a signal from new physics, one has to know the fiducial volume and efficiency.
The reason is that the detector does not reconstruct leptons with a 100$\%$ efficiency, and it does not cover the whole phase space around the interaction point.
The fiducial volume represents the phase space region which is truncated so that it mimics the detector acceptance.
It is defined by a set of cuts on the truth (generator) level. 
Kinematic cuts on the electrons are identical to the one used in the signal region definition on reconstruction level:
\begin{itemize}
 \item Leading electron $p_T > 25$ GeV
 \item Subleading electron $p_T > 20$ GeV
 \item $|\eta|<1.37$ or $1.52<|\eta|<2.47$
\end{itemize}
Requirements on the electron pair are the same as well:
\begin{itemize}
 \item Same-sign pair with $m_{ee} > 15$ GeV
 \item Veto pairs with $70 < m_{ee} < 110$ GeV
 \item No opposite-sign same-flavour pairs with $|m_{ee} - m_{Z}| < 10$ GeV
\end{itemize}
Since electrons are required to be isolated on reconstruction level, isolation has to be applied on the truth level as well.
Track-based isolation on truth level is identical to the one on reconstruction level:
all charged particles within the cone $\Delta R < 0.3$ around the electron with $p_T > 0.4$ GeV are considered and
sum of their $p_T$ has to be not larger than 10$\%$ of the electron $p_T$.
Calorimeter-based isolation was not applied on the truth level due to a significantly different behaviour at reconstruction and truth levels.

In the case of an ideal detector, fiducial volume would completely correspond to the geometrical detector acceptance.
But because the real detector does not provide a 100$\%$ reconstruction and identification efficiency, a fiducial efficiency is used in order to relate the fiducial volume with the real geometrical detector acceptance, which is defined as:
\begin{equation}
 \epsilon_{fid} = \dfrac{N_r}{N_f}
\end{equation}
where $N_f$ is the number of electron pairs which pass the fiducial volume cuts on the truth level and $N_r$ - which pass the fiducial volume cuts on the truth level 
as well as all the signal selection cuts on the reconstruction level.

In order to perform a search for new physics in a model-independent way, the fiducial efficiency has to be constant and to not depend on the type of the BSM model.
However, different models provide different $p_T$ and $\eta$ spectra, and the electron reconstruction efficiency depends on both $p_T$ and $\eta$.
As reported in~\cite{electron_tight}, the efficiency can vary up to 15$\%$ for the ``tight'' identification criteria with respect to the electron $p_T$.
Also, the presence and number of jets in the final state is model dependent, and affects the electron isolation efficiency, which will have an effect on 
the fiducial efficiency.

In order to estimate the value of the fiducial efficiency, efficiencies for four different BSM models were calculated:
\begin{itemize}
 \item Doubly charged Higgs. This model assumes production of two doubly charged scalar bosons which, decaying leptonically, will provide two pairs of same-sign leptons.
 No jets are produced in the final state of the hard process. The Higgs mass was varied between 100 GeV and 1000 GeV.
 \item Colored Zee-Babu model. Diquarks are produced, which decay to two leptoquarks with the same charge, 
 which subsequently decay to a lepton and a quark. The final state consists of one same-sign lepton pair and two jets.
 The masses considered in the model are 2.5-3.5 TeV for the diquark and 1-1.4 TeV for leptoquarks.
 \item Production of a heavy right-handed $W_R$ boson and a heavy Majorana neutrino. $W_R$ decays to a lepton and a Majorana neutrino, 
 which decays to a $W$ boson and another lepton. The final state consists of one same-sign pair and products from the W boson decay.
 The mass of $W_R$ was varied between 1 TeV and 2 TeV, while the mass of the Majorana neutrino was in the range of 0.25-1.5 TeV
 \item Pair production of the fourth generation down-type quark. Both quarks decay semi-leptonically to a $t$ quark and subsequently to a $b$ quark.
 The final state consists of two jets and four W bosons. At least two same-sign bosons have to decay leptonically in order to provide a same-sign lepton pair.
 This state is characterized by large hadron activity due to the high jet multiplicity.
 The mass of the fourth generation quark was varied from 400 GeV to 1 TeV.
\end{itemize}
The fiducial efficiencies for these models were calculated with different dilepton mass thresholds.
The obtained efficiencies are in the range of 48-74$\%$. The lowest efficiency was observed for the fourth generation down-type quark model, while the highest one is found for the heavy right-handed $W_R$ boson and heavy Majorana neutrino process. The latter model has larger efficiency with respect to the doubly charged Higgs model
because the final state of the doubly charged Higgs model contains two same-sign pairs compared to one pair in the Majorana model.
The efficiencies were measured separately for positive and negative same-sign pairs, and no significant differences were observed.

To provide a conservative cross section limit setting for new physics beyond the SM, the lowest obtained efficiency, which was 48.3$\%$, was used.

\subsection{Fiducial cross section limits}

Computed upper limits at the 95$\%$ confidence level on the fiducial cross section ($\sigma_{95}^{fid}$)
of new physics beyond the SM for the invariant mass thresholds used in \TableRef{tab:2iso_ee_SS}
are shown in \FigureRef{fig:inclusive_fid_limit}. Separate limits for positive and negative same-sign pairs are shown in \FigureRef{fig:signal_kinematics_v2}.
The expected limits are shown together with the 2$\sigma$ uncertainty bands. The limits are summarized in \TableRef{tab:limits}.

\begin{figure}[h]
\begin{center}
\includegraphics[width=0.7\textwidth]{SS/paper/limit_ee_all.eps}
\caption{Fiducial cross section upper limits at 95\% C.L. for new physics contributing to the signal region for events with $e^{\pm}e^{\pm}$ pairs.
Green and yellow bands correspond to the 1$\sigma$ and 2$\sigma$ uncertainty bands on the expected limits respectively.}
\label{fig:inclusive_fid_limit}
\end{center}
\end{figure}


\begin{figure}
\begin{subfigure}{.5\textwidth}
  \centering
  \includegraphics[width=\textwidth]{SS/support_note/limits/limit_ee_neg.eps}
\end{subfigure}%
\begin{subfigure}{.5\textwidth}
  \centering
  \includegraphics[width=\textwidth]{SS/support_note/limits/limit_ee_pos.eps}
\end{subfigure}
\caption{Fiducial cross section upper limits at 95\% C.L. for new physics contributing to the signal region
for events with $e^{+}e^{+}$ (left) and $e^{+}e^{+}$ (right) pairs. Green and yellow bands correspond to the 1$\sigma$ and 2$\sigma$ uncertainty bands on the expected limits respectively.}
  \label{fig:signal_kinematics_v2}
\end{figure}


\begin{table*}[!ht]
\begin{center}
\begin{tabular}{c||c|c||c|c||c|c}

 & \multicolumn{6}{c}{95\%  CL upper limit [fb]} \\
 & \multicolumn{2}{c||}{$e^{\pm}e^{\pm}$} & \multicolumn{2}{c||}{$e^{+}e^{+}$} & \multicolumn{2}{c}{$e^{-}e^{-}$} \\
Mass range & Expected & Observed & Expected & Observed & Expected & Observed \\
%[+0.05in]
\hline
\rule{0pt}{3ex}
  $>15$~GeV   &  $39^{+10}_{-13}$        &  32    &    $27^{+11}_{-6}$         &  28    &    $23^{+8}_{-5}$          &  19\\
  $>100$~GeV  &  $19^{+6}_{-6}$          &  14    &    $14.3^{+5.4}_{-2.8}$    &  13.5  &    $10.8^{+4.4}_{-2.4}$    &  9.0\\
  $>200$~GeV  &  $6.8^{+2.6}_{-1.7}$     &  5.3   &    $5.4^{+2.0}_{-1.4}$     &  4.6   &    $3.9^{+1.4}_{-1.2}$     &  3.5\\
  $>300$~GeV  &  $3.3^{+1.3}_{-0.4}$     &  3.3   &    $2.5^{+0.9}_{-0.6}$     &  2.0   &    $2.1^{+0.7}_{-0.5}$     &  2.6\\
  $>400$~GeV  &  $2.02^{+0.74}_{-0.21}$  &  2.03  &    $1.59^{+0.47}_{-0.34}$  &  1.64  &    $1.56^{+0.41}_{-0.31}$  &  1.35\\
  $>500$~GeV  &  $1.25^{+0.36}_{-0.26}$  &  1.10  &    $1.44^{+0.34}_{-0.36}$  &  1.55  &    $0.69^{+0.27}_{-0.17}$  &  0.64\\
  $>600$~GeV  &  $0.99^{+0.34}_{-0.20}$  &  1.02  &    $1.27^{+0.37}_{-0.26}$  &  1.10  &    $0.58^{+0.21}_{-0.08}$  &  0.61\\

\end{tabular}
\end{center}
 \caption{Upper limit at 95\% CL on the fiducial cross-section for $e^{\pm} e^{\pm}$ pairs from non-SM signals. 
 The expected limits and their $1 \sigma$ uncertainties are given together with the observed limits derived from the data. 
 Limits are given inclusively and separated by charge.}
\label{tab:limits}
\end{table*}


\section{Mass limits of doubly charged Higgs}
As an example of a BSM model which produces same-sign lepton pairs in the final state, a pair production of doubly charged Higgs bosons was studied.
The search strategy is the same as described above.
Since the final state of this model has two same-sign pairs of leptons, no jet activity and no missing transverse energy is present in the event.
Therefore, there is no need to optimize the signal selection.
Doubly charged Higgs decays should be visible as a sharp peak in the dilepton invariant mass.
Using a fiducial efficiency calculated in bins of 100 GeV (as it was done for the fiducial limit calculations) is not optimal from the point of view of the signal sensitivity.
Thus, the search is performed in mass bins with a mass-dependent width.

\subsection{MC simulation}
A few signal samples were generated with Pythia8~\cite{pythia8} with different masses of the left-handed and right-handed doubly charged Higgs bosons.
The simulated masses were produced in the range of 50-600 GeV in steps of 50 GeV and one sample had a Higgs mass of 1 TeV.
The kinematics of left- and right-handed Higgs bosons is identical, but the production rate is different due to different coupling to the Z boson mediator~\cite{dch_note}. The cross sections were calculated with NLO precision. 
% TODO [link] for NLO cross section

\subsection{Model acceptance and efficiency}
The width of the doubly charged Higgs decay is dominated by the detector momentum resolution of the electrons. Since the decay width depends from the doubly charged Higgs mass, the search was performed in the mass bins with variable width.
The idea behind the mass bin widths optimization is based on two facts. On one hand, a mass bin has to cover as much signal as possible,
while on the other hand, the background contribution in the mass bin is desired to be as small as possible.
To satisfy both conditions, the signal significance, $S$, was chosen as an optimization criterion:
\begin{equation}
 S = \sqrt{ 2((s+B)ln(1+s/B)-s } 
\end{equation}
where $s$ is the expected signal and $B=b+\delta b^2$ is the predicted background plus background systematic uncertainty squared.
A bin width is parameterized as a second degree polynomial of the Higgs mass.
Coefficients of the polynomial were the parameters to optimize.
During the optimization procedure it became clear that there is a third effect which has to be taken into account.
Due to limited statistics of the predicted background, one cannot have too many mass bins, otherwise the cross section
exclusion limit will fluctuate significantly from bin to bin. 
The optimal bin width parameterization was found to be $\pm(0.04 \times m(\dch) + 0.2 \cdot 10^{-4} \times m(\dch)^{2})$,
where $m(\dch)$ is the Higgs mass.

The next step is to define how many lepton pairs produced in Higgs boson decays are reconstructed, selected and fall into the mass bin.
The number of generated Higgs bosons is known, 
thus one only needs to count number of reconstructed same-sign pairs which pass the signal selection in a given mass bins.
The ratio of reconstructed to the total number of generated pairs corresponds to the total efficiency, which includes the effect of the acceptance and efficiency of the signal selection.
The efficiencies for each mass point were calculated, but in order to interpolate between the simulated mass points, the total efficiency $\varepsilon_{tot}$ is fitted by
an empirical piecewise function:
\begin{equation}
\varepsilon_{tot}(m) = \begin{cases} p_{0} (1-e^{-(m-p_{1})/p_{2}}) & ,\mbox{if } m < 450\mbox{ GeV} \\ 
p_{3} + p_{4} m & ,\mbox{if } m \geq 450\mbox{ GeV} \end{cases}
\label{eq:ee_eff}
\end{equation}
where $m$ is the Higgs mass and $p_{0}, p_{1}, p_{2}, p_{3}, p_{4}$ are fit parameters shown in \TableRef{tab:ee_eff_params}.
\begin{table}[htbp]
    \begin{center}
    \begin{tabular}{ l | l }
        \hline
        Parameter & Value \\
        \hline
        $p_{0}$    & $4.76 \times 10^{-1}$ \\[+0.05in]
        $p_{1}$    & $2.94 \times 10^{+1}$ \\[+0.05in]
        $p_{2}$    & $1.05 \times 10^{+2}$ \\[+0.05in]
        $p_{3}$    & $4.51 \times 10^{-1}$ \\[+0.05in]
        $p_{4}$    & set by requiring continuity \\[+0.05in]
        \hline
    \end{tabular}
    \end{center}
    \caption{Fitted parameter values for Equation~\ref{eq:ee_eff}, which gives $\varepsilon_{tot}(m)$.}
    \label{tab:ee_eff_params}
\end{table}

The computed efficiencies for available mass points and their fit are shown in \FigureRef{fig:signal_efficiency}.

\begin{figure}[h]
\begin{center}
\includegraphics[width=0.6\textwidth]{SS/support_note/DCHLimits/effic_DCH_ee_pairSF_v2.eps}
\caption{Total reconstruction efficiency ($\varepsilon_{tot}$) of the doubly charged Higgs boson decay to same-sign electron pair as a function of simulated $H^{\pm\pm}$ mass, fitted with piecewise empirical function.}
\label{fig:signal_efficiency}
\end{center}
\end{figure}

% TODO \toDo[describe systematics for the DCH signal]
% TODO [CLs = CLs+b/CLb. What is the signal in case of limit on new physics? Read Ingas thesis]

\subsection{Cross section and mass limits}

The invariant mass distribution of the same-sign electron pairs together with the signal from a left-handed Higgs 
with masses between 300 GeV and 500 GeV are shown in \FigureRef{fig:signal_mass_v2}.
The branching ratio of the $H^{\pm\pm} \to e^{\pm}e^{\pm}$ decay is assumed to be 100$\%$.

\begin{figure}[h]
\begin{center}
\includegraphics[width=0.7\textwidth]{SS/support_note/DCHPlots/ee/2isoSS_ee_mll_dch.eps}
\caption{Invariant mass distributions for $\ee$ pairs passing the full event selection. 
Open histograms show the expected signal from simulated $H^{\pm\pm}$ samples,
assuming a 100\% branching ratio to the same-sign electron pair. The last bin is an overflow bin. Only statistical uncertainties for data are shown.}
\label{fig:signal_mass_v2}
\end{center}
\end{figure}

The upper cross section limit on pair production of the doubly charged Higgs is set in the same way as the limits on the new physics described above.
The cross section is determined as:
\begin{equation}
 \sigma_{HH}\times BR =\frac{N_{H}^{rec}}{2\times A\times \epsilon \times \int \mathscr{L} dt}
\end{equation}
where BR is the branching ratio of the $H^{\pm\pm} \to e^{\pm}e^{\pm}$ decay, 
$N_{H}^{rec}$ is the number of reconstructed $H^{\pm\pm}$, $A\times \epsilon$ is the total efficiency described earlier and
$\int \mathscr{L} dt$ is the integrated luminosity.
The factor 2 in the denominator is needed to take into account presence of two same-sign pairs from $H^{++}$ and $H^{--}$ in the event.

The upper cross section limit times branching ratio (which is assumed to be 100$\%$) at 95$\%$ CL is shown in \FigureRef{fig:dch_limits_mass}.
The scatter between the mass bins is caused by fluctuations of the predicted background due to the low statistics per bin.
A good agreement between expected and observed limit lines can be seen, and all deviations are within $2\sigma$.
The theoretical cross section curves as a function of Higgs mass for left- and right-handed doubly charged Higgs bosons 
are shown as well.
Lower mass limits of the model correspond to the intersection of the theoretical curve with the expected cross section limit. The obtained
mass limits are summarized in \TableRef{tab:limits_mass}.

\begin{figure}[h]
\begin{center}
\includegraphics[width=0.6\textwidth]{SS/paper/limitDCH_ee_all.eps}
\caption{Upper limits at 95\% C.L. on the cross section as a function of $e^{\pm}e^{\pm}$ invariant mass for the production of a double charged Higgs boson 
assuming a 100\% branching ratio to $e^{\pm}e^{\pm}$. The green and yellow bands correspond to the 1$\sigma$ and 2$\sigma$ bands on the expected limits respectively.
Pair production cross sections for left and right-handed $H^{\pm\pm}$ are overlaid.}
\label{fig:dch_limits_mass}
\end{center}
\end{figure}

\begin{table}[htbp]
\begin{center}
\begin{tabular}{c||c|c}
& \multicolumn{2}{c}{95\%  C.L. upper limit [GeV]}\\
Signal & expected & observed \\
\hline
$H^{\pm\pm}_L$ & $552.6^{+11.1}_{-49.9}$ & $551.2 \pm 3.1$ \\
\hline
$H^{\pm\pm}_R$ & $424.8^{+1.0}_{-59.7}$ & $374.0 \pm 6.2$ \\
\end{tabular}
\end{center}
\caption{Upper limit at 95\% C.L. on mass of \dch, assuming 100\% branching ratio to $e^{\pm}e^{\pm}$.}
\label{tab:limits_mass}
\end{table}

These limits can also be interpreted as mass limits as a function of the branching ratio for $H^{\pm\pm}_L$ and $H^{\pm\pm}_R$ decays, which are shown in \FigureRef{fig:dch_limits_BR}.

\begin{figure}
\begin{subfigure}{.5\textwidth}
  \centering
  \includegraphics[width=\textwidth]{SS/paper/limitDCH_ee_allvsBR_LH.eps}
\end{subfigure}%
\begin{subfigure}{.5\textwidth}
  \centering
  \includegraphics[width=\textwidth]{SS/paper/limitDCH_ee_allvsBR_RH.eps}
\end{subfigure}
\caption{95\% C.L. limits on the doubly charged Higgs mass vs 
branching ratio of $H^{\pm\pm}_L$ (left) and $H^{\pm\pm}_R$ (right) for events with $e^{\pm}e^{\pm}$ pairs.}
  \label{fig:dch_limits_BR}
\end{figure}

\section{Summary and outlook}
\label{sec:ssOutlook}
% \begin{itemize}
%  \item direct comparison of 7 and 8 TeV analyses are impossible because different efficiencies have been used
%  \item two additional bins in the inclusive limit plot wrt to 7 TeV analysis. what about 13 TeV analysis?
%  \item comparison of the DCH results (7 TeV and 13 TeV)?
%  \item overview of the systematics wrt to 7 and 13 TeV analyses.
%  \item comparison of DCH mass limits as a reference?
%  \item systematization of the charge flip approach for Run 2 --> better understanding and smaller systematics.
% \end{itemize}

% fiducial limits
An inclusive search for a new physics has been performed in the final state of a same-sign electron pair using 20.3 fb$^{-1}$ of 8 TeV center of mass pp collision data. A limit on fiducial cross section has been set as a function of electron pair invariant mass with cuts ranging from $>15$ to $>600$ GeV. 
With respect to the previous ATLAS analysis performed with 7 TeV center of mass data (reported in ref.~\cite{ss_7TeV}) cross section limits have been 
extended with two additional invariant mass bins ($>500$ and $>600$ GeV).

% DCH limits
The analysis selection have been used for a narrow bin search for \dch.
The upper limit on the cross section for pair production of right- and left-handed doubly charged Higgs and exclusion mass limits have been set as a function of \dch mass.
Comparison of the obtained 95\% C.L. upper \dch mass limit, assuming 100\% branching ratio to $e^{\pm}e^{\pm}$ with results from 7 TeV~\cite{dch_7TeV_paper} and 13 TeV~\cite{dch_13TeV_conf} analyses performed by ATLAS are shown in \TableRef{tab:DCH_limit_vs_years}.
As one can see from Table mass limits benefit from the larger center of mass pp collision data though using a dataset with larger integrated luminosity will improve limits further.

\begin{table*}[]
\begin{center}
\begin{tabular}{c||c|c||c|c||c|c}

 & \multicolumn{6}{c}{95\%  C.L. upper limit [GeV]} \\
 & \multicolumn{2}{c||}{4.7 fb$^{-1}$ at 7 TeV} & \multicolumn{2}{c||}{20.3 fb$^{-1}$ at 8 TeV} & \multicolumn{2}{c}{13.9 fb$^{-1}$ at 13 TeV} \\
 \hline
Signal & Expected & Observed & Expected & Observed & Expected & Observed \\
%[+0.05in]
\hline
\rule{0pt}{3ex}
% WITH errors in 8 TeV values:
% $H^{\pm\pm}_L$ & 407 & 409 & $552.6^{+11.1}_{-49.9}$ & $551.2 \pm 3.1$ & 580 & 570 \\
% \hline
% $H^{\pm\pm}_R$ & 329 & 322 & $424.8^{+1.0}_{-59.7}$ & $374.0 \pm 6.2$  & 460 & 420 \\

% WITHOU errors in 8 TeV values:
$H^{\pm\pm}_L$ & 407 & 409 & 553 & 551 & 580 & 570 \\
\hline
$H^{\pm\pm}_R$ & 329 & 322 & 425 & 374 & 460 & 420 \\
\end{tabular}
\end{center}
 \caption{Upper limit at 95\% C.L. on mass of \dch, assuming 100\% branching ratio to $e^{\pm}e^{\pm}$ for 7, 8 and 13 TeV analyses results.}
\label{tab:DCH_limit_vs_years}
\end{table*}

% TODO check $E/p$ stuff
% improvement of signal selection
There are different ways to improve and develop analysis selection.
One of the possibility to significantly increase the sensitivity of the search is to optimize signal region to more efficiently reject charge misidentification background, which is the dominant background source. In the Run-2 LHC period, a lot of effort have been made in the ATLAS collaboration to investigate electron charge misidentification effect and to suppress it by using new variables as e.g. electron $E/p$.

% improvement of the systematics and methods
Another possibility is to investigate sources of the systematic uncertainty more deeply.
As one compare these three analyses, one can notice that the dominant sources of the systematic uncertainties arise from modelling of the charge misidentification and non-prompt backgrounds. Thus, improving method used for estimation of these backgrounds could allow reducing the systematic uncertainty. Especially this is important for high-$p_T$ electrons were charge misidentification rate is poorly understood.
Also, increased statistics available in Run-2 can be used for better understanding of the non-prompt background source.

% Other analysis improvements possible developments
As was mentioned before the analysis have been done with three channel: $e^{\pm}e^{\pm}$, $\mu^{\pm}\mu^{\pm}$. However, one can add tau lepton channel, which will complement search results. Also one can use a different definition of the fiducial region when a fiducial efficiency is counted separately per lepton then per pair, as was made in the search with three charged leptons final state reported in ref.~\cite{TheATLAScollaboration:2013cia}.




%   \chapter{Search for new charged bosons in final states with one muon and missing transverse energy}
\label{chap:Wprime}

% PLAN
% <What this chapter about - short intro to Wprime analysis>
% <What was my personal contribution to the analysis> 


This chapter describes search for new spin-1 heavy charged boson (namely $\PWprime$) in the final state with one lepton and missing transverse energy ($E_T^{miss}$).
The search was done with first $\sqrt{s}$=13~TeV data collected in 2015 by ATLAS with corresponded luminosity of 3.2~fb$^{-1}$. The search have been performed with muon and electron channels, however, this chapter is focused in the muon channel and only final results for electron channel is shown as well as combination of both channels.

% TODO describe more about the analysis.
% Analysis focuses on high-p$_T$ 

% Key features of the analysis.

% TODO move it to preface!!!
% This analysis have been made public as a conference note~\cite{ATLAS-CONF-2015-063} with early reasults and later as a paper~\cite{Aaboud:2016zkn} where final results were presented.
% 
% \toDo[carefully revisit and rewrite this]
% 
% My personal contribution to the analysis can be concluded in following three parts.
% I was involved in the analysis of muon channel. Signal selection, muon and missing transverse energy performance, validation plots and estimation of some systematic uncertainties.
% This part was done in parrallel with other collaborator to make sure that results are robust and are the same from two independent analyses codes and there are no mistaken done.
% 
% Secondly I was investigating ways to decrease systematic uncertainty caused by the limited available statistic of inclusive diboson and top backgound samples at high-$m_{T}$ region.
% One option was to generate lepton-neutrino mass-binned samples. However, it was found out that these samples doesn't populate enough high-$m_{T}$ region
% so it was decided to make samples in bins of lepton-$E_T^{miss}$ $m_T$.
% This is still ongoing project and it is planned to finish these samples for paper of the $\PZprime$ and $\PWprime$ results from 2015-2016.
% 
% The third part was related to the investigation sensitivity of $\PWprime$ signal selection for Dark Matter (DM) search.
% Main focus of this investigation was mono-W DM models, where pair of DM particles candidates are produced in final state in association with SM W boson.
% Two type of DM of models were considered: simplified and Effective Field Theory models. 


% TODO discuss with Monika - how should I position analysis - as a dedicated search of Wprime or as more model-independent search?

%*******************************************************************************
% MOTIVATION
%*******************************************************************************


\section{Search strategy}
\label{sec:wprimeIntro}
% TODO model-independent search???
% TODO experimental signature - lepton + MET
% TODO a signal discriminant: mT

As was described in \SectionRef{subsec:bsm_models} there are large number of models
which predict new spin-1 gauge charged boson. Thus it is not practical to perform
dedicated searches for all of them. This is why the so-called ``sequential'' Standard Model (SSM) is often used which is a reference benchmark model.
This model aims to provide a clear interpretation of the experimental results and
is used for results comparison between different experiments.
It assumes $\PWprime$ to be a heavy ``copy'' of the SM $W$ with the same couplings 
to the leptons, quarks and gauge bosons.

% TODO add something about mixing with SM W???
The  assumes $\PWprime$ to have the same
coupling as the SM $W$. Although it is not a gauge invariant model it is 
used by many analyses for purposes of results comparison.
Such clear interpretation allows benefit theorists to use results for other BSM models.
Considering given couplings a new decay channel (with respect to the SM $W$)
will be present $\PWprime \to WZ$. It will be dominant decay mode 
for high $\PWprime$ masses and will lead to the $\PWprime$ width larger
than its mass at  $m_{\PWprime} > 500$~GeV. Thus, branching ratio for
this channel is considered to be zero. This assumptions fits well with models, such as Left-Right Symmetric model, described previously, in which this channel is suppressed in case of $m_{W_R} \gg m_{W_L}$.
The branching ratio of $\PWprime \to \mu\nu$ or $e\nu$ as a function of $\PWprime$ mass
is shown in \FigureRef{fig:wprimeBR}. Rapid decrease at approximately 200 GeV correspond to the decay channel $\PWprime \to tb$.
\begin{figure}[!htb]
  \centering
  \includegraphics[width=7.5cm]{Wprime/WprimeBR.eps}
  \caption{Branching ratio of $\PWprime \rightarrow e\nu$ or $\mu\nu$ as a function of the $\PWprime$ mass.}
  \label{fig:wprimeBR}
\end{figure}

Considering used model one can highlight three key features important for this search:
\begin{itemize}
 \item Precise modeling of the background prediction. 
 Largest part of the background originates from the charged-current Drell-Yan process and the analysis selection tests it up to the few TeV.
 Thus it is crucially important to use the latest and most precise high-order calculations and corrections available at a time.
 \item High-$p_T$ lepton selection. Due to topologically simple event selection (one isolated lepton and missing energy) 
 this analysis use most energetic lepton candidates available at 13 TeV center-of-mass energy collisions.
 This is why reconstruction efficiency and momentum resolution of high-$p_T$ leptons are factors of special interest for the analysis.
 \item Missing transverse energy, $E_T^{miss}$. Along with the lepton momentum, $E_T^{miss}$ is used for transverse invariant mass ($m_T$) calculations,
 which is the signal discriminant and search variable in the analysis. In order to $m_T$ to be most precisely reconstructed and modelled one need
 to have not only good lepton momentum resolution but also good understanding of the missing $E_T$ performance.
\end{itemize}


\section{$\PWprime$ signal Monte Carlo}
\label{sec:wprimeSignal} 
% TODO rewrite it a little bit, cause it is directly copied from the note!!!

% MC generator
Samples for the signal process $\PWprime \to \mu \nu$ are produced with the leading-order (LO) 
{\scshape pythia-8.183}~\cite{pythia8} generator for a series of $\PWprime$ masses. 
Additionally so-called flat sample is produced, which can be reweighted
with the correct line shape for the desired $\PWprime$ mass by an appropriate reweighting function.
In order to verify validity of this sample and validity of the reweighting procedure comparison with fixed mass samples were done.
Flat sample was generated with large statistics in order to cover
wide range of the invariant transverse mass $m_T$.
\toDo[Is $m_T$ already introduced?]

% comparison with W boson background (show plot of Wprime overlaid on top of W)
The invariant mass and transverse mass distributions for the $\PWprime$ samples with pole mass of 2,3,4 and 5 TeV are shown superimposed on top of SM $W$ background in \FigureRef{fig:signal_with_W}.

\begin{figure}
\begin{subfigure}{.5\textwidth}
  \centering
  \includegraphics[width=\textwidth]{Wprime/Signal_onTopOf_W_invMass.eps}
\end{subfigure}%
\begin{subfigure}{.5\textwidth}
  \centering
  \includegraphics[width=\textwidth]{Wprime/Signal_onTopOf_W_mT.eps}
\end{subfigure}
\caption{Invariant mass (left) and transverse mass (right) spectrum of the $\PWprime$ signal 
on top of the $W$ background on generated MC level.}
  \label{fig:signal_with_W}
\end{figure}

%*******************************************************************************
% BKG PROCESSES AND MC
%*******************************************************************************


\section{Background processes}
\label{sec:wprimeBackgrounds}

In order to look for a potential signal from new physics one have firstly to examine all other SM processes which contribute to the final state of interest.

% All background predictions are obtained with MC simulation, except for non-prompt
% lepton contribution, which arising due to jets and photon being misreconstructed
% as leptons.

Since this chapter is focused on the muon decay channel, the processes which produce muon and missing transverse energy will be discussed, 
however in general the same processes are relevant for the electron decay channel as well.

The dominant expected background in the analysis is the charged current Drell-Yan.
The SM W decays to lepton and neutrino which will be reconstructed as missing transverse energy in the detector.
Contribution of this process appears like a Jacobian peak in the $m_T$ spectrum
with maximum around 80 GeV and slowly falling tail above 80 GeV.
% TODO probably I need to move it somewhere else, no?
Since $\PWprime$ conceptually is a heavier version of the SM W, it also appears in the 
transerse mass distribution as a Jacobian peak around the pole mass of the $\PWprime$ boson (as shown in \FigureRef{fig:signal_with_W}). 
% W boson bkg
The charged current Drell-Yan is simulated with \powhegbox\ v2~\cite{Alioli:2010xd}and {\scshape pythia-8.186} at next-to-leading-order (NLO) using the CT10~\cite{CT10} NLO PDFs. 
The cross section is corrected to NNLO with using the CT14NNLO PDF set by applying to the MC generator cross sections QCD and Electroweak (EW) mass dependent $K$-factors.
In order to get sufficient statistics at high transverse mass several samples, binned in invariant mass of the lepton-neutrino pair, are used if available.
% Z boson bkg
The neutral current Drell-Yan process $Z/\gamma^* \to \mu \mu$ can contribute to the muon plus $E_T^{miss}$ final state, when one of the muons is not properly reconstructed 
in the detector. In that case it will be not used in the $E_T^{miss}$ calculations as well, which will contribute to the $E_T^{miss}$ itself, as will be described in the \SectionRef{subsec:etmiss}.
Thus $Z$ boson production have to be considered as well. The process is simulated with the same MC generators and in the same way as W boson production process.
% Contribution from taus
Contribution of the processes $W \to \tau \nu$ and $Z \to \tau \tau$, which can contribute to the muon channel 
when tau decays to muon and neutrinos as $\tau^{-} \to \mu^{-} \overline{\nu_{\mu}} \nu_{\tau}$, are considered and simulated in the same way as well.

% Top
Another background which contribute to the final state of interest is the $t\bar{t}$ and single top production.
Top quark decays immediately to the $W$ boson and $b$ quark. Further leptonic decay of $W$ provides an isolated muon and $E_T^{miss}$ from escaping neutrino.
% TODO make feynman diagrams the same as in magnars thesis p.91-92
Some Feynman diagrams of the top production processes are shown in \FigureRef{???} and \FigureRef{???}.
This background is simulated with \powhegbox\ and {\scshape pythia-6.428}~\cite{Pythia} at NLO using the CT10~\cite{CT10} NLO PDFs.
All these processes are considered as a ``Top'' background in the text below.

% Diboson
More than one SM gauge boson can be produced in a single hard interaction, thus processes with $WW$, $WZ$ and $ZZ$ boson pairs produced in the final state are considered as well.
Some Feynman diagrams of the such processes are shown in \FigureRef{???}.
Contribution to the muon plus $E_T^{miss}$ final state can be provided via decays as $WZ \to \Plepton \nu \nu \nu$ or $WZ \to \Plepton \nu q \overline{q}$.
These processes are simulated with {\scshape sherpa-2.1.1}~\cite{Sherpa} using the CT10 NLO PDFs.
Such processes are grouped as a ``diboson'' background. 

% mT-binned samples
Only inclusive samples for both diboson and top backgrounds were available. 
This is why it was considered to produce samples binned in the transverse mass of lepton plus $E_T^{miss}$.
But due to technical complications these samples were not finished for 2015 analysis 
and a dedicated fitting procedure to high-$m_T$ region have been used for the estimation of these backgrounds.
However, it is planned to use these samples for 2015 and 2016 result combination study.

% Table with MC
List of all processes with used MC generators are summarized in \TableRef{tab:MC_cross}.

\begin{table}[ht]
  \begin{center}
    \begin{tabular}{l|c|c|c}

      \hline
Process &  Generator&  PDF set & Normalization \\
&  + fragmentation/ &  & based on \\
&  hadronization & &\\
\hline\hline
&   &   \multirow{4}{*}{CT14NNLO~\cite{Dulat:2015mca}} & NNLO QCD \\
$W +$ jets, & \powhegbox\ v2~\cite{Alioli:2010xd} & &  with \vrap~\cite{vrap}, \\
$Z/\gamma^* +$ jets & + {\scshape pythia-8.186}~\cite{pythia8}  & &  NLO QED \\
 & & &  with \mcsanc~\cite{Bardin:2012jk,Bondarenko:2013nu} \\
\hline
\ttbar, t-channel $t$, & \powhegbox\ & \multirow{2}{*}{CT10} & \multirow{2}{*}{NLO QCD} \\
s-channel $Wt$ & + {\scshape pythia-6.428}~\cite{Pythia} & &  \\
\hline
\multirow{2}{*}{$WW, WZ, ZZ$} & \multirow{2}{*}{{\scshape sherpa-2.1.1}~\cite{Sherpa}} & \multirow{2}{*}{CT10} & \multirow{2}{*}{???} \\
 & & &  \\
\hline
\hline
\multirow{2}{*}{$W^\prime \rightarrow \Plepton \nu$} & \multirow{2}{*}{{\scshape pythia-8.183}} &   \multirow{2}{*}{NNPDF2.3 LO} & NNLO QCD \\
& & &  with \vrap, \\
\hline
\end{tabular}
\end{center}
  \caption{List of MC generated samples used for background prediction. 
  The used MC generator, PDF set and order of cross section calculations used for the normalization are listed for each sample.
  }
\label{tab:MC_cross}
\end{table}

%*******************************************************************************
% SELECTION
%*******************************************************************************


\section{Event and lepton selection}
\label{sec:wprimeSelection}

The analysis is based on data pp collision data collected in 2015 by the ATLAS detector with 13 TeV center of mass energy.
The integrated luminosity of the data sample corresponds to 3.2 fb$^{-1}$, the mean number of interactions per bunch crossing was 14.

An event selected for the analysis need to have at least one reconstructed vertex with at least two tracks matched to it. 
If there are several vertices, the one with the highest
$\sum p^2_T$, where $p_T$ are transverse momenta of the matched tracks, is chosen.
Events should have at least one muon candidates, and fire the single muon trigger, 
which requires the presence of one muon with $p_T > 50$ GeV.

\subsection{Lepton selection}
% description of muon reconstruction in the ATLAS detector
Muon reconstruction in ATLAS is performed independently in ID and MS detectors. 
Than information from the detectors are combined to form a muon track.
There are different possibilities how to combine information from both detectors, 
thus four different muon types are used in ATLAS~\cite{muon_performance_2015}.
In this analysis so-called combined muons are used, for which track reconstruction is done 
separately in the ID and MS and than global refit is done to form a combined track.
% muon momentum reconstruction
Muon $p_T$ is measured from the track curvature.

% As well as for electrons, there are different sets of identification criteria which provide different background supression, reconstruction efficiency and momentum measurement resolution: Loose, Medium, Tight and High$-pT$.

Muons of interest are high-$p_T$ isolated muons, tracks of which originates
from the primary vertex. Candidates have to satisfy set of following criteria:
\begin{itemize}
 \item $p_T > 55$ GeV: to ensure a high and flat trigger efficiency.
 \item $|\eta|<2.5$, excluding $1.01 < |\eta| < 1.1$: muons have to be within ID acceptance. 
 Exclusion region is applied in order to reject muons whose tracks in the muon spectrometer fall into poorly aligned chambers (relative barrel-endcap aligned).
%  TODO see comment from Monika to SS chapter
 \item Transverse and longitudinal impact parameters $|d_0|/\sigma(d_0)<3$ and $|z_0\times sin\theta|<10$~mm: 
 to verify that the muon was produced close to the primary vertex and to reject muons originating from decays of long-lived particles. Recommendation by the muon combined performance (MCP) working group has
 more strict requirement $|z_0 \times sin \theta| < 0.5$ mm, however after dedicated studies it 
 was decided to relax this requirement to 10 mm as will be described in \SectionRef{subsec:wprime_cut_optimization}.
 \item Pass ``high-$p_T$'' set of muon identification criteria defined in~\cite{muon_performance_2015}. \\ 
 This set aims to maximize the momentum resolution for tracks with $p_T > 100$~GeV.
 It includes tight requirements on the MS part of the track which reduce reconstruction
 efficiency of the muons up to 20$\%$ with respect to other identification sets, however, 
 improvement of the $p_T$ resolution is reaching approximately 30$\%$.
%  TODO describe why LooseTrackOnly isolation is enough
 \item Pass ``LooseTrackOnly'' isolation requirement~\cite{muon_performance_2015}. 
 This requirement provides 99$\%$ constant efficiency over complete ($\eta$,$p_T$) phase space.
 The discriminating variable is the ratio of the sum of $p_T$ of all tracks (excluding the muon itself) with $p_T > 1.0$~GeV within $\Delta R = min(10$ GeV $/p_T^{\mu}, 0.3)$ 
 cone around the the muon track to the muon track $p_T^{\mu}$.
\end{itemize}  
 
\subsection{Optimization of the signal selection}
\label{subsec:wprime_cut_optimization}
% TODO additional lepton veto cut

In order to suppress contribution from the neutral current Drell-Yan $Z/\gamma^*$ and \ttbar  processes,
where two isolated leptons are expected in the final state an additional lepton veto requirement
is applied. If one muon pass the signal selection and second muon fail signal but pass a loosened version of the above selection with the ``high-$p_T$'' identification working point replaced by the requirement to pass either ``medium'' or
``high-$p_T$`` identification criteria and pass a lower $p_T$ cut of $20$~GeV, such event is vetoed.
Events are also vetoed if additional electron passing the selection below is found.
\begin{itemize}
\item $|\eta| < 2.47$, excluding barrel-endcap calorimeter transition region $1.37 < |\eta| < 1.52$.
\item $p_T > 20$~GeV
\item Transverse impact parameters $|d_0|/\sigma(d_0) < 5$
\item Pass the likelihood ``medium'' identification criteria~\cite{ATL-PHYS-PUB-2015-041}.
\item Pass the ``loose'' isolation criteria~\cite{ATLAS-CONF-2016-024}.
\item Electron has to not overlap with the muon: $\Delta R(e,\mu)>0.1$. If it overlaps ($\Delta R(e,\mu)>0.1$)
it is assumed that the electron candidate arise from photon radiation from the muon and the event is kept.
\end{itemize}
Veto requirements lead to a significant reduction of the dimuon ($Z$) background
at high transverse mass as well as some reduction of the $t\bar{t}$ background, it leads to a reduction of the total background level of approximately $10$--$15\%$ at high transverse mass. 
The signal efficiency is found to be essentially unaffected.
The possibility of using the ``loose'' identification working point for the additional muon veto was also considered. However, it was found to provide a tiny improvement ($1$--$3\%$ additional reduction of the total background level) with respect to using ``medium'' working point, thus latter was chosen to be used.

% TODO z sin_theta
The default recommendation by the muon combined performance (MCP) working group is to apply requirement $|z_0 \times sin \theta| < 0.5$ mm. The main purpose of this requirement is to veto events with cosmic muons.
However by making requirement on absence of the second muon in the event we ``automatically'' discard most of the events with cosmic muons. After dedicated study by the MCP group it was found that on 
$|z_0 \times sin \theta|$ requirement can be loosened to 10 mm, without significant decrease on cosmic muon rejection power.
Figure \ref{fig:Muon_LepVtxEff} shows $d_0$ significance and $|z_0 \times sin \theta|$ cut efficiencies. The efficiency shown on the left side for the recommended $|z_0 \times sin \theta|$ cut value of $0.5$ mm while on the right side - for a looser
cut value of $10$~mm. The nominal cut value leads to a reduction of the selection efficiency by about $1$\% due to picking the wrong vertex as primary vertex. This efficiency
is partially restored by loosened cut.

\begin{figure}[]
  \centering
  \includegraphics[width=0.45\textwidth]{Wprime/IPplot05mm.eps}
  \includegraphics[width=0.45\textwidth]{Wprime/IPplot10mm.eps}
  \caption{$d_{0}$ and $|z_0 \times sin \theta|$ cut efficiencies. The $d_{0}$ efficiency is shown for the cut recommended by the tracking group. In the $|z_0 \times sin \theta|$ case the recommended cut of $0.5$ mm (left) and an alternative cut of $10$ mm (right) are shown. The efficiencies are calculated for combined muons in the
$\PWprime$ flat sample passing the medium or high-$p_T$ working point requirements.}
  \label{fig:Muon_LepVtxEff}
\end{figure}


\subsection{Transverse mass and missing transverse energy}
\label{subsec:etmiss}
The missing transverse energy, $E_T^{miss}$, is calculated following the ATLAS recommendation and recommendations described in ref.~\cite{met2015_1,met2015_2}.
$E_T^{miss}$ is designed as a vector sum of the $p_T$ of selected objects:
\begin{itemize}
 \item muons which satisfy analysis signal selection.
 \item electrons which satisfy requirements described previously in
 \SectionRef{subsec:wprime_cut_optimization} with stronger transverse momentum requirements $p_T > 55$ GeV
 and likelihood ``tight'' identification criteria (which is electron signal selection).
 \item tau leptons which satisfy ``medium'' identification criteria~\cite{tau_id_8TeV} and $|\eta| < 2.5$, excluding  $1.37 < |\eta| < 1.52$ and $p_T > 20$~GeV requirements.
 \item photons which satisfy ``tight'' identification criteria~\cite{photon_id_2011}, $|\eta| < 2.37$, excluding  $1.37 < |\eta| < 1.52$ and $p_T > 25$~GeV requirements.
 \item jets reconstructed with anti-$k_t$ algorithm~\cite{jet_anti_kt} with radius parameter of 0.4.
 Jets are calibrated using the method described in ref.~\cite{jet_calib_syst_13TeV}.
 Only jets with $p_T > 20$~GeV and $|\eta| < 5$ are used.
 \item tracks originating from the primary vertex with $p_T > 0.5$ GeV and $|\eta| < 2.5$ but not belonging to any of the reconstructed physics objects listed above.
 \item \toAsk[unused calo contribution?]
\end{itemize}
The missing transverse energy is required to be larger than 55~GeV in order 
to balance the lepton transverse momentum cut.
Such high cut allows significantly suppress 
the multi-jet background which will be described below.

% TODO describe why if muon is not reconstructed it will contribute to Etmiss (see Magnars thesis, page 89)

The main variable of interest $m_T$ which is used for statistical discovery analysis are defined as:
\begin{equation}
 m_\mathrm{T} = \sqrt{2 p_\mathrm{T} p_T^{miss} (1-\cos\varphi_{\Plepton\nu})}
\end{equation}
where $\varphi_{\Plepton\nu}$ is the angle between the muon and $E_T^{miss}$ in the transverse plane.

The transverse mass has to be $m_\mathrm{T} > 110$~GeV.

%*******************************************************************************
% BACKGROUND ESTIMATION
%*******************************************************************************

\section{Background estimation}
\label{sec:wprime_backgroundEstimation}
% TODO describe sources of fakes:

The background which arise from neutral and charged current Drell-Yan processes ($W$, $Z$), diboson
production ($WW$,$WZ$,$ZZ$) and top background (in general, all processes listed in top part of \TableRef{tab:MC_cross}) are estimated with MC simulation.

Processes with multijet final state have a small possibility to be reconstructed as those which contain
an isolated muon. There are two cases when jets leads to the reconstructed muon. One of the possibilities is when hadron is not stopped in the calorimeter and go the whole way through up to muon spectrometer and is misidentified as a muon. Another possibility is when a real muon, originates from the jet activity. However, such muons are not produced in the primary vertex, because they originate from decays of heavy flavor hadrons, which have long lifetime and manage to fly for few hundreds of $\mu m$ before they decay.
Thus, reconstructed muons from jets are called ``fake'' muons.

Due to the huge cross section of the processes which lead to the multijet final state it's very difficult and time consuming to simulate their production with MC simulation, this is why a data-driven method,
the so-called Matrix Method, is used to estimate contribution of the multijet processes to the signal muon selection.

% The backgrounds which have final state with a real muon originating from the primary vertex and 
% which can contribute to the signal selection are modelled with MC simulation.
% List of all considered processes is shown in \TableRef{tab:MC_cross}.
% Such muons are called ``real'' (prompt) muons.
% 
% However, processes with multijet final states (with one or more jets) will also contribute to
% the signal selection due to probability of wrong reconstruction of the jet activity as a muon.
% Such muon candidates are called ``fake'' (non-prompt) muons.
% The ``fake'' muon can be a real muon which originate from a heavy flavor hadron decay within a jet
% or from pion or kaon decays. But because they are not originate from the primary vertex they are not
% desired to be selected in the signal region and thus are called ``fake'' muons.
% The ``fake'' muon background is estimated using a data driven technique. 
% ``Fake'' muons are expected to be in general non-isolated, although some
% fraction does pass the isolation cut and ends up in the selected event sample. Isolation variables
% hence provide a strong separation of ``fake'' and ``real'' muons, and are essential to the data-driven
% estimation of the ``fake'' background.
% The ``fake'' background is estimated using the Matrix Method, which is presented below.

\subsection{The Matrix Method}
\label{subsec:matrix_method}
The goal of the method is to get an estimation of ``fake'' muon contribution to the signal muon selection.
The ``fake'' muons in general are expected to be non-isolated, however, some fraction does pass the isolation cut. 
Thus, the idea of the method is to measure probability (efficiency) of the ``fake'' muons
which pass loosened signal selection without isolation requirement to pass nominal muon signal selection.
Also one want to measure the same efficiency for the ``real'' muons, which originate from the processes with real prompt muons from primary vertex.

The matrix method provides a connection between the number of true ``fakes'' ($N_F$) and the number of true ``real'' muons ($N_R$) and the measurable quantities from a sample in which the muons are passing loose selection ($N_L$) criteria and at the same time failing a tight selection and a sample in which the muons pass the tight selection ($N_T$) criteria via \EquationRef{eq:mm1}.
\begin{equation}
  \left(\begin{array}{c} N_T \\ N_L \end{array}\right)&=
  \begin{pmatrix}
    \epsilon_R & \epsilon_F \\
    1-\epsilon_R & 1- \epsilon_F \\
  \end{pmatrix}
  \left(\begin{array}{c} N_R \\ N_F \end{array}\right)
  \label{eq:mm1}
\end{equation} 
The vector on the right hand side of the equation corresponds to the
truth quantities which are independent from each other.
It means that quantities in the vector on the left hand side
have to be independent as well, thus $N_L$ should not contain any events
from $N_T$, and this is why former value is defined as pass loose selection but fail tight one.

Matrix consists from efficiency of ``fake'' and ``real'' muons to pass signal selection which are denoted as $\epsilon_F$ and $\epsilon_R$ respectively.
Efficiency is defined as the ratio of the number of ``fake''(``real'') muons which pass tight selection,
$N_{tight}^{fake}$ ($N_{tight}^{real}$), to the number of muons which fail tight but pass loose selection,
$N_{loose}^{fake}$ ($N_{loose}^{real}$):
\begin{equation}
 \epsilon_F = \frac{N_{\textrm tight}^{\textrm fake}}{N_{\textrm loose}^{\textrm fake}} ,\qquad & \epsilon_R = \frac{N_{\textrm tight}^{\textrm real}}{N_{\textrm loose}^{\textrm real}}.
  \label{eq:mm2}
\end{equation}
The total number of muons passed signal selection is given in the first line of the matrix:
\begin{equation}
 N_T&=N_{\textrm tight}^{\textrm real} + N_{\textrm tight}^{\textrm fake}=\epsilon_R N_R + \epsilon_F N_F\, ,
\label{eq:number_of_tight_muons}
\end{equation}
and consist from fraction of ``fake'' muons which pass signal selection and fraction of ``real'' muons.
The value of interest is the true number of ``fake'' muons which pass signal selection. One can express them by inverting the matrix in \EquationRef{eq:mm1} and using relation in \EquationRef{eq:number_of_tight_muons}:
\begin{equation}
\left(\begin{array}{c} N_R \\ N_F \end{array}\right)&=
\frac{1}{\epsilon_R(1-\epsilon_F)-\epsilon_F(1-\epsilon_R)}
\begin{pmatrix}
 1- \epsilon_F & -\epsilon_F \\
\epsilon_R-1 & \epsilon_R  \\
\end{pmatrix}
\left(\begin{array}{c} N_T \\ N_L \end{array}\right)
\end{equation} 
Thus, the true number of ``fake'' muons which pass signal selection are:
\begin{equation}
N_{\textrm tight}^{\textrm fake} = \epsilon_F N_F=\frac{\epsilon_{F}}{\epsilon_{R}-\epsilon_{F}}\left(\epsilon_{R}(N_{L}+N_{T})-N_{T}\right) \,
  \label{eq:mm5}
\end{equation}
which are expressed from measurable quantities only and can be calculated from the data.

The cut used to distinguish between loose and tight muons is the isolation cut, and the loose muons
are thus defined as passing signal muon selection cuts except the isolation cut. Tight muons correspond
to the baseline selection.

``Real'' muon efficiency is found from MC simulation, because MC reproduce efficiency of the isolation cut in data well. ``Fake'' muon efficiency is found from a control region designed to have a high purity of ``fake'' muons. The region is defined in the same way as signal selection without cuts in $E_T^{miss}$ and $m_T$ and following additional requirements:
\begin{itemize}
\item At least one jet with $p_T > 40$~GeV which does not overlap ($\Delta R > 0.2$)
with the selected muon.
\item Opening angle in the transverse plane between the muon and the $E_T^{miss}$, $\Delta\phi_{\mu,E_T^{miss}} < 0.5$.
\item No $Z$ candidate (any two muons with $80 < m_{\mu\mu} < 100$~GeV).
\item $d_0$ significance greater than $1.5$.
\item $E_T^{miss} < 55$~GeV, ensuring that the control region does not overlap with the signal region.
\end{itemize}
This region is enhanced with ``fake'' muons, however, significant ``real'' muon contamination is present in the region as well. This is why a ``real'' contribution predicted with MC modeling is subtracted.

Obtained efficiencies are shown in \FigureRef{fig:matrix_method_efficiencies}.
\begin{figure}[]
  \centering
  \includegraphics[width=0.45\textwidth]{Wprime/realEffNominal.eps}
  \includegraphics[width=0.45\textwidth]{Wprime/fakeEffNominal.eps}
  \caption{Efficiency of the ``real'' (left) and ``fake'' (right) muons as a function of muon $p_T$ used in the data-driven matrix method to estimate contribution of the multijet background to the signal selection.}
  \label{fig:matrix_method_efficiencies}
\end{figure}

Systematic uncertainty were estimated for both ``real'' and ``fake'' muon efficiencies.


``Fake'' muon efficiency uncertainty was estimated by variation of the requirements for ``fake'' muon control region. These variations are:
% TODO rewrite list!!!
\begin{itemize}
\item Removing the $Z$ veto and $\Delta\phi_{\mu,\met}$ cuts.
\item Removing the $Z$ veto and $\Delta\phi_{\mu,\met}$ cuts, but tightening the $d_0$ significance
cut to $2$.
\item Removing the $d_0$ significance cut. 
\item Using a tighter $d_0$ significance cut of $2$.
\item Removing the jet requirement.
\item Removing the jet requirement, but tightening the $d_0$ significance
cut to $2$.
\item Requiring $\met < 20~\GeV$.
\item Requiring $\met > 20~\GeV$ (but still $\met < 55~\GeV$).
\end{itemize}
Effect on the ``fake'' muon efficiency is shown in \FigureRef{fig:matrix_method_systematics} (right).

Since ``real'' muon efficiency is obtained from MC simulation one can use efficiency obtained with tag-and-probe method using muon pairs in the invariant mass window $80 < m_{\mu\mu} < 102~\GeV$ from $Z\to\mu\mu$ decays as a systematic uncertainty variation. It is shown in \FigureRef{fig:matrix_method_systematics} (left).

\begin{figure}[]
  \centering
  \includegraphics[width=0.45\textwidth]{Wprime/realEffVariations.eps}
  \includegraphics[width=0.45\textwidth]{Wprime/fakeEffVariations.eps}
  \caption{Systematic variations of the ``real'' (left) and ``fake'' (right) muon efficiencies as a function of muon $p_T$.}
  \label{fig:matrix_method_systematics}
\end{figure}

Impact of the systematic variations of the efficiencies on the final $m_T$ spectrum of the multijet background will be discussed in \SectionRef{subsec:multijet_systematcs}.

%*******************************************************************************
% VALIDATION REGIONS
%*******************************************************************************

\subsection{Multijet validation region}

\toAsk[this subsection has to be carefully reviewed]

To test data-driven prediction of the multijet background one want to find a region
where its contribution will be enhanced. Also region has to be kinematically 
close to the signal selection.
A validation region is defined in the same way as the signal selection but without the $E_T^{miss}$ and $m_T$ requirements. Also distribution with and without using isolation requirements are considered, 
which corresponds to tight and loose muon definitions used in the matrix method described above.

The validation distributions of variables, used in the definition of the enhanced ``fake'' muon control region, are shown in \FigureRef{fig:muMMval1}.
In general, reasonable agreement is observed. The most obvious discrepancy is seen in the
distribution of the number of jets. The disagreement is likely mostly due to the modeling
of jet emission in the $W\to\mu\,\nu$ MC, where only one jet emission is included at the
matrix element level calculations.
This affects in principle the MC subtraction in the ``fake'' muon control region,
but discrepancy is present only for $N_\mathrm{jet}\geq2$, while control region is 
defined with requirement $N_\mathrm{jet}\geq1$ and thus this discrepancy is not 
strongly affected the ``fake'' muon efficiency calculations.
Furthermore, as a systematic variation of the ``fake'' muon efficiency was estimated with no $N_\mathrm{jet}$ cut at all and was found to be negligible which prove that discrepancy in
the $N_\mathrm{jet}$ is negligible as well.

% TODO think about some additional description here!
% TODO check in the Magnars thesis
Distribution of the $E_T^{miss}$, muon momentum and $m_T$ are shown in \FigureRef{fig:muMMval2}.
As one can note in the loose muon sample contribution of the multijet background are significantly enhanced with respect to the tight sample. 
Taking into account size of systematic uncertainties, discussed in \SectionRef{sec:wprimeSystematics}, data and background prediction agrees reasonably which proves validity of the data-driven jet background method.
The tight muon sample includes isolation requirement and thus contribution from multijet background is significantly reduced. Also because no $E_T^{miss}$ and $m_T$ are applied 
there are significantly more statistics available than in the signal region which allows 
to test overall background prediction. In general, reasonable agreement is observed. 

\toAsk[Is explanation above good enough?]

\toAsk[is ``reasonable agreement'' sounds strange for such plots?]

\toAsk[should I focus more of the shape comparison of tight and loose plots?]

\toDo[more discussion here?]

\begin{figure}[]
  \centering
  \includegraphics[width=0.49\textwidth]{Wprime/Njet40loose.eps}
  \includegraphics[width=0.49\textwidth]{Wprime/Njet40.eps}
  \includegraphics[width=0.49\textwidth]{Wprime/deltaPhiLoose.eps}
  \includegraphics[width=0.49\textwidth]{Wprime/deltaPhi.eps}
  \includegraphics[width=0.49\textwidth]{Wprime/d0sigLoose.eps}
  \includegraphics[width=0.49\textwidth]{Wprime/d0sig.eps}
  \caption{
  Distributions of the number of jets (top), $\Delta\phi_{\mu,\met}$ (middle), 
  and $d_0$ significance (bottom) in the inclusive loose (left) and tight (right) muon samples. 
  The distributions are considered before the $E_T^{miss}$ and $m_T$ cuts.
}
  \label{fig:muMMval1}
\end{figure}
\begin{figure}[]
  \centering
  \includegraphics[width=0.49\textwidth]{Wprime/METloose.eps}
  \includegraphics[width=0.49\textwidth]{Wprime/MET.eps}
  \includegraphics[width=0.49\textwidth]{Wprime/pTloose.eps}
  \includegraphics[width=0.49\textwidth]{Wprime/pT.eps}
  \includegraphics[width=0.49\textwidth]{Wprime/mTloose.eps}
  \includegraphics[width=0.49\textwidth]{Wprime/mT.eps}
  \caption{
  Distributions of the $E_T^{miss}$ (top), $p_T$ (middle), and $m_T$ (bottom)
  in the inclusive loose (left) and tight (right) muon samples. 
  The distributions are considered before the $E_T^{miss}$ and $m_T$ cuts.
}
  \label{fig:muMMval2}
\end{figure}


\subsection{Background extrapolation}
% TODO explain why it hurts to have huge statistical fluctuation for limit settings 
% TODO (it was somewhere in the note?)
% % from app_muonSmooth.tex:
% In a single-bin statistical analysis,
% it is straight forward to take into account the statistical uncertainty of the MC background
% estimates, as one can simply add it in quadrature to the systematic uncertainties on the
% background level in the search region. However, in a multi-bin analysis, results depend on
% the shape of the background distribution, and one should avoid propagating clearly unphysical features
% in this shape to the final results.

MC simulation of the top and diboson background processes were available only as an inclusive samples,
thus they didn't provide enough statistics in the high-$m_T$ region.
Therefore these backgrounds were fitted in the low-$m_T$ region and extrapolated to obtain smooth description in the high-$m_T$ region.

% TODO this is from Marcus --> change phrasing!!!
The fit was done with functions that are commonly used to extrapolate the background, as an example in the $8~\TeV$ dilepton resonance search~\cite{Aad:2014cka}.
The first considered function is defined as::
\begin{equation}
 f(\mt) = e^{-a} m_\mathrm{T}^{b} m_\mathrm{T}^{c \log(m_\mathrm{T})}.
  \label{eq:dijetfunc}
\end{equation}
The second one is:
\begin{equation}
 f(\mt) = \frac{a}{(m_\mathrm{T}+b)^{c}}.
  \label{eq:powerlaw}
\end{equation}
These two functions were used to preform fit of the backgrounds in different ranges. 
The best fit according to $\chi^{2}/N.d.o.f$ value have been used as a central value.
The systematic uncertainty is taken as the envelope of all fits.
Statistical uncertainty of the fit was found to be negligible.

The starting range for the top background was in the range from $140$~GeV to $260$~GeV in steps of $20$~GeV. The end fit point - from $600$~GeV to $900$~GeV in steps of $25$~GeV.
For diboson background these value were from $120$~GeV to $240$~GeV and from $500$~GeV to $700$~GeV with the same step widths as for top.
The extrapolation was stitched to the background estimated with Monte Carlo at $\mt=600$~GeV in both cases.

Fits and appropriate systematic uncertainty estimations are shown in 
\FigureRef{fig:mu_extrapolate_top} fo the top and in \FigureRef{fig:mu_extrapolate_diboson} for the diboson backgrounds.
\begin{figure}[!htb]
  \centering
  \includegraphics[width=0.49\textwidth]{Wprime/top_extrapolate_fits.eps}
  \includegraphics[width=0.49\textwidth]{Wprime/top_extrapolate.eps}
  \caption{Fit and extrapolation of the top background. Both the full set of individual
fits (left) and the resulting central value and uncertainty (right) are shown.}
  \label{fig:mu_extrapolate_top}
\end{figure}
\begin{figure}[!htb]
  \centering
  \includegraphics[width=0.49\textwidth]{Wprime/diboson_extrapolate_fits.eps}
  \includegraphics[width=0.49\textwidth]{Wprime/diboson_extrapolate.eps}
  \caption{Fit and extrapolation of the diboson background. Both the full set of individual
fits (left) and the resulting central value and uncertainty (right) are shown.}
  \label{fig:mu_extrapolate_diboson}
\end{figure}

The multijet data-driven background estimation also suffers from large statistical fluctuation at high-$m_T$ region, thus, similar fitting and extrapolation is done as well.
A simple power law fit is performed, which was found to be the most appropriate way according to the 8 TeV analysis~\cite{wprime_8TeV}:
\begin{equation}
\frac{dN}{d m_T} = a\, m_T^{-b}
\end{equation}
Fits are done in the ranges $150$--$300~\GeV$ and $200$--$300~\GeV$.
The extrapolation was stitched to the multijet background estimate $\mt=300$~GeV.

% TODO where it was stitched?

%*******************************************************************************
% SYSTEMATICS
%*******************************************************************************

\section{Systematic Uncertainties}
\label{sec:wprimeSystematics}
\subsection{Muon efficiency, resolution and scale}
The muon efficiency corrections from the MCP group are obtained using the tag-and-probe method
on $Z\to\mu\mu$ and $J/\psi\to\mu\mu$ decays in data~\cite{MCP13TeV}. Systematic uncertainties
are derived from variations of the tag-and-probe selection, background subtraction etc. following
the methodology documented in ref.~\cite{MCPrun1}.

The muon momentum corrections from the MCP are obtained by fitting certain correction constants 
to match the invariant mass distribution in $Z\to\mu\mu$ and $J/\psi\to\mu\mu$ decays in MC
to that observed in data~\cite{MCP13TeV}. The dependence of the muon momentum on the fit parameters
is given by a model where each parameter is associated to a certain source of potential data/MC disagreement.
Systematic uncertainties are derived from variations of the fit procedure, alignment studies etc. 
following the methodology documented in ref~\cite{MCPrun1}.

\subsection{Jet energy scale and resolution}
% TODO write something more (about jet energy scale)
The jet energy scale and resolution uncertainties enter the analysis through 
the $\met$ calculation, since the $\met$ is calculated using calibrated jets. 
The uncertainties for the jet energy scale and resolution are provided 
by the ATLAS JetEtMiss working group ~\cite{jet_calib_syst_13TeV, JESUncer13TeV}. 
% A reduced set of uncertainties with three nuisance parameters is chosen for the jet energy scale. This reduced set of
% nuisance parameters simplifies the correlations between the different sources of the jet energy scale uncertainty (JET\_GroupedNP\_1, JET\_GroupedNP\_2, JET\_GroupedNP\_3).
% Four scenarios of correlation models are provided by the JetEtMiss group. The final result of an analysis using the reduced set must not
% depend on a specific choice of correlation model. 
The jet energy scale uncertainty has been tested for different recommended scenarios 
and was found to be negligible for all of them.\\
No resolution smearing is applied in the default scenario. 
According to working group recommendation effect of the smearing has to be used as a systematic uncertainty.\\ The jet uncertainties are fully correlated between the electron and muon channel.

\subsection{Transverse missing energy scale and resolution}
The uncertainties for the $\met$ scale and resolution are provided by the JetEtMiss group~\cite{met2015_1}. They enter
the analysis through the soft term in the $\met$ calculation, 
which corresponds to the energy deposits in the calorimeter not associated with
any reconstructed physics objects (leptons, photons, jets).
The uncertainties cover differences between data and MC and are only applied to MC. 
The $\met$ uncertainties are fully correlated between the electron and muon channel. 
Also the jet, electron and muon energy/momentum uncertainties are affecting the $\met$
calculation. These uncertainties are propagated to the $\met$ calculation in the same way.

\subsection{Background estimate uncertainty}

Uncertainty of the charged and neutral current Drell-Yan processes were estimated by variations of $\alpha_s$ and electroweak corrections as well as by using PDF error set and estimation pf difference between alternative PDF sets.
The $\alpha_s$ uncertainty was estimated by varying $\alpha_s$ by $\pm 0.0003$. This corresponds to the 90\% C.L uncertainty. The effect on the $W$ background when doing this variation
was $3\%$ at maximum and the effect is therefore neglected. 
The variation of the electroweak corrections was estimated to be larger than $3\%$ and was
taken into account during limit settings.
The PDF uncertainty of the CT14NNLO PDF is one of the main theory uncertainties
and it was calculated by using 90$\%$ CL PDF error set.
Uncertainty related to the choice of the PDF set to use was estimated by comparing
results with NNPDF3.0~\cite{Ball:2014uwa}.
The difference between CT14 and HERAPDF2.0 is not considered as the PDF does not include high-x data. 

Uncertainty on ``Top'' and ``Diboson'' background consist from theoretical uncertainty on cross section and high-$m_T$ extrapolation uncertainty. The former uncertainty affect total background
prediction less than 3$\%$ and thus are neglected, the latter one become visible at high-$m_T$
region is is taken into account during limit settings.

The detailed description of the theoretical uncertainties on MC cross section can be found in~\cite{Aaboud:2016zkn}.

\subsection{Trigger and luminosity}
Trigger systematics is evaluated by the ATLAS trigger group and it is related to the trigger
efficiency of muons with different $\eta$ and $p_T$.
Luminosity uncertainty was estimated in the same way as described in \SectionRef{sec:lucid_performance}, however current analysis is using preliminary luminosity
uncertainty equal to 5$\%$ which was obtained from preliminary calibration of the luminosity scale done with using data from vdM scans in August 2015.

\subsection{Multijet background}
\label{subsec:multijet_systematcs}

Systematic variations on the ``real'' and ``fake'' efficiency used in the matrix method were
described previously in \SectionRef{subsec:matrix_method}.
Multijet background have a very small contribution to the muon signal selection, as can be seen in 
\FigureRef{fig:muMMfinal} (left), where fraction of the multijet background to the total background
after the final selection is shown as function of $m_T$. Thus affect from the systematic
variations of the matrix method efficiencies on the total background is small as well, as shown in
\FigureRef{fig:muMMfinal} (right). 
Systematic uncertainty on the total background is at level of 1 $\%$ for $m_T<4$~TeV and less than 2$\%$ for $m_T>4$~TeV.

\begin{figure}[!htb]
  \centering
  \includegraphics[width=0.49\textwidth]{Wprime/totalBGfrac.eps}
  \includegraphics[width=0.49\textwidth]{Wprime/totalBGsys.eps}
  \caption{The fraction that the fake muon background constitutes of the total background
as function of \mt\ (left) and the effect of systematics variations on the total background
level as function of \mt\ (right). The power law fits are used for the fake muon background above
$\mt=300~\GeV$. In the left plot, the fit range $150$--$300~\GeV$ is used, and in the right plot, the
range $150$--$300~\GeV$ corresponds to solid lines and $200$--$300~\GeV$ to dashed lines.}
  \label{fig:muMMfinal}
\end{figure}

\subsection{Summary}
\TableRef{tab:syst} lists the various systematic uncertainty sources
and their size for background and a 2 or 4~TeV $\PWprime$ signal at a transverse mass at 2 or 4~TeV.
Uncertainties which do not have a number are either neglected or do not apply. 
All uncertainties below $3$\% have been neglected so far
since they do not affect the final result of the statistical analysis. 
The remaining experimental and theoretical systematics are applied to the background.
To the signal only the experimental uncertainties are applied. 

\begin{table}
\begin{center}
\centering
\small
\begin{tabular}{l|cc}
\toprule
Source &  Background  &  Signal  \\
\midrule
Trigger &\syspair{3}{4} & \syspair{4}{4}\\
Lepton reconstruction  &\multirow{2}{*}{\syspair{5}{8}} & \multirow{2}{*}{\syspair{5}{7}}\\
and identification & & \\
Lepton isolation &\syspair{5}{5} & \syspair{5}{5}\\
Lepton momentum &\multirow{2}{*}{\syspair{3}{11}} & \multirow{2}{*}{\syspair{1}{4}}\\
scale and resolution & & \\
$E_T^{miss}$ resolution and scale &\syspair{<0.5}{<0.5} &\syspair{<0.5}{<0.5}\\
Jet energy resolution &\syspair{1}{2} &\syspair{<0.5}{<0.5}\\
\midrule
Multijet background & \syspair{1}{1} & {\sc n/a} ({\sc n/a})\\
Diboson \& top-quark bkg. &\syspair{5}{15} & {\sc n/a} ({\sc n/a})\\
PDF choice for DY &\syspair{<0.5}{1} & {\sc n/a} ({\sc n/a})\\
PDF variation for DY &\syspair{8}{12} & {\sc n/a} ({\sc n/a})\\
Electroweak corrections &\syspair{4}{6} & {\sc n/a} ({\sc n/a})\\
\midrule
Luminosity &\syspair{5}{5} &\syspair{5}{5}\\
\midrule
Total &\syspair{14}{25} & \syspair{9}{12}\\
\bottomrule
\end{tabular}
\end{center}
\caption{Systematic uncertainties on the expected number of events as evaluated at $m_T = $ 2 (4)~\TeV, both for signal events 
with a \wpssm\ mass of 2~(4)~\TeV\ and for background. Uncertainties estimated to have an impact
$< 3\%$ on the expected number of events in both channels and for all values of $m_T$ are not listed.
Uncertainties that are not applicable are denoted ``n/a''. \label{tab:syst}}
\end{table}

%*******************************************************************************
% SIGNAL REGION
%*******************************************************************************
\section{Signal Region}
\label{sec:wprimeSignalRegion}
The muon $\eta$, $\phi$, $p_T$, and $E_T^{miss}$ distributions in the signal region are shown in \FigureRef{fig:mu_results_etaphi} and \FigureRef{fig:mu_results_ptmet}. 
The dominant contribution to the signal region originates from W boson background.
No visible excess is observed and good agreement between data and background estimate is observed.

The basis for the statistical analysis and the main distribution of interest 
is the transverse mass distributions which is shown in \FigureRef{fig:MT_mu_Wprime}.
The resonant $\PWprime$ signal overlaid on background prediction is shown as well.
As one can see from the ratio plot a data is systematically above the total background prediction
in low-$m_T$ region but are within $\pm 1 \sigma$ uncertainty band, which is dominated by the $E_T^{miss}$ systematic uncertainty at low $m_T$.

\begin{figure}[]
  \centering
  \includegraphics[width=0.49\textwidth]{Wprime/muon_eta.eps}
  \includegraphics[width=0.49\textwidth]{Wprime/muon_phi.eps}
  \caption{
  Muon \eta\ (left) and $\phi$ (right) distributions after final selection. The uncertainty band in the ratio plot includes all systematic uncertainties which are included in the statistical analysis except the integrated luminosity uncertainty ($5\%$).
}
  \label{fig:mu_results_etaphi}
\end{figure}

\begin{figure}[]
  \centering
  \includegraphics[width=0.49\textwidth]{Wprime/muon_pT.eps}
  \includegraphics[width=0.49\textwidth]{Wprime/muon_MET.eps}
 \caption{
 Muon $p_T$ (left) and $E_T^{miss}$ (right) distributions after final selection. The uncertainty band in the ratio plot includes all systematic uncertainties which are included in the statistical analysis except the integrated luminosity uncertainty ($5\%$).
}
  \label{fig:mu_results_ptmet}
\end{figure}


\begin{figure}[]
  \centering
  \includegraphics[width=0.65\textwidth]{Wprime/muon_mT.eps}
  \caption{
  Muon $m_T$ distribution after final selection. 
  Shown is the total background estimate with resonant $\PWprime$ signal overlaid for various pole masses. 
  The uncertainty band in the ratio plot includes all systematic uncertainties which are included in the statistical analysis except the integrated luminosity uncertainty (5$\%$).}
  \label{fig:MT_mu_Wprime}
\end{figure}

\TableRef{tab:muBkgData}
shows the contributions of individual backgrounds as well as the total background
and the data in different $\mt$ regions. The quoted uncertainties include both systematic and
statistical uncertainties except the uncertainty on the integrated luminosity ($5\%$).
One can observe good agreement between data and total background prediction in all $\mt$ regions. Charged-current DY provides the dominant contribution to the high-$m_T$ region which is above 90$\%$ of the total background for $m_T>1$ TeV region. No events with $m_T > 3$ TeV are
observed in data.

\begin{table}[]
  \centering
  \scriptsize
  \begin{tabular}{|c|c|c|c|c|c|c|c|}
    
    \multirow{2}{*}{Process} & \multicolumn{7}{c|}{$m_T$ [\GeV]} \\
& $110$--$150$ & $150$--$200$ & $200$--$400$ & $400$--$600$ & $600$--$1000$ & $1000$--$3000$ & $3000$--$7000$ \\ \hline 
$W$ & $98100\pm10000$ & $21000\pm2000$ & $7700\pm400$ & $476\pm30$ & $110\pm9$ & $13.0\pm1.2$ & $0.051\pm0.010$ \\ 
Top & $9900\pm700$ & $5410\pm340$ & $3090\pm140$ & $120\pm6$ & $13\pm5$ & $0.44\pm0.32$ & $0.00005\pm0.00030$ \\ 
$Z/\gamma^*$ & $7700\pm1000$ & $2130\pm250$ & $840\pm70$ & $37\pm4$ & $7.6\pm1.8$ & $0.64\pm0.06$ & $0.0037\pm0.0007$ \\ 
Diboson & $1140\pm80$ & $588\pm33$ & $326\pm14$ & $20.6\pm1.2$ & $3.8\pm2.1$ & $0.4\pm0.4$ & $0.002\pm0.008$ \\ 
Multi-jet & $1350\pm40$ & $551\pm23$ & $180\pm10$ & $5.6\pm1.0$ & $0.85\pm0.21$ & $0.078\pm0.028$ & $0.00038\pm0.00022$ \\ \hline 
Total SM & $118000\pm12000$ & $29700\pm2600$ & $12100\pm600$ & $660\pm40$ & $135\pm11$ & $14.6\pm1.4$ & $0.058\pm0.013$ \\ \hline 
Data & $131672$ & $31980$ & $12393$ & $631$ & $121$ & $15$ & $0$ \\ 
\end{tabular}
\caption{Contributions of individual backgrounds with uncertainties for different $m_T$ regions.
The uncertainties include both statistical and systematic uncertainty, and all weights are included
so that the total background level can be compared to data. The systematic uncertainty includes all systematic 
uncertainties which are included in the statistical analysis except the uncertainty
on the integrated luminosity ($5\%$). For the multi-jet background, only statistical uncertainty is shown,
since the multi-jet systematics are not included in the statistical analysis.}
\label{tab:muBkgData}
\end{table}

\toAsk[something more here?]

%*******************************************************************************
% XSEC and MASS LIMITS
%*******************************************************************************

\section{Cross section and mass limits}
% TODO say that Bayesian approach is used...
\toDo[some text here]


% TODO describe look for deviation
To search for a $\PWprime$ signal-like excess in the data a log likelihood ratio test is performed using RooStat~\cite{RooStat_project} framework.
The likelihood function is constructed as the product of Poisson probabilities of over all $m_T$ bins in the search region.
The effect of systematic uncertainties are described by nuisance parameters in the likelihood function.

One of the needed component for the statistical analysis to calculate the number of expected events is total efficiency (product of acceptance and reconstruction efficiency) 
of the signal selection to reconstruct $\PWprime$ decay, which is shown in \FigureRef{fig:AccEff_mu}.

\toDo[explain shape of the total efficiency curve!]

\begin{figure}[]
  \centering
  \includegraphics[width=0.65\textwidth]{Wprime/acceptance.eps}
  \caption{
%   Total signal acceptance times efficiency versus SMM \wp\ pole mass for the SSM \wp\ model in the muon channel.
  }
  \label{fig:AccEff_mu}
\end{figure}

Statistical analysis demonstrates that no excess larger than 2$\sigma$ is observed.
This is why upper limit on the cross section for the production of $\PWprime$ times branching ratio has been set. 
A Bayesian approach have been used for the limit settings and limits were calculated with help of the Bayesian Analysis Toolkit~\cite{BAT}.

\toAsk[is it OK to not explain in the detail about limit setting procedure?]

\toDo[add something here???]

Upper limits are set on the cross-section times branching ratio, $\PWprime \rightarrow \Plepton\nu$, at 95\% C.L. and mass 
limits are extracted using the relationship between the theoretical $\PWprime$ cross section and pole mass. Limits are presented for the electron, muon, 
and combined lepton channels in \TableRef{tab:limits_mass_wp}, \FigureRef{fig:wprime_limits} and \FigureRef{fig:wprime_limits_combined}.

\toAsk[which conclution whuols I made? comparison with 8 TeV?]

\begin{figure}[]
  \centering
\includegraphics[width=0.49\textwidth]{Wprime/Limit_xsec_wprime_m_Sys.eps}
\includegraphics[width=0.49\textwidth]{Wprime/Limit_xsec_wprime_e_Sys.eps}
\caption{$\PWprime$ cross section limit results for the muon (left) and electron (right) channels.}
\label{fig:wprime_limits}
\end{figure}


\begin{figure}[]
  \centering
\includegraphics[width=0.65\textwidth]{Wprime/Limit_xsec_wprime_comb_Sys.eps}
\caption{$\PWprime$ cross section limit results for combined both muon and electron channels.}
\label{fig:wprime_limits_combined}
\end{figure}


\begin{table}[]
  \centering
  \begin{tabular}{c|cc}
    \hline
    \hline
    &  \multicolumn{2}{c}{$m_{\PWprime}$ lower limit [\TeV]} \\
    Decay     &  Expected & Observed \\
    \hline
    \wpe  & 3.99 & 3.96 \\
    \wpmu & 3.72 & 3.56 \\
    \wpl  & 4.18 & 4.07 \\
    \hline
    \hline
  \end{tabular}
  \caption{Expected and observed 95\% CL lower limit on the \wpssm\ mass in the electron and muon channels and their combination.}
  \label{tab:limits_mass_wp}
\end{table}



%*******************************************************************************
%*******************************************************************************
% MONO-W STUDY
%*******************************************************************************
%*******************************************************************************

% Sources:
% 
% Dark Matter Benchmark Models for Early LHC Run-2 Searches:
% Report of the ATLAS/CMS Dark Matter Forum (July 6, 2015)
% http://arxiv.org/pdf/1507.00966.pdf
% 

\section{Search for dark matter pair-production with a leptonically decaying W boson}
\label{chap:monoW}

\subsection{Introduction}

Beside $\PWprime$ model discussed above there are other BSM models 
with lepton plus $E_{T}^{miss}$ final state signature.
One of such model is associated 
production of pair of weakly interacting massive particles (WIMP)
with SM W boson. WIMPs are one of possible dark matter candidates (DM).

% Another possible BSM signature which can be easily tested in signal region is
% associated production of pair of weakly interacting massive particles (WIMP), 
% which are candidates for dark matter (DM) particles, with SM W. 
Because WIMPs don't interact strongly or electromagnetic they will most probably escape from detector the same way as neutrino
from leptonic W decay and will contribute to $E_{T}^{miss}$ of the event.

In this section qualitative study based on using models recommended by Dark Matter Forum (link?) are presented 
which address question about sensitivity of signal selection presented above to such kind of models.


% While searching for $\PWprime$ boson one can naivly expect that distributions of 
% transferred energy of decay products will look quite similar, which
% means that distributions of transfer lepton momentum and missing energy will 
% look the same as well. But if we consider pair-production of dark matter 
% particles associated with W we expect 

\subsection{Theoretical models}

% TODO rephrase first setence
There are plethora of models which try to introduce and explain DM as a possible particles which can be produced at LHC. 
But all these models can be classified to the three distinct classes: DM Effective Field Theories (EFT), Simplified DM models 
and Complete DM models. EFT approach allows to describe the DM-SM interactions mediated by all 
kinematically inaccesible particles in a universal way. 
It allows to derive stringent bounds on the ``new physics'' scale $\Lambda$. 
Simplified models are charachterized by the most important state mediating the DM interaction with SM. Unlike EFT approach,
simplified models are able to describe correctly the full kinematics of DM production, because they resolve the EFT contact interaction in single-particle 
s- or t-channel exchange. Complete DM models allows to describe correlation between observables~\cite{arXiv:1506.03116}.

Main focus of current study will stay on EFT and Simplified DM models with W boson and DM particles produced in the final state. 
In simplfied model W is produced as initial state radiation from one of incoming quarks as shown at \FigureRef{fig:feynMonoWSimple}. 
In EFT approach W can be produces as final state particle (\FigureRef{fig:feynMonoWEFT}) or initial state radiation (\FigureRef{fig:feynMonoWEFT}).

% TODO: explanation of simplified and EFT models (take from DM forum report or fro Bell's papers).


% and mediator connect SM interaction from one side and DM 

% TODO find reference from DM forum report which explain idea of simlified models 
% (that we introduce additionla mediator which on one side interact with SM particles and on other side - with dark matter particles).
% same for EFT models. But for now skip this part.

% TODO describe possible type of interection for mono-W models.

% TODO figure out what does it mean constructive and destructive intereference for DX models?

% TODO explanation for contact interaction is needed as well, ha-ha-ha...

% TODO say that our main focus is on simplified models because they are recommended by Dark Matter forum.

\begin{figure}[]

\centering
\begin{subfigure}{.5\textwidth}
  \centering
  \includegraphics[width=0.3\textheight]{monoW/simplifiedDM_diagram.pdf}
\end{subfigure}%
\begin{subfigure}{.5\textwidth}
  \centering
  \includegraphics[width=0.3\textheight]{monoW/simplified_tChannel_model_diag.pdf}
\end{subfigure}
  \caption{Feynman diagrams of production of dark matter pairs ($\chi\overline{\chi}$) associated with a W boson in simplified model 
	   in s-channel (left) and t-channel (right) scenarious.}
  \label{fig:feynMonoWSimple}
\end{figure}

% TODO change y-axis title. because it is normalized distribution!!!

\begin{figure}[]

\centering
\begin{subfigure}{.5\textwidth}
  \centering
  \includegraphics[width=0.3\textheight]{monoW/WWxx_diagramm_v2.pdf}
\end{subfigure}%
\begin{subfigure}{.5\textwidth}
  \centering
  \includegraphics[width=0.3\textheight]{monoW/EFT_D5_model_diag_v3.pdf}
\end{subfigure}
  \caption{Representative diagrams for production of dark matter pairs ($\chi\overline{\chi}$) associated with a W boson in models where
dark matter interacts directly with W boson (left) or with quarks (right).}
  \label{fig:feynMonoWEFT}
\end{figure}

Current study cover both s- and t-channel simplified models with different mass of $Z'$ mediator and with different mass of DM particles.
EFT approach are represented by two models which correspond to diagrams which are shown at \FigureRef{fig:feynMonoWEFT}. 
They are charachterized by energy scale of ``new physics'' $\Lambda$ (or effective mass $M^{*}$) as well as mass of produced DM particles.
Both models assume vector type of interaction between DM and SM sectors. For D52 model (which correspond to the right diagram in  \FigureRef{fig:feynMonoWEFT}) 
we consider only constructive interferance with SM.

\subsection{Sensitivity studies}

Main idea of the study is to understand kinematics of DM models in final state with one lepton and missing energy and to estimate sensitivity 
of signal selection to such kind of models.
Signal selection are designed with focus of high-$m_{T}$ region in order to get rid of dominant SM W background.
Kinematic distribution of DM models were studied to evaluate contribution to the signal selection and to make comparison
with signal from $\PWprime$ model.
It's worth to mention that due to the fact that mass of DM particles doesn't affect kinematic distribution of the event 
all simulated DM samples used are with mass of DM particle was set to 1 GeV.
(should some plots to be shown to proof this???).

In \FigureRef{fig:kinematicsSChannel} normalized kinematic distributions of transverse mass of lepton and $E_{T}^{miss}$ of the event
and $E_{T}^{miss}$ of the event for s-channel simplified model as well as EFT models with comparison with $\PWprime$ model are shown.
First distinguishable charachteristic of all DM models is that there is no clear peak structure in any kinematic distribution, 
which is expected because transverse missing energy is formed by
neutrino from W boson decay and two DM particles which are independent between each other. 
Second feature of DM models that main contribution are tends to be in low-$m_{T}$ region.
Especially it concern simplified models, for which dominant contribution is outside of signal region for any parameter of mediator mass.
But with increasing of mediator's mass $m_{T}$ distribution tends to become more flat and moves towards high-$m_{T}$ region.
Also with increasing mediator mass cross section of process significantly decreasing (see \TableRef{tab:TriggerDetails}) 
and for mass equal 10 TeV cross section is a few order of magnitude lower than background (need proof, find plot ???) which 
makes signal selection not sensitive for s-channel simplified DM model.

% TODO reprhase it. It not visible from table - only from non-normalized plot. Plot should be mono-W with W backgound only? Or I should also include other backgounds?

\begin{figure}[]

\begin{subfigure}{.5\textwidth}
  \centering
  \includegraphics[width=\textwidth]{monoW/mT_kinemComparison_simplS_EFT_Wprime.png}
\end{subfigure}%
% \begin{subfigure}{.333\textwidth}
%   \centering
%   \includegraphics[width=0.25\textheight]{monoW/pT_kinemComparison_simplS_EFT_Wprime.png}
% \end{subfigure}
\begin{subfigure}{.5\textwidth}
  \centering
  \includegraphics[width=\textwidth]{monoW/EtMiss_kinemComparison_simplS_EFT_Wprime.png}
\end{subfigure}

% \includegraphics[width=0.75\textheight]{monoW/kinematics_simplifiedSChannel_EFT_Wprime.png}
\caption{Normalized kinematic distributions of transverse mass (left) and transverse missing energy in the event (right) of simplified model in s-channel as well as EFT models compared to $\PWprime$ distribution.}
  \label{fig:kinematicsSChannel}
\end{figure}

In \FigureRef{fig:scaledKin} transverse mass distribution are show for s- and t-channels simplified models as well as EFT and $\PWprime$ signals with comparison to SM W background 
scaled to according cross process section. Distribution for t-channel model looks almost identical to one for s-channel. 
But cross sections for t-channel processes are for one-two order of magnitude lower compared to s-channel (see \TableRef{tab:TriggerDetails}) which leads to conclution
that signal selection even less sensitive for t-channel simplified model. Aslo SM W backround are for a few orders higher in all range of $m_{T}$ than any sample of simplified model.
For D52 EFT model excess over SM W backgound can be seen in high-$m_{T}$ region. While for WW$\chi\chi$ EFT model it is not the case, because cross section is very low.

% TODO us or D5 or D52. harmonize stuff!!!

\begin{figure}[]
\begin{subfigure}{.5\textwidth}
  \centering
  \includegraphics[width=\textwidth]{monoW/dm_final_S_vs_T_channel_pad1.png} 
\end{subfigure}%
% \begin{subfigure}{.33\textwidth}
%   \centering
%   \includegraphics[width=0.95\textwidth]{monoW/pT_kinemComparison_simplT_EFT_Wprime.png}
% \end{subfigure}
\begin{subfigure}{.5\textwidth}
  \centering
  \includegraphics[width=\textwidth]{monoW/dm_final_EFT_vs_SMW_pad1.png}
\end{subfigure}
% \includegraphics[width=0.75\textheight]{monoW/kinematics_simplifiedTChannel_EFT_Wprime.png}
\caption{Kinematic distributions of transverse mass of simplified model in s- and t-channels (left) 
and EFT models with $\PWprime$ signal (right) in comparison with SM W background scaled to according process cross section.}
  \label{fig:scaledKin}
\end{figure}

Transverse mass distribution for both EFT models looks similar and majority (???) of the signal lays in the signal region. 
% Cross section for both processes strongly depends on energy scale of new physics $\Lambda$ (or effective mediator mass for D52 model).
% Dependence of cross section for WW$\chi\chi$ model versus energy scale is shown in \FigureRef{fig:lambdaScan}. 

% TODO describe what is D52 model. That it is vector interaction constructive interference

\begin{table}[]
  \begin{tabular}{r|c|c|c}
    Model 	& Channel 	  & Parameters	    & Cross section, [nb] \\
    \midrule
    Simplified  & s-channel	  & $M_{Z'}$=1TeV    & .05181 \\
		&		  & $M_{Z'}$=100GeV  & .001989 \\
		&		  & $M_{Z'}$=10TeV   & 7.5E-11 \\
		& t-channel	  & $M_{Z'}$=10GeV   & .0018895 \\
		&		  & $M_{Z'}$=100GeV  & 9.2E-5 \\
		&		  & $M_{Z'}$=2TeV    & 4.874E-8 \\
    \midrule
%     EFT 	& WW$\chi\chi$	  & $m_{\chi}$=1GeV; $\Lambda$=3TeV    & 3.6E-10 \\
% 		& D52		  & $m_{\chi}$=1GeV; $M^{*}$=1TeV	& 4.4E-4 \\
EFT 	& WW$\chi\chi$	  & $\Lambda$=3TeV    & 3.6E-10 \\
	& D52		  & $M^{*}$=1TeV	& 4.4E-4 \\
    \midrule	
$\PWprime$ 	& 	  & $m_{\PWprime} =$ 2 GeV   & 1.1E-4 \\    
		%     Wprime ($m_{\PWprime}$=2TeV) & 1.1E-4 \\
  \end{tabular}
  \caption{Mono-W cross section for different theoretical models.}
  \label{tab:TriggerDetails}
\end{table}


\begin{figure}[]
 \includegraphics[width=0.6\textheight]{monoW/WWxx_Wlv_DM1_LambdaScan.pdf}
  \caption{Cross section of WW$\chi\chi$ EFT model as a function of energy scale of new physics, $\Lambda$.}
  \label{fig:lambdaScan}
\end{figure}

\subsection{Validity of EFT approach and summary}

Cross section for both EFT processes strongly depends on energy scale of new physics $\Lambda$ (or effective mediator mass $M^{*}$ for D52 model).
Dependence of cross section for WW$\chi\chi$ model versus energy scale is shown in \FigureRef{fig:lambdaScan}.
In order for WW$\chi\chi$ process have sizeable cross section compared with $\PWprime$ ($M_{\PWprime}$=2TeV) model which is used as a reference in this study, 
energy scale of new physics $\Lambda$ has to be of order 200-300 GeV. But according to ~\cite{arXiv:1512.00476}: 
``The EFT approximation is valid when the momentum transfer in a given
process of interest is much smaller than the mass of the mediating
particle. For momentum transfer larger than or comparable to
$\Lambda$, the EFT description will break down.''
Moment transfer for 13 TeV collisions at LHC correspond to scale of few TeV, 
which mean that $\Lambda$ has to be of order of TeV in order for model to be valid.

% TODO cross check this statement with Dark matter report...
% http://arxiv.org/pdf/1507.00966.pdf
Also for WW$\chi\chi$ model there is no straightforward way to compare the
results with non-collider searches for DM
which make this model less appealing comparing with other models [add non-collider DM production diagram].

D5 constructive (D52) model violate weak gauge invariance as described at ~\cite{arXiv:1503.07874}.
It leads to spurious cross section enhancements at LHC energies. It is not recommended to be used anymore by Dark Matter Forum (some link???).

\subsection{Conclusion}

Transverse mass and $E_{T}^{miss}$ for all presented DM models with comparison with $\PWprime$ ($M_{W'}$ = 2TeV) signal are shown at \FigureRef{fig:scaledKin}.
All distribution are scaled according to the cross section of the process. It's clearly seen that simplified models tends to contribute to low-$m_{T}$ region
outside of the signal selection. With increasing of mediator mass $Z'$ cross section drops significantly and become indistinguishable with SM W background.
On other hand EFT DM samples contribute to high-$m_{T}$ region but as desribed above D52 model has physicaly unmotivated high cross section and WW$\chi\chi$ 
significantly smaller comparing with $\PWprime$ signal for physicaly motivated values of scale of new physics $\Lambda$. Which leads to the conclusion that
lepton and $E_{T}^{miss}$ signal selection is not sensitive for DM searches.



Similar studies was done by Bell and collaborators~\cite{arXiv:1512.00476}, where authors estimated approximate upper limit on for 3000 fb$^{-1}$. 
At \FigureRef{fig:bellExclLim} exclusion limit
as a function of mass of DM particle and mass of DM-SM mediator $Z'$ is shown. One can notice that obtained limits for mono-lepton channel are significantly 
worse than from all other channel, especially than from di-jet analysis. Conclusion of authors are identical to the conclusion of this study that mono-lepton channel
are not sensitive enough for DM searches and is significantly worse comparing to all other hadronic channels.

\begin{figure}[]
 \includegraphics[width=0.8\textwidth]{monoW/schan1.pdf}
  \caption{Exclusion limit for the s-channel $Z'$ model as a function of mass of dark matter particle, $m_{\chi}$, 
  and mass of DM-SM mediator, $m_{Z'}$, reported in~\cite{arXiv:1512.00476}.
  Exclusions are shown as shaded regions for LUX and for mono-jet and di-jets at 8 TeV, 
  and the reaches are shown for the mono lepton and mono fat jet searches at 14 TeV 3000 fb$^{-1}$.}
  \label{fig:bellExclLim}
\end{figure}

\section{Outlook}
\label{sec:wprimeConclusion}

  %% To ignore a specific chapter while working on another, making the build faster, comment it out:
  %\chapter{Same-sign dilepton}
\label{chap:SS}
\section{Introduction}

\section{Background processes}
\label{sec:wprimeBackgrounds}

\section{Event Selection}
\label{sec:wprimeEventSelection}

\section{Data driven multijet background estimate}
\label{sec:wprimeMultijetBackground}

\section{Signal Region}
\label{sec:wprimeSignalRegion}

\section{Systematic Uncertainties}
\label{sec:wprimeSystematics}

\section{Conclusion}
\label{sec:wprimeConclusion}


\end{mainmatter}

%% Produce the appendices
% \begin{appendices}
%   %% The "\appendix" call has already been made in the declaration
%% of the "appendices" environment (see thesis.tex).
\chapter{Extra}
\label{app:Extra}

\section{Possible bibliography to use}
TRT:\\
~\cite{CERN-THESIS-2006-025} \\

LUCID:\\


SSdileptom:\\

$\PWprime$:\\
~\cite{arXiv:1503.07874} \\
~\cite{arXiv:1512.00476} \\
~\cite{arXiv:1506.03116} \\

%% Big appendixes should be split off into separate files, just like chapters
%\input{app-myreallybigappendix}

% \end{appendices}

%% Produce the un-numbered back matter (e.g. colophon,
%% bibliography, tables of figures etc., index...)
\begin{backmatter}
  
%% You're recommended to use the eprint-aware biblio styles which
%% can be obtained from e.g. www.arxiv.org. The file mythesis.bib
%% is derived from the source using the SPIRES Bibtex service.
\bibliographystyle{h-physrev}
\bibliography{mythesis}

%% I prefer to put these tables here rather than making the
%% front matter seemingly interminable. No-one cares, anyway!
\listoffigures
\listoftables

%% If you have time and interest to generate a (decent) index,
%% then you've clearly spent more time on the write-up than the 
%% research ;-)
%\printindex

\end{backmatter}

%% Close
\end{document}
