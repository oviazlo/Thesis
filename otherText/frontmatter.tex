%% Title
\titlepage[Lund University]{%
  A dissertation submitted to the Lund University\\ for the degree of Doctor of Philosophy}

% Abstract
\begin{abstract}%[\smaller \thetitle\\ \vspace*{1cm} \smaller {\theauthor}]
  %\thispagestyle{empty}
  
  This thesis covers four different topics related to the physics analysis in the ATLAS experiment that uses proton-proton collisions data provided by the Large Hadron Collider.
  
  The first topic is focused on the numerical simulation of the Transition Radiation Tracker which is one of the ATLAS tracking detectors.
  The implementation of the alternative Argon based gas mixtures used in the drift tubes is described.
  A performance study with focus on the hit and track parameters with respect to the gas mixture is discussed as well.
  
  The second topic is related to the ATLAS luminosity monitor called LUCID.
  A number of studies are presented from the design phase of the detector and of its calibration system as well as detector operation, the performance and luminosity measurements.
  
  The two last topics are analyses that searches for beyond Standard Model physics with the ATLAS detector.
  The first search presented in the thesis is done in a final state with same-sign electron pairs using the data collected at a center of mass energy of $\sqrt{s} = $8~TeV. 
  No significant excess above the predictions of the Standard Model is observed.
  Limits on the fiducial cross section for new physics as a function of the invariant mass of electron pairs have been set as well as mass limits for doubly charged Higgs models.
  
  The second analysis is focused on a search for a new heavy charged gauge boson in the final state with one lepton and missing transverse momentum at $\sqrt{s} = $13~TeV.   
  Since no significant deviation from the Standard Model is observed, limits on a new heavy charged gauge boson mass are set.

\end{abstract}


% Acknowledgements
\begin{acknowledgements}
  
During my PhD studies, a lot of people were guiding, helping and supporting me. Here I would like to express my sincere gratitude to all of them.

Firstly, this thesis would be impossible without my supervisor, Else Lytken, whom I am really grateful to for giving me a chance to pursue a PhD and helping me along the whole way. Also, I really would like to thank for allowing me to work on many different topics, starting from the physics data analysis and ending with detector development and operation. This gave me a broad experience in the field and made my PhD studies very interesting.
Thanks to my co-supervisor Oxana Smirnova for always finding a time to answer my questions, for support and proofreading of the thesis.
Also, I am very grateful to Monika Wielers for guidance in the analysis and comments to the thesis as well as tips about the future career planning and recommendations of skiing and hiking places in The Alps.  
Thanks to Peter Christiansen and Will Kalderon for reading my thesis and sharing their opinion about it. 

In the middle of my studies, I joined LUCID project. I obtained a large amount of hardware experience while working together with the LUCID group.
I would like to thank my co-supervisor Vincent Hedberg for teaching me many new things and spending many hours proofreading and commenting this thesis. Also, I am grateful for giving me a possibility to take part in the detector development starting from the design phase and ending with entrusting me the role of the LUCID Run Coordinator. Thanks to Davide Caforio for support and for your initial push which led me to involvement in the detector operation activities. Thanks to Carla Sbarra, Benedetto Giacobbe, Giulio Avoni and others for your explanations and discussions. It was incredibly interesting to work together with you on LUCID.

Also, I would like to thank many people from the TRT community. Alejandro Alonso whom I always considered my inofficial TRT supervisor. He was always glad to help even though he was extremely overloaded by his duties as TRT software convener. 
Anatoli Romaniouk for his guidance and comments,
Andrew Beddall for sharing his knowledge about the TRT simulation package, my TRT Argon buddy Konstantin Vorobev for interesting discussions, and many other people. 

I would like to thank Torsten \AA{}kesson for discussions about my future career and organization of many interesting PhD-schools.
Thanks to senior ATLAS PhD students Anthony Hawkins, Anders Floderus and Lene Bryngemark for sharing their experience with me, as well as to a young generation of ATLAS students Trine Poulsen, Edgar Kellermann, Katja Karppinen, Eric Corrigan, Eva Hansen -- you look like a great team!
Also thanks to Martin Ljunggren, Tuva Richert, Vytautas Vislavicius, Ben Folsom and other students for great discussions, board games and nice time at summer schools.
And very big thanks to everyone at the particle physics division -- it was a pleasure to work together with you!

Also, I would like to thank all people who always make my stays at CERN fun and interesting: Seppo Heikkila, Sergey Senyukov, Olena Karacheban, Stanislav Suchek, Misha Lisovyi, Illia Khvastunov, Pavlo Svirin, Evgeny Soldatov and many others.
And my friends who were always motivating me to keep going: Oleh Kivernyk, Misha Rakov, Borys Movchan, Maryna Kozubska, Olga Gogota, Oleksandr Volynets, Olena Bachynska, Mykola Savitskyi, Ganna Dolinska, Ievgen Korol, Oleksandr Kondratyk and many many others.

Last but not the least, I want to thank my family for the infinite amount of their support.

\end{acknowledgements}


%% Preface
\begin{preface}

The studies reported in this thesis were performed during my work with the \ATLAS experiment at the Large Hadron Collider (\LHC) in 2012-2016. All the studies are based on data collected by the \ATLAS experiment or related to the \ATLAS detector performance.
\ATLAS is a multi-purpose detector which is registering results of hadron collisions at the \LHC. It consists of a set of subdetectors which perform dedicated tasks providing specific information about particles created in collisions.
While LHC can collide both proton and lead beams, this thesis focuses on proton-proton collision data.

% ***TRT***
The first study concerns one of the \ATLAS tracking subdetectors -- the Transition Radiation Tracker (TRT).
It uses a gas mixture as an active detector volume to detect charged particles. The TRT was designed to use only one type of gas mixture, based on Xenon. However, some parts of the detector started to leak after few years of running. Thus, to decrease the operational costs of the detector, it was decided to switch to a cheaper gas mixture based on Argon. My task was to rewrite the TRT simulation software, used for Monte Carlo (MC) simulation of physics processes, to support the new gas mixture. It was a very challenging yet interesting task, because I had to review the whole chain of the numerical simulation of the TRT detector and add support for the new gas in each step. After the implementation was completed, I did a performance study of the TRT with focus on the hit and track parameters for the Argon and Xenon gas mixtures.

% ***LUCID***
Later in my PhD studies in 2014, I joined the LUCID group, which was designing the new LUCID-2 detector. My involvement with LUCID lasted for more than two years. The LUCID detector is a luminosity detector which measures the collision rate of the hadron-hadron interactions in ATLAS. The luminosity measurements are crucial because luminosity is used for all analyses which measure or put limits on cross sections of processes.
During the design phase, many tests were done to find an optimal design and the optimal parameters for various detector components.
During the assembly and installation phase, a number of tests were done to make sure that all components performed as they should.
An overall test of the system was done to make sure that no damage has been done during the installation of the detector.
During the operation phase, which is still ongoing, many studies have been made to understand the performance of the calibration system and the detector.

I contributed to all the steps mentioned above. I took part in the development of the LUCID design and particularly the design of the calibration system. 
I made a series of tests to find the optimal design parameters of the LED and laser diffusers used to distribute LED and laser signals evenly and deliver them to all 
% TODO ``detector channels'' - maybe rephrase?
detector channels. 
I spent a lot of time investigating the behavior of the LED system as well as the PMT and PIN-diode signal behavior.
I also conducted tests with Bi-207 radioactive sources which are used to monitor the photomultiplier (PMT) gain.
I participated in the detector assembly in the clean room and did detector testing during this process.
Testing of the LED and laser diffusers were done to cross-check the integrity of fibers and the homogeneity of signals between all PMTs.
Also, a temperature stress test was performed in order to understand what maximum temperature can be tolerated without destroying the detector during the 
beampipe bake-out procedure.
In the operational phase, the main focus was on understanding the aging of the PMTs and the possibility of improving the calibration system.
Lastly, I became the ATLAS Forward Detectors Run Coordinator for five months, where my task was to ensure the successful operation of LUCID and to verify the correctness of the measured luminosity during the ATLAS data taking.

% The LUCID group published a paper with a description of the choice and the characterization of photomultipliers for the new LUCID detector for Run-2~\cite{Alberghi:2016tad}.
% My contribution was in understanding and developing the monitoring using a Bi-207 source.

% ***Same-Sign***
During the second half of my studies, I was doing searches for new physics beyond the Standard Model (BSM) in final states with isolated leptons. The first search I did was an inclusive search with same-sign lepton pairs. The analysis was performed in three channels: $e^{\pm}e^{\pm}$, $\mu^{\pm}\mu^{\pm}$ and $e^{\pm}\mu^{\pm}$. My task was to perform a complete analysis with the $e^{\pm}e^{\pm}$ channel and to make cross-checks for other channels. To achieve this, I was doing event and lepton selections, a study of the MC inputs, testing of the charge flip and non-prompt background modelling, optimization of the selection criteria, verification of prompt background predictions, estimation of the systematic uncertainties and provision of final numbers for the limit settings. Also, I was studying the hypothetical doubly charged Higgs signal, calculating the total efficiency of the signal selection and making the bin width optimization for the limit setting for the doubly charged Higgs.

% ***WPRIME***
The second analysis was a search for a new heavy spin-1 gauge boson, namely, $\PWprime$. The final state of interest was a high momentum lepton and significant missing transverse momentum observed in the detector (which corresponds to a particle which escapes the detector without interacting with it, for example, a neutrino). The analysis was performed in the electron and muon channels.
I was working on the muon channel, making an event and lepton selection, studying the behavior of the missing transverse momentum, testing fake background predictions and performing cross-checks both in muon and electron channels. Also, I was working on a production of the new MC simulation samples of the top and diboson backgrounds which would provide enough statistics in the whole signal region and significantly reduce the dominant systematic uncertainty. However, the problem appeared to be much deeper than expected, thus it was not solved completely in time for the publication and I had to hand it over to the analysis team which started working with the new 2016 data.
I also made a study of the signal selection sensitivity to the so-called Simplified Dark Matter models, which are the recommended 
% benchmark models for the current running period searches at the LHC. 
benchmark models for searches during the current running period at the LHC.

The results of the research activities described above were included in the publications:
\begin{itemize}
\item G.~L. Alberghi {et~al.}, {\em {Choice and characterization of photomultipliers
  for the new ATLAS LUCID detector}},
  \href{http://dx.doi.org/10.1088/1748-0221/11/05/P05014}{JINST {\bfseries 11}
  no.~05, (2016) P05014}.
\item {ATLAS} Collaboration, G.~Aad {et~al.}, {\em {Search for anomalous production
  of prompt same-sign lepton pairs and pair-produced doubly charged Higgs
  bosons with $ \sqrt{s}=8 $ TeV $pp$ collisions using the ATLAS detector}},
  \href{http://dx.doi.org/10.1007/JHEP03(2015)041}{JHEP {\bfseries 03} (2015)
  041},
  \href{http://arxiv.org/abs/1412.0237}{{\ttfamily arXiv:1412.0237 [hep-ex]}}.
\item {ATLAS} Collaboration, M.~Aaboud {et~al.}, {\em {Search for new resonances in
  events with one lepton and missing transverse momentum in $pp$ collisions at
  $\sqrt{s} = 13$ TeV with the ATLAS detector}},
  \href{http://arxiv.org/abs/1606.03977}{{\ttfamily arXiv:1606.03977 [hep-ex]}}.
\end{itemize}

Also, I presented my results at two international conferences with published proceedings:
\begin{itemize}
 \item O.~Viazlo, {\em {ATLAS LUCID detector upgrade for LHC Run 2}},
PoS {\bfseries EPS-HEP2015} (2015) 275.
 \item O.~Viazlo, {\em {Searches for new physics in high-mass
  fermionic final states and jets with the ATLAS detector at the LHC}},
PoS {\bfseries DIS2016} (2016) 109.
\end{itemize}












%% ***WPRIME***
% This analysis have been made public as a conference note~\cite{ATLAS-CONF-2015-063} with early reasults and later as a paper~\cite{Aaboud:2016zkn} where final results were presented.
% 
% \toDo[carefully revisit and rewrite this]
% 
% My personal contribution to the analysis can be concluded in following three parts.
% I was involved in the analysis of muon channel. Signal selection, muon and missing transverse momentum performance, validation plots and estimation of some systematic uncertainties.
% This part was done in parrallel with other collaborator to make sure that results are robust and are the same from two independent analyses codes and there are no mistaken done.
% 
% SECOND WPRIME: MT BINNED MC
% Secondly I was investigating ways to decrease systematic uncertainty caused by the limited available statistic of inclusive diboson and top backgound samples at high-$m_{T}$ region.
% One option was to generate lepton-neutrino mass-binned samples. However, it was found out that these samples doesn't populate enough high-$m_{T}$ region
% so it was decided to make samples in bins of lepton-$E_T^{miss}$ $m_T$.
% This is still ongoing project and it is planned to finish these samples for paper of the $\PZprime$ and $\PWprime$ results from 2015-2016.
% 

% My second contribution was related to the design of the $m_T$-binned diboson and top background samples. Since the main source of the systematic uncertainty is caused by the limited statistic of the available inclusive samples for the top and diboson processes it was crucially important to make it ready this samples.
% However simple in the beginning problem became more and more deeper and harder and 
% 
% I wrote custom selector for the Sherpa generator which allow to make filters on the transverse mass of the lepton momentum plus missing transverse momentum variable (all definitions are done in the body text)
% 
% However months were spent for the debugging of the problem 
% and I had to hand it over to other people to finish due to starting work on the writing this thesis.

% THIRD WPRIME: MONO-W
% The third part was related to the investigation of sensitivity of the analysis signal selection for the so-called Dark Matter simplified models, which are recommended benchmark models for Run-2 searches at LHC.
% In these models a pair of DM particles candidates are produced in final state in association with SM W boson.
% The study was done in a way of comparison of the signals from so-called Effective Field Theory (used for Run-1 search) and $\PWprime$ models signals with signals from simplified models. 




  
\end{preface}

%% ToC
\tableofcontents

%% Strictly optional!
% \frontquote{%
%   Writing in English is the most ingenious torture\\
%   ever devised for sins committed in previous lives.}%
%   {James Joyce}
%% I don't want a page number on the following blank page either.
% \thispagestyle{empty}
